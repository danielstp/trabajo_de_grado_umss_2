\chapter{Introducción}\label{cap:intro}
\pagenumbering{arabic}
\section{Antecedentes}
El  Programa de Alimentos y Productos Naturales (PAPN) fue creado el 13 de
febrero de 1987 como un programa, pero rápidamente se convirtió, por iniciativa
y esfuerzo de su dirección y sus miembros, en un centro superior de
investigación. Este centro depende de la Facultad de Ciencias y Tecnología de la
Universidad Mayor de San Simón (UMSS) y esta situado en el campus principal.
Dentro de sus actividades de ciencia y tecnología en el campo agroalimentario y
de productos naturales, la UMSS ejecuta la investigación a través de convenios y
proyectos en diferentes ámbitos del desarrollo regional y nacional, a través del
centro superior PAPN.

De acuerdo a posibilidades y disponibilidades, se han incorporado al PAPN
practicas sobre secado y análisis sensorial para los estudiantes de ingeniería
industrial y alimentos, así como se han realizado diversos estudios en el secado
de alimentos andinos y el diseño de secadores solares, convencional, deseando
implementar 2 nuevos tipos de deshidratadores: cama fluidizada. y spray.

En este sentido, para mejorar el estudio y la investigación del secado de
alimentos es que este centro precisa la construcción de tres secadores
experimentales automatizados que permitan obtener datos de las variables de
secado de diferentes tipos de alimentos de la forma más rápida y precisa.

El propósito de un control automático en un sistema es producir una salida
deseada cuando las entradas del sistema son modificadas. Estas entradas se
modifican por señales de mando, y también por perturbaciones, que se espera que
el control automático minimice.

%\pagebreak % poner esto al final de cada seccion
\section{Identificación del problema}

Es complicado analizar todas las variables que intervienen en el proceso de
deshidratado, para que, dado un producto que se desea secar; se pueda escoger el
mejor proceso para este producto.

\subsection{Definición del problema}

¿Como mejorar la obtención y análisis de las variables de secado, de distintos
tipos de alimentos, por medio del control de las variables de entrada?

\section{Objetivos}

\subsection{Objetivos General}
\begin{itemize}
  \item Construir un sistema de control automático para el proceso de
        deshidratación experimental de alimentos.
  %\item blah blah blah blah.

\end{itemize}

\subsection{Objetivos Específicos}
Los objetivos a cumplir durante el desarrollo del proyecto son:
\begin{itemize}
  \item Determinar el modelo del sistema y de control.
  \item Seleccionar sensores, controladores y actuadores a utilizar.
  \item Desarrollar el modulo para funcionamiento básico del microcontrolador.
  \item Diseñar el modelo unificado el sistema electrónico y el sistema de
        información.
  \item Desarrollar el modulo de control supervisado.
  \item Desarrollar el modulo de cambio de parámetros de control.
  \item Implementar el control automático.
\end{itemize}

\section{Alcance}
El proyecto tendrá el siguiente alcance:
\begin{itemize}
  \item Sistema electrónico de control.
  \item Sistema informático de análisis y seguimiento del proceso de secado.
\end{itemize}

%\pagebreak % poner esto al final de cada seccion
\section{Justificación}
La desecación de vegetales por paso de aire caliente sobre superficie húmeda,
representa la forma mas común de secado de vegetales destinados a la
alimentación, siendo uno de los procesos mas antiguos de conservación que se
conocen.

Mediante este método es posible preservar el color, los principios activos otras
características de calidad que se pierden por acción de las encimas después de
la cosecha, permitiendo de esta manera, una mejor conservación por tiempos
relativamente largos sin perdidas de estas características.

Ante la necesidad de conocer las distintas cualidades y capacidades de los
distintos tipos de deshidratadores, y de obtener los cambios de sus variables
durante el proceso de deshidratación de manera rápida y casi desatendida por una
persona se ve que es conveniente la construcción de un sistema de control
automático para cada tipo de deshidratador que se cuenta en la planta piloto.


