El ciclo inicial empiezo el día Viernes 26 de Septiembre del 2014. Este ciclo
consta de 2 semanas para completar las historias que serán definidas en la
planificación del sprint. El ciclo concluirá el Jueves 9 de Octubre del 2014.

%\begin{landscape}
\subsection{Pila del Ciclo (Sprint backlog)}
En este ciclo se eligió trabajar en las características 1 y 2 de la pila de
productos, por el hecho de tener alta prioridad. Cada característica se divide
en historias de usuario, y estas son estimadas con puntos de historia. En los
Cuadros \ref{table:eus1} y \ref{table:eus2} se definen las estimaciones. Para
cada historia de usuario se definen criterios de aceptación, como se muestra en
el Cuadro~\ref{table:criteriosSA}.

Cada historia de usuario se divide en tareas, y se definen en los Cuadros
\ref{table:tareasUS1}, \ref{table:tareasUS2} y \ref{table:tareasUS3_1}.

%--------------------------------------------------------------------
\begin{table}[ht]
\centering
\begin{tabular}{|l|p{6cm}|c|p{5cm}|r|}
\hline
\textbf{No.} & \textbf{Feature} & \textbf{Id.} & \textbf{Historia de usuario} & \textbf{Estimación} \\
\hline
1 & \pbtemp & US1 & Controlar Temperatura. & 3 \\
\hline
\end{tabular}
\caption{Estimación de US1}
\label{table:eus1}
\end{table}

%--------------------------------------------------------------------

\subsection{Demostración de fin de sprint}


\subsection{Gráfico burn down del sprint}
La Figura~\ref{fig:sprintA} muestra el gráfico \emph{burn down} del presente
sprint.

\subsection{Retrospectiva del sprint}
\begin{itemize}
  \item ¿Qué salió bien?
    \begin{itemize}
        \item Se completaron la mayoria de las tareas de la iteración.
        \item Se escribieron pruebas de unidad para el código desarrollado.
    \end{itemize}

  \item ¿Qué podría haber sido mejor?
    \begin{itemize}
        \item La estimación de la historia US3 no fue buena.
    \end{itemize}

  \item ¿Qué  se puede mejorar en el futuro?
    \begin{itemize}
        \item Mejorar las estimaciones de las historias, si la historia es muy larga es mejor dividirla.
    \end{itemize}
\end{itemize}

