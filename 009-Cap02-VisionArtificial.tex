\chapter{Visi�n Artificial}
\section{Introducci�n}

blah blah blah blah blah blah blah blah blah blah blah blah blah blah blah blah blah blah blah blah blah blah blah blah blah blah blah blah blah blah blah blah blah blah blah blah blah blah blah blah blah blah blah blah blah blah blah blah blah blah blah blah blah blah blah blah blah blah blah blah blah blah blah blah blah blah blah blah blah blah blah blah blah blah blah blah blah blah blah blah blah blah blah blah blah blah blah blah blah blah blah blah blah blah blah blah.

\section{Componentes de un sistema de visi�n}

La visi�n artificial es un tema complejo, por lo tanto es necesario dividirlo en varios componentes:
%\cite{comp_cv}
\begin{itemize}
\item \textbf{Una fuente de radiaci�n} puede ser la luz visible. La luz visible esta formada por radiaci�n electromagn�tica cuyas longitudes de onda est�n comprendidas entre 390 y 700 nm. \cite{luz_visible} Si la radiaci�n no es emitida desde la escena o del objeto inter�s, nada puede ser observado o procesado. Por lo tanto es necesaria una apropiada iluminaci�n para objetos que por si mismos no son luminosos.

\item \textbf{Una c�mara} obtiene la proyecci�n plana de una imagen externa sobre el sensor.\cite{camara} En el caso mas simple la c�mara solo estar�a compuesto de lentes �pticos.

\item \textbf{El sensor} convierte la radiaci�n colectada en se�ales el�ctricas para luego procesarlas.

\item \textbf{La unidad} de procesamiento procesa las se�ales colectadas, extrayendo caracter�sticas adecuadas que pueden ser usadas para medir las propiedades de un objeto y categorizarlo en clases.

\item \textbf{Los actores} reaccionan al resultado de la observaci�n visual. Ellos llegan a ser una parte integral del sistema de visi�n cuando el sistema de visi�n esta respondiendo activamente a la observaci�n, por ejemplo, rastreando un objeto de inter�s o usando navegaci�n guiada por visi�n.
\end{itemize}

\section{Aplicaciones de la visi�n artificial}

\section{Problemas t�picos en la visi�n artificial}

\section{Conclusiones}
blah blah.
