\chapter{Control Automático}
\section{Introducción}

El control es un área de la ingeniería y forma parte de la Ingeniería de
Control. Se centra en el control de los sistemas dinámicos mediante el principio
de la retroalimentación, para conseguir que las salidas de los mismos se
acerquen lo más posible a un comportamiento predefinido. Esta rama de la
ingeniería tiene como herramientas los métodos de la teoría de sistemas
matemáticos.

Las bases de esta ingeniería se sentaron a mediados del Siglo XX a partir de la
cibernética. Sus principales aportaciones corresponden a Norbert Wiener, Rudolf
Kalman y David G. Luenberger.

La ingeniería de control es una ciencia interdisciplinar relacionada con muchos
otros campos, principalmente las matemáticas y la informática. Las aplicaciones
son de lo más variado: desde tecnología de fabricación, instrumentación médica,
Subestación eléctrica, ingeniería de procesos, robótica hasta economía y
sociología. Aplicaciones típicas son, por ejemplo, el piloto automático de
aviones y barcos y el Sistema Anti-Bloqueos de frenos de los automóviles(ABS).
En la biología se pueden encontrar también sistemas de control retro-alimentados
,como por ejemplo el habla humana, donde el oído recoge la propia voz para
regularla.

El control de temperatura en una habitación, es un ejemplo claro y típico de una
aplicación de ingeniería de control. El objetivo es mantener la temperatura de
una habitación en un valor deseado, aunque la apertura de puertas y ventanas; y
la temperatura en el exterior hagan que la cantidad de calor que pierde la
habitación sean variables (perturbaciones externas). Para alcanzar el objetivo,
el sistema de calefacción debe modificarse para compensar esas perturbaciones.
Esto se hace a través del termostato, que mide la temperatura actual y la
temperatura deseada, y modifica la temperatura del agua del sistema de
calefacción para reducir la diferencia entre las dos temperaturas.

\section{Teoría de Control}
La teoría de control es un campo interdisciplinario de la ingeniería y las
matemáticas, que trata con el comportamiento de sistemas dinámicos. A la entrada
de un sistema se le llama referencia. Cuando una o más variables de salida de un
sistema necesitan seguir cierta referencia sobre el tiempo, un controlador
manipula la entrada al sistema para obtener el efecto deseado en la salida del
sistema (retroalimentación). La retroalimentación puede ser negativa (regulación
autocompensatoria) o positiva (efecto «bola de nieve» o «círculo vicioso»). Es
de gran importancia en el estudio de la ecología trófica y de poblaciones.

\section{Control Supervisado}
Un sistema de control supervisado (DCS) ha sido desarrollado para resolver la
adquisición de grandes volúmenes de información, su tratamiento en centros
de supervisión y mando, y la actuación en tiempo real sobre el proceso a
controlar.

Se trata de un sistema abierto, que permite la integración con equipos de otros
fabricantes que realicen funciones especificas, y hace la función de canalizador
de datos recogidos para, a través de líneas de comunicaron de alta velocidad,
ponerlos a disposición de los usuarios de planta.

El sistema está especialmente recomendado para llevar acabo la supervisión de
plantas de diferentes procesos que en ellas se desarrollan permitiendo a los
usuarios disponer de una información precedente de distintos puntos del proceso
A su vez este sistema dispone de módulos de software para la resolución de
problemas específicos dentro de las plantas, como pueden ser el cálculo de
rendimientos, calculo de consumos o el modulo de mantenimiento.

\section{Sistema de control distribuido}

El desarrollo de los microprocesadores, micro controladores y controladores
lógicos programables (PLC’s) dio lugar a la aparición del control distribuido.
En este tipo de esquema un PLC o µP controla una o más variables del sistema
realizado un control directo de las mismas. Estos equipos de control local se
comunican con otros elementos de su nivel y con el nivel superior de
supervisión. Cabe aclarar q el control distribuido son aplicaciones de control
de arquitectura distribuida.

En este hay varias unidades de control que llevan a cabo las tareas. En este
caso de avería o sobrecarga de trabajo, será `posible transferir parte o todo
el trabajo que desarrollaba la misma a otras unidades

\section{Lazo abierto y cerrado}

Existen dos tipos de sistemas principalmente: los de lazo abierto o no
realimentados y los de lazo cerrado o realimentados. Los sistemas de lazo
cerrado funcionan de tal manera que hacen que la salida vuelva al principio para
que se analice la diferencia con un valor de referencia y en una segunda opción
la salida se vaya ajustando, así hasta que el error sea 0. Cualquier sistema que
tenga como objeto controlar una cantidad como por ejemplo temperatura,
velocidad, presión, caudal, fuerza, posición, etc. son normalmente de lazo
cerrado. Los sistemas de lazo abierto no se comparan a la variable controlada
con una entrada de referencia. Cada ajuste de entrada determina una posición de
funcionamiento fijo en los elementos de control (por ejemplo con temporizadores).

\begin{figure}[h]
  \centering
  %LaTeX with PSTricks extensions
%%Creator: inkscape 0.91
%%Please note this file requires PSTricks extensions
\psset{xunit=.5pt,yunit=.5pt,runit=.5pt}
\begin{pspicture}(500,126)
{
\newrgbcolor{curcolor}{1 1 1}
\pscustom[linestyle=none,fillstyle=solid,fillcolor=curcolor]
{
\newpath
\moveto(155.64737859,111.10280239)
\lineto(239.63309221,111.10280239)
\lineto(239.63309221,71.18201285)
\lineto(155.64737859,71.18201285)
\lineto(155.64737859,111.10280239)
\closepath
}
}
{
\newrgbcolor{curcolor}{0 0 0}
\pscustom[linewidth=2.49269314,linecolor=curcolor]
{
\newpath
\moveto(155.64737859,111.10280239)
\lineto(239.63309221,111.10280239)
\lineto(239.63309221,71.18201285)
\lineto(155.64737859,71.18201285)
\lineto(155.64737859,111.10280239)
\closepath
}
}
{
\newrgbcolor{curcolor}{0 0 0}
\pscustom[linewidth=2.49269314,linecolor=curcolor]
{
\newpath
\moveto(90.89855651,91.14240762)
\lineto(147.26193413,91.14240762)
}
}
{
\newrgbcolor{curcolor}{0 0 0}
\pscustom[linestyle=none,fillstyle=solid,fillcolor=curcolor]
{
\newpath
\moveto(152.86519575,91.14240762)
\lineto(145.39418026,87.39983296)
\lineto(147.26193413,91.14240762)
\lineto(145.39418026,94.884981)
\lineto(152.86519575,91.14240762)
\closepath
}
}
{
\newrgbcolor{curcolor}{0 0 0}
\pscustom[linewidth=2.49269314,linecolor=curcolor]
{
\newpath
\moveto(152.86519575,91.14240762)
\lineto(145.39418026,87.39983296)
\lineto(147.26193413,91.14240762)
\lineto(145.39418026,94.884981)
\lineto(152.86519575,91.14240762)
\closepath
}
}
{
\newrgbcolor{curcolor}{0 0 0}
\pscustom[linestyle=none,fillstyle=solid,fillcolor=curcolor]
{
\newpath
\moveto(85.91787867,96.13250631)
\curveto(92.96088145,96.13250631)(92.96088145,86.15230893)(85.91787867,86.15230893)
\curveto(78.87489999,86.15230893)(78.87489999,96.13250631)(85.91787867,96.13250631)
}
}
{
\newrgbcolor{curcolor}{0 0 0}
\pscustom[linewidth=2.49269314,linecolor=curcolor]
{
\newpath
\moveto(85.91787867,96.13250631)
\curveto(92.96088145,96.13250631)(92.96088145,86.15230893)(85.91787867,86.15230893)
\curveto(78.87489999,86.15230893)(78.87489999,96.13250631)(85.91787867,96.13250631)
}
}
{
\newrgbcolor{curcolor}{0 0 0}
\pscustom[linewidth=0.24926931,linecolor=curcolor]
{
\newpath
\moveto(85.91787867,96.13250631)
\curveto(92.96088145,96.13250631)(92.96088145,86.15230893)(85.91787867,86.15230893)
\curveto(78.87489999,86.15230893)(78.87489999,96.13250631)(85.91787867,96.13250631)
}
}
{
\newrgbcolor{curcolor}{0 0 0}
\pscustom[linewidth=2.49269314,linecolor=curcolor]
{
\newpath
\moveto(1.24634663,91.14240762)
\lineto(72.5517551,91.14240762)
}
}
{
\newrgbcolor{curcolor}{0 0 0}
\pscustom[linestyle=none,fillstyle=solid,fillcolor=curcolor]
{
\newpath
\moveto(78.15501799,91.14240762)
\lineto(70.68400123,87.39983296)
\lineto(72.5517551,91.14240762)
\lineto(70.68400123,94.884981)
\lineto(78.15501799,91.14240762)
\closepath
}
}
{
\newrgbcolor{curcolor}{0 0 0}
\pscustom[linewidth=2.49269314,linecolor=curcolor]
{
\newpath
\moveto(78.15501799,91.14240762)
\lineto(70.68400123,87.39983296)
\lineto(72.5517551,91.14240762)
\lineto(70.68400123,94.884981)
\lineto(78.15501799,91.14240762)
\closepath
}
}
{
\newrgbcolor{curcolor}{0 0 0}
\pscustom[linewidth=2.49269314,linecolor=curcolor]
{
\newpath
\moveto(394.03412558,91.14240762)
\lineto(481.64834187,91.14240762)
}
}
{
\newrgbcolor{curcolor}{0 0 0}
\pscustom[linestyle=none,fillstyle=solid,fillcolor=curcolor]
{
\newpath
\moveto(487.25160349,91.14240762)
\lineto(479.780588,87.39983296)
\lineto(481.64834187,91.14240762)
\lineto(479.780588,94.884981)
\lineto(487.25160349,91.14240762)
\closepath
}
}
{
\newrgbcolor{curcolor}{0 0 0}
\pscustom[linewidth=2.49269314,linecolor=curcolor]
{
\newpath
\moveto(487.25160349,91.14240762)
\lineto(479.780588,87.39983296)
\lineto(481.64834187,91.14240762)
\lineto(479.780588,94.884981)
\lineto(487.25160349,91.14240762)
\closepath
}
}
{
\newrgbcolor{curcolor}{0 0 0}
\pscustom[linewidth=2.49269314,linecolor=curcolor]
{
\newpath
\moveto(442.03638616,91.14240762)
\lineto(442.03638616,26.27112462)
\lineto(327.71425014,26.27112462)
}
}
{
\newrgbcolor{curcolor}{0 0 0}
\pscustom[linestyle=none,fillstyle=solid,fillcolor=curcolor]
{
\newpath
\moveto(322.11098852,26.27112462)
\lineto(329.58200402,30.013698)
\lineto(327.71425014,26.27112462)
\lineto(329.58200402,22.52854997)
\lineto(322.11098852,26.27112462)
\closepath
}
}
{
\newrgbcolor{curcolor}{0 0 0}
\pscustom[linewidth=2.49269314,linecolor=curcolor]
{
\newpath
\moveto(322.11098852,26.27112462)
\lineto(329.58200402,30.013698)
\lineto(327.71425014,26.27112462)
\lineto(329.58200402,22.52854997)
\lineto(322.11098852,26.27112462)
\closepath
}
}
{
\newrgbcolor{curcolor}{1 1 1}
\pscustom[linestyle=none,fillstyle=solid,fillcolor=curcolor]
{
\newpath
\moveto(235.33823165,46.23151939)
\lineto(319.32394528,46.23151939)
\lineto(319.32394528,6.31072985)
\lineto(235.33823165,6.31072985)
\lineto(235.33823165,46.23151939)
\closepath
}
}
{
\newrgbcolor{curcolor}{0 0 0}
\pscustom[linewidth=2.49269314,linecolor=curcolor]
{
\newpath
\moveto(235.33823165,46.23151939)
\lineto(319.32394528,46.23151939)
\lineto(319.32394528,6.31072985)
\lineto(235.33823165,6.31072985)
\lineto(235.33823165,46.23151939)
\closepath
}
}
{
\newrgbcolor{curcolor}{0 0 0}
\pscustom[linestyle=none,fillstyle=solid,fillcolor=curcolor]
{
\newpath
\moveto(259.73090666,32.38204316)
\lineto(259.73090666,31.07604077)
\curveto(259.21046437,31.31969855)(258.72407,31.50487737)(258.27172355,31.62183281)
\curveto(257.81451222,31.73878825)(257.37675791,31.79726597)(256.94873086,31.79726597)
\curveto(256.21913868,31.79726597)(255.65492246,31.65107105)(255.25607845,31.36842936)
\curveto(254.85237082,31.0809136)(254.65294944,30.6764424)(254.65294944,30.15988984)
\curveto(254.65294944,29.71643349)(254.77941135,29.38018692)(255.04206493,29.14627604)
\curveto(255.31444578,28.92211083)(255.82029592,28.74180484)(256.55961537,28.60048399)
\lineto(257.37675791,28.44454341)
\curveto(258.3884582,28.24961768)(259.13264127,27.9036242)(259.61417201,27.41143704)
\curveto(260.09083911,26.92899555)(260.33403568,26.28574064)(260.33403568,25.48167231)
\curveto(260.33403568,24.50704366)(260.00815114,23.76632589)(259.36124694,23.25951899)
\curveto(258.70947786,22.76245838)(257.75614614,22.51880121)(256.50124804,22.51880121)
\curveto(256.03430945,22.51880121)(255.5284593,22.57727893)(254.98369761,22.69423437)
\curveto(254.44866442,22.79657032)(253.89903785,22.95251065)(253.32995675,23.16205612)
\lineto(253.32995675,24.54602881)
\curveto(253.88444571,24.23414764)(254.42434377,24.00023676)(254.94478606,23.84429618)
\curveto(255.47495561,23.68835559)(255.9953979,23.6103853)(256.50124804,23.6103853)
\curveto(257.27947903,23.6103853)(257.87288079,23.75658022)(258.29117932,24.05871448)
\curveto(258.70461423,24.37059565)(258.91376412,24.80430509)(258.91376412,25.36471687)
\curveto(258.91376412,25.85690403)(258.76298155,26.24188266)(258.4662813,26.51477868)
\curveto(258.16471616,26.79742161)(257.67345816,27.01183867)(256.98764241,27.15803359)
\lineto(256.17049987,27.31397418)
\curveto(255.15879958,27.50889991)(254.42434377,27.82078107)(253.97199732,28.24961768)
\curveto(253.51478723,28.67845429)(253.2910452,29.27297714)(253.2910452,30.0429344)
\curveto(253.2910452,30.92497301)(253.6023376,31.62183281)(254.22492239,32.12863971)
\curveto(254.84750718,32.64519352)(255.70842491,32.90834263)(256.81254044,32.90834263)
\curveto(257.27947903,32.90834263)(257.75128126,32.85961057)(258.232812,32.77189462)
\curveto(258.72407,32.67930459)(259.22505651,32.54772941)(259.73090666,32.38204316)
\closepath
}
}
{
\newrgbcolor{curcolor}{0 0 0}
\pscustom[linestyle=none,fillstyle=solid,fillcolor=curcolor]
{
\newpath
\moveto(268.95541388,26.76818213)
\lineto(268.95541388,26.16391237)
\lineto(263.29378342,26.16391237)
\curveto(263.34242348,25.31598607)(263.5953473,24.67272991)(264.05255863,24.23414764)
\curveto(264.50490509,23.79069129)(265.14208202,23.57140016)(265.95922456,23.57140016)
\curveto(266.43589042,23.57140016)(266.89796539,23.62987788)(267.34058457,23.74683331)
\curveto(267.77834013,23.86378875)(268.22095807,24.02947625)(268.66357726,24.25364021)
\lineto(268.66357726,23.1035784)
\curveto(268.22095807,22.90865267)(267.76861162,22.76245838)(267.30167303,22.6747418)
\curveto(266.83473443,22.5724061)(266.36779583,22.51880121)(265.90085724,22.51880121)
\curveto(264.6945992,22.51880121)(263.74126624,22.85992187)(263.04085834,23.55190758)
\curveto(262.35017772,24.25364021)(262.00970228,25.19903063)(262.00970228,26.39782324)
\curveto(262.00970228,27.61610968)(262.34045045,28.58586425)(263.00194679,29.30221662)
\curveto(263.66344314,30.02831466)(264.55840878,30.39380071)(265.68684372,30.39380071)
\curveto(266.68395186,30.39380071)(267.476775,30.0672998)(268.06044824,29.41917206)
\curveto(268.65384875,28.76617024)(268.95541388,27.88413162)(268.95541388,26.76818213)
\closepath
\moveto(267.7102443,27.13854102)
\curveto(267.7102443,27.81103541)(267.52055018,28.35195338)(267.14602683,28.75642458)
\curveto(266.78123074,29.1560217)(266.29483637,29.36069434)(265.68684372,29.36069434)
\curveto(264.99616309,29.36069434)(264.44653776,29.16576861)(264.03310286,28.77591715)
\curveto(263.62939522,28.38606569)(263.39592592,27.84027365)(263.33269497,27.13854102)
\lineto(267.7102443,27.13854102)
\closepath
}
}
{
\newrgbcolor{curcolor}{0 0 0}
\pscustom[linestyle=none,fillstyle=solid,fillcolor=curcolor]
{
\newpath
\moveto(277.34324679,27.23600388)
\lineto(277.34324679,22.71372694)
\lineto(276.11753298,22.71372694)
\lineto(276.11753298,27.19701874)
\curveto(276.11753298,27.90849703)(275.98134256,28.44454341)(275.70896171,28.79540972)
\curveto(275.43658086,29.14627604)(275.01828108,29.3217092)(274.46379212,29.3217092)
\curveto(273.80229578,29.3217092)(273.27698986,29.10729089)(272.88787436,28.67845429)
\curveto(272.50848738,28.25936334)(272.3236569,27.68920589)(272.3236569,26.96310786)
\lineto(272.3236569,22.71372694)
\lineto(271.07848731,22.71372694)
\lineto(271.07848731,30.21836755)
\lineto(272.3236569,30.21836755)
\lineto(272.3236569,29.04881317)
\curveto(272.60576501,29.50201518)(272.95110532,29.838263)(273.35481296,30.06242697)
\curveto(273.75365697,30.28171811)(274.21573068,30.39380071)(274.73617297,30.39380071)
\curveto(275.59222706,30.39380071)(276.23913126,30.12577752)(276.68175045,29.59460522)
\curveto(277.11950476,29.05856008)(277.34324679,28.27398308)(277.34324679,27.23600388)
\closepath
}
}
{
\newrgbcolor{curcolor}{0 0 0}
\pscustom[linestyle=none,fillstyle=solid,fillcolor=curcolor]
{
\newpath
\moveto(284.74474412,30.00394925)
\lineto(284.74474412,28.83439487)
\curveto(284.39454017,29.01470086)(284.02974409,29.14627604)(283.65522073,29.22424633)
\curveto(283.2758325,29.31196229)(282.886717,29.36069434)(282.48787424,29.36069434)
\curveto(281.87501672,29.36069434)(281.41780663,29.26323148)(281.10651423,29.06830575)
\curveto(280.80495035,28.88312568)(280.65903141,28.61022966)(280.65903141,28.24961768)
\curveto(280.65903141,27.96210192)(280.7660388,27.73306512)(280.98977958,27.56737762)
\curveto(281.20865674,27.41143704)(281.65127593,27.25549646)(282.31277227,27.09955587)
\lineto(282.72134354,27.00209301)
\curveto(283.60171704,26.81691294)(284.22430183,26.55376383)(284.58909792,26.20289751)
\curveto(284.94902913,25.86177811)(285.13385962,25.38908228)(285.13385962,24.77993968)
\curveto(285.13385962,24.08795272)(284.86147877,23.53728784)(284.31671708,23.12307098)
\curveto(283.77195538,22.71860009)(283.01318016,22.51880121)(282.04039142,22.51880121)
\curveto(281.63668379,22.51880121)(281.21352037,22.55778636)(280.77576606,22.63575665)
\curveto(280.34773902,22.71372694)(279.89052769,22.83068238)(279.41386183,22.98662297)
\lineto(279.41386183,24.25364021)
\curveto(279.86620828,24.01972934)(280.3136911,23.84429618)(280.75631029,23.72734074)
\curveto(281.19406584,23.6103853)(281.63668379,23.55190758)(282.07930297,23.55190758)
\curveto(282.64838407,23.55190758)(283.09100326,23.64937045)(283.40229566,23.84429618)
\curveto(283.71358806,24.04896757)(283.86923425,24.3316105)(283.86923425,24.68247682)
\curveto(283.86923425,25.01872339)(283.75736324,25.27700092)(283.53848608,25.46217974)
\curveto(283.31474405,25.64248573)(282.82348605,25.81304605)(282.0598472,25.96898664)
\lineto(281.63182015,26.08594208)
\curveto(280.86331643,26.24188266)(280.3136911,26.48066637)(279.9780793,26.80716728)
\curveto(279.63760386,27.14341385)(279.47222915,27.60636277)(279.47222915,28.19113996)
\curveto(279.47222915,28.89287259)(279.71542696,29.42891772)(280.21154859,29.80902352)
\curveto(280.71739874,30.19887498)(281.42753514,30.39380071)(282.35168382,30.39380071)
\curveto(282.80403027,30.39380071)(283.23205732,30.3596884)(283.63576496,30.29633785)
\curveto(284.03460896,30.22811447)(284.40426744,30.13065035)(284.74474412,30.00394925)
\closepath
}
}
{
\newrgbcolor{curcolor}{0 0 0}
\pscustom[linestyle=none,fillstyle=solid,fillcolor=curcolor]
{
\newpath
\moveto(290.14478561,29.36069434)
\curveto(289.48328927,29.36069434)(288.95798335,29.09754523)(288.56886786,28.58099142)
\curveto(288.18948087,28.05956478)(288.00465039,27.35295932)(288.00465039,26.45630096)
\curveto(288.00465039,25.5596426)(288.18948087,24.84816431)(288.56886786,24.3316105)
\curveto(288.94339121,23.82480361)(289.46869713,23.57140016)(290.14478561,23.57140016)
\curveto(290.79168982,23.57140016)(291.30240484,23.82480361)(291.68179182,24.3316105)
\curveto(292.07090732,24.84816431)(292.26546507,25.5596426)(292.26546507,26.45630096)
\curveto(292.26546507,27.35295932)(292.07090732,28.05956478)(291.68179182,28.58099142)
\curveto(291.30240484,29.09754523)(290.79168982,29.36069434)(290.14478561,29.36069434)
\closepath
\moveto(290.14478561,30.39380071)
\curveto(291.20512472,30.39380071)(292.04172428,30.0429344)(292.65458056,29.34120177)
\curveto(293.26257321,28.6492148)(293.56900198,27.68920589)(293.56900198,26.45630096)
\curveto(293.56900198,25.23314169)(293.26257321,24.27313279)(292.65458056,23.57140016)
\curveto(292.04172428,22.86966753)(291.20512472,22.51880121)(290.14478561,22.51880121)
\curveto(289.06498949,22.51880121)(288.22352754,22.86966753)(287.61553489,23.57140016)
\curveto(287.00267736,24.27313279)(286.70111348,25.23314169)(286.70111348,26.45630096)
\curveto(286.70111348,27.68920589)(287.00267736,28.6492148)(287.61553489,29.34120177)
\curveto(288.22352754,30.0429344)(289.06498949,30.39380071)(290.14478561,30.39380071)
\closepath
}
}
{
\newrgbcolor{curcolor}{0 0 0}
\pscustom[linestyle=none,fillstyle=solid,fillcolor=curcolor]
{
\newpath
\moveto(300.09977739,29.06830575)
\curveto(299.96845059,29.14627604)(299.81766803,29.20475376)(299.65229456,29.2437389)
\curveto(299.49664837,29.28272405)(299.32154639,29.30221662)(299.12698865,29.30221662)
\curveto(298.42658075,29.30221662)(297.88668269,29.07317858)(297.51215934,28.61997657)
\curveto(297.14736325,28.16190173)(296.96739764,27.51377274)(296.96739764,26.67071927)
\lineto(296.96739764,22.71372694)
\lineto(295.72222806,22.71372694)
\lineto(295.72222806,30.21836755)
\lineto(296.96739764,30.21836755)
\lineto(296.96739764,29.04881317)
\curveto(297.21059421,29.50201518)(297.54134362,29.838263)(297.95964216,30.06242697)
\curveto(298.37307706,30.28171811)(298.87892721,30.39380071)(299.47719259,30.39380071)
\curveto(299.5647442,30.39380071)(299.6571582,30.38405505)(299.74957344,30.37430814)
\curveto(299.85171595,30.37430814)(299.96845059,30.3596884)(300.09977739,30.33532299)
\lineto(300.09977739,29.06830575)
\closepath
}
}
{
\newrgbcolor{curcolor}{0 0 0}
\pscustom[linewidth=2.49269314,linecolor=curcolor]
{
\newpath
\moveto(235.33823165,26.27112462)
\lineto(235.33823165,26.27112462)
\lineto(85.91787867,26.27112462)
\lineto(85.91787867,77.7461381)
}
}
{
\newrgbcolor{curcolor}{0 0 0}
\pscustom[linestyle=none,fillstyle=solid,fillcolor=curcolor]
{
\newpath
\moveto(85.91787867,83.35999944)
\lineto(89.65338768,75.87485141)
\lineto(85.91787867,77.7461381)
\lineto(82.18236965,75.87485141)
\lineto(85.91787867,83.35999944)
\closepath
}
}
{
\newrgbcolor{curcolor}{0 0 0}
\pscustom[linewidth=2.49269314,linecolor=curcolor]
{
\newpath
\moveto(85.91787867,83.35999944)
\lineto(89.65338768,75.87485141)
\lineto(85.91787867,77.7461381)
\lineto(82.18236965,75.87485141)
\lineto(85.91787867,83.35999944)
\closepath
}
}
{
\newrgbcolor{curcolor}{0 0 0}
\pscustom[linestyle=none,fillstyle=solid,fillcolor=curcolor]
{
\newpath
\moveto(76.39914073,105.30863196)
\lineto(76.39914073,101.58555051)
\lineto(80.11519372,101.58555051)
\lineto(80.11519372,100.4354887)
\lineto(76.39914073,100.4354887)
\lineto(76.39914073,96.71240726)
\lineto(75.2707058,96.71240726)
\lineto(75.2707058,100.4354887)
\lineto(71.55465281,100.4354887)
\lineto(71.55465281,101.58555051)
\lineto(75.2707058,101.58555051)
\lineto(75.2707058,105.30863196)
\lineto(76.39914073,105.30863196)
\closepath
}
}
{
\newrgbcolor{curcolor}{0 0 0}
\pscustom[linestyle=none,fillstyle=solid,fillcolor=curcolor]
{
\newpath
\moveto(68.98387908,77.09313252)
\lineto(76.71042897,77.09313252)
\lineto(76.71042897,76.00154843)
\lineto(68.98387908,76.00154843)
\lineto(68.98387908,77.09313252)
\closepath
}
}
{
\newrgbcolor{curcolor}{1 1 1}
\pscustom[linestyle=none,fillstyle=solid,fillcolor=curcolor]
{
\newpath
\moveto(310.04839927,111.10280239)
\lineto(394.03412558,111.10280239)
\lineto(394.03412558,71.18201285)
\lineto(310.04839927,71.18201285)
\lineto(310.04839927,111.10280239)
\closepath
}
}
{
\newrgbcolor{curcolor}{0 0 0}
\pscustom[linewidth=2.49269314,linecolor=curcolor]
{
\newpath
\moveto(310.04839927,111.10280239)
\lineto(394.03412558,111.10280239)
\lineto(394.03412558,71.18201285)
\lineto(310.04839927,71.18201285)
\lineto(310.04839927,111.10280239)
\closepath
}
}
{
\newrgbcolor{curcolor}{0 0 0}
\pscustom[linewidth=2.49269314,linecolor=curcolor]
{
\newpath
\moveto(239.63309221,91.14240762)
\lineto(301.66295481,91.14240762)
}
}
{
\newrgbcolor{curcolor}{0 0 0}
\pscustom[linestyle=none,fillstyle=solid,fillcolor=curcolor]
{
\newpath
\moveto(307.26621643,91.14240762)
\lineto(299.79520093,87.39983296)
\lineto(301.66295481,91.14240762)
\lineto(299.79520093,94.884981)
\lineto(307.26621643,91.14240762)
\closepath
}
}
{
\newrgbcolor{curcolor}{0 0 0}
\pscustom[linewidth=2.49269314,linecolor=curcolor]
{
\newpath
\moveto(307.26621643,91.14240762)
\lineto(299.79520093,87.39983296)
\lineto(301.66295481,91.14240762)
\lineto(299.79520093,94.884981)
\lineto(307.26621643,91.14240762)
\closepath
}
}
{
\newrgbcolor{curcolor}{0 0 0}
\pscustom[linestyle=none,fillstyle=solid,fillcolor=curcolor]
{
\newpath
\moveto(6.80123155,100.34597809)
\curveto(7.06571724,100.25645986)(7.32206492,100.06521638)(7.57027457,99.77224763)
\curveto(7.82255324,99.47927889)(8.0748319,99.07644687)(8.32711057,98.56375156)
\lineto(9.57833136,96.07351724)
\lineto(8.25386837,96.07351724)
\lineto(7.0880968,98.41116368)
\curveto(6.78699001,99.02151523)(6.49402124,99.42638176)(6.20919049,99.62576326)
\curveto(5.92842875,99.82514477)(5.54390724,99.92483552)(5.05562595,99.92483552)
\lineto(3.71285241,99.92483552)
\lineto(3.71285241,96.07351724)
\lineto(2.47994216,96.07351724)
\lineto(2.47994216,105.18606589)
\lineto(5.2631455,105.18606589)
\curveto(6.30481224,105.18606589)(7.08199329,104.96837383)(7.59468864,104.53298973)
\curveto(8.10738399,104.09760562)(8.36373166,103.44046045)(8.36373166,102.56155422)
\curveto(8.36373166,101.98782376)(8.22945431,101.51174955)(7.9608996,101.13333159)
\curveto(7.6964139,100.75491363)(7.30985789,100.49246246)(6.80123155,100.34597809)
\closepath
\moveto(3.71285241,104.17288231)
\lineto(3.71285241,100.9380191)
\lineto(5.2631455,100.9380191)
\curveto(5.85722106,100.9380191)(6.30481224,101.07433094)(6.60591903,101.34695463)
\curveto(6.91109484,101.62364734)(7.06368274,102.02851387)(7.06368274,102.56155422)
\curveto(7.06368274,103.09459457)(6.91109484,103.49539209)(6.60591903,103.76394677)
\curveto(6.30481224,104.03657047)(5.85722106,104.17288231)(5.2631455,104.17288231)
\lineto(3.71285241,104.17288231)
\closepath
}
}
{
\newrgbcolor{curcolor}{0 0 0}
\pscustom[linestyle=none,fillstyle=solid,fillcolor=curcolor]
{
\newpath
\moveto(16.40816566,99.77224763)
\lineto(16.40816566,99.22293124)
\lineto(11.24459106,99.22293124)
\curveto(11.29341919,98.44981928)(11.5253528,97.85981278)(11.94039189,97.45291174)
\curveto(12.35949999,97.05007972)(12.94136853,96.84866371)(13.68599749,96.84866371)
\curveto(14.11731262,96.84866371)(14.53438622,96.90156084)(14.93721828,97.00735511)
\curveto(15.34411936,97.11314938)(15.74695142,97.27184078)(16.14571447,97.48342932)
\lineto(16.14571447,96.42141762)
\curveto(15.74288241,96.25051919)(15.32987782,96.12031086)(14.9067007,96.03079263)
\curveto(14.48352359,95.9412744)(14.05424296,95.89651529)(13.61885881,95.89651529)
\curveto(12.52836394,95.89651529)(11.66369916,96.2138981)(11.02486448,96.84866371)
\curveto(10.39009881,97.48342932)(10.07271597,98.3419905)(10.07271597,99.42434725)
\curveto(10.07271597,100.54332509)(10.37382276,101.43036935)(10.97603635,102.08548001)
\curveto(11.58231895,102.74465968)(12.3981556,103.07424952)(13.4235463,103.07424952)
\curveto(14.34314272,103.07424952)(15.06946113,102.77721177)(15.60250154,102.18313626)
\curveto(16.13961095,101.59312976)(16.40816566,100.78950022)(16.40816566,99.77224763)
\closepath
\moveto(15.2851187,100.10183747)
\curveto(15.27698068,100.71625803)(15.10404772,101.20657378)(14.76631983,101.57278471)
\curveto(14.43266095,101.93899564)(13.98913879,102.1221011)(13.43575333,102.1221011)
\curveto(12.80912568,102.1221011)(12.30660285,101.94509915)(11.92818486,101.59109525)
\curveto(11.55383587,101.23709136)(11.3381783,100.73863759)(11.28121215,100.09573396)
\lineto(15.2851187,100.10183747)
\closepath
}
}
{
\newrgbcolor{curcolor}{0 0 0}
\pscustom[linestyle=none,fillstyle=solid,fillcolor=curcolor]
{
\newpath
\moveto(21.71212023,105.57058736)
\lineto(21.71212023,104.63674949)
\lineto(20.6379014,104.63674949)
\curveto(20.23506933,104.63674949)(19.9543076,104.55536928)(19.79561618,104.39260887)
\curveto(19.64099377,104.22984846)(19.56368257,103.93687971)(19.56368257,103.51370264)
\lineto(19.56368257,102.9094546)
\lineto(21.41304794,102.9094546)
\lineto(21.41304794,102.03665189)
\lineto(19.56368257,102.03665189)
\lineto(19.56368257,96.07351724)
\lineto(18.43453209,96.07351724)
\lineto(18.43453209,102.03665189)
\lineto(17.36031326,102.03665189)
\lineto(17.36031326,102.9094546)
\lineto(18.43453209,102.9094546)
\lineto(18.43453209,103.38552881)
\curveto(18.43453209,104.14643375)(18.61153406,104.69981915)(18.96553799,105.04568503)
\curveto(19.31954192,105.39561992)(19.8810654,105.57058736)(20.65010843,105.57058736)
\lineto(21.71212023,105.57058736)
\closepath
}
}
{
\newrgbcolor{curcolor}{0 0 0}
\pscustom[linestyle=none,fillstyle=solid,fillcolor=curcolor]
{
\newpath
\moveto(28.49312763,99.77224763)
\lineto(28.49312763,99.22293124)
\lineto(23.32955303,99.22293124)
\curveto(23.37838115,98.44981928)(23.61031477,97.85981278)(24.02535386,97.45291174)
\curveto(24.44446196,97.05007972)(25.0263305,96.84866371)(25.77095946,96.84866371)
\curveto(26.20227459,96.84866371)(26.61934819,96.90156084)(27.02218025,97.00735511)
\curveto(27.42908132,97.11314938)(27.83191339,97.27184078)(28.23067644,97.48342932)
\lineto(28.23067644,96.42141762)
\curveto(27.82784437,96.25051919)(27.41483979,96.12031086)(26.99166267,96.03079263)
\curveto(26.56848556,95.9412744)(26.13920493,95.89651529)(25.70382078,95.89651529)
\curveto(24.61332591,95.89651529)(23.74866113,96.2138981)(23.10982645,96.84866371)
\curveto(22.47506078,97.48342932)(22.15767794,98.3419905)(22.15767794,99.42434725)
\curveto(22.15767794,100.54332509)(22.45878473,101.43036935)(23.06099832,102.08548001)
\curveto(23.66728092,102.74465968)(24.48311756,103.07424952)(25.50850827,103.07424952)
\curveto(26.42810469,103.07424952)(27.1544231,102.77721177)(27.6874635,102.18313626)
\curveto(28.22457292,101.59312976)(28.49312763,100.78950022)(28.49312763,99.77224763)
\closepath
\moveto(27.37008067,100.10183747)
\curveto(27.36194265,100.71625803)(27.18900969,101.20657378)(26.8512818,101.57278471)
\curveto(26.51762292,101.93899564)(26.07410076,102.1221011)(25.5207153,102.1221011)
\curveto(24.89408765,102.1221011)(24.39156482,101.94509915)(24.01314683,101.59109525)
\curveto(23.63879784,101.23709136)(23.42314027,100.73863759)(23.36617412,100.09573396)
\lineto(27.37008067,100.10183747)
\closepath
}
}
{
\newrgbcolor{curcolor}{0 0 0}
\pscustom[linestyle=none,fillstyle=solid,fillcolor=curcolor]
{
\newpath
\moveto(34.29756991,101.85964994)
\curveto(34.17143058,101.93289212)(34.03308421,101.98578926)(33.88253081,102.01834134)
\curveto(33.73604643,102.05496243)(33.573286,102.07327298)(33.39424953,102.07327298)
\curveto(32.75948386,102.07327298)(32.27120257,101.86575345)(31.92940567,101.4507144)
\curveto(31.59167778,101.03974435)(31.42281384,100.44770335)(31.42281384,99.67459139)
\lineto(31.42281384,96.07351724)
\lineto(30.29366336,96.07351724)
\lineto(30.29366336,102.9094546)
\lineto(31.42281384,102.9094546)
\lineto(31.42281384,101.84744291)
\curveto(31.65881646,102.26248196)(31.96602677,102.56969224)(32.34444476,102.76907375)
\curveto(32.72286276,102.97252426)(33.18266097,103.07424952)(33.7238394,103.07424952)
\curveto(33.8011506,103.07424952)(33.88659983,103.06814601)(33.98018707,103.05593898)
\curveto(34.07377432,103.04780095)(34.17753409,103.03355942)(34.29146639,103.01321437)
\lineto(34.29756991,101.85964994)
\closepath
}
}
{
\newrgbcolor{curcolor}{0 0 0}
\pscustom[linestyle=none,fillstyle=solid,fillcolor=curcolor]
{
\newpath
\moveto(41.06637147,99.77224763)
\lineto(41.06637147,99.22293124)
\lineto(35.90279687,99.22293124)
\curveto(35.951625,98.44981928)(36.18355861,97.85981278)(36.5985977,97.45291174)
\curveto(37.0177058,97.05007972)(37.59957434,96.84866371)(38.3442033,96.84866371)
\curveto(38.77551843,96.84866371)(39.19259203,96.90156084)(39.59542409,97.00735511)
\curveto(40.00232517,97.11314938)(40.40515723,97.27184078)(40.80392028,97.48342932)
\lineto(40.80392028,96.42141762)
\curveto(40.40108822,96.25051919)(39.98808363,96.12031086)(39.56490651,96.03079263)
\curveto(39.1417294,95.9412744)(38.71244877,95.89651529)(38.27706462,95.89651529)
\curveto(37.18656975,95.89651529)(36.32190497,96.2138981)(35.68307029,96.84866371)
\curveto(35.04830462,97.48342932)(34.73092178,98.3419905)(34.73092178,99.42434725)
\curveto(34.73092178,100.54332509)(35.03202857,101.43036935)(35.63424216,102.08548001)
\curveto(36.24052476,102.74465968)(37.05636141,103.07424952)(38.08175211,103.07424952)
\curveto(39.00134853,103.07424952)(39.72766694,102.77721177)(40.26070735,102.18313626)
\curveto(40.79781676,101.59312976)(41.06637147,100.78950022)(41.06637147,99.77224763)
\closepath
\moveto(39.94332451,100.10183747)
\curveto(39.93518649,100.71625803)(39.76225353,101.20657378)(39.42452564,101.57278471)
\curveto(39.09086676,101.93899564)(38.6473446,102.1221011)(38.09395914,102.1221011)
\curveto(37.46733149,102.1221011)(36.96480866,101.94509915)(36.58639067,101.59109525)
\curveto(36.21204168,101.23709136)(35.99638411,100.73863759)(35.93941796,100.09573396)
\lineto(39.94332451,100.10183747)
\closepath
}
}
{
\newrgbcolor{curcolor}{0 0 0}
\pscustom[linestyle=none,fillstyle=solid,fillcolor=curcolor]
{
\newpath
\moveto(48.59200649,100.19949372)
\lineto(48.59200649,96.07351724)
\lineto(47.46895953,96.07351724)
\lineto(47.46895953,100.16287263)
\curveto(47.46895953,100.80984527)(47.3428202,101.2940575)(47.09054154,101.61550932)
\curveto(46.83826287,101.93696113)(46.45984488,102.09768704)(45.95528755,102.09768704)
\curveto(45.34900495,102.09768704)(44.87089619,101.90440905)(44.52096127,101.51785307)
\curveto(44.17102635,101.13129709)(43.99605889,100.60436025)(43.99605889,99.93704255)
\lineto(43.99605889,96.07351724)
\lineto(42.86690841,96.07351724)
\lineto(42.86690841,102.9094546)
\lineto(43.99605889,102.9094546)
\lineto(43.99605889,101.84744291)
\curveto(44.26461359,102.25841295)(44.57996193,102.56562323)(44.94210388,102.76907375)
\curveto(45.30831484,102.97252426)(45.72945745,103.07424952)(46.20553171,103.07424952)
\curveto(46.99085078,103.07424952)(47.58492634,102.8301089)(47.9877584,102.34182766)
\curveto(48.39059046,101.85761543)(48.59200649,101.14350412)(48.59200649,100.19949372)
\closepath
}
}
{
\newrgbcolor{curcolor}{0 0 0}
\pscustom[linestyle=none,fillstyle=solid,fillcolor=curcolor]
{
\newpath
\moveto(55.76363758,102.64700344)
\lineto(55.76363758,101.59719877)
\curveto(55.44625474,101.77216621)(55.1268374,101.90237455)(54.80538555,101.98782376)
\curveto(54.48800272,102.07734199)(54.16655087,102.1221011)(53.84103001,102.1221011)
\curveto(53.11267709,102.1221011)(52.5470846,101.89016751)(52.14425254,101.42630034)
\curveto(51.74142048,100.96650217)(51.54000445,100.31952952)(51.54000445,99.48538241)
\curveto(51.54000445,98.65123529)(51.74142048,98.00222814)(52.14425254,97.53836096)
\curveto(52.5470846,97.07856279)(53.11267709,96.84866371)(53.84103001,96.84866371)
\curveto(54.16655087,96.84866371)(54.48800272,96.89138832)(54.80538555,96.97683753)
\curveto(55.1268374,97.06635576)(55.44625474,97.1985986)(55.76363758,97.37356604)
\lineto(55.76363758,96.33596841)
\curveto(55.45032375,96.18948403)(55.12480289,96.07962075)(54.787075,96.00637857)
\curveto(54.45341612,95.93313638)(54.09737769,95.89651529)(53.71895969,95.89651529)
\curveto(52.68949998,95.89651529)(51.87162882,96.22000161)(51.26534623,96.86697425)
\curveto(50.65906363,97.5139469)(50.35592233,98.38674962)(50.35592233,99.48538241)
\curveto(50.35592233,100.60029124)(50.66109814,101.47716296)(51.27144974,102.11599759)
\curveto(51.88587036,102.75483221)(52.72612108,103.07424952)(53.79220188,103.07424952)
\curveto(54.13806779,103.07424952)(54.47579568,103.03762843)(54.80538555,102.96438624)
\curveto(55.13497542,102.89521307)(55.45439276,102.7894188)(55.76363758,102.64700344)
\closepath
}
}
{
\newrgbcolor{curcolor}{0 0 0}
\pscustom[linestyle=none,fillstyle=solid,fillcolor=curcolor]
{
\newpath
\moveto(57.72896755,102.9094546)
\lineto(58.85201451,102.9094546)
\lineto(58.85201451,96.07351724)
\lineto(57.72896755,96.07351724)
\lineto(57.72896755,102.9094546)
\closepath
\moveto(57.72896755,105.57058736)
\lineto(58.85201451,105.57058736)
\lineto(58.85201451,104.14846825)
\lineto(57.72896755,104.14846825)
\lineto(57.72896755,105.57058736)
\closepath
}
}
{
\newrgbcolor{curcolor}{0 0 0}
\pscustom[linestyle=none,fillstyle=solid,fillcolor=curcolor]
{
\newpath
\moveto(64.30245713,99.50979647)
\curveto(63.39506774,99.50979647)(62.76640559,99.4060367)(62.41647067,99.19851718)
\curveto(62.06653575,98.99099765)(61.89156828,98.63699375)(61.89156828,98.13650548)
\curveto(61.89156828,97.73774247)(62.02177663,97.42035966)(62.28219331,97.18435706)
\curveto(62.54667901,96.95242347)(62.90475195,96.83645668)(63.35641214,96.83645668)
\curveto(63.97897078,96.83645668)(64.4774246,97.05618324)(64.85177358,97.49563635)
\curveto(65.23019158,97.93915848)(65.41940058,98.52713047)(65.41940058,99.25955233)
\lineto(65.41940058,99.50979647)
\lineto(64.30245713,99.50979647)
\closepath
\moveto(66.54244753,99.97366365)
\lineto(66.54244753,96.07351724)
\lineto(65.41940058,96.07351724)
\lineto(65.41940058,97.11111487)
\curveto(65.1630529,96.69607582)(64.84363556,96.38886554)(64.46114855,96.18948403)
\curveto(64.07866155,95.99417154)(63.61072531,95.89651529)(63.05733985,95.89651529)
\curveto(62.35747001,95.89651529)(61.80001554,96.09182779)(61.38497645,96.48245278)
\curveto(60.97400637,96.87714678)(60.76852133,97.40408362)(60.76852133,98.06326329)
\curveto(60.76852133,98.83230625)(61.024869,99.41214022)(61.53756435,99.80276521)
\curveto(62.05432871,100.1933902)(62.82337174,100.3887027)(63.84469343,100.3887027)
\lineto(65.41940058,100.3887027)
\lineto(65.41940058,100.49856598)
\curveto(65.41940058,101.01533029)(65.24850213,101.4140933)(64.90670523,101.69485502)
\curveto(64.56897734,101.97968574)(64.09290308,102.1221011)(63.47848246,102.1221011)
\curveto(63.08785744,102.1221011)(62.70740493,102.07530748)(62.33712496,101.98172025)
\curveto(61.96684498,101.88813301)(61.61080654,101.74775215)(61.26900964,101.56057768)
\lineto(61.26900964,102.59817531)
\curveto(61.67997973,102.75686672)(62.07874278,102.87486802)(62.4652988,102.95217921)
\curveto(62.85185481,103.03355942)(63.22823831,103.07424952)(63.59444927,103.07424952)
\curveto(64.58321887,103.07424952)(65.32174432,102.81790187)(65.81002561,102.30520657)
\curveto(66.29830689,101.79251127)(66.54244753,101.01533029)(66.54244753,99.97366365)
\closepath
}
}
{
\newrgbcolor{curcolor}{0 0 0}
\pscustom[linestyle=none,fillstyle=solid,fillcolor=curcolor]
{
\newpath
\moveto(93.54092502,120.64137025)
\lineto(99.3026442,120.64137025)
\lineto(99.3026442,119.60377262)
\lineto(94.77383527,119.60377262)
\lineto(94.77383527,116.90601877)
\lineto(99.1134352,116.90601877)
\lineto(99.1134352,115.86842113)
\lineto(94.77383527,115.86842113)
\lineto(94.77383527,112.56641924)
\lineto(99.41250749,112.56641924)
\lineto(99.41250749,111.52882161)
\lineto(93.54092502,111.52882161)
\lineto(93.54092502,120.64137025)
\closepath
}
}
{
\newrgbcolor{curcolor}{0 0 0}
\pscustom[linestyle=none,fillstyle=solid,fillcolor=curcolor]
{
\newpath
\moveto(105.36343572,117.3149543)
\curveto(105.23729638,117.38819649)(105.09895002,117.44109362)(104.94839662,117.47364571)
\curveto(104.80191224,117.5102668)(104.63915181,117.52857735)(104.46011534,117.52857735)
\curveto(103.82534967,117.52857735)(103.33706838,117.32105782)(102.99527148,116.90601877)
\curveto(102.65754359,116.49504872)(102.48867964,115.90300772)(102.48867964,115.12989575)
\lineto(102.48867964,111.52882161)
\lineto(101.35952917,111.52882161)
\lineto(101.35952917,118.36475897)
\lineto(102.48867964,118.36475897)
\lineto(102.48867964,117.30274727)
\curveto(102.72468227,117.71778633)(103.03189258,118.02499661)(103.41031057,118.22437811)
\curveto(103.78872857,118.42782863)(104.24852678,118.52955389)(104.78970521,118.52955389)
\curveto(104.86701641,118.52955389)(104.95246563,118.52345037)(105.04605288,118.51124334)
\curveto(105.13964013,118.50310532)(105.2433999,118.48886379)(105.3573322,118.46851873)
\lineto(105.36343572,117.3149543)
\closepath
}
}
{
\newrgbcolor{curcolor}{0 0 0}
\pscustom[linestyle=none,fillstyle=solid,fillcolor=curcolor]
{
\newpath
\moveto(110.29507639,117.3149543)
\curveto(110.16893706,117.38819649)(110.0305907,117.44109362)(109.8800373,117.47364571)
\curveto(109.73355292,117.5102668)(109.57079249,117.52857735)(109.39175602,117.52857735)
\curveto(108.75699034,117.52857735)(108.26870906,117.32105782)(107.92691216,116.90601877)
\curveto(107.58918427,116.49504872)(107.42032032,115.90300772)(107.42032032,115.12989575)
\lineto(107.42032032,111.52882161)
\lineto(106.29116985,111.52882161)
\lineto(106.29116985,118.36475897)
\lineto(107.42032032,118.36475897)
\lineto(107.42032032,117.30274727)
\curveto(107.65632294,117.71778633)(107.96353325,118.02499661)(108.34195125,118.22437811)
\curveto(108.72036925,118.42782863)(109.18016746,118.52955389)(109.72134588,118.52955389)
\curveto(109.79865709,118.52955389)(109.88410631,118.52345037)(109.97769356,118.51124334)
\curveto(110.07128081,118.50310532)(110.17504058,118.48886379)(110.28897288,118.46851873)
\lineto(110.29507639,117.3149543)
\closepath
}
}
{
\newrgbcolor{curcolor}{0 0 0}
\pscustom[linestyle=none,fillstyle=solid,fillcolor=curcolor]
{
\newpath
\moveto(113.86563432,117.57740547)
\curveto(113.26342073,117.57740547)(112.78734648,117.34140287)(112.43741156,116.86939767)
\curveto(112.08747664,116.40146148)(111.91250918,115.75855785)(111.91250918,114.94068677)
\curveto(111.91250918,114.1228157)(112.08544213,113.47787756)(112.43130804,113.00587236)
\curveto(112.78124296,112.53793617)(113.25935172,112.30396808)(113.86563432,112.30396808)
\curveto(114.4637789,112.30396808)(114.93781864,112.53997067)(115.28775357,113.01197587)
\curveto(115.63768849,113.48398107)(115.81265595,114.12688471)(115.81265595,114.94068677)
\curveto(115.81265595,115.75041983)(115.63768849,116.39128896)(115.28775357,116.86329416)
\curveto(114.93781864,117.33936837)(114.4637789,117.57740547)(113.86563432,117.57740547)
\closepath
\moveto(113.86563432,118.52955389)
\curveto(114.84219689,118.52955389)(115.60920541,118.21217108)(116.16665988,117.57740547)
\curveto(116.72411435,116.94263986)(117.00284158,116.06373363)(117.00284158,114.94068677)
\curveto(117.00284158,113.82170893)(116.72411435,112.9428027)(116.16665988,112.30396808)
\curveto(115.60920541,111.66920246)(114.84219689,111.35181966)(113.86563432,111.35181966)
\curveto(112.88500274,111.35181966)(112.11595971,111.66920246)(111.55850524,112.30396808)
\curveto(111.00511979,112.9428027)(110.72842706,113.82170893)(110.72842706,114.94068677)
\curveto(110.72842706,116.06373363)(111.00511979,116.94263986)(111.55850524,117.57740547)
\curveto(112.11595971,118.21217108)(112.88500274,118.52955389)(113.86563432,118.52955389)
\closepath
}
}
{
\newrgbcolor{curcolor}{0 0 0}
\pscustom[linestyle=none,fillstyle=solid,fillcolor=curcolor]
{
\newpath
\moveto(122.81949163,117.3149543)
\curveto(122.69335229,117.38819649)(122.55500593,117.44109362)(122.40445253,117.47364571)
\curveto(122.25796815,117.5102668)(122.09520772,117.52857735)(121.91617125,117.52857735)
\curveto(121.28140557,117.52857735)(120.79312429,117.32105782)(120.45132739,116.90601877)
\curveto(120.1135995,116.49504872)(119.94473555,115.90300772)(119.94473555,115.12989575)
\lineto(119.94473555,111.52882161)
\lineto(118.81558508,111.52882161)
\lineto(118.81558508,118.36475897)
\lineto(119.94473555,118.36475897)
\lineto(119.94473555,117.30274727)
\curveto(120.18073817,117.71778633)(120.48794848,118.02499661)(120.86636648,118.22437811)
\curveto(121.24478448,118.42782863)(121.70458269,118.52955389)(122.24576111,118.52955389)
\curveto(122.32307232,118.52955389)(122.40852154,118.52345037)(122.50210879,118.51124334)
\curveto(122.59569604,118.50310532)(122.69945581,118.48886379)(122.81338811,118.46851873)
\lineto(122.81949163,117.3149543)
\closepath
}
}
{
\newrgbcolor{curcolor}{0 0 0}
\pscustom[linestyle=none,fillstyle=solid,fillcolor=curcolor]
{
\newpath
\moveto(132.48746171,117.32716133)
\lineto(132.48746171,121.02589173)
\lineto(133.61050866,121.02589173)
\lineto(133.61050866,111.52882161)
\lineto(132.48746171,111.52882161)
\lineto(132.48746171,112.55421221)
\curveto(132.25145908,112.14731118)(131.9523868,111.84416991)(131.59024484,111.6447884)
\curveto(131.2321719,111.44947591)(130.80085676,111.35181966)(130.29629943,111.35181966)
\curveto(129.47029026,111.35181966)(128.79686898,111.68140949)(128.27603561,112.34058917)
\curveto(127.75927125,112.99976884)(127.50088907,113.86646804)(127.50088907,114.94068677)
\curveto(127.50088907,116.0149055)(127.75927125,116.8816047)(128.27603561,117.54078438)
\curveto(128.79686898,118.19996405)(129.47029026,118.52955389)(130.29629943,118.52955389)
\curveto(130.80085676,118.52955389)(131.2321719,118.42986314)(131.59024484,118.23048163)
\curveto(131.9523868,118.03516913)(132.25145908,117.73406237)(132.48746171,117.32716133)
\closepath
\moveto(128.66055713,114.94068677)
\curveto(128.66055713,114.11467767)(128.82942107,113.46567053)(129.16714896,112.99366533)
\curveto(129.50894586,112.52572914)(129.97688209,112.29176104)(130.57095766,112.29176104)
\curveto(131.16503322,112.29176104)(131.63296945,112.52572914)(131.97476635,112.99366533)
\curveto(132.31656326,113.46567053)(132.48746171,114.11467767)(132.48746171,114.94068677)
\curveto(132.48746171,115.76669587)(132.31656326,116.41366851)(131.97476635,116.8816047)
\curveto(131.63296945,117.3536099)(131.16503322,117.5896125)(130.57095766,117.5896125)
\curveto(129.97688209,117.5896125)(129.50894586,117.3536099)(129.16714896,116.8816047)
\curveto(128.82942107,116.41366851)(128.66055713,115.76669587)(128.66055713,114.94068677)
\closepath
}
}
{
\newrgbcolor{curcolor}{0 0 0}
\pscustom[linestyle=none,fillstyle=solid,fillcolor=curcolor]
{
\newpath
\moveto(141.77090935,115.227552)
\lineto(141.77090935,114.67823561)
\lineto(136.60733475,114.67823561)
\curveto(136.65616288,113.90512364)(136.88809649,113.31511714)(137.30313558,112.90821611)
\curveto(137.72224369,112.50538409)(138.30411222,112.30396808)(139.04874118,112.30396808)
\curveto(139.48005632,112.30396808)(139.89712992,112.35686521)(140.29996198,112.46265948)
\curveto(140.70686305,112.56845375)(141.10969511,112.72714515)(141.50845816,112.93873369)
\lineto(141.50845816,111.87672199)
\curveto(141.1056261,111.70582356)(140.69262151,111.57561523)(140.2694444,111.486097)
\curveto(139.84626728,111.39657877)(139.41698665,111.35181966)(138.9816025,111.35181966)
\curveto(137.89110763,111.35181966)(137.02644285,111.66920246)(136.38760817,112.30396808)
\curveto(135.7528425,112.93873369)(135.43545966,113.79729487)(135.43545966,114.87965162)
\curveto(135.43545966,115.99862946)(135.73656646,116.88567371)(136.33878004,117.54078438)
\curveto(136.94506264,118.19996405)(137.76089929,118.52955389)(138.78628999,118.52955389)
\curveto(139.70588641,118.52955389)(140.43220482,118.23251613)(140.96524523,117.63844063)
\curveto(141.50235464,117.04843413)(141.77090935,116.24480459)(141.77090935,115.227552)
\closepath
\moveto(140.64786239,115.55714184)
\curveto(140.63972437,116.1715624)(140.46679142,116.66187814)(140.12906353,117.02808908)
\curveto(139.79540465,117.39430001)(139.35188248,117.57740547)(138.79849702,117.57740547)
\curveto(138.17186937,117.57740547)(137.66934655,117.40040352)(137.29092855,117.04639962)
\curveto(136.91657956,116.69239572)(136.700922,116.19394196)(136.64395585,115.55103832)
\lineto(140.64786239,115.55714184)
\closepath
}
}
{
\newrgbcolor{curcolor}{0 0 0}
\pscustom[linestyle=none,fillstyle=solid,fillcolor=curcolor]
{
\newpath
\moveto(98.81436291,101.42750345)
\curveto(99.09512465,101.93206073)(99.43081803,102.30437518)(99.82144306,102.54444679)
\curveto(100.21206809,102.7845184)(100.6718663,102.9045542)(101.2008377,102.9045542)
\curveto(101.91291457,102.9045542)(102.46223102,102.65431006)(102.84878704,102.15382179)
\curveto(103.23534305,101.65740253)(103.42862106,100.94939473)(103.42862106,100.0297984)
\lineto(103.42862106,95.90382192)
\lineto(102.29947059,95.90382192)
\lineto(102.29947059,99.99317731)
\curveto(102.29947059,100.64828797)(102.18350378,101.1345347)(101.95157017,101.45191751)
\curveto(101.71963656,101.76930032)(101.36563263,101.92799172)(100.88955838,101.92799172)
\curveto(100.30768984,101.92799172)(99.84789163,101.73471373)(99.51016374,101.34815775)
\curveto(99.17243585,100.96160177)(99.00357191,100.43466493)(99.00357191,99.76734723)
\lineto(99.00357191,95.90382192)
\lineto(97.87442143,95.90382192)
\lineto(97.87442143,99.99317731)
\curveto(97.87442143,100.65235698)(97.75845463,101.13860371)(97.52652102,101.45191751)
\curveto(97.29458741,101.76930032)(96.93651446,101.92799172)(96.45230219,101.92799172)
\curveto(95.87857168,101.92799172)(95.42284248,101.73267922)(95.08511459,101.34205423)
\curveto(94.7473867,100.95549825)(94.57852275,100.43059592)(94.57852275,99.76734723)
\lineto(94.57852275,95.90382192)
\lineto(93.44937228,95.90382192)
\lineto(93.44937228,102.73975928)
\lineto(94.57852275,102.73975928)
\lineto(94.57852275,101.67774758)
\curveto(94.83487043,102.09685565)(95.14208074,102.40610043)(95.50015368,102.60548194)
\curveto(95.85822662,102.80486345)(96.28343824,102.9045542)(96.77578854,102.9045542)
\curveto(97.27220785,102.9045542)(97.69335046,102.77841488)(98.03921637,102.52613624)
\curveto(98.38915129,102.2738576)(98.64753347,101.90764667)(98.81436291,101.42750345)
\closepath
}
}
{
\newrgbcolor{curcolor}{0 0 0}
\pscustom[linestyle=none,fillstyle=solid,fillcolor=curcolor]
{
\newpath
\moveto(111.52188388,99.60255231)
\lineto(111.52188388,99.05323592)
\lineto(106.35830928,99.05323592)
\curveto(106.40713741,98.28012395)(106.63907102,97.69011746)(107.05411011,97.28321642)
\curveto(107.47321821,96.8803844)(108.05508675,96.67896839)(108.79971571,96.67896839)
\curveto(109.23103084,96.67896839)(109.64810444,96.73186552)(110.0509365,96.83765979)
\curveto(110.45783758,96.94345406)(110.86066964,97.10214546)(111.25943269,97.313734)
\lineto(111.25943269,96.2517223)
\curveto(110.85660063,96.08082387)(110.44359604,95.95061554)(110.02041892,95.86109731)
\curveto(109.59724181,95.77157908)(109.16796118,95.72681997)(108.73257703,95.72681997)
\curveto(107.64208216,95.72681997)(106.77741738,96.04420277)(106.1385827,96.67896839)
\curveto(105.50381703,97.313734)(105.18643419,98.17229518)(105.18643419,99.25465193)
\curveto(105.18643419,100.37362977)(105.48754098,101.26067402)(106.08975457,101.91578469)
\curveto(106.69603717,102.57496436)(107.51187382,102.9045542)(108.53726452,102.9045542)
\curveto(109.45686094,102.9045542)(110.18317935,102.60751645)(110.71621976,102.01344094)
\curveto(111.25332917,101.42343444)(111.52188388,100.6198049)(111.52188388,99.60255231)
\closepath
\moveto(110.39883692,99.93214215)
\curveto(110.3906989,100.54656271)(110.21776594,101.03687846)(109.88003805,101.40308939)
\curveto(109.54637917,101.76930032)(109.10285701,101.95240578)(108.54947155,101.95240578)
\curveto(107.9228439,101.95240578)(107.42032107,101.77540383)(107.04190308,101.42139993)
\curveto(106.66755409,101.06739603)(106.45189652,100.56894227)(106.39493037,99.92603863)
\lineto(110.39883692,99.93214215)
\closepath
}
}
{
\newrgbcolor{curcolor}{0 0 0}
\pscustom[linestyle=none,fillstyle=solid,fillcolor=curcolor]
{
\newpath
\moveto(117.86343678,101.70216165)
\lineto(117.86343678,105.40089204)
\lineto(118.98648374,105.40089204)
\lineto(118.98648374,95.90382192)
\lineto(117.86343678,95.90382192)
\lineto(117.86343678,96.92921252)
\curveto(117.62743416,96.52231149)(117.32836187,96.21917022)(116.96621992,96.01978871)
\curveto(116.60814698,95.82447622)(116.17683184,95.72681997)(115.67227451,95.72681997)
\curveto(114.84626534,95.72681997)(114.17284406,96.05640981)(113.65201069,96.71558948)
\curveto(113.13524633,97.37476915)(112.87686415,98.24146836)(112.87686415,99.31568708)
\curveto(112.87686415,100.38990581)(113.13524633,101.25660501)(113.65201069,101.91578469)
\curveto(114.17284406,102.57496436)(114.84626534,102.9045542)(115.67227451,102.9045542)
\curveto(116.17683184,102.9045542)(116.60814698,102.80486345)(116.96621992,102.60548194)
\curveto(117.32836187,102.41016944)(117.62743416,102.10906268)(117.86343678,101.70216165)
\closepath
\moveto(114.0365322,99.31568708)
\curveto(114.0365322,98.48967799)(114.20539615,97.84067084)(114.54312404,97.36866564)
\curveto(114.88492094,96.90072945)(115.35285717,96.66676136)(115.94693274,96.66676136)
\curveto(116.5410083,96.66676136)(117.00894453,96.90072945)(117.35074143,97.36866564)
\curveto(117.69253833,97.84067084)(117.86343678,98.48967799)(117.86343678,99.31568708)
\curveto(117.86343678,100.14169618)(117.69253833,100.78866883)(117.35074143,101.25660501)
\curveto(117.00894453,101.72861021)(116.5410083,101.96461281)(115.94693274,101.96461281)
\curveto(115.35285717,101.96461281)(114.88492094,101.72861021)(114.54312404,101.25660501)
\curveto(114.20539615,100.78866883)(114.0365322,100.14169618)(114.0365322,99.31568708)
\closepath
}
}
{
\newrgbcolor{curcolor}{0 0 0}
\pscustom[linestyle=none,fillstyle=solid,fillcolor=curcolor]
{
\newpath
\moveto(121.29971603,102.73975928)
\lineto(122.42276299,102.73975928)
\lineto(122.42276299,95.90382192)
\lineto(121.29971603,95.90382192)
\lineto(121.29971603,102.73975928)
\closepath
\moveto(121.29971603,105.40089204)
\lineto(122.42276299,105.40089204)
\lineto(122.42276299,103.97877293)
\lineto(121.29971603,103.97877293)
\lineto(121.29971603,105.40089204)
\closepath
}
}
{
\newrgbcolor{curcolor}{0 0 0}
\pscustom[linestyle=none,fillstyle=solid,fillcolor=curcolor]
{
\newpath
\moveto(129.68594747,102.47730812)
\lineto(129.68594747,101.42750345)
\curveto(129.36856463,101.60247089)(129.04914729,101.73267922)(128.72769544,101.81812844)
\curveto(128.41031261,101.90764667)(128.08886076,101.95240578)(127.7633399,101.95240578)
\curveto(127.03498698,101.95240578)(126.46939449,101.72047219)(126.06656243,101.25660501)
\curveto(125.66373037,100.79680685)(125.46231434,100.1498342)(125.46231434,99.31568708)
\curveto(125.46231434,98.48153997)(125.66373037,97.83253282)(126.06656243,97.36866564)
\curveto(126.46939449,96.90886747)(127.03498698,96.67896839)(127.7633399,96.67896839)
\curveto(128.08886076,96.67896839)(128.41031261,96.721693)(128.72769544,96.80714221)
\curveto(129.04914729,96.89666044)(129.36856463,97.02890328)(129.68594747,97.20387072)
\lineto(129.68594747,96.16627308)
\curveto(129.37263364,96.01978871)(129.04711278,95.90992543)(128.70938489,95.83668325)
\curveto(128.37572602,95.76344106)(128.01968758,95.72681997)(127.64126958,95.72681997)
\curveto(126.61180987,95.72681997)(125.79393872,96.05030629)(125.18765612,96.69727893)
\curveto(124.58137352,97.34425158)(124.27823222,98.21705429)(124.27823222,99.31568708)
\curveto(124.27823222,100.43059592)(124.58340803,101.30746764)(125.19375963,101.94630227)
\curveto(125.80818025,102.58513689)(126.64843097,102.9045542)(127.71451177,102.9045542)
\curveto(128.06037769,102.9045542)(128.39810557,102.86793311)(128.72769544,102.79469092)
\curveto(129.05728531,102.72551775)(129.37670265,102.61972348)(129.68594747,102.47730812)
\closepath
}
}
{
\newrgbcolor{curcolor}{0 0 0}
\pscustom[linestyle=none,fillstyle=solid,fillcolor=curcolor]
{
\newpath
\moveto(131.65127986,102.73975928)
\lineto(132.77432682,102.73975928)
\lineto(132.77432682,95.90382192)
\lineto(131.65127986,95.90382192)
\lineto(131.65127986,102.73975928)
\closepath
\moveto(131.65127986,105.40089204)
\lineto(132.77432682,105.40089204)
\lineto(132.77432682,103.97877293)
\lineto(131.65127986,103.97877293)
\lineto(131.65127986,105.40089204)
\closepath
}
}
{
\newrgbcolor{curcolor}{0 0 0}
\pscustom[linestyle=none,fillstyle=solid,fillcolor=curcolor]
{
\newpath
\moveto(137.7670009,101.95240578)
\curveto(137.16478731,101.95240578)(136.68871306,101.71640318)(136.33877814,101.24439798)
\curveto(135.98884322,100.77646179)(135.81387575,100.13355816)(135.81387575,99.31568708)
\curveto(135.81387575,98.49781601)(135.98680871,97.85287787)(136.33267462,97.38087267)
\curveto(136.68260954,96.91293648)(137.1607183,96.67896839)(137.7670009,96.67896839)
\curveto(138.36514547,96.67896839)(138.83918522,96.91497099)(139.18912015,97.38697619)
\curveto(139.53905507,97.85898138)(139.71402253,98.50188502)(139.71402253,99.31568708)
\curveto(139.71402253,100.12542014)(139.53905507,100.76628927)(139.18912015,101.23829447)
\curveto(138.83918522,101.71436868)(138.36514547,101.95240578)(137.7670009,101.95240578)
\closepath
\moveto(137.7670009,102.9045542)
\curveto(138.74356347,102.9045542)(139.51057199,102.58717139)(140.06802646,101.95240578)
\curveto(140.62548093,101.31764017)(140.90420816,100.43873394)(140.90420816,99.31568708)
\curveto(140.90420816,98.19670924)(140.62548093,97.31780301)(140.06802646,96.67896839)
\curveto(139.51057199,96.04420277)(138.74356347,95.72681997)(137.7670009,95.72681997)
\curveto(136.78636932,95.72681997)(136.01732629,96.04420277)(135.45987182,96.67896839)
\curveto(134.90648636,97.31780301)(134.62979364,98.19670924)(134.62979364,99.31568708)
\curveto(134.62979364,100.43873394)(134.90648636,101.31764017)(135.45987182,101.95240578)
\curveto(136.01732629,102.58717139)(136.78636932,102.9045542)(137.7670009,102.9045542)
\closepath
\moveto(138.61538963,105.90138031)
\lineto(139.82998933,105.90138031)
\lineto(137.84024309,103.60645848)
\lineto(136.90640513,103.60645848)
\lineto(138.61538963,105.90138031)
\closepath
}
}
{
\newrgbcolor{curcolor}{0 0 0}
\pscustom[linestyle=none,fillstyle=solid,fillcolor=curcolor]
{
\newpath
\moveto(148.44205216,100.0297984)
\lineto(148.44205216,95.90382192)
\lineto(147.3190052,95.90382192)
\lineto(147.3190052,99.99317731)
\curveto(147.3190052,100.64014995)(147.19286587,101.12436218)(146.9405872,101.44581399)
\curveto(146.68830854,101.76726581)(146.30989054,101.92799172)(145.80533321,101.92799172)
\curveto(145.19905062,101.92799172)(144.72094186,101.73471373)(144.37100693,101.34815775)
\curveto(144.02107201,100.96160177)(143.84610455,100.43466493)(143.84610455,99.76734723)
\lineto(143.84610455,95.90382192)
\lineto(142.71695408,95.90382192)
\lineto(142.71695408,102.73975928)
\lineto(143.84610455,102.73975928)
\lineto(143.84610455,101.67774758)
\curveto(144.11465926,102.08871763)(144.43000759,102.39592791)(144.79214954,102.59937843)
\curveto(145.15836051,102.80282894)(145.57950312,102.9045542)(146.05557737,102.9045542)
\curveto(146.84089644,102.9045542)(147.43497201,102.66041358)(147.83780407,102.17213234)
\curveto(148.24063613,101.68792011)(148.44205216,100.9738088)(148.44205216,100.0297984)
\closepath
}
}
{
\newrgbcolor{curcolor}{0 0 0}
\pscustom[linestyle=none,fillstyle=solid,fillcolor=curcolor]
{
\newpath
\moveto(250.60025635,111.03402066)
\lineto(256.36197552,111.03402066)
\lineto(256.36197552,109.99642302)
\lineto(251.83316659,109.99642302)
\lineto(251.83316659,107.29866917)
\lineto(256.17276652,107.29866917)
\lineto(256.17276652,106.26107153)
\lineto(251.83316659,106.26107153)
\lineto(251.83316659,102.95906964)
\lineto(256.47183881,102.95906964)
\lineto(256.47183881,101.92147201)
\lineto(250.60025635,101.92147201)
\lineto(250.60025635,111.03402066)
\closepath
}
}
{
\newrgbcolor{curcolor}{0 0 0}
\pscustom[linestyle=none,fillstyle=solid,fillcolor=curcolor]
{
\newpath
\moveto(264.14395858,106.04744849)
\lineto(264.14395858,101.92147201)
\lineto(263.02091162,101.92147201)
\lineto(263.02091162,106.0108274)
\curveto(263.02091162,106.65780004)(262.89477229,107.14201227)(262.64249362,107.46346409)
\curveto(262.39021496,107.7849159)(262.01179696,107.94564181)(261.50723963,107.94564181)
\curveto(260.90095703,107.94564181)(260.42284828,107.75236382)(260.07291335,107.36580784)
\curveto(259.72297843,106.97925186)(259.54801097,106.45231502)(259.54801097,105.78499732)
\lineto(259.54801097,101.92147201)
\lineto(258.4188605,101.92147201)
\lineto(258.4188605,108.75740937)
\lineto(259.54801097,108.75740937)
\lineto(259.54801097,107.69539768)
\curveto(259.81656568,108.10636772)(260.13191401,108.413578)(260.49405596,108.61702852)
\curveto(260.86026693,108.82047903)(261.28140954,108.92220429)(261.75748379,108.92220429)
\curveto(262.54280286,108.92220429)(263.13687842,108.67806367)(263.53971049,108.18978243)
\curveto(263.94254255,107.7055702)(264.14395858,106.99145889)(264.14395858,106.04744849)
\closepath
}
}
{
\newrgbcolor{curcolor}{0 0 0}
\pscustom[linestyle=none,fillstyle=solid,fillcolor=curcolor]
{
\newpath
\moveto(267.50699623,110.6983273)
\lineto(267.50699623,108.75740937)
\lineto(269.82022883,108.75740937)
\lineto(269.82022883,107.88460666)
\lineto(267.50699623,107.88460666)
\lineto(267.50699623,104.17366923)
\curveto(267.50699623,103.61621481)(267.58227293,103.2581419)(267.73282633,103.0994505)
\curveto(267.88744874,102.9407591)(268.19872806,102.8614134)(268.66666429,102.8614134)
\lineto(269.82022883,102.8614134)
\lineto(269.82022883,101.92147201)
\lineto(268.66666429,101.92147201)
\curveto(267.799965,101.92147201)(267.20182043,102.08219792)(266.87223056,102.40364973)
\curveto(266.54264069,102.72917056)(266.37784576,103.31917706)(266.37784576,104.17366923)
\lineto(266.37784576,107.88460666)
\lineto(265.55387109,107.88460666)
\lineto(265.55387109,108.75740937)
\lineto(266.37784576,108.75740937)
\lineto(266.37784576,110.6983273)
\lineto(267.50699623,110.6983273)
\closepath
}
}
{
\newrgbcolor{curcolor}{0 0 0}
\pscustom[linestyle=none,fillstyle=solid,fillcolor=curcolor]
{
\newpath
\moveto(275.26456401,107.70760471)
\curveto(275.13842468,107.78084689)(275.00007831,107.83374403)(274.84952492,107.86629611)
\curveto(274.70304053,107.9029172)(274.5402801,107.92122775)(274.36124363,107.92122775)
\curveto(273.72647796,107.92122775)(273.23819667,107.71370822)(272.89639977,107.29866917)
\curveto(272.55867188,106.88769912)(272.38980794,106.29565812)(272.38980794,105.52254616)
\lineto(272.38980794,101.92147201)
\lineto(271.26065746,101.92147201)
\lineto(271.26065746,108.75740937)
\lineto(272.38980794,108.75740937)
\lineto(272.38980794,107.69539768)
\curveto(272.62581056,108.11043673)(272.93302087,108.41764701)(273.31143886,108.61702852)
\curveto(273.68985686,108.82047903)(274.14965507,108.92220429)(274.6908335,108.92220429)
\curveto(274.7681447,108.92220429)(274.85359393,108.91610078)(274.94718117,108.90389375)
\curveto(275.04076842,108.89575572)(275.14452819,108.88151419)(275.25846049,108.86116914)
\lineto(275.26456401,107.70760471)
\closepath
}
}
{
\newrgbcolor{curcolor}{0 0 0}
\pscustom[linestyle=none,fillstyle=solid,fillcolor=curcolor]
{
\newpath
\moveto(279.56144058,105.35775124)
\curveto(278.65405119,105.35775124)(278.02538903,105.25399147)(277.67545411,105.04647195)
\curveto(277.32551919,104.83895242)(277.15055173,104.48494852)(277.15055173,103.98446025)
\curveto(277.15055173,103.58569724)(277.28076007,103.26831443)(277.54117676,103.03231183)
\curveto(277.80566245,102.80037824)(278.1637354,102.68441145)(278.61539559,102.68441145)
\curveto(279.23795423,102.68441145)(279.73640804,102.90413801)(280.11075703,103.34359112)
\curveto(280.48917502,103.78711325)(280.67838402,104.37508524)(280.67838402,105.1075071)
\lineto(280.67838402,105.35775124)
\lineto(279.56144058,105.35775124)
\closepath
\moveto(281.80143098,105.82161842)
\lineto(281.80143098,101.92147201)
\lineto(280.67838402,101.92147201)
\lineto(280.67838402,102.95906964)
\curveto(280.42203635,102.54403059)(280.102619,102.23682031)(279.720132,102.0374388)
\curveto(279.33764499,101.84212631)(278.86970876,101.74447006)(278.3163233,101.74447006)
\curveto(277.61645346,101.74447006)(277.05899899,101.93978256)(276.64395989,102.33040755)
\curveto(276.23298981,102.72510155)(276.02750477,103.25203839)(276.02750477,103.91121806)
\curveto(276.02750477,104.68026102)(276.28385245,105.26009499)(276.7965478,105.65071998)
\curveto(277.31331216,106.04134497)(278.08235518,106.23665747)(279.10367687,106.23665747)
\lineto(280.67838402,106.23665747)
\lineto(280.67838402,106.34652075)
\curveto(280.67838402,106.86328506)(280.50748557,107.26204807)(280.16568867,107.54280979)
\curveto(279.82796078,107.82764051)(279.35188653,107.97005587)(278.73746591,107.97005587)
\curveto(278.34684088,107.97005587)(277.96638838,107.92326225)(277.5961084,107.82967502)
\curveto(277.22582843,107.73608778)(276.86978999,107.59570692)(276.52799309,107.40853245)
\lineto(276.52799309,108.44613008)
\curveto(276.93896317,108.60482149)(277.33772622,108.72282279)(277.72428224,108.80013398)
\curveto(278.11083826,108.88151419)(278.48722175,108.92220429)(278.85343271,108.92220429)
\curveto(279.84220232,108.92220429)(280.58072776,108.66585664)(281.06900905,108.15316134)
\curveto(281.55729034,107.64046604)(281.80143098,106.86328506)(281.80143098,105.82161842)
\closepath
}
}
{
\newrgbcolor{curcolor}{0 0 0}
\pscustom[linestyle=none,fillstyle=solid,fillcolor=curcolor]
{
\newpath
\moveto(288.61905911,107.71981174)
\lineto(288.61905911,111.41854213)
\lineto(289.74210607,111.41854213)
\lineto(289.74210607,101.92147201)
\lineto(288.61905911,101.92147201)
\lineto(288.61905911,102.94686261)
\curveto(288.38305649,102.53996158)(288.0839842,102.23682031)(287.72184225,102.0374388)
\curveto(287.3637693,101.84212631)(286.93245417,101.74447006)(286.42789684,101.74447006)
\curveto(285.60188766,101.74447006)(284.92846639,102.0740599)(284.40763302,102.73323957)
\curveto(283.89086865,103.39241925)(283.63248647,104.25911845)(283.63248647,105.33333718)
\curveto(283.63248647,106.4075559)(283.89086865,107.27425511)(284.40763302,107.93343478)
\curveto(284.92846639,108.59261445)(285.60188766,108.92220429)(286.42789684,108.92220429)
\curveto(286.93245417,108.92220429)(287.3637693,108.82251354)(287.72184225,108.62313203)
\curveto(288.0839842,108.42781954)(288.38305649,108.12671277)(288.61905911,107.71981174)
\closepath
\moveto(284.79215453,105.33333718)
\curveto(284.79215453,104.50732808)(284.96101847,103.85832093)(285.29874636,103.38631573)
\curveto(285.64054326,102.91837954)(286.1084795,102.68441145)(286.70255506,102.68441145)
\curveto(287.29663063,102.68441145)(287.76456686,102.91837954)(288.10636376,103.38631573)
\curveto(288.44816066,103.85832093)(288.61905911,104.50732808)(288.61905911,105.33333718)
\curveto(288.61905911,106.15934627)(288.44816066,106.80631892)(288.10636376,107.27425511)
\curveto(287.76456686,107.7462603)(287.29663063,107.9822629)(286.70255506,107.9822629)
\curveto(286.1084795,107.9822629)(285.64054326,107.7462603)(285.29874636,107.27425511)
\curveto(284.96101847,106.80631892)(284.79215453,106.15934627)(284.79215453,105.33333718)
\closepath
}
}
{
\newrgbcolor{curcolor}{0 0 0}
\pscustom[linestyle=none,fillstyle=solid,fillcolor=curcolor]
{
\newpath
\moveto(295.16202561,105.35775124)
\curveto(294.25463622,105.35775124)(293.62597407,105.25399147)(293.27603915,105.04647195)
\curveto(292.92610423,104.83895242)(292.75113676,104.48494852)(292.75113676,103.98446025)
\curveto(292.75113676,103.58569724)(292.88134511,103.26831443)(293.14176179,103.03231183)
\curveto(293.40624749,102.80037824)(293.76432043,102.68441145)(294.21598062,102.68441145)
\curveto(294.83853926,102.68441145)(295.33699308,102.90413801)(295.71134206,103.34359112)
\curveto(296.08976006,103.78711325)(296.27896906,104.37508524)(296.27896906,105.1075071)
\lineto(296.27896906,105.35775124)
\lineto(295.16202561,105.35775124)
\closepath
\moveto(297.40201602,105.82161842)
\lineto(297.40201602,101.92147201)
\lineto(296.27896906,101.92147201)
\lineto(296.27896906,102.95906964)
\curveto(296.02262138,102.54403059)(295.70320404,102.23682031)(295.32071703,102.0374388)
\curveto(294.93823003,101.84212631)(294.47029379,101.74447006)(293.91690834,101.74447006)
\curveto(293.21703849,101.74447006)(292.65958402,101.93978256)(292.24454493,102.33040755)
\curveto(291.83357485,102.72510155)(291.62808981,103.25203839)(291.62808981,103.91121806)
\curveto(291.62808981,104.68026102)(291.88443748,105.26009499)(292.39713283,105.65071998)
\curveto(292.91389719,106.04134497)(293.68294022,106.23665747)(294.70426191,106.23665747)
\lineto(296.27896906,106.23665747)
\lineto(296.27896906,106.34652075)
\curveto(296.27896906,106.86328506)(296.10807061,107.26204807)(295.76627371,107.54280979)
\curveto(295.42854582,107.82764051)(294.95247156,107.97005587)(294.33805094,107.97005587)
\curveto(293.94742592,107.97005587)(293.56697341,107.92326225)(293.19669344,107.82967502)
\curveto(292.82641346,107.73608778)(292.47037503,107.59570692)(292.12857812,107.40853245)
\lineto(292.12857812,108.44613008)
\curveto(292.53954821,108.60482149)(292.93831126,108.72282279)(293.32486728,108.80013398)
\curveto(293.71142329,108.88151419)(294.08780679,108.92220429)(294.45401775,108.92220429)
\curveto(295.44278735,108.92220429)(296.1813128,108.66585664)(296.66959409,108.15316134)
\curveto(297.15787537,107.64046604)(297.40201602,106.86328506)(297.40201602,105.82161842)
\closepath
}
}
{
\newrgbcolor{curcolor}{0 0 0}
\pscustom[linestyle=none,fillstyle=solid,fillcolor=curcolor]
{
\newpath
\moveto(168.45156332,95.71223671)
\lineto(168.45156332,94.41218791)
\curveto(168.03652423,94.79874389)(167.59300206,95.08764362)(167.12099681,95.27888711)
\curveto(166.65306058,95.47013059)(166.15460677,95.56575234)(165.62563537,95.56575234)
\curveto(164.58396863,95.56575234)(163.78644253,95.24633503)(163.23305707,94.6075004)
\curveto(162.67967162,93.97273479)(162.40297889,93.05313845)(162.40297889,91.8487114)
\curveto(162.40297889,90.64835335)(162.67967162,89.72875701)(163.23305707,89.08992239)
\curveto(163.78644253,88.45515678)(164.58396863,88.13777397)(165.62563537,88.13777397)
\curveto(166.15460677,88.13777397)(166.65306058,88.23339571)(167.12099681,88.4246392)
\curveto(167.59300206,88.61588268)(168.03652423,88.90478242)(168.45156332,89.2913384)
\lineto(168.45156332,88.00349663)
\curveto(168.02024818,87.71052788)(167.56248448,87.49080133)(167.0782722,87.34431695)
\curveto(166.59812894,87.19783258)(166.0895026,87.1245904)(165.55239318,87.1245904)
\curveto(164.17299855,87.1245904)(163.08657269,87.54573297)(162.2931156,88.38801811)
\curveto(161.49965851,89.23437225)(161.10292996,90.38793669)(161.10292996,91.8487114)
\curveto(161.10292996,93.31355512)(161.49965851,94.46711955)(162.2931156,95.30940469)
\curveto(163.08657269,96.15575884)(164.17299855,96.57893591)(165.55239318,96.57893591)
\curveto(166.09764062,96.57893591)(166.61033597,96.50569372)(167.09047923,96.35920935)
\curveto(167.57469151,96.21679399)(168.0283862,96.00113644)(168.45156332,95.71223671)
\closepath
}
}
{
\newrgbcolor{curcolor}{0 0 0}
\pscustom[linestyle=none,fillstyle=solid,fillcolor=curcolor]
{
\newpath
\moveto(172.968165,93.35017621)
\curveto(172.36595141,93.35017621)(171.88987716,93.11417361)(171.53994224,92.64216841)
\curveto(171.19000731,92.17423222)(171.01503985,91.53132859)(171.01503985,90.71345751)
\curveto(171.01503985,89.89558643)(171.18797281,89.2506483)(171.53383872,88.7786431)
\curveto(171.88377364,88.31070691)(172.3618824,88.07673881)(172.968165,88.07673881)
\curveto(173.56630957,88.07673881)(174.04034932,88.31274141)(174.39028424,88.78474661)
\curveto(174.74021917,89.25675181)(174.91518663,89.89965544)(174.91518663,90.71345751)
\curveto(174.91518663,91.52319057)(174.74021917,92.1640597)(174.39028424,92.6360649)
\curveto(174.04034932,93.1121391)(173.56630957,93.35017621)(172.968165,93.35017621)
\closepath
\moveto(172.968165,94.30232463)
\curveto(173.94472757,94.30232463)(174.71173609,93.98494182)(175.26919056,93.35017621)
\curveto(175.82664503,92.7154106)(176.10537226,91.83650436)(176.10537226,90.71345751)
\curveto(176.10537226,89.59447967)(175.82664503,88.71557344)(175.26919056,88.07673881)
\curveto(174.71173609,87.4419732)(173.94472757,87.1245904)(172.968165,87.1245904)
\curveto(171.98753341,87.1245904)(171.21849039,87.4419732)(170.66103592,88.07673881)
\curveto(170.10765046,88.71557344)(169.83095773,89.59447967)(169.83095773,90.71345751)
\curveto(169.83095773,91.83650436)(170.10765046,92.7154106)(170.66103592,93.35017621)
\curveto(171.21849039,93.98494182)(171.98753341,94.30232463)(172.968165,94.30232463)
\closepath
}
}
{
\newrgbcolor{curcolor}{0 0 0}
\pscustom[linestyle=none,fillstyle=solid,fillcolor=curcolor]
{
\newpath
\moveto(183.64321444,91.42756883)
\lineto(183.64321444,87.30159235)
\lineto(182.52016748,87.30159235)
\lineto(182.52016748,91.39094773)
\curveto(182.52016748,92.03792038)(182.39402815,92.52213261)(182.14174949,92.84358442)
\curveto(181.88947082,93.16503624)(181.51105282,93.32576215)(181.0064955,93.32576215)
\curveto(180.4002129,93.32576215)(179.92210414,93.13248416)(179.57216922,92.74592817)
\curveto(179.2222343,92.35937219)(179.04726684,91.83243535)(179.04726684,91.16511766)
\lineto(179.04726684,87.30159235)
\lineto(177.91811636,87.30159235)
\lineto(177.91811636,94.13752971)
\lineto(179.04726684,94.13752971)
\lineto(179.04726684,93.07551801)
\curveto(179.31582154,93.48648806)(179.63116987,93.79369834)(179.99331183,93.99714885)
\curveto(180.35952279,94.20059937)(180.7806654,94.30232463)(181.25673965,94.30232463)
\curveto(182.04205872,94.30232463)(182.63613429,94.05818401)(183.03896635,93.56990277)
\curveto(183.44179841,93.08569054)(183.64321444,92.37157922)(183.64321444,91.42756883)
\closepath
}
}
{
\newrgbcolor{curcolor}{0 0 0}
\pscustom[linestyle=none,fillstyle=solid,fillcolor=curcolor]
{
\newpath
\moveto(187.0062527,96.07844764)
\lineto(187.0062527,94.13752971)
\lineto(189.3194853,94.13752971)
\lineto(189.3194853,93.26472699)
\lineto(187.0062527,93.26472699)
\lineto(187.0062527,89.55378957)
\curveto(187.0062527,88.99633515)(187.0815294,88.63826224)(187.2320828,88.47957084)
\curveto(187.3867052,88.32087943)(187.69798452,88.24153373)(188.16592076,88.24153373)
\lineto(189.3194853,88.24153373)
\lineto(189.3194853,87.30159235)
\lineto(188.16592076,87.30159235)
\curveto(187.29922147,87.30159235)(186.7010769,87.46231825)(186.37148703,87.78377007)
\curveto(186.04189716,88.1092909)(185.87710223,88.6992974)(185.87710223,89.55378957)
\lineto(185.87710223,93.26472699)
\lineto(185.05312756,93.26472699)
\lineto(185.05312756,94.13752971)
\lineto(185.87710223,94.13752971)
\lineto(185.87710223,96.07844764)
\lineto(187.0062527,96.07844764)
\closepath
}
}
{
\newrgbcolor{curcolor}{0 0 0}
\pscustom[linestyle=none,fillstyle=solid,fillcolor=curcolor]
{
\newpath
\moveto(194.76382048,93.08772504)
\curveto(194.63768115,93.16096723)(194.49933478,93.21386436)(194.34878138,93.24641645)
\curveto(194.202297,93.28303754)(194.03953657,93.30134809)(193.8605001,93.30134809)
\curveto(193.22573443,93.30134809)(192.73745314,93.09382856)(192.39565624,92.6787895)
\curveto(192.05792835,92.26781946)(191.88906441,91.67577846)(191.88906441,90.90266649)
\lineto(191.88906441,87.30159235)
\lineto(190.75991393,87.30159235)
\lineto(190.75991393,94.13752971)
\lineto(191.88906441,94.13752971)
\lineto(191.88906441,93.07551801)
\curveto(192.12506703,93.49055707)(192.43227734,93.79776735)(192.81069533,93.99714885)
\curveto(193.18911333,94.20059937)(193.64891154,94.30232463)(194.19008997,94.30232463)
\curveto(194.26740117,94.30232463)(194.3528504,94.29622111)(194.44643764,94.28401408)
\curveto(194.54002489,94.27587606)(194.64378466,94.26163452)(194.75771696,94.24128947)
\lineto(194.76382048,93.08772504)
\closepath
}
}
{
\newrgbcolor{curcolor}{0 0 0}
\pscustom[linestyle=none,fillstyle=solid,fillcolor=curcolor]
{
\newpath
\moveto(198.33437719,93.35017621)
\curveto(197.73216361,93.35017621)(197.25608935,93.11417361)(196.90615443,92.64216841)
\curveto(196.55621951,92.17423222)(196.38125205,91.53132859)(196.38125205,90.71345751)
\curveto(196.38125205,89.89558643)(196.554185,89.2506483)(196.90005092,88.7786431)
\curveto(197.24998584,88.31070691)(197.7280946,88.07673881)(198.33437719,88.07673881)
\curveto(198.93252177,88.07673881)(199.40656152,88.31274141)(199.75649644,88.78474661)
\curveto(200.10643136,89.25675181)(200.28139882,89.89965544)(200.28139882,90.71345751)
\curveto(200.28139882,91.52319057)(200.10643136,92.1640597)(199.75649644,92.6360649)
\curveto(199.40656152,93.1121391)(198.93252177,93.35017621)(198.33437719,93.35017621)
\closepath
\moveto(198.33437719,94.30232463)
\curveto(199.31093977,94.30232463)(200.07794829,93.98494182)(200.63540275,93.35017621)
\curveto(201.19285722,92.7154106)(201.47158446,91.83650436)(201.47158446,90.71345751)
\curveto(201.47158446,89.59447967)(201.19285722,88.71557344)(200.63540275,88.07673881)
\curveto(200.07794829,87.4419732)(199.31093977,87.1245904)(198.33437719,87.1245904)
\curveto(197.35374561,87.1245904)(196.58470258,87.4419732)(196.02724812,88.07673881)
\curveto(195.47386266,88.71557344)(195.19716993,89.59447967)(195.19716993,90.71345751)
\curveto(195.19716993,91.83650436)(195.47386266,92.7154106)(196.02724812,93.35017621)
\curveto(196.58470258,93.98494182)(197.35374561,94.30232463)(198.33437719,94.30232463)
\closepath
}
}
{
\newrgbcolor{curcolor}{0 0 0}
\pscustom[linestyle=none,fillstyle=solid,fillcolor=curcolor]
{
\newpath
\moveto(203.32705256,96.79866247)
\lineto(204.45009952,96.79866247)
\lineto(204.45009952,87.30159235)
\lineto(203.32705256,87.30159235)
\lineto(203.32705256,96.79866247)
\closepath
}
}
{
\newrgbcolor{curcolor}{0 0 0}
\pscustom[linestyle=none,fillstyle=solid,fillcolor=curcolor]
{
\newpath
\moveto(209.90054215,90.73787157)
\curveto(208.99315276,90.73787157)(208.3644906,90.63411181)(208.01455568,90.42659228)
\curveto(207.66462076,90.21907276)(207.4896533,89.86506886)(207.4896533,89.36458059)
\curveto(207.4896533,88.96581757)(207.61986164,88.64843477)(207.88027833,88.41243217)
\curveto(208.14476402,88.18049858)(208.50283697,88.06453178)(208.95449716,88.06453178)
\curveto(209.5770558,88.06453178)(210.07550961,88.28425834)(210.4498586,88.72371146)
\curveto(210.82827659,89.16723358)(211.01748559,89.75520558)(211.01748559,90.48762744)
\lineto(211.01748559,90.73787157)
\lineto(209.90054215,90.73787157)
\closepath
\moveto(212.14053255,91.20173875)
\lineto(212.14053255,87.30159235)
\lineto(211.01748559,87.30159235)
\lineto(211.01748559,88.33918998)
\curveto(210.76113792,87.92415093)(210.44172057,87.61694065)(210.05923357,87.41755914)
\curveto(209.67674656,87.22224664)(209.20881033,87.1245904)(208.65542487,87.1245904)
\curveto(207.95555503,87.1245904)(207.39810056,87.31990289)(206.98306146,87.71052788)
\curveto(206.57209138,88.10522189)(206.36660634,88.63215873)(206.36660634,89.2913384)
\curveto(206.36660634,90.06038135)(206.62295402,90.64021533)(207.13564937,91.03084032)
\curveto(207.65241373,91.42146531)(208.42145675,91.61677781)(209.44277844,91.61677781)
\lineto(211.01748559,91.61677781)
\lineto(211.01748559,91.72664109)
\curveto(211.01748559,92.2434054)(210.84658714,92.64216841)(210.50479024,92.92293012)
\curveto(210.16706235,93.20776085)(209.6909881,93.35017621)(209.07656748,93.35017621)
\curveto(208.68594245,93.35017621)(208.30548995,93.30338259)(207.93520997,93.20979535)
\curveto(207.56493,93.11620812)(207.20889156,92.97582726)(206.86709466,92.78865278)
\lineto(206.86709466,93.82625042)
\curveto(207.27806474,93.98494182)(207.67682779,94.10294312)(208.06338381,94.18025432)
\curveto(208.44993983,94.26163452)(208.82632332,94.30232463)(209.19253428,94.30232463)
\curveto(210.18130389,94.30232463)(210.91982933,94.04597698)(211.40811062,93.53328167)
\curveto(211.89639191,93.02058637)(212.14053255,92.2434054)(212.14053255,91.20173875)
\closepath
}
}
{
\newrgbcolor{curcolor}{0 0 0}
\pscustom[linestyle=none,fillstyle=solid,fillcolor=curcolor]
{
\newpath
\moveto(218.95816068,93.09993207)
\lineto(218.95816068,96.79866247)
\lineto(220.08120764,96.79866247)
\lineto(220.08120764,87.30159235)
\lineto(218.95816068,87.30159235)
\lineto(218.95816068,88.32698295)
\curveto(218.72215806,87.92008192)(218.42308577,87.61694065)(218.06094381,87.41755914)
\curveto(217.70287087,87.22224664)(217.27155574,87.1245904)(216.76699841,87.1245904)
\curveto(215.94098923,87.1245904)(215.26756796,87.45418023)(214.74673459,88.11335991)
\curveto(214.22997022,88.77253958)(213.97158804,89.63923878)(213.97158804,90.71345751)
\curveto(213.97158804,91.78767624)(214.22997022,92.65437544)(214.74673459,93.31355512)
\curveto(215.26756796,93.97273479)(215.94098923,94.30232463)(216.76699841,94.30232463)
\curveto(217.27155574,94.30232463)(217.70287087,94.20263387)(218.06094381,94.00325237)
\curveto(218.42308577,93.80793987)(218.72215806,93.50683311)(218.95816068,93.09993207)
\closepath
\moveto(215.1312561,90.71345751)
\curveto(215.1312561,89.88744841)(215.30012004,89.23844127)(215.63784793,88.76643607)
\curveto(215.97964483,88.29849988)(216.44758107,88.06453178)(217.04165663,88.06453178)
\curveto(217.63573219,88.06453178)(218.10366843,88.29849988)(218.44546533,88.76643607)
\curveto(218.78726223,89.23844127)(218.95816068,89.88744841)(218.95816068,90.71345751)
\curveto(218.95816068,91.53946661)(218.78726223,92.18643925)(218.44546533,92.65437544)
\curveto(218.10366843,93.12638064)(217.63573219,93.36238324)(217.04165663,93.36238324)
\curveto(216.44758107,93.36238324)(215.97964483,93.12638064)(215.63784793,92.65437544)
\curveto(215.30012004,92.18643925)(215.1312561,91.53946661)(215.1312561,90.71345751)
\closepath
}
}
{
\newrgbcolor{curcolor}{0 0 0}
\pscustom[linestyle=none,fillstyle=solid,fillcolor=curcolor]
{
\newpath
\moveto(225.0433659,93.35017621)
\curveto(224.44115231,93.35017621)(223.96507806,93.11417361)(223.61514314,92.64216841)
\curveto(223.26520822,92.17423222)(223.09024075,91.53132859)(223.09024075,90.71345751)
\curveto(223.09024075,89.89558643)(223.26317371,89.2506483)(223.60903962,88.7786431)
\curveto(223.95897454,88.31070691)(224.4370833,88.07673881)(225.0433659,88.07673881)
\curveto(225.64151047,88.07673881)(226.11555022,88.31274141)(226.46548515,88.78474661)
\curveto(226.81542007,89.25675181)(226.99038753,89.89965544)(226.99038753,90.71345751)
\curveto(226.99038753,91.52319057)(226.81542007,92.1640597)(226.46548515,92.6360649)
\curveto(226.11555022,93.1121391)(225.64151047,93.35017621)(225.0433659,93.35017621)
\closepath
\moveto(225.0433659,94.30232463)
\curveto(226.01992847,94.30232463)(226.78693699,93.98494182)(227.34439146,93.35017621)
\curveto(227.90184593,92.7154106)(228.18057316,91.83650436)(228.18057316,90.71345751)
\curveto(228.18057316,89.59447967)(227.90184593,88.71557344)(227.34439146,88.07673881)
\curveto(226.78693699,87.4419732)(226.01992847,87.1245904)(225.0433659,87.1245904)
\curveto(224.06273432,87.1245904)(223.29369129,87.4419732)(222.73623682,88.07673881)
\curveto(222.18285136,88.71557344)(221.90615864,89.59447967)(221.90615864,90.71345751)
\curveto(221.90615864,91.83650436)(222.18285136,92.7154106)(222.73623682,93.35017621)
\curveto(223.29369129,93.98494182)(224.06273432,94.30232463)(225.0433659,94.30232463)
\closepath
}
}
{
\newrgbcolor{curcolor}{0 0 0}
\pscustom[linestyle=none,fillstyle=solid,fillcolor=curcolor]
{
\newpath
\moveto(233.99722078,93.08772504)
\curveto(233.87108145,93.16096723)(233.73273509,93.21386436)(233.58218169,93.24641645)
\curveto(233.4356973,93.28303754)(233.27293688,93.30134809)(233.0939004,93.30134809)
\curveto(232.45913473,93.30134809)(231.97085345,93.09382856)(231.62905655,92.6787895)
\curveto(231.29132866,92.26781946)(231.12246471,91.67577846)(231.12246471,90.90266649)
\lineto(231.12246471,87.30159235)
\lineto(229.99331424,87.30159235)
\lineto(229.99331424,94.13752971)
\lineto(231.12246471,94.13752971)
\lineto(231.12246471,93.07551801)
\curveto(231.35846733,93.49055707)(231.66567764,93.79776735)(232.04409564,93.99714885)
\curveto(232.42251364,94.20059937)(232.88231185,94.30232463)(233.42349027,94.30232463)
\curveto(233.50080148,94.30232463)(233.5862507,94.29622111)(233.67983795,94.28401408)
\curveto(233.77342519,94.27587606)(233.87718497,94.26163452)(233.99111727,94.24128947)
\lineto(233.99722078,93.08772504)
\closepath
}
}
{
\newrgbcolor{curcolor}{0 0 0}
\pscustom[linestyle=none,fillstyle=solid,fillcolor=curcolor]
{
\newpath
\moveto(333.90916199,96.11506873)
\lineto(333.90916199,94.91267618)
\curveto(333.44122575,95.13647175)(332.99973809,95.30330117)(332.584699,95.41316445)
\curveto(332.1696599,95.52302773)(331.76886235,95.57795937)(331.38230633,95.57795937)
\curveto(330.71091956,95.57795937)(330.1921207,95.44775104)(329.82590973,95.18733438)
\curveto(329.46376778,94.92691771)(329.2826968,94.55663777)(329.2826968,94.07649455)
\curveto(329.2826968,93.67366253)(329.40273262,93.36848676)(329.64280425,93.16096723)
\curveto(329.88694489,92.95751671)(330.3467431,92.79272179)(331.02219888,92.66658247)
\lineto(331.76682784,92.51399459)
\curveto(332.68642427,92.33902714)(333.36391455,92.02978236)(333.7992987,91.58626023)
\curveto(334.23875185,91.14680711)(334.45847843,90.55680061)(334.45847843,89.81624073)
\curveto(334.45847843,88.93326549)(334.16144065,88.26391329)(333.56736509,87.80818413)
\curveto(332.97735853,87.35245497)(332.11065925,87.1245904)(330.96726724,87.1245904)
\curveto(330.5359521,87.1245904)(330.07615389,87.17341852)(329.5878726,87.27107477)
\curveto(329.10366033,87.36873102)(328.6011375,87.51318088)(328.08030413,87.70442437)
\lineto(328.08030413,88.97395559)
\curveto(328.58079245,88.69319388)(329.07110824,88.48160534)(329.55125151,88.33918998)
\curveto(330.03139477,88.19677462)(330.50340002,88.12556694)(330.96726724,88.12556694)
\curveto(331.67120609,88.12556694)(332.21441902,88.26391329)(332.59690603,88.54060599)
\curveto(332.97939304,88.8172987)(333.17063654,89.2119927)(333.17063654,89.724688)
\curveto(333.17063654,90.17227914)(333.03229018,90.52221403)(332.75559745,90.77449267)
\curveto(332.48297373,91.02677131)(332.03334804,91.21598029)(331.40672039,91.34211961)
\lineto(330.65598792,91.48860398)
\curveto(329.73639149,91.67170945)(329.07110824,91.95857467)(328.66013816,92.34919967)
\curveto(328.24916808,92.73982466)(328.04368304,93.28303754)(328.04368304,93.97883831)
\curveto(328.04368304,94.78450235)(328.32647928,95.41926797)(328.89207177,95.88313514)
\curveto(329.46173327,96.34700232)(330.24501783,96.57893591)(331.24192546,96.57893591)
\curveto(331.66917159,96.57893591)(332.10455573,96.54028031)(332.5480779,96.46296912)
\curveto(332.99160007,96.38565792)(333.44529476,96.26969113)(333.90916199,96.11506873)
\closepath
}
}
{
\newrgbcolor{curcolor}{0 0 0}
\pscustom[linestyle=none,fillstyle=solid,fillcolor=curcolor]
{
\newpath
\moveto(336.33225817,94.13752971)
\lineto(337.45530512,94.13752971)
\lineto(337.45530512,87.30159235)
\lineto(336.33225817,87.30159235)
\lineto(336.33225817,94.13752971)
\closepath
\moveto(336.33225817,96.79866247)
\lineto(337.45530512,96.79866247)
\lineto(337.45530512,95.37654336)
\lineto(336.33225817,95.37654336)
\lineto(336.33225817,96.79866247)
\closepath
}
}
{
\newrgbcolor{curcolor}{0 0 0}
\pscustom[linestyle=none,fillstyle=solid,fillcolor=curcolor]
{
\newpath
\moveto(344.15696552,93.9361137)
\lineto(344.15696552,92.874102)
\curveto(343.83958269,93.03686241)(343.50999282,93.15893272)(343.16819592,93.24031293)
\curveto(342.82639902,93.32169314)(342.47239508,93.36238324)(342.10618412,93.36238324)
\curveto(341.54872965,93.36238324)(341.12962155,93.27693402)(340.84885981,93.10603559)
\curveto(340.57216708,92.93513716)(340.43382071,92.6787895)(340.43382071,92.33699264)
\curveto(340.43382071,92.07657597)(340.53351148,91.87109095)(340.732893,91.72053757)
\curveto(340.93227453,91.5740532)(341.33307208,91.43367234)(341.93528567,91.299395)
\lineto(342.31980718,91.21394578)
\curveto(343.11733328,91.04304735)(343.68292577,90.80094123)(344.01658465,90.48762744)
\curveto(344.35431254,90.17838265)(344.52317649,89.74503305)(344.52317649,89.18757864)
\curveto(344.52317649,88.55281302)(344.27089782,88.05029025)(343.76634049,87.68001031)
\curveto(343.26585217,87.30973037)(342.57615486,87.1245904)(341.69724854,87.1245904)
\curveto(341.33103758,87.1245904)(340.94855057,87.16121149)(340.54978752,87.23445367)
\curveto(340.15509348,87.30362685)(339.73801988,87.40942112)(339.29856672,87.55183648)
\lineto(339.29856672,88.71150443)
\curveto(339.71360582,88.49584688)(340.12254139,88.33308647)(340.52537346,88.22322319)
\curveto(340.92820552,88.11742892)(341.32696857,88.06453178)(341.72166261,88.06453178)
\curveto(342.250634,88.06453178)(342.65753507,88.15405001)(342.94236582,88.33308647)
\curveto(343.22719657,88.51619193)(343.36961195,88.77253958)(343.36961195,89.10212942)
\curveto(343.36961195,89.40730519)(343.26585217,89.64127329)(343.05833263,89.8040337)
\curveto(342.85488209,89.96679412)(342.40525641,90.12345101)(341.70945557,90.2740044)
\lineto(341.31883055,90.36555713)
\curveto(340.62302971,90.5120415)(340.12050689,90.73583707)(339.81126207,91.03694383)
\curveto(339.50201726,91.34211961)(339.34739485,91.75919317)(339.34739485,92.28816451)
\curveto(339.34739485,92.93106814)(339.57525945,93.42748741)(340.03098865,93.77742229)
\curveto(340.48671785,94.12735718)(341.13369056,94.30232463)(341.97190677,94.30232463)
\curveto(342.38694586,94.30232463)(342.77757089,94.27180705)(343.14378185,94.2107719)
\curveto(343.50999282,94.14973674)(343.84772071,94.05818401)(344.15696552,93.9361137)
\closepath
}
}
{
\newrgbcolor{curcolor}{0 0 0}
\pscustom[linestyle=none,fillstyle=solid,fillcolor=curcolor]
{
\newpath
\moveto(347.42845037,96.07844764)
\lineto(347.42845037,94.13752971)
\lineto(349.74168296,94.13752971)
\lineto(349.74168296,93.26472699)
\lineto(347.42845037,93.26472699)
\lineto(347.42845037,89.55378957)
\curveto(347.42845037,88.99633515)(347.50372707,88.63826224)(347.65428046,88.47957084)
\curveto(347.80890287,88.32087943)(348.12018219,88.24153373)(348.58811842,88.24153373)
\lineto(349.74168296,88.24153373)
\lineto(349.74168296,87.30159235)
\lineto(348.58811842,87.30159235)
\curveto(347.72141914,87.30159235)(347.12327457,87.46231825)(346.7936847,87.78377007)
\curveto(346.46409483,88.1092909)(346.2992999,88.6992974)(346.2992999,89.55378957)
\lineto(346.2992999,93.26472699)
\lineto(345.47532523,93.26472699)
\lineto(345.47532523,94.13752971)
\lineto(346.2992999,94.13752971)
\lineto(346.2992999,96.07844764)
\lineto(347.42845037,96.07844764)
\closepath
}
}
{
\newrgbcolor{curcolor}{0 0 0}
\pscustom[linestyle=none,fillstyle=solid,fillcolor=curcolor]
{
\newpath
\moveto(357.07200703,91.00032274)
\lineto(357.07200703,90.45100635)
\lineto(351.90843243,90.45100635)
\curveto(351.95726056,89.67789438)(352.18919417,89.08788788)(352.60423327,88.68098685)
\curveto(353.02334137,88.27815483)(353.6052099,88.07673881)(354.34983886,88.07673881)
\curveto(354.781154,88.07673881)(355.1982276,88.12963595)(355.60105966,88.23543022)
\curveto(356.00796073,88.34122449)(356.41079279,88.49991589)(356.80955584,88.71150443)
\lineto(356.80955584,87.64949273)
\curveto(356.40672378,87.4785943)(355.99371919,87.34838596)(355.57054208,87.25886774)
\curveto(355.14736496,87.16934951)(354.71808433,87.1245904)(354.28270019,87.1245904)
\curveto(353.19220531,87.1245904)(352.32754054,87.4419732)(351.68870585,88.07673881)
\curveto(351.05394018,88.71150443)(350.73655735,89.57006561)(350.73655735,90.65242236)
\curveto(350.73655735,91.7714002)(351.03766414,92.65844445)(351.63987773,93.31355512)
\curveto(352.24616032,93.97273479)(353.06199697,94.30232463)(354.08738767,94.30232463)
\curveto(355.00698409,94.30232463)(355.73330251,94.00528687)(356.26634291,93.41121136)
\curveto(356.80345233,92.82120487)(357.07200703,92.01757532)(357.07200703,91.00032274)
\closepath
\moveto(355.94896008,91.32991258)
\curveto(355.94082205,91.94433314)(355.7678891,92.43464888)(355.43016121,92.80085981)
\curveto(355.09650233,93.16707074)(354.65298016,93.35017621)(354.0995947,93.35017621)
\curveto(353.47296705,93.35017621)(352.97044423,93.17317426)(352.59202623,92.81917036)
\curveto(352.21767725,92.46516646)(352.00201968,91.9667127)(351.94505353,91.32380906)
\lineto(355.94896008,91.32991258)
\closepath
}
}
{
\newrgbcolor{curcolor}{0 0 0}
\pscustom[linestyle=none,fillstyle=solid,fillcolor=curcolor]
{
\newpath
\moveto(364.23753461,92.82527388)
\curveto(364.51829635,93.32983116)(364.85398973,93.7021456)(365.24461476,93.94221721)
\curveto(365.63523979,94.18228882)(366.095038,94.30232463)(366.6240094,94.30232463)
\curveto(367.33608627,94.30232463)(367.88540272,94.05208049)(368.27195874,93.55159222)
\curveto(368.65851475,93.05517296)(368.85179276,92.34716516)(368.85179276,91.42756883)
\lineto(368.85179276,87.30159235)
\lineto(367.72264229,87.30159235)
\lineto(367.72264229,91.39094773)
\curveto(367.72264229,92.0460584)(367.60667548,92.53230513)(367.37474187,92.84968794)
\curveto(367.14280826,93.16707074)(366.78880433,93.32576215)(366.31273008,93.32576215)
\curveto(365.73086154,93.32576215)(365.27106333,93.13248416)(364.93333544,92.74592817)
\curveto(364.59560755,92.35937219)(364.42674361,91.83243535)(364.42674361,91.16511766)
\lineto(364.42674361,87.30159235)
\lineto(363.29759313,87.30159235)
\lineto(363.29759313,91.39094773)
\curveto(363.29759313,92.05012741)(363.18162633,92.53637414)(362.94969272,92.84968794)
\curveto(362.71775911,93.16707074)(362.35968616,93.32576215)(361.87547389,93.32576215)
\curveto(361.30174338,93.32576215)(360.84601418,93.13044965)(360.50828629,92.73982466)
\curveto(360.1705584,92.35326868)(360.00169445,91.82836634)(360.00169445,91.16511766)
\lineto(360.00169445,87.30159235)
\lineto(358.87254398,87.30159235)
\lineto(358.87254398,94.13752971)
\lineto(360.00169445,94.13752971)
\lineto(360.00169445,93.07551801)
\curveto(360.25804213,93.49462608)(360.56525244,93.80387086)(360.92332538,94.00325237)
\curveto(361.28139832,94.20263387)(361.70660994,94.30232463)(362.19896024,94.30232463)
\curveto(362.69537955,94.30232463)(363.11652216,94.17618531)(363.46238807,93.92390667)
\curveto(363.81232299,93.67162803)(364.07070517,93.3054171)(364.23753461,92.82527388)
\closepath
}
}
{
\newrgbcolor{curcolor}{0 0 0}
\pscustom[linestyle=none,fillstyle=solid,fillcolor=curcolor]
{
\newpath
\moveto(374.20457444,90.73787157)
\curveto(373.29718505,90.73787157)(372.66852289,90.63411181)(372.31858797,90.42659228)
\curveto(371.96865305,90.21907276)(371.79368559,89.86506886)(371.79368559,89.36458059)
\curveto(371.79368559,88.96581757)(371.92389393,88.64843477)(372.18431062,88.41243217)
\curveto(372.44879631,88.18049858)(372.80686926,88.06453178)(373.25852945,88.06453178)
\curveto(373.88108809,88.06453178)(374.3795419,88.28425834)(374.75389089,88.72371146)
\curveto(375.13230888,89.16723358)(375.32151788,89.75520558)(375.32151788,90.48762744)
\lineto(375.32151788,90.73787157)
\lineto(374.20457444,90.73787157)
\closepath
\moveto(376.44456484,91.20173875)
\lineto(376.44456484,87.30159235)
\lineto(375.32151788,87.30159235)
\lineto(375.32151788,88.33918998)
\curveto(375.06517021,87.92415093)(374.74575286,87.61694065)(374.36326586,87.41755914)
\curveto(373.98077885,87.22224664)(373.51284262,87.1245904)(372.95945716,87.1245904)
\curveto(372.25958732,87.1245904)(371.70213285,87.31990289)(371.28709375,87.71052788)
\curveto(370.87612367,88.10522189)(370.67063863,88.63215873)(370.67063863,89.2913384)
\curveto(370.67063863,90.06038135)(370.92698631,90.64021533)(371.43968166,91.03084032)
\curveto(371.95644602,91.42146531)(372.72548904,91.61677781)(373.74681073,91.61677781)
\lineto(375.32151788,91.61677781)
\lineto(375.32151788,91.72664109)
\curveto(375.32151788,92.2434054)(375.15061943,92.64216841)(374.80882253,92.92293012)
\curveto(374.47109464,93.20776085)(373.99502039,93.35017621)(373.38059977,93.35017621)
\curveto(372.98997474,93.35017621)(372.60952224,93.30338259)(372.23924226,93.20979535)
\curveto(371.86896229,93.11620812)(371.51292385,92.97582726)(371.17112695,92.78865278)
\lineto(371.17112695,93.82625042)
\curveto(371.58209703,93.98494182)(371.98086008,94.10294312)(372.3674161,94.18025432)
\curveto(372.75397212,94.26163452)(373.13035561,94.30232463)(373.49656657,94.30232463)
\curveto(374.48533618,94.30232463)(375.22386162,94.04597698)(375.71214291,93.53328167)
\curveto(376.2004242,93.02058637)(376.44456484,92.2434054)(376.44456484,91.20173875)
\closepath
}
}
{
\newrgbcolor{curcolor}{0 0 0}
\pscustom[linestyle=none,fillstyle=solid,fillcolor=curcolor]
{
\newpath
\moveto(424.81812584,110.73495082)
\lineto(424.81812584,109.53255826)
\curveto(424.3501896,109.75635383)(423.90870194,109.92318325)(423.49366285,110.03304653)
\curveto(423.07862375,110.14290981)(422.6778262,110.19784145)(422.29127018,110.19784145)
\curveto(421.61988341,110.19784145)(421.10108455,110.06763312)(420.73487358,109.80721646)
\curveto(420.37273163,109.5467998)(420.19166065,109.17651986)(420.19166065,108.69637664)
\curveto(420.19166065,108.29354462)(420.31169647,107.98836884)(420.5517681,107.78084931)
\curveto(420.79590874,107.5773988)(421.25570695,107.41260388)(421.93116273,107.28646456)
\lineto(422.67579169,107.13387667)
\curveto(423.59538812,106.95890923)(424.2728784,106.64966444)(424.70826255,106.20614231)
\curveto(425.1477157,105.7666892)(425.36744228,105.1766827)(425.36744228,104.43612282)
\curveto(425.36744228,103.55314757)(425.0704045,102.88379537)(424.47632894,102.42806622)
\curveto(423.88632238,101.97233706)(423.0196231,101.74447248)(421.87623109,101.74447248)
\curveto(421.44491595,101.74447248)(420.98511774,101.7933006)(420.49683645,101.89095685)
\curveto(420.01262418,101.9886131)(419.51010135,102.13306297)(418.98926798,102.32430645)
\lineto(418.98926798,103.59383768)
\curveto(419.4897563,103.31307596)(419.98007209,103.10148743)(420.46021536,102.95907207)
\curveto(420.94035862,102.8166567)(421.41236387,102.74544902)(421.87623109,102.74544902)
\curveto(422.58016994,102.74544902)(423.12338287,102.88379537)(423.50586988,103.16048808)
\curveto(423.88835689,103.43718078)(424.07960039,103.83187478)(424.07960039,104.34457008)
\curveto(424.07960039,104.79216122)(423.94125403,105.14209611)(423.6645613,105.39437475)
\curveto(423.39193758,105.64665339)(422.9423119,105.83586237)(422.31568424,105.96200169)
\lineto(421.56495177,106.10848607)
\curveto(420.64535535,106.29159153)(419.98007209,106.57845676)(419.56910201,106.96908175)
\curveto(419.15813193,107.35970674)(418.95264689,107.90291962)(418.95264689,108.59872039)
\curveto(418.95264689,109.40438444)(419.23544313,110.03915005)(419.80103562,110.50301723)
\curveto(420.37069712,110.96688441)(421.15398168,111.198818)(422.15088931,111.198818)
\curveto(422.57813544,111.198818)(423.01351958,111.1601624)(423.45704175,111.0828512)
\curveto(423.90056392,111.00554)(424.35425861,110.88957321)(424.81812584,110.73495082)
\closepath
}
}
{
\newrgbcolor{curcolor}{0 0 0}
\pscustom[linestyle=none,fillstyle=solid,fillcolor=curcolor]
{
\newpath
\moveto(430.3479117,105.35775366)
\curveto(429.44052231,105.35775366)(428.81186015,105.25399389)(428.46192523,105.04647437)
\curveto(428.11199031,104.83895484)(427.93702285,104.48495094)(427.93702285,103.98446267)
\curveto(427.93702285,103.58569966)(428.06723119,103.26831685)(428.32764788,103.03231425)
\curveto(428.59213357,102.80038066)(428.95020652,102.68441387)(429.40186671,102.68441387)
\curveto(430.02442535,102.68441387)(430.52287916,102.90414043)(430.89722815,103.34359354)
\curveto(431.27564614,103.78711567)(431.46485514,104.37508766)(431.46485514,105.10750952)
\lineto(431.46485514,105.35775366)
\lineto(430.3479117,105.35775366)
\closepath
\moveto(432.5879021,105.82162084)
\lineto(432.5879021,101.92147443)
\lineto(431.46485514,101.92147443)
\lineto(431.46485514,102.95907207)
\curveto(431.20850747,102.54403301)(430.88909013,102.23682273)(430.50660312,102.03744122)
\curveto(430.12411611,101.84212873)(429.65617988,101.74447248)(429.10279442,101.74447248)
\curveto(428.40292458,101.74447248)(427.84547011,101.93978498)(427.43043101,102.33040997)
\curveto(427.01946093,102.72510397)(426.81397589,103.25204081)(426.81397589,103.91122048)
\curveto(426.81397589,104.68026344)(427.07032357,105.26009741)(427.58301892,105.6507224)
\curveto(428.09978328,106.04134739)(428.8688263,106.23665989)(429.89014799,106.23665989)
\lineto(431.46485514,106.23665989)
\lineto(431.46485514,106.34652317)
\curveto(431.46485514,106.86328748)(431.29395669,107.2620505)(430.95215979,107.54281221)
\curveto(430.6144319,107.82764293)(430.13835765,107.97005829)(429.52393703,107.97005829)
\curveto(429.133312,107.97005829)(428.7528595,107.92326467)(428.38257952,107.82967744)
\curveto(428.01229955,107.7360902)(427.65626111,107.59570934)(427.31446421,107.40853487)
\lineto(427.31446421,108.4461325)
\curveto(427.72543429,108.60482391)(428.12419734,108.72282521)(428.51075336,108.8001364)
\curveto(428.89730938,108.88151661)(429.27369287,108.92220671)(429.63990383,108.92220671)
\curveto(430.62867344,108.92220671)(431.36719888,108.66585906)(431.85548017,108.15316376)
\curveto(432.34376146,107.64046846)(432.5879021,106.86328748)(432.5879021,105.82162084)
\closepath
}
}
{
\newrgbcolor{curcolor}{0 0 0}
\pscustom[linestyle=none,fillstyle=solid,fillcolor=curcolor]
{
\newpath
\moveto(434.90723828,111.41854455)
\lineto(436.03028523,111.41854455)
\lineto(436.03028523,101.92147443)
\lineto(434.90723828,101.92147443)
\lineto(434.90723828,111.41854455)
\closepath
}
}
{
\newrgbcolor{curcolor}{0 0 0}
\pscustom[linestyle=none,fillstyle=solid,fillcolor=curcolor]
{
\newpath
\moveto(438.37403576,108.75741179)
\lineto(439.49708271,108.75741179)
\lineto(439.49708271,101.92147443)
\lineto(438.37403576,101.92147443)
\lineto(438.37403576,108.75741179)
\closepath
\moveto(438.37403576,111.41854455)
\lineto(439.49708271,111.41854455)
\lineto(439.49708271,109.99642544)
\lineto(438.37403576,109.99642544)
\lineto(438.37403576,111.41854455)
\closepath
}
}
{
\newrgbcolor{curcolor}{0 0 0}
\pscustom[linestyle=none,fillstyle=solid,fillcolor=curcolor]
{
\newpath
\moveto(446.33912459,107.71981416)
\lineto(446.33912459,111.41854455)
\lineto(447.46217154,111.41854455)
\lineto(447.46217154,101.92147443)
\lineto(446.33912459,101.92147443)
\lineto(446.33912459,102.94686503)
\curveto(446.10312196,102.539964)(445.80404968,102.23682273)(445.44190772,102.03744122)
\curveto(445.08383478,101.84212873)(444.65251964,101.74447248)(444.14796231,101.74447248)
\curveto(443.32195314,101.74447248)(442.64853187,102.07406232)(442.12769849,102.73324199)
\curveto(441.61093413,103.39242167)(441.35255195,104.25912087)(441.35255195,105.3333396)
\curveto(441.35255195,106.40755832)(441.61093413,107.27425753)(442.12769849,107.9334372)
\curveto(442.64853187,108.59261688)(443.32195314,108.92220671)(444.14796231,108.92220671)
\curveto(444.65251964,108.92220671)(445.08383478,108.82251596)(445.44190772,108.62313445)
\curveto(445.80404968,108.42782196)(446.10312196,108.12671519)(446.33912459,107.71981416)
\closepath
\moveto(442.51222001,105.3333396)
\curveto(442.51222001,104.5073305)(442.68108395,103.85832335)(443.01881184,103.38631815)
\curveto(443.36060874,102.91838196)(443.82854497,102.68441387)(444.42262054,102.68441387)
\curveto(445.0166961,102.68441387)(445.48463234,102.91838196)(445.82642924,103.38631815)
\curveto(446.16822614,103.85832335)(446.33912459,104.5073305)(446.33912459,105.3333396)
\curveto(446.33912459,106.15934869)(446.16822614,106.80632134)(445.82642924,107.27425753)
\curveto(445.48463234,107.74626273)(445.0166961,107.98226532)(444.42262054,107.98226532)
\curveto(443.82854497,107.98226532)(443.36060874,107.74626273)(443.01881184,107.27425753)
\curveto(442.68108395,106.80632134)(442.51222001,106.15934869)(442.51222001,105.3333396)
\closepath
}
}
{
\newrgbcolor{curcolor}{0 0 0}
\pscustom[linestyle=none,fillstyle=solid,fillcolor=curcolor]
{
\newpath
\moveto(452.88209351,105.35775366)
\curveto(451.97470412,105.35775366)(451.34604197,105.25399389)(450.99610705,105.04647437)
\curveto(450.64617212,104.83895484)(450.47120466,104.48495094)(450.47120466,103.98446267)
\curveto(450.47120466,103.58569966)(450.60141301,103.26831685)(450.86182969,103.03231425)
\curveto(451.12631539,102.80038066)(451.48438833,102.68441387)(451.93604852,102.68441387)
\curveto(452.55860716,102.68441387)(453.05706097,102.90414043)(453.43140996,103.34359354)
\curveto(453.80982796,103.78711567)(453.99903696,104.37508766)(453.99903696,105.10750952)
\lineto(453.99903696,105.35775366)
\lineto(452.88209351,105.35775366)
\closepath
\moveto(455.12208391,105.82162084)
\lineto(455.12208391,101.92147443)
\lineto(453.99903696,101.92147443)
\lineto(453.99903696,102.95907207)
\curveto(453.74268928,102.54403301)(453.42327194,102.23682273)(453.04078493,102.03744122)
\curveto(452.65829792,101.84212873)(452.19036169,101.74447248)(451.63697623,101.74447248)
\curveto(450.93710639,101.74447248)(450.37965192,101.93978498)(449.96461283,102.33040997)
\curveto(449.55364275,102.72510397)(449.34815771,103.25204081)(449.34815771,103.91122048)
\curveto(449.34815771,104.68026344)(449.60450538,105.26009741)(450.11720073,105.6507224)
\curveto(450.63396509,106.04134739)(451.40300812,106.23665989)(452.42432981,106.23665989)
\lineto(453.99903696,106.23665989)
\lineto(453.99903696,106.34652317)
\curveto(453.99903696,106.86328748)(453.82813851,107.2620505)(453.48634161,107.54281221)
\curveto(453.14861372,107.82764293)(452.67253946,107.97005829)(452.05811884,107.97005829)
\curveto(451.66749381,107.97005829)(451.28704131,107.92326467)(450.91676134,107.82967744)
\curveto(450.54648136,107.7360902)(450.19044292,107.59570934)(449.84864602,107.40853487)
\lineto(449.84864602,108.4461325)
\curveto(450.25961611,108.60482391)(450.65837916,108.72282521)(451.04493517,108.8001364)
\curveto(451.43149119,108.88151661)(451.80787468,108.92220671)(452.17408565,108.92220671)
\curveto(453.16285525,108.92220671)(453.9013807,108.66585906)(454.38966198,108.15316376)
\curveto(454.87794327,107.64046846)(455.12208391,106.86328748)(455.12208391,105.82162084)
\closepath
}
}
{
\newrgbcolor{curcolor}{0 0 0}
\pscustom[linestyle=none,fillstyle=solid,fillcolor=curcolor]
{
\newpath
\moveto(115.26189915,42.11171076)
\lineto(117.09905749,42.11171076)
\lineto(119.42449712,35.910539)
\lineto(121.76214378,42.11171076)
\lineto(123.59930211,42.11171076)
\lineto(123.59930211,32.99916211)
\lineto(122.39690945,32.99916211)
\lineto(122.39690945,41.00087093)
\lineto(120.04705576,34.75087106)
\lineto(118.80804199,34.75087106)
\lineto(116.45818831,41.00087093)
\lineto(116.45818831,32.99916211)
\lineto(115.26189915,32.99916211)
\lineto(115.26189915,42.11171076)
\closepath
}
}
{
\newrgbcolor{curcolor}{0 0 0}
\pscustom[linestyle=none,fillstyle=solid,fillcolor=curcolor]
{
\newpath
\moveto(131.85125592,36.6978925)
\lineto(131.85125592,36.14857611)
\lineto(126.68768132,36.14857611)
\curveto(126.73650944,35.37546415)(126.96844306,34.78545765)(127.38348215,34.37855661)
\curveto(127.80259025,33.97572459)(128.38445879,33.77430858)(129.12908775,33.77430858)
\curveto(129.56040288,33.77430858)(129.97747648,33.82720571)(130.38030854,33.93299998)
\curveto(130.78720961,34.03879425)(131.19004168,34.19748565)(131.58880473,34.40907419)
\lineto(131.58880473,33.34706249)
\curveto(131.18597266,33.17616406)(130.77296808,33.04595573)(130.34979096,32.9564375)
\curveto(129.92661385,32.86691927)(129.49733322,32.82216016)(129.06194907,32.82216016)
\curveto(127.9714542,32.82216016)(127.10678942,33.13954297)(126.46795474,33.77430858)
\curveto(125.83318907,34.40907419)(125.51580623,35.26763537)(125.51580623,36.34999212)
\curveto(125.51580623,37.46896996)(125.81691302,38.35601422)(126.41912661,39.01112488)
\curveto(127.02540921,39.67030456)(127.84124585,39.99989439)(128.86663656,39.99989439)
\curveto(129.78623298,39.99989439)(130.51255139,39.70285664)(131.04559179,39.10878113)
\curveto(131.58270121,38.51877463)(131.85125592,37.71514509)(131.85125592,36.6978925)
\closepath
\moveto(130.72820896,37.02748234)
\curveto(130.72007094,37.6419029)(130.54713798,38.13221865)(130.20941009,38.49842958)
\curveto(129.87575121,38.86464051)(129.43222905,39.04774597)(128.87884359,39.04774597)
\curveto(128.25221594,39.04774597)(127.74969311,38.87074402)(127.37127512,38.51674012)
\curveto(126.99692613,38.16273623)(126.78126856,37.66428246)(126.72430241,37.02137883)
\lineto(130.72820896,37.02748234)
\closepath
}
}
{
\newrgbcolor{curcolor}{0 0 0}
\pscustom[linestyle=none,fillstyle=solid,fillcolor=curcolor]
{
\newpath
\moveto(138.19280882,38.79750184)
\lineto(138.19280882,42.49623223)
\lineto(139.31585578,42.49623223)
\lineto(139.31585578,32.99916211)
\lineto(138.19280882,32.99916211)
\lineto(138.19280882,34.02455271)
\curveto(137.9568062,33.61765168)(137.65773391,33.31451041)(137.29559196,33.1151289)
\curveto(136.93751902,32.91981641)(136.50620388,32.82216016)(136.00164655,32.82216016)
\curveto(135.17563738,32.82216016)(134.5022161,33.15175)(133.98138273,33.81092967)
\curveto(133.46461837,34.47010935)(133.20623619,35.33680855)(133.20623619,36.41102728)
\curveto(133.20623619,37.485246)(133.46461837,38.35194521)(133.98138273,39.01112488)
\curveto(134.5022161,39.67030456)(135.17563738,39.99989439)(136.00164655,39.99989439)
\curveto(136.50620388,39.99989439)(136.93751902,39.90020364)(137.29559196,39.70082213)
\curveto(137.65773391,39.50550964)(137.9568062,39.20440287)(138.19280882,38.79750184)
\closepath
\moveto(134.36590424,36.41102728)
\curveto(134.36590424,35.58501818)(134.53476819,34.93601103)(134.87249608,34.46400583)
\curveto(135.21429298,33.99606964)(135.68222921,33.76210155)(136.27630477,33.76210155)
\curveto(136.87038034,33.76210155)(137.33831657,33.99606964)(137.68011347,34.46400583)
\curveto(138.02191037,34.93601103)(138.19280882,35.58501818)(138.19280882,36.41102728)
\curveto(138.19280882,37.23703637)(138.02191037,37.88400902)(137.68011347,38.35194521)
\curveto(137.33831657,38.82395041)(136.87038034,39.059953)(136.27630477,39.059953)
\curveto(135.68222921,39.059953)(135.21429298,38.82395041)(134.87249608,38.35194521)
\curveto(134.53476819,37.88400902)(134.36590424,37.23703637)(134.36590424,36.41102728)
\closepath
}
}
{
\newrgbcolor{curcolor}{0 0 0}
\pscustom[linestyle=none,fillstyle=solid,fillcolor=curcolor]
{
\newpath
\moveto(141.62908807,39.83509947)
\lineto(142.75213502,39.83509947)
\lineto(142.75213502,32.99916211)
\lineto(141.62908807,32.99916211)
\lineto(141.62908807,39.83509947)
\closepath
\moveto(141.62908807,42.49623223)
\lineto(142.75213502,42.49623223)
\lineto(142.75213502,41.07411312)
\lineto(141.62908807,41.07411312)
\lineto(141.62908807,42.49623223)
\closepath
}
}
{
\newrgbcolor{curcolor}{0 0 0}
\pscustom[linestyle=none,fillstyle=solid,fillcolor=curcolor]
{
\newpath
\moveto(149.5941769,38.79750184)
\lineto(149.5941769,42.49623223)
\lineto(150.71722385,42.49623223)
\lineto(150.71722385,32.99916211)
\lineto(149.5941769,32.99916211)
\lineto(149.5941769,34.02455271)
\curveto(149.35817427,33.61765168)(149.05910199,33.31451041)(148.69696003,33.1151289)
\curveto(148.33888709,32.91981641)(147.90757195,32.82216016)(147.40301463,32.82216016)
\curveto(146.57700545,32.82216016)(145.90358418,33.15175)(145.3827508,33.81092967)
\curveto(144.86598644,34.47010935)(144.60760426,35.33680855)(144.60760426,36.41102728)
\curveto(144.60760426,37.485246)(144.86598644,38.35194521)(145.3827508,39.01112488)
\curveto(145.90358418,39.67030456)(146.57700545,39.99989439)(147.40301463,39.99989439)
\curveto(147.90757195,39.99989439)(148.33888709,39.90020364)(148.69696003,39.70082213)
\curveto(149.05910199,39.50550964)(149.35817427,39.20440287)(149.5941769,38.79750184)
\closepath
\moveto(145.76727232,36.41102728)
\curveto(145.76727232,35.58501818)(145.93613626,34.93601103)(146.27386415,34.46400583)
\curveto(146.61566105,33.99606964)(147.08359728,33.76210155)(147.67767285,33.76210155)
\curveto(148.27174841,33.76210155)(148.73968465,33.99606964)(149.08148155,34.46400583)
\curveto(149.42327845,34.93601103)(149.5941769,35.58501818)(149.5941769,36.41102728)
\curveto(149.5941769,37.23703637)(149.42327845,37.88400902)(149.08148155,38.35194521)
\curveto(148.73968465,38.82395041)(148.27174841,39.059953)(147.67767285,39.059953)
\curveto(147.08359728,39.059953)(146.61566105,38.82395041)(146.27386415,38.35194521)
\curveto(145.93613626,37.88400902)(145.76727232,37.23703637)(145.76727232,36.41102728)
\closepath
}
}
{
\newrgbcolor{curcolor}{0 0 0}
\pscustom[linestyle=none,fillstyle=solid,fillcolor=curcolor]
{
\newpath
\moveto(156.13714582,36.43544134)
\curveto(155.22975643,36.43544134)(154.60109428,36.33168157)(154.25115936,36.12416205)
\curveto(153.90122443,35.91664252)(153.72625697,35.56263862)(153.72625697,35.06215035)
\curveto(153.72625697,34.66338734)(153.85646532,34.34600453)(154.116882,34.11000193)
\curveto(154.3813677,33.87806834)(154.73944064,33.76210155)(155.19110083,33.76210155)
\curveto(155.81365947,33.76210155)(156.31211328,33.98182811)(156.68646227,34.42128122)
\curveto(157.06488027,34.86480335)(157.25408927,35.45277534)(157.25408927,36.1851972)
\lineto(157.25408927,36.43544134)
\lineto(156.13714582,36.43544134)
\closepath
\moveto(158.37713622,36.89930852)
\lineto(158.37713622,32.99916211)
\lineto(157.25408927,32.99916211)
\lineto(157.25408927,34.03675975)
\curveto(156.99774159,33.62172069)(156.67832425,33.31451041)(156.29583724,33.1151289)
\curveto(155.91335023,32.91981641)(155.445414,32.82216016)(154.89202854,32.82216016)
\curveto(154.1921587,32.82216016)(153.63470423,33.01747266)(153.21966514,33.40809765)
\curveto(152.80869506,33.80279165)(152.60321002,34.32972849)(152.60321002,34.98890816)
\curveto(152.60321002,35.75795112)(152.85955769,36.33778509)(153.37225304,36.72841008)
\curveto(153.8890174,37.11903507)(154.65806043,37.31434757)(155.67938212,37.31434757)
\lineto(157.25408927,37.31434757)
\lineto(157.25408927,37.42421085)
\curveto(157.25408927,37.94097516)(157.08319082,38.33973818)(156.74139392,38.62049989)
\curveto(156.40366603,38.90533061)(155.92759177,39.04774597)(155.31317115,39.04774597)
\curveto(154.92254612,39.04774597)(154.54209362,39.00095235)(154.17181365,38.90736512)
\curveto(153.80153367,38.81377788)(153.44549523,38.67339702)(153.10369833,38.48622255)
\lineto(153.10369833,39.52382018)
\curveto(153.51466842,39.68251159)(153.91343147,39.80051289)(154.29998748,39.87782408)
\curveto(154.6865435,39.95920429)(155.06292699,39.99989439)(155.42913796,39.99989439)
\curveto(156.41790756,39.99989439)(157.15643301,39.74354674)(157.64471429,39.23085144)
\curveto(158.13299558,38.71815614)(158.37713622,37.94097516)(158.37713622,36.89930852)
\closepath
}
}
{
\newrgbcolor{curcolor}{0 0 0}
\pscustom[linestyle=none,fillstyle=solid,fillcolor=curcolor]
{
\newpath
\moveto(167.32489325,39.04774597)
\curveto(166.72267966,39.04774597)(166.24660541,38.81174337)(165.89667048,38.33973818)
\curveto(165.54673556,37.87180199)(165.3717681,37.22889835)(165.3717681,36.41102728)
\curveto(165.3717681,35.5931562)(165.54470106,34.94821806)(165.89056697,34.47621286)
\curveto(166.24050189,34.00827667)(166.71861065,33.77430858)(167.32489325,33.77430858)
\curveto(167.92303782,33.77430858)(168.39707757,34.01031118)(168.74701249,34.48231638)
\curveto(169.09694741,34.95432158)(169.27191487,35.59722521)(169.27191487,36.41102728)
\curveto(169.27191487,37.22076033)(169.09694741,37.86162946)(168.74701249,38.33363466)
\curveto(168.39707757,38.80970887)(167.92303782,39.04774597)(167.32489325,39.04774597)
\closepath
\moveto(167.32489325,39.99989439)
\curveto(168.30145582,39.99989439)(169.06846434,39.68251159)(169.62591881,39.04774597)
\curveto(170.18337328,38.41298036)(170.46210051,37.53407413)(170.46210051,36.41102728)
\curveto(170.46210051,35.29204943)(170.18337328,34.4131432)(169.62591881,33.77430858)
\curveto(169.06846434,33.13954297)(168.30145582,32.82216016)(167.32489325,32.82216016)
\curveto(166.34426166,32.82216016)(165.57521864,33.13954297)(165.01776417,33.77430858)
\curveto(164.46437871,34.4131432)(164.18768598,35.29204943)(164.18768598,36.41102728)
\curveto(164.18768598,37.53407413)(164.46437871,38.41298036)(165.01776417,39.04774597)
\curveto(165.57521864,39.68251159)(166.34426166,39.99989439)(167.32489325,39.99989439)
\closepath
}
}
{
\newrgbcolor{curcolor}{0 0 0}
\pscustom[linestyle=none,fillstyle=solid,fillcolor=curcolor]
{
\newpath
\moveto(177.22479312,36.41102728)
\curveto(177.22479312,37.23703637)(177.05389467,37.88400902)(176.71209777,38.35194521)
\curveto(176.37436988,38.82395041)(175.90846815,39.059953)(175.31439259,39.059953)
\curveto(174.72031702,39.059953)(174.25238079,38.82395041)(173.91058389,38.35194521)
\curveto(173.572856,37.88400902)(173.40399206,37.23703637)(173.40399206,36.41102728)
\curveto(173.40399206,35.58501818)(173.572856,34.93601103)(173.91058389,34.46400583)
\curveto(174.25238079,33.99606964)(174.72031702,33.76210155)(175.31439259,33.76210155)
\curveto(175.90846815,33.76210155)(176.37436988,33.99606964)(176.71209777,34.46400583)
\curveto(177.05389467,34.93601103)(177.22479312,35.58501818)(177.22479312,36.41102728)
\closepath
\moveto(173.40399206,38.79750184)
\curveto(173.63999468,39.20440287)(173.93703246,39.50550964)(174.2951054,39.70082213)
\curveto(174.65724736,39.90020364)(175.08856249,39.99989439)(175.58905081,39.99989439)
\curveto(176.419129,39.99989439)(177.09255027,39.67030456)(177.60931463,39.01112488)
\curveto(178.13014801,38.35194521)(178.39056469,37.485246)(178.39056469,36.41102728)
\curveto(178.39056469,35.33680855)(178.13014801,34.47010935)(177.60931463,33.81092967)
\curveto(177.09255027,33.15175)(176.419129,32.82216016)(175.58905081,32.82216016)
\curveto(175.08856249,32.82216016)(174.65724736,32.91981641)(174.2951054,33.1151289)
\curveto(173.93703246,33.31451041)(173.63999468,33.61765168)(173.40399206,34.02455271)
\lineto(173.40399206,32.99916211)
\lineto(172.27484158,32.99916211)
\lineto(172.27484158,42.49623223)
\lineto(173.40399206,42.49623223)
\lineto(173.40399206,38.79750184)
\closepath
}
}
{
\newrgbcolor{curcolor}{0 0 0}
\pscustom[linestyle=none,fillstyle=solid,fillcolor=curcolor]
{
\newpath
\moveto(181.36297671,41.7760174)
\lineto(181.36297671,39.83509947)
\lineto(183.67620931,39.83509947)
\lineto(183.67620931,38.96229676)
\lineto(181.36297671,38.96229676)
\lineto(181.36297671,35.25135933)
\curveto(181.36297671,34.69390491)(181.43825341,34.33583201)(181.58880681,34.1771406)
\curveto(181.74342922,34.0184492)(182.05470854,33.9391035)(182.52264477,33.9391035)
\lineto(183.67620931,33.9391035)
\lineto(183.67620931,32.99916211)
\lineto(182.52264477,32.99916211)
\curveto(181.65594549,32.99916211)(181.05780091,33.15988802)(180.72821104,33.48133983)
\curveto(180.39862117,33.80686066)(180.23382624,34.39686716)(180.23382624,35.25135933)
\lineto(180.23382624,38.96229676)
\lineto(179.40985157,38.96229676)
\lineto(179.40985157,39.83509947)
\lineto(180.23382624,39.83509947)
\lineto(180.23382624,41.7760174)
\lineto(181.36297671,41.7760174)
\closepath
}
}
{
\newrgbcolor{curcolor}{0 0 0}
\pscustom[linestyle=none,fillstyle=solid,fillcolor=curcolor]
{
\newpath
\moveto(191.0065358,36.6978925)
\lineto(191.0065358,36.14857611)
\lineto(185.8429612,36.14857611)
\curveto(185.89178933,35.37546415)(186.12372294,34.78545765)(186.53876203,34.37855661)
\curveto(186.95787013,33.97572459)(187.53973867,33.77430858)(188.28436763,33.77430858)
\curveto(188.71568276,33.77430858)(189.13275636,33.82720571)(189.53558842,33.93299998)
\curveto(189.9424895,34.03879425)(190.34532156,34.19748565)(190.74408461,34.40907419)
\lineto(190.74408461,33.34706249)
\curveto(190.34125255,33.17616406)(189.92824796,33.04595573)(189.50507084,32.9564375)
\curveto(189.08189373,32.86691927)(188.6526131,32.82216016)(188.21722895,32.82216016)
\curveto(187.12673408,32.82216016)(186.2620693,33.13954297)(185.62323462,33.77430858)
\curveto(184.98846895,34.40907419)(184.67108611,35.26763537)(184.67108611,36.34999212)
\curveto(184.67108611,37.46896996)(184.9721929,38.35601422)(185.57440649,39.01112488)
\curveto(186.18068909,39.67030456)(186.99652574,39.99989439)(188.02191644,39.99989439)
\curveto(188.94151286,39.99989439)(189.66783127,39.70285664)(190.20087168,39.10878113)
\curveto(190.73798109,38.51877463)(191.0065358,37.71514509)(191.0065358,36.6978925)
\closepath
\moveto(189.88348884,37.02748234)
\curveto(189.87535082,37.6419029)(189.70241786,38.13221865)(189.36468997,38.49842958)
\curveto(189.0310311,38.86464051)(188.58750893,39.04774597)(188.03412347,39.04774597)
\curveto(187.40749582,39.04774597)(186.90497299,38.87074402)(186.526555,38.51674012)
\curveto(186.15220601,38.16273623)(185.93654844,37.66428246)(185.87958229,37.02137883)
\lineto(189.88348884,37.02748234)
\closepath
}
}
{
\newrgbcolor{curcolor}{0 0 0}
\pscustom[linestyle=none,fillstyle=solid,fillcolor=curcolor]
{
\newpath
\moveto(198.53217082,37.12513859)
\lineto(198.53217082,32.99916211)
\lineto(197.40912387,32.99916211)
\lineto(197.40912387,37.0885175)
\curveto(197.40912387,37.73549014)(197.28298453,38.21970237)(197.03070587,38.54115419)
\curveto(196.7784272,38.862606)(196.40000921,39.02333191)(195.89545188,39.02333191)
\curveto(195.28916928,39.02333191)(194.81106052,38.83005392)(194.4611256,38.44349794)
\curveto(194.11119068,38.05694196)(193.93622322,37.53000512)(193.93622322,36.86268742)
\lineto(193.93622322,32.99916211)
\lineto(192.80707274,32.99916211)
\lineto(192.80707274,39.83509947)
\lineto(193.93622322,39.83509947)
\lineto(193.93622322,38.77308778)
\curveto(194.20477792,39.18405782)(194.52012626,39.4912681)(194.88226821,39.69471862)
\curveto(195.24847917,39.89816913)(195.66962178,39.99989439)(196.14569604,39.99989439)
\curveto(196.93101511,39.99989439)(197.52509067,39.75575377)(197.92792273,39.26747253)
\curveto(198.33075479,38.7832603)(198.53217082,38.06914899)(198.53217082,37.12513859)
\closepath
}
}
{
\newrgbcolor{curcolor}{0 0 0}
\pscustom[linestyle=none,fillstyle=solid,fillcolor=curcolor]
{
\newpath
\moveto(200.78436311,39.83509947)
\lineto(201.90741007,39.83509947)
\lineto(201.90741007,32.99916211)
\lineto(200.78436311,32.99916211)
\lineto(200.78436311,39.83509947)
\closepath
\moveto(200.78436311,42.49623223)
\lineto(201.90741007,42.49623223)
\lineto(201.90741007,41.07411312)
\lineto(200.78436311,41.07411312)
\lineto(200.78436311,42.49623223)
\closepath
}
}
{
\newrgbcolor{curcolor}{0 0 0}
\pscustom[linestyle=none,fillstyle=solid,fillcolor=curcolor]
{
\newpath
\moveto(208.7494592,38.79750184)
\lineto(208.7494592,42.49623223)
\lineto(209.87250616,42.49623223)
\lineto(209.87250616,32.99916211)
\lineto(208.7494592,32.99916211)
\lineto(208.7494592,34.02455271)
\curveto(208.51345658,33.61765168)(208.21438429,33.31451041)(207.85224234,33.1151289)
\curveto(207.49416939,32.91981641)(207.06285426,32.82216016)(206.55829693,32.82216016)
\curveto(205.73228775,32.82216016)(205.05886648,33.15175)(204.53803311,33.81092967)
\curveto(204.02126874,34.47010935)(203.76288656,35.33680855)(203.76288656,36.41102728)
\curveto(203.76288656,37.485246)(204.02126874,38.35194521)(204.53803311,39.01112488)
\curveto(205.05886648,39.67030456)(205.73228775,39.99989439)(206.55829693,39.99989439)
\curveto(207.06285426,39.99989439)(207.49416939,39.90020364)(207.85224234,39.70082213)
\curveto(208.21438429,39.50550964)(208.51345658,39.20440287)(208.7494592,38.79750184)
\closepath
\moveto(204.92255462,36.41102728)
\curveto(204.92255462,35.58501818)(205.09141856,34.93601103)(205.42914645,34.46400583)
\curveto(205.77094335,33.99606964)(206.23887959,33.76210155)(206.83295515,33.76210155)
\curveto(207.42703072,33.76210155)(207.89496695,33.99606964)(208.23676385,34.46400583)
\curveto(208.57856075,34.93601103)(208.7494592,35.58501818)(208.7494592,36.41102728)
\curveto(208.7494592,37.23703637)(208.57856075,37.88400902)(208.23676385,38.35194521)
\curveto(207.89496695,38.82395041)(207.42703072,39.059953)(206.83295515,39.059953)
\curveto(206.23887959,39.059953)(205.77094335,38.82395041)(205.42914645,38.35194521)
\curveto(205.09141856,37.88400902)(204.92255462,37.23703637)(204.92255462,36.41102728)
\closepath
}
}
{
\newrgbcolor{curcolor}{0 0 0}
\pscustom[linestyle=none,fillstyle=solid,fillcolor=curcolor]
{
\newpath
\moveto(215.29242813,36.43544134)
\curveto(214.38503874,36.43544134)(213.75637658,36.33168157)(213.40644166,36.12416205)
\curveto(213.05650674,35.91664252)(212.88153928,35.56263862)(212.88153928,35.06215035)
\curveto(212.88153928,34.66338734)(213.01174762,34.34600453)(213.2721643,34.11000193)
\curveto(213.53665,33.87806834)(213.89472294,33.76210155)(214.34638313,33.76210155)
\curveto(214.96894177,33.76210155)(215.46739559,33.98182811)(215.84174457,34.42128122)
\curveto(216.22016257,34.86480335)(216.40937157,35.45277534)(216.40937157,36.1851972)
\lineto(216.40937157,36.43544134)
\lineto(215.29242813,36.43544134)
\closepath
\moveto(217.53241853,36.89930852)
\lineto(217.53241853,32.99916211)
\lineto(216.40937157,32.99916211)
\lineto(216.40937157,34.03675975)
\curveto(216.15302389,33.62172069)(215.83360655,33.31451041)(215.45111954,33.1151289)
\curveto(215.06863254,32.91981641)(214.6006963,32.82216016)(214.04731085,32.82216016)
\curveto(213.347441,32.82216016)(212.78998653,33.01747266)(212.37494744,33.40809765)
\curveto(211.96397736,33.80279165)(211.75849232,34.32972849)(211.75849232,34.98890816)
\curveto(211.75849232,35.75795112)(212.01483999,36.33778509)(212.52753534,36.72841008)
\curveto(213.0442997,37.11903507)(213.81334273,37.31434757)(214.83466442,37.31434757)
\lineto(216.40937157,37.31434757)
\lineto(216.40937157,37.42421085)
\curveto(216.40937157,37.94097516)(216.23847312,38.33973818)(215.89667622,38.62049989)
\curveto(215.55894833,38.90533061)(215.08287407,39.04774597)(214.46845346,39.04774597)
\curveto(214.07782843,39.04774597)(213.69737592,39.00095235)(213.32709595,38.90736512)
\curveto(212.95681597,38.81377788)(212.60077754,38.67339702)(212.25898064,38.48622255)
\lineto(212.25898064,39.52382018)
\curveto(212.66995072,39.68251159)(213.06871377,39.80051289)(213.45526979,39.87782408)
\curveto(213.8418258,39.95920429)(214.2182093,39.99989439)(214.58442026,39.99989439)
\curveto(215.57318987,39.99989439)(216.31171531,39.74354674)(216.7999966,39.23085144)
\curveto(217.28827788,38.71815614)(217.53241853,37.94097516)(217.53241853,36.89930852)
\closepath
}
}
\end{pspicture}

  \caption{Control de Lazo Cerrado}
  para controlar el comportamiento dinámico de la referencia; se trata de
  realimentación negativa, pues al valor sensado se le resta el valor deseado
  para crear la señal de error, que es amplificada por el controlador.
  \label{fig:ControLazoCerrado}
\end{figure}

Es así que, la realimentación es un mecanismo o proceso cuya señal se mueve
dentro de un sistema y vuelve al principio de éste como en un bucle, que se
llama "bucle de realimentación". En un sistema de control (que tiene entradas y
salidas), parte de la señal de salida vuelve de nuevo al sistema como parte de
su entrada; a esto se le llama "realimentación" o retroalimentación.

La realimentación comprende todas aquellas soluciones de aplicación que hacen
referencia a la captura de información de un proceso o planta, no necesariamente
industrial, para que, con esta información, sea posible realizar una serie de
análisis o estudios con los que se pueden obtener valiosos indicadores que
 permitan una retroalimentación sobre un operador o sobre el propio proceso,
 tales como:
 \begin{itemize}
   \item Indicadores sin retroalimentación inherente (no afectan al proceso,
   sólo al operador):
   \begin{itemize}
     \item Estado actual del proceso. Valores instantáneos;
     \item Desviación o deriva del proceso. Evolución histórica y acumulada;
     \begin{itemize}
        \item Medición de los parámetros que tu creas necesarios
       \end{itemize}
   \end{itemize}
   \item Indicadores con retroalimentación inherente (afectan al proceso,
         después al operador):
   \begin{itemize}
     \item Generación de alarmas;
     \item HMI Human Machine Interface (Interfaces hombre-máquina);
     \item Toma de decisiones:
     \begin{itemize}
       \item Mediante operatoria humana;
       \item Automática (mediante la utilización de sistemas basados en el
             conocimiento o sistemas expertos).
     \end{itemize}
   \end{itemize}
 \end{itemize}


 \section{SCADA} acrónimo de Supervisory Control And Data Acquisition
 (Supervisión, Control y Adquisición de Datos) es un software para ordenadores
 que permite controlar y supervisar procesos industriales a distancia. Facilita
 retroalimentación en tiempo real con los dispositivos de campo (sensores y
 actuadores), y controla el proceso automáticamente. Provee de toda la
 información que se genera en el proceso productivo (supervisión, control
 calidad, control de producción, almacenamiento de datos, etc.) y permite su
 gestión e intervención.
 \begin{figure}[h]
   \centering
   %LaTeX with PSTricks extensions
%%Creator: inkscape 0.91
%%Please note this file requires PSTricks extensions
\psset{xunit=.5pt,yunit=.5pt,runit=.5pt}
\begin{pspicture}(483.8631897,366.53445435)
{
\newrgbcolor{curcolor}{0 0 1}
\pscustom[linestyle=none,fillstyle=solid,fillcolor=curcolor]
{
\newpath
\moveto(417.371256,240.03283735)
\lineto(417.371256,239.03283735)
\lineto(435.158806,239.03283735)
\lineto(435.158806,83.09533735)
\lineto(436.158806,83.09533735)
\lineto(436.158806,239.53283735)
\lineto(436.158806,240.03283735)
\lineto(435.658806,240.03283735)
\lineto(417.371256,240.03283735)
\closepath
}
}
{
\newrgbcolor{curcolor}{0 0 1}
\pscustom[linestyle=none,fillstyle=solid,fillcolor=curcolor]
{
\newpath
\moveto(368.721306,240.03283735)
\lineto(368.721306,239.53283735)
\lineto(368.721306,102.90783735)
\lineto(362.756386,102.90783735)
\lineto(362.756386,101.90783735)
\lineto(369.221306,101.90783735)
\lineto(369.721306,101.90783735)
\lineto(369.721306,102.40783735)
\lineto(369.721306,239.03283735)
\lineto(396.377556,239.03283735)
\lineto(396.377556,240.03283735)
\lineto(369.221306,240.03283735)
\lineto(368.721306,240.03283735)
\closepath
}
}
{
\newrgbcolor{curcolor}{0.22745098 0.85490197 0.87843138}
\pscustom[linestyle=none,fillstyle=solid,fillcolor=curcolor]
{
\newpath
\moveto(209.940056,338.82244153)
\lineto(250.65434662,338.82244153)
\lineto(250.65434662,303.46529813)
\lineto(209.940056,303.46529813)
\closepath
}
}
{
\newrgbcolor{curcolor}{0 0 0}
\pscustom[linewidth=2,linecolor=curcolor]
{
\newpath
\moveto(209.940056,338.82244153)
\lineto(250.65434662,338.82244153)
\lineto(250.65434662,303.46529813)
\lineto(209.940056,303.46529813)
\closepath
}
}
{
\newrgbcolor{curcolor}{0.40000001 0.40000001 0.40000001}
\pscustom[linestyle=none,fillstyle=solid,fillcolor=curcolor]
{
\newpath
\moveto(221.251286,302.74540735)
\lineto(217.845036,299.37040735)
\lineto(242.532536,299.37040735)
\lineto(239.126286,302.74540735)
\lineto(221.251286,302.74540735)
\closepath
}
}
{
\newrgbcolor{curcolor}{0 0 0}
\pscustom[linewidth=1,linecolor=curcolor]
{
\newpath
\moveto(221.251286,302.74540735)
\lineto(217.845036,299.37040735)
\lineto(242.532536,299.37040735)
\lineto(239.126286,302.74540735)
\lineto(221.251286,302.74540735)
\closepath
}
}
{
\newrgbcolor{curcolor}{0 0 0}
\pscustom[linewidth=0.98478031,linecolor=curcolor]
{
\newpath
\moveto(251.614176,318.76322735)
\lineto(279.778766,318.76322735)
}
}
{
\newrgbcolor{curcolor}{0.59607846 0.59607846 0.59607846}
\pscustom[linestyle=none,fillstyle=solid,fillcolor=curcolor]
{
\newpath
\moveto(280.76372543,337.70359082)
\lineto(306.77515427,337.70359082)
\lineto(306.77515427,294.01449247)
\lineto(280.76372543,294.01449247)
\closepath
}
}
{
\newrgbcolor{curcolor}{0 0 0}
\pscustom[linewidth=2,linecolor=curcolor]
{
\newpath
\moveto(280.76372543,337.70359082)
\lineto(306.77515427,337.70359082)
\lineto(306.77515427,294.01449247)
\lineto(280.76372543,294.01449247)
\closepath
}
}
{
\newrgbcolor{curcolor}{0.40000001 0.40000001 0.40000001}
\pscustom[linestyle=none,fillstyle=solid,fillcolor=curcolor]
{
\newpath
\moveto(282.91030136,331.01133008)
\lineto(304.12350479,331.01133008)
\lineto(304.12350479,328.61221786)
\lineto(282.91030136,328.61221786)
\closepath
}
}
{
\newrgbcolor{curcolor}{0 0 0}
\pscustom[linewidth=2,linecolor=curcolor]
{
\newpath
\moveto(282.91030136,331.01133008)
\lineto(304.12350479,331.01133008)
\lineto(304.12350479,328.61221786)
\lineto(282.91030136,328.61221786)
\closepath
}
}
{
\newrgbcolor{curcolor}{0.40000001 0.40000001 0.40000001}
\pscustom[linestyle=none,fillstyle=solid,fillcolor=curcolor]
{
\newpath
\moveto(282.91030136,325.01133008)
\lineto(304.12350479,325.01133008)
\lineto(304.12350479,322.61221786)
\lineto(282.91030136,322.61221786)
\closepath
}
}
{
\newrgbcolor{curcolor}{0 0 0}
\pscustom[linewidth=2,linecolor=curcolor]
{
\newpath
\moveto(282.91030136,325.01133008)
\lineto(304.12350479,325.01133008)
\lineto(304.12350479,322.61221786)
\lineto(282.91030136,322.61221786)
\closepath
}
}
{
\newrgbcolor{curcolor}{0 0.70980394 0}
\pscustom[linestyle=none,fillstyle=solid,fillcolor=curcolor]
{
\newpath
\moveto(299.07274643,319.89965101)
\lineto(303.87097088,319.89965101)
\lineto(303.87097088,316.1115789)
\lineto(299.07274643,316.1115789)
\closepath
}
}
{
\newrgbcolor{curcolor}{0 0 0}
\pscustom[linewidth=1,linecolor=curcolor]
{
\newpath
\moveto(299.07274643,319.89965101)
\lineto(303.87097088,319.89965101)
\lineto(303.87097088,316.1115789)
\lineto(299.07274643,316.1115789)
\closepath
}
}
{
\newrgbcolor{curcolor}{0 0 0}
\pscustom[linewidth=0.97328591,linecolor=curcolor]
{
\newpath
\moveto(307.738656,318.38441935)
\lineto(336.445736,318.38441935)
}
}
{
\newrgbcolor{curcolor}{0.60000002 0.60000002 0.60000002}
\pscustom[linestyle=none,fillstyle=solid,fillcolor=curcolor]
{
\newpath
\moveto(336.808706,327.23635835)
\lineto(336.808706,313.01760835)
\curveto(336.667416,312.53122535)(336.589956,312.02576135)(336.589956,311.51760835)
\curveto(336.589956,306.32491735)(344.239286,302.11135735)(353.683706,302.11135735)
\curveto(363.128126,302.11135735)(370.808706,306.32491735)(370.808706,311.51760835)
\lineto(370.777506,311.73635835)
\curveto(370.763906,312.05598235)(370.722306,312.36310735)(370.652506,312.67385835)
\lineto(370.652506,327.23635835)
\lineto(336.808756,327.23635835)
\closepath
}
}
{
\newrgbcolor{curcolor}{0 0 0}
\pscustom[linewidth=1,linecolor=curcolor]
{
\newpath
\moveto(336.808706,327.23635835)
\lineto(336.808706,313.01760835)
\curveto(336.667416,312.53122535)(336.589956,312.02576135)(336.589956,311.51760835)
\curveto(336.589956,306.32491735)(344.239286,302.11135735)(353.683706,302.11135735)
\curveto(363.128126,302.11135735)(370.808706,306.32491735)(370.808706,311.51760835)
\lineto(370.777506,311.73635835)
\curveto(370.763906,312.05598235)(370.722306,312.36310735)(370.652506,312.67385835)
\lineto(370.652506,327.23635835)
\lineto(336.808756,327.23635835)
\closepath
}
}
{
\newrgbcolor{curcolor}{0.60000002 0.60000002 0.60000002}
\pscustom[linestyle=none,fillstyle=solid,fillcolor=curcolor]
{
\newpath
\moveto(370.793596,325.89743135)
\curveto(370.80114346,320.72350263)(363.20702789,316.51742375)(353.80187933,316.48335289)
\curveto(344.39720639,316.44928376)(336.71629963,320.59975513)(336.59873039,325.77249615)
\curveto(336.48116114,330.94523718)(343.97102984,335.19994382)(353.37431435,335.29510028)
\curveto(362.77807441,335.39026156)(370.56098535,331.28983505)(370.788596,326.11741835)
}
}
{
\newrgbcolor{curcolor}{0 0 0}
\pscustom[linewidth=1,linecolor=curcolor]
{
\newpath
\moveto(370.793596,325.89743135)
\curveto(370.80114346,320.72350263)(363.20702789,316.51742375)(353.80187933,316.48335289)
\curveto(344.39720639,316.44928376)(336.71629963,320.59975513)(336.59873039,325.77249615)
\curveto(336.48116114,330.94523718)(343.97102984,335.19994382)(353.37431435,335.29510028)
\curveto(362.77807441,335.39026156)(370.56098535,331.28983505)(370.788596,326.11741835)
}
}
{
\newrgbcolor{curcolor}{0 0 0}
\pscustom[linewidth=1,linecolor=curcolor]
{
\newpath
\moveto(298.511486,293.82243735)
\lineto(298.511486,269.53672735)
\lineto(161.368626,269.53672735)
}
}
{
\newrgbcolor{curcolor}{0 0 0}
\pscustom[linewidth=1.00874472,linecolor=curcolor]
{
\newpath
\moveto(180.297196,269.88508735)
\lineto(180.297196,250.62401735)
}
}
{
\newrgbcolor{curcolor}{0 0 0}
\pscustom[linewidth=0.99840677,linecolor=curcolor]
{
\newpath
\moveto(298.661976,269.50989735)
\lineto(426.824036,269.50989735)
}
}
{
\newrgbcolor{curcolor}{0 0 0}
\pscustom[linewidth=0.99794096,linecolor=curcolor]
{
\newpath
\moveto(406.725766,269.34499735)
\lineto(406.725766,250.13860735)
}
}
{
\newrgbcolor{curcolor}{0 0 0}
\pscustom[linewidth=1,linecolor=curcolor]
{
\newpath
\moveto(396.55248276,249.94657941)
\lineto(417.00807205,249.94657941)
\lineto(417.00807205,229.6172585)
\lineto(396.55248276,229.6172585)
\closepath
}
}
{
\newrgbcolor{curcolor}{0 0 0}
\pscustom[linewidth=0.69109042,linecolor=curcolor]
{
\newpath
\moveto(396.75190956,239.81417453)
\lineto(406.74803905,249.81030402)
\lineto(416.68246489,239.87587819)
\lineto(406.68633539,229.87974869)
\closepath
}
}
{
\newrgbcolor{curcolor}{0 0 0}
\pscustom[linewidth=1,linecolor=curcolor]
{
\newpath
\moveto(396.426206,239.84505735)
\lineto(416.629256,239.89385735)
}
}
{
\newrgbcolor{curcolor}{0 0 1}
\pscustom[linestyle=none,fillstyle=solid,fillcolor=curcolor]
{
\newpath
\moveto(190.929436,240.75158735)
\lineto(190.929436,239.75158735)
\lineto(248.002556,239.75158735)
\lineto(248.002556,157.67920735)
\lineto(249.002556,157.84422735)
\lineto(249.002556,240.25158735)
\lineto(249.002556,240.75158735)
\lineto(248.502556,240.75158735)
\lineto(190.929436,240.75158735)
\closepath
}
}
{
\newrgbcolor{curcolor}{0 0 0}
\pscustom[linestyle=none,fillstyle=solid,fillcolor=curcolor]
{
\newpath
\moveto(101.686926,240.75158735)
\lineto(101.686926,240.25291735)
\lineto(101.686926,147.65536735)
\lineto(102.684276,147.65536735)
\lineto(102.684276,239.75423735)
\lineto(169.849436,239.75423735)
\lineto(169.849436,240.75158735)
\lineto(102.185596,240.75158735)
\lineto(101.686926,240.75158735)
\closepath
}
}
{
\newrgbcolor{curcolor}{0 0 0}
\pscustom[linewidth=0.98895234,linecolor=curcolor]
{
\newpath
\moveto(101.858166,148.10815735)
\lineto(119.964366,148.10815735)
}
}
{
\newrgbcolor{curcolor}{0 0 0}
\pscustom[linestyle=none,fillstyle=solid,fillcolor=curcolor]
{
\newpath
\moveto(116.00855665,148.10815735)
\lineto(114.03065197,146.13025267)
\lineto(120.95331834,148.10815735)
\lineto(114.03065197,150.08606202)
\lineto(116.00855665,148.10815735)
\closepath
}
}
{
\newrgbcolor{curcolor}{0 0 0}
\pscustom[linewidth=0.49447617,linecolor=curcolor]
{
\newpath
\moveto(116.00855665,148.10815735)
\lineto(114.03065197,146.13025267)
\lineto(120.95331834,148.10815735)
\lineto(114.03065197,150.08606202)
\lineto(116.00855665,148.10815735)
\closepath
}
}
{
\newrgbcolor{curcolor}{0 0 0}
\pscustom[linewidth=1,linecolor=curcolor]
{
\newpath
\moveto(102.165426,148.10815735)
\lineto(317.082916,148.10815735)
\lineto(317.082916,142.39386735)
}
}
{
\newrgbcolor{curcolor}{0 0 0}
\pscustom[linewidth=0.96579844,linecolor=curcolor]
{
\newpath
\moveto(337.589176,123.12525735)
\lineto(353.528586,123.12525735)
\lineto(353.528586,112.37676735)
}
}
{
\newrgbcolor{curcolor}{0 0 0}
\pscustom[linewidth=1,linecolor=curcolor]
{
\newpath
\moveto(338.257816,83.27317735)
\lineto(353.257816,83.27317735)
\lineto(353.257816,91.84460735)
}
}
{
\newrgbcolor{curcolor}{0 0 0}
\pscustom[linewidth=1.01016271,linecolor=curcolor]
{
\newpath
\moveto(336.378836,73.10815735)
\lineto(426.030776,73.10815735)
}
}
{
\newrgbcolor{curcolor}{0 0 0}
\pscustom[linewidth=0.5929069,linecolor=curcolor]
{
\newpath
\moveto(426.38863655,79.04520197)
\lineto(430.90363445,76.5212388)
\lineto(435.34694991,73.87311499)
\lineto(430.90363445,71.22499119)
\lineto(426.38863655,68.70102802)
\lineto(426.31694817,73.87311499)
\lineto(426.38863655,79.04520197)
\closepath
}
}
{
\newrgbcolor{curcolor}{0 0 0}
\pscustom[linewidth=0.5929069,linecolor=curcolor]
{
\newpath
\moveto(444.94522545,79.04520197)
\lineto(440.43022755,76.5212388)
\lineto(435.98691209,73.87311499)
\lineto(440.43022755,71.22499119)
\lineto(444.94522545,68.70102802)
\lineto(445.01691383,73.87311499)
\lineto(444.94522545,79.04520197)
\closepath
}
}
{
\newrgbcolor{curcolor}{0 0 0}
\pscustom[linewidth=1,linecolor=curcolor]
{
\newpath
\moveto(435.619896,73.78178735)
\lineto(435.619896,81.10321735)
}
}
{
\newrgbcolor{curcolor}{0 0 0}
\pscustom[linewidth=0.36949679,linecolor=curcolor]
{
\newpath
\moveto(431.526176,81.01007735)
\curveto(435.679196,80.88163735)(435.655776,80.99387735)(439.696386,80.92457735)
\curveto(439.696386,82.18112735)(437.752196,83.13780735)(435.512256,83.13780735)
\curveto(433.365466,83.13780735)(431.666316,82.21176735)(431.526176,81.01003735)
\closepath
}
}
{
\newrgbcolor{curcolor}{0 0 0}
\pscustom[linewidth=0.9671053,linecolor=curcolor]
{
\newpath
\moveto(362.90201509,102.30549503)
\curveto(362.90201541,96.77355858)(358.41749785,92.28904898)(352.88556527,92.28904898)
\curveto(347.35363269,92.28904898)(342.86911513,96.77355858)(342.86911545,102.30549503)
\curveto(342.86911577,107.83743103)(347.35363314,112.32193993)(352.88556527,112.32193993)
\curveto(358.4174974,112.32193993)(362.90201477,107.83743103)(362.90201509,102.30549503)
\closepath
}
}
{
\newrgbcolor{curcolor}{0 0 0}
\pscustom[linewidth=1,linecolor=curcolor]
{
\newpath
\moveto(342.971456,102.39326735)
\lineto(362.808286,102.39326735)
}
}
{
\newrgbcolor{curcolor}{0 0 0}
\pscustom[linewidth=1,linecolor=curcolor]
{
\newpath
\moveto(312.90227333,141.95605512)
\lineto(322.14342705,141.95605512)
\curveto(330.9198829,141.95605512)(337.98540512,134.8905329)(337.98540512,126.11407705)
\lineto(337.98540512,79.24822088)
\curveto(337.98540512,70.47176502)(330.9198829,63.4062428)(322.14342705,63.4062428)
\lineto(312.90227333,63.4062428)
\curveto(304.12581748,63.4062428)(297.06029526,70.47176502)(297.06029526,79.24822088)
\lineto(297.06029526,126.11407705)
\curveto(297.06029526,134.8905329)(304.12581748,141.95605512)(312.90227333,141.95605512)
\closepath
}
}
{
\newrgbcolor{curcolor}{0 0 0}
\pscustom[linewidth=0.9671053,linecolor=curcolor]
{
\newpath
\moveto(258.81950509,147.79068003)
\curveto(258.81950541,142.25874358)(254.33498785,137.77423398)(248.80305527,137.77423398)
\curveto(243.27112269,137.77423398)(238.78660513,142.25874358)(238.78660545,147.79068003)
\curveto(238.78660577,153.32261603)(243.27112314,157.80712493)(248.80305527,157.80712493)
\curveto(254.3349874,157.80712493)(258.81950477,153.32261603)(258.81950509,147.79068003)
\closepath
}
}
{
\newrgbcolor{curcolor}{0 0 0}
\pscustom[linewidth=0.9534992,linecolor=curcolor]
{
\newpath
\moveto(121.93390678,127.51673642)
\curveto(121.93871078,116.56388866)(113.06102522,107.68232475)(102.10818409,107.68232475)
\curveto(91.15534295,107.68232475)(82.27765739,116.56388866)(82.28246139,127.51673642)
\curveto(82.27765739,138.46958419)(91.15534295,147.3511481)(102.10818409,147.3511481)
\curveto(113.06102522,147.3511481)(121.93871078,138.46958419)(121.93390678,127.51673642)
\closepath
}
}
{
\newrgbcolor{curcolor}{0 0 0}
\pscustom[linestyle=none,fillstyle=solid,fillcolor=curcolor]
{
\newpath
\moveto(72.2188369,128.08656675)
\lineto(72.2188369,127.08656675)
\lineto(99.9375869,127.08656675)
\lineto(99.9375869,128.08656675)
\lineto(72.2188369,128.08656675)
\closepath
}
}
{
\newrgbcolor{curcolor}{0 0 0}
\pscustom[linestyle=none,fillstyle=solid,fillcolor=curcolor]
{
\newpath
\moveto(98.7431669,127.59925675)
\lineto(97.7431669,126.59925675)
\lineto(101.2431669,127.59925675)
\lineto(97.7431669,128.59925675)
\lineto(98.7431669,127.59925675)
\closepath
}
}
{
\newrgbcolor{curcolor}{0 0 0}
\pscustom[linewidth=0.25,linecolor=curcolor]
{
\newpath
\moveto(98.7431669,127.59925675)
\lineto(97.7431669,126.59925675)
\lineto(101.2431669,127.59925675)
\lineto(97.7431669,128.59925675)
\lineto(98.7431669,127.59925675)
\closepath
}
}
{
\newrgbcolor{curcolor}{0 0 0}
\pscustom[linewidth=1,linecolor=curcolor]
{
\newpath
\moveto(91.527116,110.51961735)
\lineto(81.347566,104.64245735)
\lineto(122.716016,104.64245735)
\lineto(112.343706,110.50668735)
}
}
{
\newrgbcolor{curcolor}{0 0 0}
\pscustom[linewidth=1,linecolor=curcolor]
{
\newpath
\moveto(170.06264276,250.29664101)
\lineto(190.51823205,250.29664101)
\lineto(190.51823205,229.9673201)
\lineto(170.06264276,229.9673201)
\closepath
}
}
{
\newrgbcolor{curcolor}{0 0 0}
\pscustom[linewidth=0.69109042,linecolor=curcolor]
{
\newpath
\moveto(170.26206956,240.16423613)
\lineto(180.25819905,250.16036562)
\lineto(190.19262489,240.22593979)
\lineto(180.19649539,230.22981029)
\closepath
}
}
{
\newrgbcolor{curcolor}{0 0 0}
\pscustom[linewidth=1,linecolor=curcolor]
{
\newpath
\moveto(169.936366,240.19511895)
\lineto(190.139416,240.24391895)
}
}
{
\newrgbcolor{curcolor}{0 0 0}
\pscustom[linewidth=1.13841295,linecolor=curcolor,linestyle=dashed,dash=6.83047766 2.27682589]
{
\newpath
\moveto(445.422806,243.19437035)
\lineto(445.422806,90.15493035)
}
}
{
\newrgbcolor{curcolor}{0 0 0}
\pscustom[linestyle=none,fillstyle=solid,fillcolor=curcolor]
{
\newpath
\moveto(450.47406804,102.3250732)
\lineto(445.44285867,88.64309694)
\lineto(440.41165041,102.32507395)
\curveto(443.38207375,100.13926617)(447.44612841,100.15186084)(450.47406804,102.3250732)
\closepath
}
}
{
\newrgbcolor{curcolor}{1 1 1}
\pscustom[linestyle=none,fillstyle=solid,fillcolor=curcolor]
{
\newpath
\moveto(442.08864609,224.59132904)
\lineto(451.42362133,224.59132904)
\lineto(451.42362133,114.97720855)
\lineto(442.08864609,114.97720855)
\closepath
}
}
{
\newrgbcolor{curcolor}{0 0 0}
\pscustom[linestyle=none,fillstyle=solid,fillcolor=curcolor]
{
\newpath
\moveto(450.50679794,120.07335446)
\lineto(441.75875037,116.73351044)
\lineto(441.75875037,117.96983866)
\lineto(449.12398533,120.74132326)
\lineto(441.75875037,123.51866723)
\lineto(441.75875037,124.74913608)
\lineto(450.50679794,121.41515144)
\lineto(450.50679794,120.07335446)
\closepath
}
}
{
\newrgbcolor{curcolor}{0 0 0}
\pscustom[linestyle=none,fillstyle=solid,fillcolor=curcolor]
{
\newpath
\moveto(447.20796956,128.01866749)
\curveto(447.20796956,127.14757367)(447.30757894,126.544058)(447.5067977,126.20812047)
\curveto(447.70601647,125.87218294)(448.04586025,125.70421418)(448.52632904,125.70421418)
\curveto(448.90914157,125.70421418)(449.21382909,125.82921419)(449.44039161,126.07921421)
\curveto(449.66304788,126.33312048)(449.77437601,126.67687051)(449.77437601,127.11046429)
\curveto(449.77437601,127.70812059)(449.56343849,128.18663625)(449.14156346,128.54601128)
\curveto(448.71578218,128.90929256)(448.15132901,129.0909332)(447.44820395,129.0909332)
\lineto(447.20796956,129.0909332)
\lineto(447.20796956,128.01866749)
\closepath
\moveto(446.76265702,130.16905828)
\lineto(450.50679794,130.16905828)
\lineto(450.50679794,129.0909332)
\lineto(449.51070411,129.0909332)
\curveto(449.90914165,128.84483943)(450.20406354,128.53819878)(450.39546981,128.17101125)
\curveto(450.58296982,127.80382372)(450.67671983,127.35460493)(450.67671983,126.82335489)
\curveto(450.67671983,126.15147984)(450.48921982,125.61632355)(450.11421979,125.21788601)
\curveto(449.73531351,124.82335473)(449.22945409,124.62608909)(448.59664154,124.62608909)
\curveto(447.85836023,124.62608909)(447.30171956,124.87218286)(446.92671953,125.3643704)
\curveto(446.5517195,125.86046419)(446.36421949,126.5987455)(446.36421949,127.57921433)
\lineto(446.36421949,129.0909332)
\lineto(446.25875073,129.0909332)
\curveto(445.76265694,129.0909332)(445.37984441,128.92687068)(445.11031314,128.59874566)
\curveto(444.83687562,128.27452688)(444.70015686,127.8174956)(444.70015686,127.2276518)
\curveto(444.70015686,126.85265177)(444.74507874,126.48741737)(444.83492249,126.13194859)
\curveto(444.92476625,125.77647981)(445.05953189,125.43468291)(445.2392194,125.10655788)
\lineto(444.24312557,125.10655788)
\curveto(444.09078181,125.50108916)(443.97750055,125.88390169)(443.90328179,126.25499547)
\curveto(443.82515679,126.62608925)(443.78609428,126.98741741)(443.78609428,127.33897993)
\curveto(443.78609428,128.28819876)(444.03218805,128.99718319)(444.52437559,129.46593323)
\curveto(445.01656313,129.93468326)(445.76265694,130.16905828)(446.76265702,130.16905828)
\closepath
}
}
{
\newrgbcolor{curcolor}{0 0 0}
\pscustom[linestyle=none,fillstyle=solid,fillcolor=curcolor]
{
\newpath
\moveto(441.38960972,132.39562085)
\lineto(441.38960972,133.47374594)
\lineto(450.50679794,133.47374594)
\lineto(450.50679794,132.39562085)
\lineto(441.38960972,132.39562085)
\closepath
}
}
{
\newrgbcolor{curcolor}{0 0 0}
\pscustom[linestyle=none,fillstyle=solid,fillcolor=curcolor]
{
\newpath
\moveto(443.94429742,134.95030924)
\lineto(443.94429742,136.09288746)
\lineto(449.45211036,138.14366887)
\lineto(443.94429742,140.19445029)
\lineto(443.94429742,141.3370285)
\lineto(450.50679794,138.87609081)
\lineto(450.50679794,137.41124694)
\lineto(443.94429742,134.95030924)
\closepath
}
}
{
\newrgbcolor{curcolor}{0 0 0}
\pscustom[linestyle=none,fillstyle=solid,fillcolor=curcolor]
{
\newpath
\moveto(447.91695399,142.71398117)
\lineto(443.94429742,142.71398117)
\lineto(443.94429742,143.79210626)
\lineto(447.87593836,143.79210626)
\curveto(448.49703216,143.79210626)(448.96382907,143.91320002)(449.2763291,144.15538754)
\curveto(449.58492287,144.39757506)(449.73921976,144.76085633)(449.73921976,145.24523137)
\curveto(449.73921976,145.82726267)(449.55367287,146.28624708)(449.18257909,146.62218461)
\curveto(448.81148531,146.96202838)(448.30562589,147.13195027)(447.66500084,147.13195027)
\lineto(443.94429742,147.13195027)
\lineto(443.94429742,148.21007536)
\lineto(450.50679794,148.21007536)
\lineto(450.50679794,147.13195027)
\lineto(449.49898536,147.13195027)
\curveto(449.89742289,146.8702315)(450.19429792,146.56554398)(450.38961043,146.2178877)
\curveto(450.5810167,145.87413767)(450.67671983,145.47374702)(450.67671983,145.01671573)
\curveto(450.67671983,144.26280942)(450.44234481,143.69054375)(449.97359478,143.29991872)
\curveto(449.50484474,142.90929369)(448.81929781,142.71398117)(447.91695399,142.71398117)
\closepath
}
}
{
\newrgbcolor{curcolor}{0 0 0}
\pscustom[linestyle=none,fillstyle=solid,fillcolor=curcolor]
{
\newpath
\moveto(441.38960972,150.44249871)
\lineto(441.38960972,151.5206238)
\lineto(450.50679794,151.5206238)
\lineto(450.50679794,150.44249871)
\lineto(441.38960972,150.44249871)
\closepath
}
}
{
\newrgbcolor{curcolor}{0 0 0}
\pscustom[linestyle=none,fillstyle=solid,fillcolor=curcolor]
{
\newpath
\moveto(447.20796956,156.75304583)
\curveto(447.20796956,155.88195201)(447.30757894,155.27843633)(447.5067977,154.94249881)
\curveto(447.70601647,154.60656128)(448.04586025,154.43859252)(448.52632904,154.43859252)
\curveto(448.90914157,154.43859252)(449.21382909,154.56359253)(449.44039161,154.81359255)
\curveto(449.66304788,155.06749882)(449.77437601,155.41124884)(449.77437601,155.84484263)
\curveto(449.77437601,156.44249893)(449.56343849,156.92101459)(449.14156346,157.28038962)
\curveto(448.71578218,157.6436709)(448.15132901,157.82531154)(447.44820395,157.82531154)
\lineto(447.20796956,157.82531154)
\lineto(447.20796956,156.75304583)
\closepath
\moveto(446.76265702,158.90343662)
\lineto(450.50679794,158.90343662)
\lineto(450.50679794,157.82531154)
\lineto(449.51070411,157.82531154)
\curveto(449.90914165,157.57921777)(450.20406354,157.27257712)(450.39546981,156.90538959)
\curveto(450.58296982,156.53820206)(450.67671983,156.08898327)(450.67671983,155.55773323)
\curveto(450.67671983,154.88585818)(450.48921982,154.35070188)(450.11421979,153.95226435)
\curveto(449.73531351,153.55773307)(449.22945409,153.36046743)(448.59664154,153.36046743)
\curveto(447.85836023,153.36046743)(447.30171956,153.6065612)(446.92671953,154.09874874)
\curveto(446.5517195,154.59484253)(446.36421949,155.33312384)(446.36421949,156.31359267)
\lineto(446.36421949,157.82531154)
\lineto(446.25875073,157.82531154)
\curveto(445.76265694,157.82531154)(445.37984441,157.66124902)(445.11031314,157.333124)
\curveto(444.83687562,157.00890522)(444.70015686,156.55187393)(444.70015686,155.96203014)
\curveto(444.70015686,155.58703011)(444.74507874,155.2217957)(444.83492249,154.86632693)
\curveto(444.92476625,154.51085815)(445.05953189,154.16906124)(445.2392194,153.84093622)
\lineto(444.24312557,153.84093622)
\curveto(444.09078181,154.2354675)(443.97750055,154.61828003)(443.90328179,154.98937381)
\curveto(443.82515679,155.36046759)(443.78609428,155.72179574)(443.78609428,156.07335827)
\curveto(443.78609428,157.0225771)(444.03218805,157.73156153)(444.52437559,158.20031157)
\curveto(445.01656313,158.6690616)(445.76265694,158.90343662)(446.76265702,158.90343662)
\closepath
}
}
{
\newrgbcolor{curcolor}{0 0 0}
\pscustom[linestyle=none,fillstyle=solid,fillcolor=curcolor]
{
\newpath
\moveto(444.94039125,169.26867093)
\lineto(441.38960972,169.26867093)
\lineto(441.38960972,170.34679602)
\lineto(450.50679794,170.34679602)
\lineto(450.50679794,169.26867093)
\lineto(449.52242287,169.26867093)
\curveto(449.9130479,169.04210841)(450.20406354,168.75499902)(450.39546981,168.40734274)
\curveto(450.58296982,168.06359271)(450.67671983,167.64953018)(450.67671983,167.16515514)
\curveto(450.67671983,166.37218633)(450.36031356,165.7257019)(449.72750101,165.22570186)
\curveto(449.09468846,164.72960807)(448.26265714,164.48156118)(447.23140706,164.48156118)
\curveto(446.20015698,164.48156118)(445.36812566,164.72960807)(444.73531311,165.22570186)
\curveto(444.10250056,165.7257019)(443.78609428,166.37218633)(443.78609428,167.16515514)
\curveto(443.78609428,167.64953018)(443.88179742,168.06359271)(444.07320368,168.40734274)
\curveto(444.2607037,168.75499902)(444.54976622,169.04210841)(444.94039125,169.26867093)
\closepath
\moveto(447.23140706,165.59484251)
\curveto(448.02437587,165.59484251)(448.6474228,165.7569519)(449.10054783,166.08117068)
\curveto(449.54976662,166.4092957)(449.77437601,166.85851449)(449.77437601,167.42882704)
\curveto(449.77437601,167.99913958)(449.54976662,168.44835837)(449.10054783,168.77648339)
\curveto(448.6474228,169.10460842)(448.02437587,169.26867093)(447.23140706,169.26867093)
\curveto(446.43843825,169.26867093)(445.81734445,169.10460842)(445.36812566,168.77648339)
\curveto(444.91500062,168.44835837)(444.68843811,167.99913958)(444.68843811,167.42882704)
\curveto(444.68843811,166.85851449)(444.91500062,166.4092957)(445.36812566,166.08117068)
\curveto(445.81734445,165.7569519)(446.43843825,165.59484251)(447.23140706,165.59484251)
\closepath
}
}
{
\newrgbcolor{curcolor}{0 0 0}
\pscustom[linestyle=none,fillstyle=solid,fillcolor=curcolor]
{
\newpath
\moveto(446.95601641,178.18078041)
\lineto(447.4833602,178.18078041)
\lineto(447.4833602,173.22374877)
\curveto(448.22554776,173.27062377)(448.79195406,173.49328004)(449.18257909,173.89171757)
\curveto(449.56929787,174.29406135)(449.76265726,174.85265514)(449.76265726,175.56749895)
\curveto(449.76265726,175.98156148)(449.71187601,176.38195214)(449.6103135,176.76867092)
\curveto(449.50875099,177.15929595)(449.35640723,177.54601473)(449.15328221,177.92882726)
\lineto(450.17281354,177.92882726)
\curveto(450.33687605,177.54210848)(450.46187606,177.14562408)(450.54781357,176.73937404)
\curveto(450.63375108,176.33312401)(450.67671983,175.9210146)(450.67671983,175.50304582)
\curveto(450.67671983,174.45617074)(450.37203231,173.62609255)(449.76265726,173.01281125)
\curveto(449.15328221,172.4034362)(448.3290634,172.09874868)(447.29000081,172.09874868)
\curveto(446.21578198,172.09874868)(445.36421941,172.3878112)(444.73531311,172.96593624)
\curveto(444.10250056,173.54796754)(443.78609428,174.33117073)(443.78609428,175.31554581)
\curveto(443.78609428,176.19835838)(444.07125056,176.89562406)(444.6415631,177.40734285)
\curveto(445.2079694,177.92296789)(445.97945383,178.18078041)(446.95601641,178.18078041)
\closepath
\moveto(446.63961014,177.10265532)
\curveto(446.04976634,177.09484282)(445.57906318,176.92882718)(445.22750065,176.60460841)
\curveto(444.87593812,176.28429588)(444.70015686,175.8585146)(444.70015686,175.32726456)
\curveto(444.70015686,174.72570201)(444.87007875,174.2432801)(445.20992252,173.87999882)
\curveto(445.5497663,173.52062379)(446.02828196,173.31359252)(446.64546951,173.25890502)
\lineto(446.63961014,177.10265532)
\closepath
}
}
{
\newrgbcolor{curcolor}{0 0 0}
\pscustom[linestyle=none,fillstyle=solid,fillcolor=curcolor]
{
\newpath
\moveto(444.19625057,188.49328415)
\lineto(445.20406315,188.49328415)
\curveto(445.03609438,188.18859663)(444.91109437,187.88195598)(444.82906312,187.57336221)
\curveto(444.74312561,187.26867468)(444.70015686,186.96008091)(444.70015686,186.64758088)
\curveto(444.70015686,185.94836208)(444.92281312,185.40539328)(445.36812566,185.0186745)
\curveto(445.80953195,184.63195572)(446.43062574,184.43859633)(447.23140706,184.43859633)
\curveto(448.03218837,184.43859633)(448.6552353,184.63195572)(449.10054783,185.0186745)
\curveto(449.54195412,185.40539328)(449.76265726,185.94836208)(449.76265726,186.64758088)
\curveto(449.76265726,186.96008091)(449.72164163,187.26867468)(449.63961037,187.57336221)
\curveto(449.55367287,187.88195598)(449.42671973,188.18859663)(449.25875097,188.49328415)
\lineto(450.2548448,188.49328415)
\curveto(450.39546981,188.19250288)(450.50093857,187.88000285)(450.57125107,187.55578408)
\curveto(450.64156358,187.23547155)(450.67671983,186.89367465)(450.67671983,186.53039337)
\curveto(450.67671983,185.54211204)(450.36617293,184.75695573)(449.74507913,184.17492443)
\curveto(449.12398533,183.59289314)(448.28609464,183.30187749)(447.23140706,183.30187749)
\curveto(446.16109447,183.30187749)(445.31929753,183.59484626)(444.70601623,184.18078381)
\curveto(444.09273493,184.77062761)(443.78609428,185.5772683)(443.78609428,186.60070588)
\curveto(443.78609428,186.93273715)(443.82125054,187.25695593)(443.89156304,187.57336221)
\curveto(443.9579693,187.88976848)(444.05953181,188.19640913)(444.19625057,188.49328415)
\closepath
}
}
{
\newrgbcolor{curcolor}{0 0 0}
\pscustom[linestyle=none,fillstyle=solid,fillcolor=curcolor]
{
\newpath
\moveto(444.70015686,192.92297148)
\curveto(444.70015686,192.34484643)(444.92671937,191.88781515)(445.37984441,191.55187762)
\curveto(445.8290632,191.21594009)(446.44625075,191.04797133)(447.23140706,191.04797133)
\curveto(448.01656337,191.04797133)(448.63570404,191.21398697)(449.08882908,191.54601825)
\curveto(449.53804787,191.88195577)(449.76265726,192.34094018)(449.76265726,192.92297148)
\curveto(449.76265726,193.49719028)(449.53609474,193.95226844)(449.08296971,194.28820596)
\curveto(448.62984467,194.62414349)(448.01265712,194.79211225)(447.23140706,194.79211225)
\curveto(446.45406325,194.79211225)(445.83882882,194.62414349)(445.38570379,194.28820596)
\curveto(444.9286725,193.95226844)(444.70015686,193.49719028)(444.70015686,192.92297148)
\closepath
\moveto(443.78609428,192.92297148)
\curveto(443.78609428,193.86047155)(444.09078181,194.59679974)(444.70015686,195.13195603)
\curveto(445.30953191,195.66711232)(446.15328197,195.93469047)(447.23140706,195.93469047)
\curveto(448.30562589,195.93469047)(449.14937596,195.66711232)(449.76265726,195.13195603)
\curveto(450.37203231,194.59679974)(450.67671983,193.86047155)(450.67671983,192.92297148)
\curveto(450.67671983,191.98156516)(450.37203231,191.24328385)(449.76265726,190.70812755)
\curveto(449.14937596,190.17687751)(448.30562589,189.91125249)(447.23140706,189.91125249)
\curveto(446.15328197,189.91125249)(445.30953191,190.17687751)(444.70015686,190.70812755)
\curveto(444.09078181,191.24328385)(443.78609428,191.98156516)(443.78609428,192.92297148)
\closepath
}
}
{
\newrgbcolor{curcolor}{0 0 0}
\pscustom[linestyle=none,fillstyle=solid,fillcolor=curcolor]
{
\newpath
\moveto(446.54586013,203.17101859)
\lineto(450.50679794,203.17101859)
\lineto(450.50679794,202.0928935)
\lineto(446.58101638,202.0928935)
\curveto(445.95992258,202.0928935)(445.4950788,201.97179974)(445.18648502,201.72961222)
\curveto(444.87789125,201.4874247)(444.72359436,201.12414342)(444.72359436,200.63976839)
\curveto(444.72359436,200.05773709)(444.90914125,199.59875268)(445.28023503,199.26281515)
\curveto(445.65132881,198.92687762)(446.15718822,198.75890886)(446.79781327,198.75890886)
\lineto(450.50679794,198.75890886)
\lineto(450.50679794,197.6749244)
\lineto(443.94429742,197.6749244)
\lineto(443.94429742,198.75890886)
\lineto(444.96382875,198.75890886)
\curveto(444.56929747,199.01672138)(444.27437557,199.31945578)(444.07906306,199.66711206)
\curveto(443.88375054,200.01867459)(443.78609428,200.42297149)(443.78609428,200.88000278)
\curveto(443.78609428,201.63390909)(444.0204693,202.20422163)(444.48921934,202.59094042)
\curveto(444.95406313,202.9776592)(445.63961006,203.17101859)(446.54586013,203.17101859)
\closepath
}
}
{
\newrgbcolor{curcolor}{0 0 0}
\pscustom[linestyle=none,fillstyle=solid,fillcolor=curcolor]
{
\newpath
\moveto(442.08101602,206.39953386)
\lineto(443.94429742,206.39953386)
\lineto(443.94429742,208.62023716)
\lineto(444.78218811,208.62023716)
\lineto(444.78218811,206.39953386)
\lineto(448.3446884,206.39953386)
\curveto(448.87984469,206.39953386)(449.22359472,206.47179949)(449.37593848,206.61633075)
\curveto(449.52828224,206.76476827)(449.60445412,207.06359642)(449.60445412,207.5128152)
\lineto(449.60445412,208.62023716)
\lineto(450.50679794,208.62023716)
\lineto(450.50679794,207.5128152)
\curveto(450.50679794,206.68078388)(450.35250106,206.10656509)(450.04390728,205.79015881)
\curveto(449.73140726,205.47375254)(449.16500096,205.3155494)(448.3446884,205.3155494)
\lineto(444.78218811,205.3155494)
\lineto(444.78218811,204.52453371)
\lineto(443.94429742,204.52453371)
\lineto(443.94429742,205.3155494)
\lineto(442.08101602,205.3155494)
\lineto(442.08101602,206.39953386)
\closepath
}
}
{
\newrgbcolor{curcolor}{0 0 0}
\pscustom[linestyle=none,fillstyle=solid,fillcolor=curcolor]
{
\newpath
\moveto(444.95211,213.84679971)
\curveto(444.8817975,213.72570595)(444.83101624,213.59289343)(444.79976624,213.44836217)
\curveto(444.76460999,213.30773716)(444.74703186,213.15148715)(444.74703186,212.97961214)
\curveto(444.74703186,212.37023709)(444.94625063,211.90148705)(445.34468816,211.57336202)
\curveto(445.73921944,211.24914325)(446.30757886,211.08703386)(447.04976642,211.08703386)
\lineto(450.50679794,211.08703386)
\lineto(450.50679794,210.0030494)
\lineto(443.94429742,210.0030494)
\lineto(443.94429742,211.08703386)
\lineto(444.96382875,211.08703386)
\curveto(444.56539122,211.31359638)(444.27046932,211.60851828)(444.07906306,211.97179956)
\curveto(443.88375054,212.33508083)(443.78609428,212.77648712)(443.78609428,213.29601841)
\curveto(443.78609428,213.37023717)(443.79195366,213.45226842)(443.80367241,213.54211218)
\curveto(443.81148491,213.63195594)(443.82515679,213.73156532)(443.84468804,213.84094033)
\lineto(444.95211,213.84679971)
\closepath
}
}
{
\newrgbcolor{curcolor}{0 0 0}
\pscustom[linestyle=none,fillstyle=solid,fillcolor=curcolor]
{
\newpath
\moveto(444.70015686,217.28625273)
\curveto(444.70015686,216.70812768)(444.92671937,216.2510964)(445.37984441,215.91515887)
\curveto(445.8290632,215.57922134)(446.44625075,215.41125258)(447.23140706,215.41125258)
\curveto(448.01656337,215.41125258)(448.63570404,215.57726822)(449.08882908,215.9092995)
\curveto(449.53804787,216.24523702)(449.76265726,216.70422143)(449.76265726,217.28625273)
\curveto(449.76265726,217.86047153)(449.53609474,218.31554969)(449.08296971,218.65148721)
\curveto(448.62984467,218.98742474)(448.01265712,219.1553935)(447.23140706,219.1553935)
\curveto(446.45406325,219.1553935)(445.83882882,218.98742474)(445.38570379,218.65148721)
\curveto(444.9286725,218.31554969)(444.70015686,217.86047153)(444.70015686,217.28625273)
\closepath
\moveto(443.78609428,217.28625273)
\curveto(443.78609428,218.2237528)(444.09078181,218.96008099)(444.70015686,219.49523728)
\curveto(445.30953191,220.03039357)(446.15328197,220.29797172)(447.23140706,220.29797172)
\curveto(448.30562589,220.29797172)(449.14937596,220.03039357)(449.76265726,219.49523728)
\curveto(450.37203231,218.96008099)(450.67671983,218.2237528)(450.67671983,217.28625273)
\curveto(450.67671983,216.34484641)(450.37203231,215.6065651)(449.76265726,215.0714088)
\curveto(449.14937596,214.54015876)(448.30562589,214.27453374)(447.23140706,214.27453374)
\curveto(446.15328197,214.27453374)(445.30953191,214.54015876)(444.70015686,215.0714088)
\curveto(444.09078181,215.6065651)(443.78609428,216.34484641)(443.78609428,217.28625273)
\closepath
}
}
{
\newrgbcolor{curcolor}{0 0 0}
\pscustom[linestyle=none,fillstyle=solid,fillcolor=curcolor]
{
\newpath
\moveto(441.38960972,222.09094003)
\lineto(441.38960972,223.16906511)
\lineto(450.50679794,223.16906511)
\lineto(450.50679794,222.09094003)
\lineto(441.38960972,222.09094003)
\closepath
}
}
{
\newrgbcolor{curcolor}{0 0 0}
\pscustom[linewidth=0.8943699,linecolor=curcolor,linestyle=dashed,dash=5.3662194 1.7887398]
{
\newpath
\moveto(361.408256,132.96377735)
\lineto(361.408256,226.13173735)
}
}
{
\newrgbcolor{curcolor}{0 0 0}
\pscustom[linestyle=none,fillstyle=solid,fillcolor=curcolor]
{
\newpath
\moveto(357.43983952,216.5705237)
\lineto(361.39250204,227.31947707)
\lineto(365.34516371,216.57052311)
\curveto(363.01151385,218.28775647)(359.81867582,218.27786173)(357.43983952,216.5705237)
\closepath
}
}
{
\newrgbcolor{curcolor}{1 1 1}
\pscustom[linestyle=none,fillstyle=solid,fillcolor=curcolor]
{
\newpath
\moveto(356.74053207,196.38896222)
\lineto(366.07550732,196.38896222)
\lineto(366.07550732,160.44931073)
\lineto(356.74053207,160.44931073)
\closepath
}
}
{
\newrgbcolor{curcolor}{0 0 0}
\pscustom[linestyle=none,fillstyle=solid,fillcolor=curcolor]
{
\newpath
\moveto(357.74397987,163.6233068)
\lineto(357.74397987,165.21705693)
\lineto(365.06233982,169.09596348)
\lineto(357.74397987,169.09596348)
\lineto(357.74397987,170.24440108)
\lineto(366.49202744,170.24440108)
\lineto(366.49202744,168.65065095)
\lineto(359.17366748,164.77174439)
\lineto(366.49202744,164.77174439)
\lineto(366.49202744,163.6233068)
\lineto(357.74397987,163.6233068)
\closepath
}
}
{
\newrgbcolor{curcolor}{0 0 0}
\pscustom[linestyle=none,fillstyle=solid,fillcolor=curcolor]
{
\newpath
\moveto(359.92952691,172.55299525)
\lineto(359.92952691,173.63112034)
\lineto(366.49202744,173.63112034)
\lineto(366.49202744,172.55299525)
\lineto(359.92952691,172.55299525)
\closepath
\moveto(357.37483921,172.55299525)
\lineto(357.37483921,173.63112034)
\lineto(358.74007369,173.63112034)
\lineto(358.74007369,172.55299525)
\lineto(357.37483921,172.55299525)
\closepath
}
}
{
\newrgbcolor{curcolor}{0 0 0}
\pscustom[linestyle=none,fillstyle=solid,fillcolor=curcolor]
{
\newpath
\moveto(359.92952691,175.10768269)
\lineto(359.92952691,176.2502609)
\lineto(365.43733985,178.30104232)
\lineto(359.92952691,180.35182373)
\lineto(359.92952691,181.49440195)
\lineto(366.49202744,179.03346425)
\lineto(366.49202744,177.56862038)
\lineto(359.92952691,175.10768269)
\closepath
}
}
{
\newrgbcolor{curcolor}{0 0 0}
\pscustom[linestyle=none,fillstyle=solid,fillcolor=curcolor]
{
\newpath
\moveto(362.9412459,188.5959654)
\lineto(363.4685897,188.5959654)
\lineto(363.4685897,183.63893376)
\curveto(364.21077725,183.68580876)(364.77718355,183.90846503)(365.16780858,184.30690256)
\curveto(365.55452736,184.70924634)(365.74788675,185.26784013)(365.74788675,185.98268394)
\curveto(365.74788675,186.39674647)(365.6971055,186.79713713)(365.59554299,187.18385591)
\curveto(365.49398048,187.57448094)(365.34163672,187.96119972)(365.1385117,188.34401225)
\lineto(366.15804303,188.34401225)
\curveto(366.32210555,187.95729347)(366.44710556,187.56080907)(366.53304306,187.15455903)
\curveto(366.61898057,186.748309)(366.66194932,186.33619959)(366.66194932,185.91823081)
\curveto(366.66194932,184.87135573)(366.3572618,184.04127754)(365.74788675,183.42799624)
\curveto(365.1385117,182.81862119)(364.31429289,182.51393367)(363.2752303,182.51393367)
\curveto(362.20101147,182.51393367)(361.3494489,182.80299619)(360.7205426,183.38112123)
\curveto(360.08773005,183.96315253)(359.77132378,184.74635572)(359.77132378,185.7307308)
\curveto(359.77132378,186.61354337)(360.05648005,187.31080905)(360.62679259,187.82252784)
\curveto(361.19319889,188.33815288)(361.96468333,188.5959654)(362.9412459,188.5959654)
\closepath
\moveto(362.62483963,187.51784031)
\curveto(362.03499583,187.51002781)(361.56429267,187.34401217)(361.21273014,187.0197934)
\curveto(360.86116761,186.69948087)(360.68538635,186.27369959)(360.68538635,185.74244955)
\curveto(360.68538635,185.140887)(360.85530824,184.65846509)(361.19515201,184.29518381)
\curveto(361.53499579,183.93580878)(362.01351145,183.72877751)(362.630699,183.67409001)
\lineto(362.62483963,187.51784031)
\closepath
}
}
{
\newrgbcolor{curcolor}{0 0 0}
\pscustom[linestyle=none,fillstyle=solid,fillcolor=curcolor]
{
\newpath
\moveto(357.37483921,190.3654962)
\lineto(357.37483921,191.44362129)
\lineto(366.49202744,191.44362129)
\lineto(366.49202744,190.3654962)
\lineto(357.37483921,190.3654962)
\closepath
}
}
{
\newrgbcolor{curcolor}{0 0 0}
\pscustom[linewidth=0.73291481,linecolor=curcolor,linestyle=dashed,dash=4.39748888 1.46582963]
{
\newpath
\moveto(257.789196,168.24448735)
\lineto(257.789196,229.58679735)
}
}
{
\newrgbcolor{curcolor}{0 0 0}
\pscustom[linestyle=none,fillstyle=solid,fillcolor=curcolor]
{
\newpath
\moveto(254.53717333,221.75161077)
\lineto(257.77628601,230.56012173)
\lineto(261.01539798,221.75161029)
\curveto(259.10302757,223.15884207)(256.48657295,223.15073357)(254.53717333,221.75161077)
\closepath
}
}
{
\newrgbcolor{curcolor}{1 1 1}
\pscustom[linestyle=none,fillstyle=solid,fillcolor=curcolor]
{
\newpath
\moveto(253.12230698,211.32492109)
\lineto(262.45728222,211.32492109)
\lineto(262.45728222,181.91975255)
\lineto(253.12230698,181.91975255)
\closepath
}
}
{
\newrgbcolor{curcolor}{0 0 0}
\pscustom[linestyle=none,fillstyle=solid,fillcolor=curcolor]
{
\newpath
\moveto(254.12575477,182.76002555)
\lineto(254.12575477,187.7873697)
\lineto(255.1218486,187.7873697)
\lineto(255.1218486,183.94361939)
\lineto(257.6999738,183.94361939)
\lineto(257.6999738,187.41236967)
\lineto(258.69606763,187.41236967)
\lineto(258.69606763,183.94361939)
\lineto(262.87380234,183.94361939)
\lineto(262.87380234,182.76002555)
\lineto(254.12575477,182.76002555)
\closepath
}
}
{
\newrgbcolor{curcolor}{0 0 0}
\pscustom[linestyle=none,fillstyle=solid,fillcolor=curcolor]
{
\newpath
\moveto(253.75661411,189.62721352)
\lineto(253.75661411,190.70533861)
\lineto(262.87380234,190.70533861)
\lineto(262.87380234,189.62721352)
\lineto(253.75661411,189.62721352)
\closepath
}
}
{
\newrgbcolor{curcolor}{0 0 0}
\pscustom[linestyle=none,fillstyle=solid,fillcolor=curcolor]
{
\newpath
\moveto(260.28395838,192.84401087)
\lineto(256.31130182,192.84401087)
\lineto(256.31130182,193.92213595)
\lineto(260.24294275,193.92213595)
\curveto(260.86403655,193.92213595)(261.33083347,194.04322971)(261.64333349,194.28541723)
\curveto(261.95192726,194.52760475)(262.10622415,194.89088603)(262.10622415,195.37526107)
\curveto(262.10622415,195.95729236)(261.92067726,196.41627677)(261.54958348,196.7522143)
\curveto(261.1784897,197.09205808)(260.67263029,197.26197997)(260.03200524,197.26197997)
\lineto(256.31130182,197.26197997)
\lineto(256.31130182,198.34010505)
\lineto(262.87380234,198.34010505)
\lineto(262.87380234,197.26197997)
\lineto(261.86598976,197.26197997)
\curveto(262.26442729,197.0002612)(262.56130231,196.69557367)(262.75661483,196.34791739)
\curveto(262.94802109,196.00416737)(263.04372423,195.60377671)(263.04372423,195.14674542)
\curveto(263.04372423,194.39283911)(262.80934921,193.82057344)(262.34059917,193.42994841)
\curveto(261.87184913,193.03932338)(261.1863022,192.84401087)(260.28395838,192.84401087)
\closepath
}
}
{
\newrgbcolor{curcolor}{0 0 0}
\pscustom[linestyle=none,fillstyle=solid,fillcolor=curcolor]
{
\newpath
\moveto(256.31130182,200.57252745)
\lineto(256.31130182,201.65065254)
\lineto(262.99098985,201.65065254)
\curveto(263.82692741,201.65065254)(264.43239621,201.49049628)(264.80739624,201.17018375)
\curveto(265.18239627,200.85377748)(265.36989629,200.34205868)(265.36989629,199.63502738)
\lineto(265.36989629,199.2248711)
\lineto(264.45583371,199.2248711)
\lineto(264.45583371,199.51198049)
\curveto(264.45583371,199.92213678)(264.36013058,200.20143367)(264.16872432,200.34987119)
\curveto(263.9812243,200.4983087)(263.58864614,200.57252745)(262.99098985,200.57252745)
\lineto(256.31130182,200.57252745)
\closepath
\moveto(253.75661411,200.57252745)
\lineto(253.75661411,201.65065254)
\lineto(255.1218486,201.65065254)
\lineto(255.1218486,200.57252745)
\lineto(253.75661411,200.57252745)
\closepath
}
}
{
\newrgbcolor{curcolor}{0 0 0}
\pscustom[linestyle=none,fillstyle=solid,fillcolor=curcolor]
{
\newpath
\moveto(257.06716125,206.44362141)
\curveto(257.06716125,205.86549636)(257.29372377,205.40846507)(257.74684881,205.07252755)
\curveto(258.19606759,204.73659002)(258.81325514,204.56862126)(259.59841145,204.56862126)
\curveto(260.38356776,204.56862126)(261.00270844,204.73463689)(261.45583348,205.06666817)
\curveto(261.90505226,205.4026057)(262.12966165,205.86159011)(262.12966165,206.44362141)
\curveto(262.12966165,207.0178402)(261.90309914,207.47291836)(261.4499741,207.80885589)
\curveto(260.99684906,208.14479342)(260.37966151,208.31276218)(259.59841145,208.31276218)
\curveto(258.82106764,208.31276218)(258.20583322,208.14479342)(257.75270818,207.80885589)
\curveto(257.29567689,207.47291836)(257.06716125,207.0178402)(257.06716125,206.44362141)
\closepath
\moveto(256.15309868,206.44362141)
\curveto(256.15309868,207.38112148)(256.4577862,208.11744966)(257.06716125,208.65260596)
\curveto(257.6765363,209.18776225)(258.52028637,209.45534039)(259.59841145,209.45534039)
\curveto(260.67263029,209.45534039)(261.51638036,209.18776225)(262.12966165,208.65260596)
\curveto(262.7390367,208.11744966)(263.04372423,207.38112148)(263.04372423,206.44362141)
\curveto(263.04372423,205.50221508)(262.7390367,204.76393377)(262.12966165,204.22877748)
\curveto(261.51638036,203.69752744)(260.67263029,203.43190242)(259.59841145,203.43190242)
\curveto(258.52028637,203.43190242)(257.6765363,203.69752744)(257.06716125,204.22877748)
\curveto(256.4577862,204.76393377)(256.15309868,205.50221508)(256.15309868,206.44362141)
\closepath
}
}
{
\newrgbcolor{curcolor}{0 0 0}
\pscustom[linewidth=0.73291481,linecolor=curcolor,linestyle=dashed,dash=4.39748888 1.46582963]
{
\newpath
\moveto(89.292886,225.99292735)
\lineto(89.292886,164.65061735)
}
}
{
\newrgbcolor{curcolor}{0 0 0}
\pscustom[linestyle=none,fillstyle=solid,fillcolor=curcolor]
{
\newpath
\moveto(92.54490867,172.48580392)
\lineto(89.30579599,163.67729296)
\lineto(86.06668402,172.4858044)
\curveto(87.97905443,171.07857262)(90.59550905,171.08668112)(92.54490867,172.48580392)
\closepath
}
}
{
\newrgbcolor{curcolor}{0 0 0}
\pscustom[linestyle=none,fillstyle=solid,fillcolor=curcolor]
{
\newpath
\moveto(77.4350581,186.04356489)
\lineto(78.68310508,186.04356489)
\curveto(78.3120113,185.64512736)(78.03466753,185.21934607)(77.85107376,184.76622104)
\curveto(77.66748,184.31700225)(77.57568312,183.83848659)(77.57568312,183.33067405)
\curveto(77.57568312,182.33067397)(77.88232377,181.56504891)(78.49560506,181.03379887)
\curveto(79.10498011,180.50254882)(79.98779268,180.2369238)(81.14404277,180.2369238)
\curveto(82.29638662,180.2369238)(83.17919919,180.50254882)(83.79248049,181.03379887)
\curveto(84.40185553,181.56504891)(84.70654306,182.33067397)(84.70654306,183.33067405)
\curveto(84.70654306,183.83848659)(84.61474618,184.31700225)(84.43115241,184.76622104)
\curveto(84.24755865,185.21934607)(83.97021487,185.64512736)(83.59912109,186.04356489)
\lineto(84.83544932,186.04356489)
\curveto(85.11669934,185.62950236)(85.32763686,185.1900492)(85.46826187,184.72520541)
\curveto(85.60888688,184.26426787)(85.67919939,183.77598658)(85.67919939,183.26036154)
\curveto(85.67919939,181.93614269)(85.27490248,180.89317385)(84.46630866,180.13145504)
\curveto(83.6538086,179.36973623)(82.54638664,178.98887683)(81.14404277,178.98887683)
\curveto(79.73779266,178.98887683)(78.6303707,179.36973623)(77.82177689,180.13145504)
\curveto(77.00927682,180.89317385)(76.60302679,181.93614269)(76.60302679,183.26036154)
\curveto(76.60302679,183.78379908)(76.67333929,184.27598662)(76.81396431,184.73692416)
\curveto(76.95068307,185.20176795)(77.15771433,185.63731486)(77.4350581,186.04356489)
\closepath
}
}
{
\newrgbcolor{curcolor}{0 0 0}
\pscustom[linestyle=none,fillstyle=solid,fillcolor=curcolor]
{
\newpath
\moveto(79.70263641,190.37950302)
\curveto(79.70263641,189.80137797)(79.92919893,189.34434669)(80.38232396,189.00840916)
\curveto(80.83154275,188.67247163)(81.4487303,188.50450287)(82.23388661,188.50450287)
\curveto(83.01904292,188.50450287)(83.6381836,188.67051851)(84.09130863,189.00254979)
\curveto(84.54052742,189.33848731)(84.76513681,189.79747172)(84.76513681,190.37950302)
\curveto(84.76513681,190.95372182)(84.53857429,191.40879998)(84.08544926,191.7447375)
\curveto(83.63232422,192.08067503)(83.01513667,192.24864379)(82.23388661,192.24864379)
\curveto(81.4565428,192.24864379)(80.84130838,192.08067503)(80.38818334,191.7447375)
\curveto(79.93115205,191.40879998)(79.70263641,190.95372182)(79.70263641,190.37950302)
\closepath
\moveto(78.78857384,190.37950302)
\curveto(78.78857384,191.31700309)(79.09326136,192.05333128)(79.70263641,192.58848757)
\curveto(80.31201146,193.12364386)(81.15576153,193.39122201)(82.23388661,193.39122201)
\curveto(83.30810545,193.39122201)(84.15185551,193.12364386)(84.76513681,192.58848757)
\curveto(85.37451186,192.05333128)(85.67919939,191.31700309)(85.67919939,190.37950302)
\curveto(85.67919939,189.4380967)(85.37451186,188.69981539)(84.76513681,188.16465909)
\curveto(84.15185551,187.63340905)(83.30810545,187.36778403)(82.23388661,187.36778403)
\curveto(81.15576153,187.36778403)(80.31201146,187.63340905)(79.70263641,188.16465909)
\curveto(79.09326136,188.69981539)(78.78857384,189.4380967)(78.78857384,190.37950302)
\closepath
}
}
{
\newrgbcolor{curcolor}{0 0 0}
\pscustom[linestyle=none,fillstyle=solid,fillcolor=curcolor]
{
\newpath
\moveto(81.54833968,200.62755108)
\lineto(85.5092775,200.62755108)
\lineto(85.5092775,199.54942599)
\lineto(81.58349593,199.54942599)
\curveto(80.96240214,199.54942599)(80.49755835,199.42833223)(80.18896457,199.18614472)
\curveto(79.8803708,198.9439572)(79.72607391,198.58067592)(79.72607391,198.09630088)
\curveto(79.72607391,197.51426958)(79.9116208,197.05528517)(80.28271458,196.71934764)
\curveto(80.65380836,196.38341012)(81.15966778,196.21544135)(81.80029283,196.21544135)
\lineto(85.5092775,196.21544135)
\lineto(85.5092775,195.13145689)
\lineto(78.94677698,195.13145689)
\lineto(78.94677698,196.21544135)
\lineto(79.96630831,196.21544135)
\curveto(79.57177702,196.47325388)(79.27685513,196.77598827)(79.08154261,197.12364455)
\curveto(78.8862301,197.47520708)(78.78857384,197.87950399)(78.78857384,198.33653527)
\curveto(78.78857384,199.09044158)(79.02294886,199.66075413)(79.49169889,200.04747291)
\curveto(79.95654268,200.43419169)(80.64208961,200.62755108)(81.54833968,200.62755108)
\closepath
}
}
{
\newrgbcolor{curcolor}{0 0 0}
\pscustom[linestyle=none,fillstyle=solid,fillcolor=curcolor]
{
\newpath
\moveto(77.08349558,203.85606636)
\lineto(78.94677698,203.85606636)
\lineto(78.94677698,206.07676966)
\lineto(79.78466767,206.07676966)
\lineto(79.78466767,203.85606636)
\lineto(83.34716795,203.85606636)
\curveto(83.88232424,203.85606636)(84.22607427,203.92833199)(84.37841803,204.07286325)
\curveto(84.53076179,204.22130076)(84.60693367,204.52012891)(84.60693367,204.96934769)
\lineto(84.60693367,206.07676966)
\lineto(85.5092775,206.07676966)
\lineto(85.5092775,204.96934769)
\curveto(85.5092775,204.13731638)(85.35498061,203.56309758)(85.04638683,203.24669131)
\curveto(84.73388681,202.93028503)(84.16748051,202.7720819)(83.34716795,202.7720819)
\lineto(79.78466767,202.7720819)
\lineto(79.78466767,201.98106621)
\lineto(78.94677698,201.98106621)
\lineto(78.94677698,202.7720819)
\lineto(77.08349558,202.7720819)
\lineto(77.08349558,203.85606636)
\closepath
}
}
{
\newrgbcolor{curcolor}{0 0 0}
\pscustom[linestyle=none,fillstyle=solid,fillcolor=curcolor]
{
\newpath
\moveto(79.95458956,211.3033322)
\curveto(79.88427705,211.18223844)(79.8334958,211.04942593)(79.80224579,210.90489467)
\curveto(79.76708954,210.76426966)(79.74951141,210.60801964)(79.74951141,210.43614463)
\curveto(79.74951141,209.82676958)(79.94873018,209.35801954)(80.34716771,209.02989452)
\curveto(80.74169899,208.70567574)(81.31005841,208.54356635)(82.05224597,208.54356635)
\lineto(85.5092775,208.54356635)
\lineto(85.5092775,207.45958189)
\lineto(78.94677698,207.45958189)
\lineto(78.94677698,208.54356635)
\lineto(79.96630831,208.54356635)
\curveto(79.56787077,208.77012887)(79.27294888,209.06505077)(79.08154261,209.42833205)
\curveto(78.8862301,209.79161333)(78.78857384,210.23301961)(78.78857384,210.75255091)
\curveto(78.78857384,210.82676966)(78.79443321,210.90880092)(78.80615196,210.99864467)
\curveto(78.81396446,211.08848843)(78.82763634,211.18809781)(78.84716759,211.29747282)
\lineto(79.95458956,211.3033322)
\closepath
}
}
{
\newrgbcolor{curcolor}{0 0 0}
\pscustom[linestyle=none,fillstyle=solid,fillcolor=curcolor]
{
\newpath
\moveto(79.70263641,214.74278713)
\curveto(79.70263641,214.16466209)(79.92919893,213.7076308)(80.38232396,213.37169327)
\curveto(80.83154275,213.03575575)(81.4487303,212.86778698)(82.23388661,212.86778698)
\curveto(83.01904292,212.86778698)(83.6381836,213.03380262)(84.09130863,213.3658339)
\curveto(84.54052742,213.70177142)(84.76513681,214.16075583)(84.76513681,214.74278713)
\curveto(84.76513681,215.31700593)(84.53857429,215.77208409)(84.08544926,216.10802161)
\curveto(83.63232422,216.44395914)(83.01513667,216.6119279)(82.23388661,216.6119279)
\curveto(81.4565428,216.6119279)(80.84130838,216.44395914)(80.38818334,216.10802161)
\curveto(79.93115205,215.77208409)(79.70263641,215.31700593)(79.70263641,214.74278713)
\closepath
\moveto(78.78857384,214.74278713)
\curveto(78.78857384,215.68028721)(79.09326136,216.41661539)(79.70263641,216.95177168)
\curveto(80.31201146,217.48692797)(81.15576153,217.75450612)(82.23388661,217.75450612)
\curveto(83.30810545,217.75450612)(84.15185551,217.48692797)(84.76513681,216.95177168)
\curveto(85.37451186,216.41661539)(85.67919939,215.68028721)(85.67919939,214.74278713)
\curveto(85.67919939,213.80138081)(85.37451186,213.0630995)(84.76513681,212.52794321)
\curveto(84.15185551,211.99669316)(83.30810545,211.73106814)(82.23388661,211.73106814)
\curveto(81.15576153,211.73106814)(80.31201146,211.99669316)(79.70263641,212.52794321)
\curveto(79.09326136,213.0630995)(78.78857384,213.80138081)(78.78857384,214.74278713)
\closepath
}
}
{
\newrgbcolor{curcolor}{0 0 0}
\pscustom[linestyle=none,fillstyle=solid,fillcolor=curcolor]
{
\newpath
\moveto(76.39208927,219.54747443)
\lineto(76.39208927,220.62559951)
\lineto(85.5092775,220.62559951)
\lineto(85.5092775,219.54747443)
\lineto(76.39208927,219.54747443)
\closepath
}
}
{
\newrgbcolor{curcolor}{0 0 0}
\pscustom[linestyle=none,fillstyle=solid,fillcolor=curcolor]
{
\newpath
\moveto(96.33154398,180.67637696)
\lineto(99.53662236,180.67637696)
\lineto(99.53662236,182.57481461)
\curveto(99.53662236,183.21153341)(99.40576298,183.68223658)(99.14404421,183.9869241)
\curveto(98.87841918,184.29551787)(98.47412228,184.44981476)(97.93115348,184.44981476)
\curveto(97.38427844,184.44981476)(96.98193466,184.29551787)(96.72412214,183.9869241)
\curveto(96.46240337,183.68223658)(96.33154398,183.21153341)(96.33154398,182.57481461)
\lineto(96.33154398,180.67637696)
\closepath
\moveto(92.73388745,180.67637696)
\lineto(95.37060641,180.67637696)
\lineto(95.37060641,182.42833023)
\curveto(95.37060641,183.00645527)(95.26318452,183.43614281)(95.04834075,183.71739283)
\curveto(94.82959074,184.0025491)(94.49755946,184.14512724)(94.05224693,184.14512724)
\curveto(93.61084064,184.14512724)(93.28076249,184.0025491)(93.06201247,183.71739283)
\curveto(92.84326245,183.43614281)(92.73388745,183.00645527)(92.73388745,182.42833023)
\lineto(92.73388745,180.67637696)
\closepath
\moveto(91.76123112,179.49278312)
\lineto(91.76123112,182.51622086)
\curveto(91.76123112,183.41856468)(91.94873113,184.11387724)(92.32373116,184.60215852)
\curveto(92.69873119,185.09043981)(93.23193436,185.33458046)(93.92334067,185.33458046)
\curveto(94.45849696,185.33458046)(94.88427824,185.20958045)(95.20068452,184.95958043)
\curveto(95.51709079,184.70958041)(95.71435643,184.34239288)(95.79248144,183.85801784)
\curveto(95.91748145,184.44004914)(96.17920022,184.89122105)(96.57763775,185.21153357)
\curveto(96.97216903,185.53575235)(97.4663097,185.69786174)(98.06005974,185.69786174)
\curveto(98.84130981,185.69786174)(99.44482548,185.43223672)(99.87060676,184.90098667)
\curveto(100.29638805,184.36973663)(100.50927869,183.6138772)(100.50927869,182.63340837)
\lineto(100.50927869,179.49278312)
\lineto(91.76123112,179.49278312)
\closepath
}
}
{
\newrgbcolor{curcolor}{0 0 0}
\pscustom[linestyle=none,fillstyle=solid,fillcolor=curcolor]
{
\newpath
\moveto(94.7026376,190.21544052)
\curveto(94.7026376,189.63731547)(94.92920012,189.18028419)(95.38232516,188.84434666)
\curveto(95.83154394,188.50840913)(96.44873149,188.34044037)(97.2338878,188.34044037)
\curveto(98.01904412,188.34044037)(98.63818479,188.50645601)(99.09130983,188.83848729)
\curveto(99.54052861,189.17442481)(99.765138,189.63340922)(99.765138,190.21544052)
\curveto(99.765138,190.78965932)(99.53857549,191.24473748)(99.08545045,191.580675)
\curveto(98.63232541,191.91661253)(98.01513787,192.08458129)(97.2338878,192.08458129)
\curveto(96.45654399,192.08458129)(95.84130957,191.91661253)(95.38818453,191.580675)
\curveto(94.93115325,191.24473748)(94.7026376,190.78965932)(94.7026376,190.21544052)
\closepath
\moveto(93.78857503,190.21544052)
\curveto(93.78857503,191.15294059)(94.09326255,191.88926878)(94.7026376,192.42442507)
\curveto(95.31201265,192.95958136)(96.15576272,193.22715951)(97.2338878,193.22715951)
\curveto(98.30810664,193.22715951)(99.15185671,192.95958136)(99.765138,192.42442507)
\curveto(100.37451305,191.88926878)(100.67920058,191.15294059)(100.67920058,190.21544052)
\curveto(100.67920058,189.2740342)(100.37451305,188.53575289)(99.765138,188.00059659)
\curveto(99.15185671,187.46934655)(98.30810664,187.20372153)(97.2338878,187.20372153)
\curveto(96.15576272,187.20372153)(95.31201265,187.46934655)(94.7026376,188.00059659)
\curveto(94.09326255,188.53575289)(93.78857503,189.2740342)(93.78857503,190.21544052)
\closepath
}
}
{
\newrgbcolor{curcolor}{0 0 0}
\pscustom[linestyle=none,fillstyle=solid,fillcolor=curcolor]
{
\newpath
\moveto(95.20654389,200.11778543)
\curveto(94.72216885,200.3873167)(94.36474695,200.70958235)(94.13427818,201.08458238)
\curveto(93.90380941,201.45958241)(93.78857503,201.90098869)(93.78857503,202.40880123)
\curveto(93.78857503,203.09239504)(94.02880942,203.61973883)(94.50927821,203.99083261)
\curveto(94.98584075,204.36192639)(95.6655283,204.54747328)(96.54834087,204.54747328)
\lineto(100.50927869,204.54747328)
\lineto(100.50927869,203.46348882)
\lineto(96.58349713,203.46348882)
\curveto(95.95459083,203.46348882)(95.48779391,203.35216068)(95.18310639,203.12950442)
\curveto(94.87841887,202.90684815)(94.7260751,202.56700437)(94.7260751,202.10997309)
\curveto(94.7260751,201.55137929)(94.91162199,201.10997301)(95.28271577,200.78575423)
\curveto(95.65380955,200.46153546)(96.15966897,200.29942607)(96.80029402,200.29942607)
\lineto(100.50927869,200.29942607)
\lineto(100.50927869,199.21544161)
\lineto(96.58349713,199.21544161)
\curveto(95.95068458,199.21544161)(95.48388766,199.10411347)(95.18310639,198.8814572)
\curveto(94.87841887,198.65880094)(94.7260751,198.31505091)(94.7260751,197.85020712)
\curveto(94.7260751,197.29942583)(94.91357512,196.86192579)(95.28857515,196.53770702)
\curveto(95.65966893,196.21348824)(96.16357522,196.05137885)(96.80029402,196.05137885)
\lineto(100.50927869,196.05137885)
\lineto(100.50927869,194.96739439)
\lineto(93.94677817,194.96739439)
\lineto(93.94677817,196.05137885)
\lineto(94.9663095,196.05137885)
\curveto(94.56396572,196.29747262)(94.26709069,196.59239452)(94.07568443,196.93614455)
\curveto(93.88427816,197.27989458)(93.78857503,197.68809773)(93.78857503,198.16075402)
\curveto(93.78857503,198.63731656)(93.90966879,199.04161347)(94.15185631,199.37364474)
\curveto(94.39404383,199.70958227)(94.74560636,199.95762916)(95.20654389,200.11778543)
\closepath
}
}
{
\newrgbcolor{curcolor}{0 0 0}
\pscustom[linestyle=none,fillstyle=solid,fillcolor=curcolor]
{
\newpath
\moveto(97.2338878,211.4146604)
\curveto(96.44091899,211.4146604)(95.81982519,211.25059788)(95.37060641,210.92247286)
\curveto(94.91748137,210.59825408)(94.69091885,210.15098842)(94.69091885,209.58067588)
\curveto(94.69091885,209.01036333)(94.91748137,208.56114454)(95.37060641,208.23301952)
\curveto(95.81982519,207.90880074)(96.44091899,207.74669135)(97.2338878,207.74669135)
\curveto(98.02685662,207.74669135)(98.64990354,207.90880074)(99.10302858,208.23301952)
\curveto(99.55224736,208.56114454)(99.77685676,209.01036333)(99.77685676,209.58067588)
\curveto(99.77685676,210.15098842)(99.55224736,210.59825408)(99.10302858,210.92247286)
\curveto(98.64990354,211.25059788)(98.02685662,211.4146604)(97.2338878,211.4146604)
\closepath
\moveto(94.942872,207.74669135)
\curveto(94.55224697,207.97325387)(94.26318444,208.25841015)(94.07568443,208.60216017)
\curveto(93.88427816,208.94981645)(93.78857503,209.36387898)(93.78857503,209.84434777)
\curveto(93.78857503,210.64122283)(94.1049813,211.28770726)(94.73779385,211.78380105)
\curveto(95.37060641,212.28380109)(96.20263772,212.53380111)(97.2338878,212.53380111)
\curveto(98.26513789,212.53380111)(99.0971692,212.28380109)(99.72998175,211.78380105)
\curveto(100.3627943,211.28770726)(100.67920058,210.64122283)(100.67920058,209.84434777)
\curveto(100.67920058,209.36387898)(100.58545057,208.94981645)(100.39795055,208.60216017)
\curveto(100.20654429,208.25841015)(99.91552864,207.97325387)(99.52490361,207.74669135)
\lineto(100.50927869,207.74669135)
\lineto(100.50927869,206.66270689)
\lineto(91.39209046,206.66270689)
\lineto(91.39209046,207.74669135)
\lineto(94.942872,207.74669135)
\closepath
}
}
{
\newrgbcolor{curcolor}{0 0 0}
\pscustom[linestyle=none,fillstyle=solid,fillcolor=curcolor]
{
\newpath
\moveto(97.2104503,217.30333404)
\curveto(97.2104503,216.43224022)(97.31005968,215.82872455)(97.50927845,215.49278702)
\curveto(97.70849722,215.1568495)(98.04834099,214.98888073)(98.52880978,214.98888073)
\curveto(98.91162231,214.98888073)(99.21630984,215.11388074)(99.44287235,215.36388076)
\curveto(99.66552862,215.61778703)(99.77685676,215.96153706)(99.77685676,216.39513084)
\curveto(99.77685676,216.99278714)(99.56591924,217.4713028)(99.14404421,217.83067783)
\curveto(98.71826292,218.19395911)(98.15380975,218.37559975)(97.4506847,218.37559975)
\lineto(97.2104503,218.37559975)
\lineto(97.2104503,217.30333404)
\closepath
\moveto(96.76513777,219.45372484)
\lineto(100.50927869,219.45372484)
\lineto(100.50927869,218.37559975)
\lineto(99.51318486,218.37559975)
\curveto(99.91162239,218.12950598)(100.20654429,217.82286533)(100.39795055,217.4556778)
\curveto(100.58545057,217.08849027)(100.67920058,216.63927149)(100.67920058,216.10802145)
\curveto(100.67920058,215.43614639)(100.49170056,214.9009901)(100.11670053,214.50255257)
\curveto(99.73779425,214.10802129)(99.23193484,213.91075565)(98.59912229,213.91075565)
\curveto(97.86084098,213.91075565)(97.30420031,214.15684942)(96.92920028,214.64903696)
\curveto(96.55420025,215.14513074)(96.36670023,215.88341205)(96.36670023,216.86388088)
\lineto(96.36670023,218.37559975)
\lineto(96.26123148,218.37559975)
\curveto(95.76513769,218.37559975)(95.38232516,218.21153724)(95.11279388,217.88341221)
\curveto(94.83935636,217.55919344)(94.7026376,217.10216215)(94.7026376,216.51231835)
\curveto(94.7026376,216.13731832)(94.74755948,215.77208392)(94.83740324,215.41661514)
\curveto(94.927247,215.06114636)(95.06201263,214.71934946)(95.24170015,214.39122443)
\lineto(94.24560632,214.39122443)
\curveto(94.09326255,214.78575572)(93.97998129,215.16856825)(93.90576254,215.53966203)
\curveto(93.82763753,215.91075581)(93.78857503,216.27208396)(93.78857503,216.62364649)
\curveto(93.78857503,217.57286531)(94.0346688,218.28184974)(94.52685634,218.75059978)
\curveto(95.01904388,219.21934982)(95.76513769,219.45372484)(96.76513777,219.45372484)
\closepath
}
}
{
\newrgbcolor{curcolor}{0 0 0}
\pscustom[linestyle=none,fillstyle=solid,fillcolor=curcolor]
{
\newpath
\moveto(94.20523087,101.98550458)
\lineto(99.73648087,101.98550458)
\lineto(99.73648087,100.98941083)
\lineto(95.38882462,100.98941083)
\lineto(95.38882462,98.39956708)
\lineto(99.55484024,98.39956708)
\lineto(99.55484024,97.40347333)
\lineto(95.38882462,97.40347333)
\lineto(95.38882462,94.23355146)
\lineto(99.84194962,94.23355146)
\lineto(99.84194962,93.23745771)
\lineto(94.20523087,93.23745771)
\lineto(94.20523087,101.98550458)
\closepath
}
}
{
\newrgbcolor{curcolor}{0 0 0}
\pscustom[linestyle=none,fillstyle=solid,fillcolor=curcolor]
{
\newpath
\moveto(101.20718399,97.00503583)
\lineto(104.36538712,97.00503583)
\lineto(104.36538712,96.04409833)
\lineto(101.20718399,96.04409833)
\lineto(101.20718399,97.00503583)
\closepath
}
}
{
\newrgbcolor{curcolor}{0 0 0}
\pscustom[linestyle=none,fillstyle=solid,fillcolor=curcolor]
{
\newpath
\moveto(106.44546524,94.23355146)
\lineto(108.37905899,94.23355146)
\lineto(108.37905899,100.90737958)
\lineto(106.27554337,100.48550458)
\lineto(106.27554337,101.56362958)
\lineto(108.36734024,101.98550458)
\lineto(109.55093399,101.98550458)
\lineto(109.55093399,94.23355146)
\lineto(111.48452774,94.23355146)
\lineto(111.48452774,93.23745771)
\lineto(106.44546524,93.23745771)
\lineto(106.44546524,94.23355146)
\closepath
}
}
{
\newrgbcolor{curcolor}{0 0 0}
\pscustom[linestyle=none,fillstyle=solid,fillcolor=curcolor]
{
\newpath
\moveto(244.49831169,146.88392329)
\lineto(247.70005466,146.88392329)
\lineto(247.70005466,146.24954531)
\lineto(245.25210199,146.24954531)
\lineto(245.25210199,144.60762584)
\lineto(247.46123001,144.60762584)
\lineto(247.46123001,143.97324786)
\lineto(245.25210199,143.97324786)
\lineto(245.25210199,141.31259198)
\lineto(244.49831169,141.31259198)
\lineto(244.49831169,146.88392329)
\closepath
}
}
{
\newrgbcolor{curcolor}{0 0 0}
\pscustom[linestyle=none,fillstyle=solid,fillcolor=curcolor]
{
\newpath
\moveto(248.52474591,143.7120334)
\lineto(250.53609727,143.7120334)
\lineto(250.53609727,143.10004523)
\lineto(248.52474591,143.10004523)
\lineto(248.52474591,143.7120334)
\closepath
}
}
{
\newrgbcolor{curcolor}{0 0 0}
\pscustom[linestyle=none,fillstyle=solid,fillcolor=curcolor]
{
\newpath
\moveto(251.86082796,141.94696996)
\lineto(253.09226757,141.94696996)
\lineto(253.09226757,146.19730242)
\lineto(251.75261054,145.92862469)
\lineto(251.75261054,146.61524556)
\lineto(253.0848043,146.88392329)
\lineto(253.8385946,146.88392329)
\lineto(253.8385946,141.94696996)
\lineto(255.07003421,141.94696996)
\lineto(255.07003421,141.31259198)
\lineto(251.86082796,141.31259198)
\lineto(251.86082796,141.94696996)
\closepath
}
}
{
\newrgbcolor{curcolor}{0 0 0}
\pscustom[linestyle=none,fillstyle=solid,fillcolor=curcolor]
{
\newpath
\moveto(166.40236679,227.23576398)
\lineto(166.40236679,223.94865461)
\lineto(167.89064804,223.94865461)
\curveto(168.44142929,223.94865461)(168.86721054,224.09123273)(169.16799179,224.37638898)
\curveto(169.46877304,224.66154523)(169.61916367,225.06779523)(169.61916367,225.59513898)
\curveto(169.61916367,226.11857648)(169.46877304,226.52287336)(169.16799179,226.80802961)
\curveto(168.86721054,227.09318586)(168.44142929,227.23576398)(167.89064804,227.23576398)
\lineto(166.40236679,227.23576398)
\closepath
\moveto(165.21877304,228.20842023)
\lineto(167.89064804,228.20842023)
\curveto(168.87111679,228.20842023)(169.61135117,227.98576398)(170.11135117,227.54045148)
\curveto(170.61525742,227.09904523)(170.86721054,226.45060773)(170.86721054,225.59513898)
\curveto(170.86721054,224.73185773)(170.61525742,224.07951398)(170.11135117,223.63810773)
\curveto(169.61135117,223.19670148)(168.87111679,222.97599836)(167.89064804,222.97599836)
\lineto(166.40236679,222.97599836)
\lineto(166.40236679,219.46037336)
\lineto(165.21877304,219.46037336)
\lineto(165.21877304,228.20842023)
\closepath
}
}
{
\newrgbcolor{curcolor}{0 0 0}
\pscustom[linestyle=none,fillstyle=solid,fillcolor=curcolor]
{
\newpath
\moveto(172.46096054,228.20842023)
\lineto(173.64455429,228.20842023)
\lineto(173.64455429,220.45646711)
\lineto(177.90431992,220.45646711)
\lineto(177.90431992,219.46037336)
\lineto(172.46096054,219.46037336)
\lineto(172.46096054,228.20842023)
\closepath
}
}
{
\newrgbcolor{curcolor}{0 0 0}
\pscustom[linestyle=none,fillstyle=solid,fillcolor=curcolor]
{
\newpath
\moveto(185.69142929,227.53459211)
\lineto(185.69142929,226.28654523)
\curveto(185.29299179,226.65763898)(184.86721054,226.93498273)(184.41408554,227.11857648)
\curveto(183.96486679,227.30217023)(183.48635117,227.39396711)(182.97853867,227.39396711)
\curveto(181.97853867,227.39396711)(181.21291367,227.08732648)(180.68166367,226.47404523)
\curveto(180.15041367,225.86467023)(179.88478867,224.98185773)(179.88478867,223.82560773)
\curveto(179.88478867,222.67326398)(180.15041367,221.79045148)(180.68166367,221.17717023)
\curveto(181.21291367,220.56779523)(181.97853867,220.26310773)(182.97853867,220.26310773)
\curveto(183.48635117,220.26310773)(183.96486679,220.35490461)(184.41408554,220.53849836)
\curveto(184.86721054,220.72209211)(185.29299179,220.99943586)(185.69142929,221.37052961)
\lineto(185.69142929,220.13420148)
\curveto(185.27736679,219.85295148)(184.83791367,219.64201398)(184.37306992,219.50138898)
\curveto(183.91213242,219.36076398)(183.42385117,219.29045148)(182.90822617,219.29045148)
\curveto(181.58400742,219.29045148)(180.54103867,219.69474836)(179.77931992,220.50334211)
\curveto(179.01760117,221.31584211)(178.63674179,222.42326398)(178.63674179,223.82560773)
\curveto(178.63674179,225.23185773)(179.01760117,226.33927961)(179.77931992,227.14787336)
\curveto(180.54103867,227.96037336)(181.58400742,228.36662336)(182.90822617,228.36662336)
\curveto(183.43166367,228.36662336)(183.92385117,228.29631086)(184.38478867,228.15568586)
\curveto(184.84963242,228.01896711)(185.28517929,227.81193586)(185.69142929,227.53459211)
\closepath
}
}
{
\newrgbcolor{curcolor}{0 0 0}
\pscustom[linestyle=none,fillstyle=solid,fillcolor=curcolor]
{
\newpath
\moveto(186.93947617,223.22795148)
\lineto(190.09767929,223.22795148)
\lineto(190.09767929,222.26701398)
\lineto(186.93947617,222.26701398)
\lineto(186.93947617,223.22795148)
\closepath
}
}
{
\newrgbcolor{curcolor}{0 0 0}
\pscustom[linestyle=none,fillstyle=solid,fillcolor=curcolor]
{
\newpath
\moveto(192.17775742,220.45646711)
\lineto(194.11135117,220.45646711)
\lineto(194.11135117,227.13029523)
\lineto(192.00783554,226.70842023)
\lineto(192.00783554,227.78654523)
\lineto(194.09963242,228.20842023)
\lineto(195.28322617,228.20842023)
\lineto(195.28322617,220.45646711)
\lineto(197.21681992,220.45646711)
\lineto(197.21681992,219.46037336)
\lineto(192.17775742,219.46037336)
\lineto(192.17775742,220.45646711)
\closepath
}
}
{
\newrgbcolor{curcolor}{0 0 0}
\pscustom[linestyle=none,fillstyle=solid,fillcolor=curcolor]
{
\newpath
\moveto(391.37526718,225.83551069)
\lineto(391.37526718,222.54840131)
\lineto(392.86354843,222.54840131)
\curveto(393.41432968,222.54840131)(393.84011093,222.69097944)(394.14089218,222.97613569)
\curveto(394.44167343,223.26129194)(394.59206406,223.66754194)(394.59206406,224.19488569)
\curveto(394.59206406,224.71832319)(394.44167343,225.12262006)(394.14089218,225.40777631)
\curveto(393.84011093,225.69293256)(393.41432968,225.83551069)(392.86354843,225.83551069)
\lineto(391.37526718,225.83551069)
\closepath
\moveto(390.19167343,226.80816694)
\lineto(392.86354843,226.80816694)
\curveto(393.84401718,226.80816694)(394.58425156,226.58551069)(395.08425156,226.14019819)
\curveto(395.58815781,225.69879194)(395.84011093,225.05035444)(395.84011093,224.19488569)
\curveto(395.84011093,223.33160444)(395.58815781,222.67926069)(395.08425156,222.23785444)
\curveto(394.58425156,221.79644819)(393.84401718,221.57574506)(392.86354843,221.57574506)
\lineto(391.37526718,221.57574506)
\lineto(391.37526718,218.06012006)
\lineto(390.19167343,218.06012006)
\lineto(390.19167343,226.80816694)
\closepath
}
}
{
\newrgbcolor{curcolor}{0 0 0}
\pscustom[linestyle=none,fillstyle=solid,fillcolor=curcolor]
{
\newpath
\moveto(397.43386093,226.80816694)
\lineto(398.61745468,226.80816694)
\lineto(398.61745468,219.05621381)
\lineto(402.87722031,219.05621381)
\lineto(402.87722031,218.06012006)
\lineto(397.43386093,218.06012006)
\lineto(397.43386093,226.80816694)
\closepath
}
}
{
\newrgbcolor{curcolor}{0 0 0}
\pscustom[linestyle=none,fillstyle=solid,fillcolor=curcolor]
{
\newpath
\moveto(410.66432968,226.13433881)
\lineto(410.66432968,224.88629194)
\curveto(410.26589218,225.25738569)(409.84011093,225.53472944)(409.38698593,225.71832319)
\curveto(408.93776718,225.90191694)(408.45925156,225.99371381)(407.95143906,225.99371381)
\curveto(406.95143906,225.99371381)(406.18581406,225.68707319)(405.65456406,225.07379194)
\curveto(405.12331406,224.46441694)(404.85768906,223.58160444)(404.85768906,222.42535444)
\curveto(404.85768906,221.27301069)(405.12331406,220.39019819)(405.65456406,219.77691694)
\curveto(406.18581406,219.16754194)(406.95143906,218.86285444)(407.95143906,218.86285444)
\curveto(408.45925156,218.86285444)(408.93776718,218.95465131)(409.38698593,219.13824506)
\curveto(409.84011093,219.32183881)(410.26589218,219.59918256)(410.66432968,219.97027631)
\lineto(410.66432968,218.73394819)
\curveto(410.25026718,218.45269819)(409.81081406,218.24176069)(409.34597031,218.10113569)
\curveto(408.88503281,217.96051069)(408.39675156,217.89019819)(407.88112656,217.89019819)
\curveto(406.55690781,217.89019819)(405.51393906,218.29449506)(404.75222031,219.10308881)
\curveto(403.99050156,219.91558881)(403.60964218,221.02301069)(403.60964218,222.42535444)
\curveto(403.60964218,223.83160444)(403.99050156,224.93902631)(404.75222031,225.74762006)
\curveto(405.51393906,226.56012006)(406.55690781,226.96637006)(407.88112656,226.96637006)
\curveto(408.40456406,226.96637006)(408.89675156,226.89605756)(409.35768906,226.75543256)
\curveto(409.82253281,226.61871381)(410.25807968,226.41168256)(410.66432968,226.13433881)
\closepath
}
}
{
\newrgbcolor{curcolor}{0 0 0}
\pscustom[linestyle=none,fillstyle=solid,fillcolor=curcolor]
{
\newpath
\moveto(411.91237656,221.82769819)
\lineto(415.07057968,221.82769819)
\lineto(415.07057968,220.86676069)
\lineto(411.91237656,220.86676069)
\lineto(411.91237656,221.82769819)
\closepath
}
}
{
\newrgbcolor{curcolor}{0 0 0}
\pscustom[linestyle=none,fillstyle=solid,fillcolor=curcolor]
{
\newpath
\moveto(417.96511093,219.05621381)
\lineto(422.09597031,219.05621381)
\lineto(422.09597031,218.06012006)
\lineto(416.54128281,218.06012006)
\lineto(416.54128281,219.05621381)
\curveto(416.99050156,219.52105756)(417.60182968,220.14410444)(418.37526718,220.92535444)
\curveto(419.15261093,221.71051069)(419.64089218,222.21637006)(419.84011093,222.44293256)
\curveto(420.21901718,222.86871381)(420.48268906,223.22808881)(420.63112656,223.52105756)
\curveto(420.78347031,223.81793256)(420.85964218,224.10894819)(420.85964218,224.39410444)
\curveto(420.85964218,224.85894819)(420.69557968,225.23785444)(420.36745468,225.53082319)
\curveto(420.04323593,225.82379194)(419.61940781,225.97027631)(419.09597031,225.97027631)
\curveto(418.72487656,225.97027631)(418.33229843,225.90582319)(417.91823593,225.77691694)
\curveto(417.50807968,225.64801069)(417.06862656,225.45269819)(416.59987656,225.19097944)
\lineto(416.59987656,226.38629194)
\curveto(417.07643906,226.57769819)(417.52175156,226.72222944)(417.93581406,226.81988569)
\curveto(418.34987656,226.91754194)(418.72878281,226.96637006)(419.07253281,226.96637006)
\curveto(419.97878281,226.96637006)(420.70143906,226.73980756)(421.24050156,226.28668256)
\curveto(421.77956406,225.83355756)(422.04909531,225.22808881)(422.04909531,224.47027631)
\curveto(422.04909531,224.11090131)(421.98073593,223.76910444)(421.84401718,223.44488569)
\curveto(421.71120468,223.12457319)(421.46706406,222.74566694)(421.11159531,222.30816694)
\curveto(421.01393906,222.19488569)(420.70339218,221.86676069)(420.17995468,221.32379194)
\curveto(419.65651718,220.78472944)(418.91823593,220.02887006)(417.96511093,219.05621381)
\closepath
}
}
{
\newrgbcolor{curcolor}{0 0 0}
\pscustom[linestyle=none,fillstyle=solid,fillcolor=curcolor]
{
\newpath
\moveto(351.38380459,101.38575211)
\lineto(352.31479756,101.38575211)
\lineto(352.31479756,95.28820907)
\lineto(355.66545046,95.28820907)
\lineto(355.66545046,94.50470014)
\lineto(351.38380459,94.50470014)
\lineto(351.38380459,101.38575211)
\closepath
}
}
{
\newrgbcolor{curcolor}{0 0 0}
\pscustom[linestyle=none,fillstyle=solid,fillcolor=curcolor]
{
\newpath
\moveto(430.18645493,58.23129316)
\lineto(426.84661118,66.97934003)
\lineto(428.0829393,66.97934003)
\lineto(430.85442368,59.61410566)
\lineto(433.63176743,66.97934003)
\lineto(434.86223618,66.97934003)
\lineto(431.5282518,58.23129316)
\lineto(430.18645493,58.23129316)
\closepath
}
}
{
\newrgbcolor{curcolor}{0 0 0}
\pscustom[linestyle=none,fillstyle=solid,fillcolor=curcolor]
{
\newpath
\moveto(434.83879868,61.99887128)
\lineto(437.9970018,61.99887128)
\lineto(437.9970018,61.03793378)
\lineto(434.83879868,61.03793378)
\lineto(434.83879868,61.99887128)
\closepath
}
}
{
\newrgbcolor{curcolor}{0 0 0}
\pscustom[linestyle=none,fillstyle=solid,fillcolor=curcolor]
{
\newpath
\moveto(440.89153305,59.22738691)
\lineto(445.02239243,59.22738691)
\lineto(445.02239243,58.23129316)
\lineto(439.46770493,58.23129316)
\lineto(439.46770493,59.22738691)
\curveto(439.91692368,59.69223066)(440.5282518,60.31527753)(441.3016893,61.09652753)
\curveto(442.07903305,61.88168378)(442.5673143,62.38754316)(442.76653305,62.61410566)
\curveto(443.1454393,63.03988691)(443.40911118,63.39926191)(443.55754868,63.69223066)
\curveto(443.70989243,63.98910566)(443.7860643,64.28012128)(443.7860643,64.56527753)
\curveto(443.7860643,65.03012128)(443.6220018,65.40902753)(443.2938768,65.70199628)
\curveto(442.96965805,65.99496503)(442.54582993,66.14144941)(442.02239243,66.14144941)
\curveto(441.65129868,66.14144941)(441.25872055,66.07699628)(440.84465805,65.94809003)
\curveto(440.4345018,65.81918378)(439.99504868,65.62387128)(439.52629868,65.36215253)
\lineto(439.52629868,66.55746503)
\curveto(440.00286118,66.74887128)(440.44817368,66.89340253)(440.86223618,66.99105878)
\curveto(441.27629868,67.08871503)(441.65520493,67.13754316)(441.99895493,67.13754316)
\curveto(442.90520493,67.13754316)(443.62786118,66.91098066)(444.16692368,66.45785566)
\curveto(444.70598618,66.00473066)(444.97551743,65.39926191)(444.97551743,64.64144941)
\curveto(444.97551743,64.28207441)(444.90715805,63.94027753)(444.7704393,63.61605878)
\curveto(444.6376268,63.29574628)(444.39348618,62.91684003)(444.03801743,62.47934003)
\curveto(443.94036118,62.36605878)(443.6298143,62.03793378)(443.1063768,61.49496503)
\curveto(442.5829393,60.95590253)(441.84465805,60.20004316)(440.89153305,59.22738691)
\closepath
}
}
{
\newrgbcolor{curcolor}{0 0 0}
\pscustom[linestyle=none,fillstyle=solid,fillcolor=curcolor]
{
\newpath
\moveto(110.45491806,365.49645276)
\lineto(115.98616806,365.49645276)
\lineto(115.98616806,364.50035901)
\lineto(111.63851181,364.50035901)
\lineto(111.63851181,361.91051526)
\lineto(115.80452744,361.91051526)
\lineto(115.80452744,360.91442151)
\lineto(111.63851181,360.91442151)
\lineto(111.63851181,357.74449964)
\lineto(116.09163681,357.74449964)
\lineto(116.09163681,356.74840589)
\lineto(110.45491806,356.74840589)
\lineto(110.45491806,365.49645276)
\closepath
}
}
{
\newrgbcolor{curcolor}{0 0 0}
\pscustom[linestyle=none,fillstyle=solid,fillcolor=curcolor]
{
\newpath
\moveto(118.00179306,365.86559339)
\lineto(119.07991806,365.86559339)
\lineto(119.07991806,356.74840589)
\lineto(118.00179306,356.74840589)
\lineto(118.00179306,365.86559339)
\closepath
}
}
{
\newrgbcolor{curcolor}{0 0 0}
\pscustom[linestyle=none,fillstyle=solid,fillcolor=curcolor]
{
\newpath
\moveto(129.33382431,363.11754651)
\lineto(129.33382431,362.09801526)
\curveto(129.02913681,362.25426526)(128.71273056,362.37145276)(128.38460556,362.44957776)
\curveto(128.05648056,362.52770276)(127.71663681,362.56676526)(127.36507431,362.56676526)
\curveto(126.82991806,362.56676526)(126.42757431,362.48473401)(126.15804306,362.32067151)
\curveto(125.89241806,362.15660901)(125.75960556,361.91051526)(125.75960556,361.58239026)
\curveto(125.75960556,361.33239026)(125.85530869,361.13512464)(126.04671494,360.99059339)
\curveto(126.23812119,360.84996839)(126.62288681,360.71520276)(127.20101181,360.58629651)
\lineto(127.57015244,360.50426526)
\curveto(128.33577744,360.34020276)(128.87874619,360.10778089)(129.19905869,359.80699964)
\curveto(129.52327744,359.51012464)(129.68538681,359.09410901)(129.68538681,358.55895276)
\curveto(129.68538681,357.94957776)(129.44319931,357.46715589)(128.95882431,357.11168714)
\curveto(128.47835556,356.75621839)(127.81624619,356.57848401)(126.97249619,356.57848401)
\curveto(126.62093369,356.57848401)(126.25374619,356.61364026)(125.87093369,356.68395276)
\curveto(125.49202744,356.75035901)(125.09163681,356.85192151)(124.66976181,356.98864026)
\lineto(124.66976181,358.10192151)
\curveto(125.06819931,357.89489026)(125.46077744,357.73864026)(125.84749619,357.63317151)
\curveto(126.23421494,357.53160901)(126.61702744,357.48082776)(126.99593369,357.48082776)
\curveto(127.50374619,357.48082776)(127.89437119,357.56676526)(128.16780869,357.73864026)
\curveto(128.44124619,357.91442151)(128.57796494,358.16051526)(128.57796494,358.47692151)
\curveto(128.57796494,358.76989026)(128.47835556,358.99449964)(128.27913681,359.15074964)
\curveto(128.08382431,359.30699964)(127.65218369,359.45739026)(126.98421494,359.60192151)
\lineto(126.60921494,359.68981214)
\curveto(125.94124619,359.83043714)(125.45882431,360.04528089)(125.16194931,360.33434339)
\curveto(124.86507431,360.62731214)(124.71663681,361.02770276)(124.71663681,361.53551526)
\curveto(124.71663681,362.15270276)(124.93538681,362.62926526)(125.37288681,362.96520276)
\curveto(125.81038681,363.30114026)(126.43148056,363.46910901)(127.23616806,363.46910901)
\curveto(127.63460556,363.46910901)(128.00960556,363.43981214)(128.36116806,363.38121839)
\curveto(128.71273056,363.32262464)(129.03694931,363.23473401)(129.33382431,363.11754651)
\closepath
}
}
{
\newrgbcolor{curcolor}{0 0 0}
\pscustom[linestyle=none,fillstyle=solid,fillcolor=curcolor]
{
\newpath
\moveto(131.40804306,363.31090589)
\lineto(132.48616806,363.31090589)
\lineto(132.48616806,356.74840589)
\lineto(131.40804306,356.74840589)
\lineto(131.40804306,363.31090589)
\closepath
\moveto(131.40804306,365.86559339)
\lineto(132.48616806,365.86559339)
\lineto(132.48616806,364.50035901)
\lineto(131.40804306,364.50035901)
\lineto(131.40804306,365.86559339)
\closepath
}
}
{
\newrgbcolor{curcolor}{0 0 0}
\pscustom[linestyle=none,fillstyle=solid,fillcolor=curcolor]
{
\newpath
\moveto(138.91976181,363.11754651)
\lineto(138.91976181,362.09801526)
\curveto(138.61507431,362.25426526)(138.29866806,362.37145276)(137.97054306,362.44957776)
\curveto(137.64241806,362.52770276)(137.30257431,362.56676526)(136.95101181,362.56676526)
\curveto(136.41585556,362.56676526)(136.01351181,362.48473401)(135.74398056,362.32067151)
\curveto(135.47835556,362.15660901)(135.34554306,361.91051526)(135.34554306,361.58239026)
\curveto(135.34554306,361.33239026)(135.44124619,361.13512464)(135.63265244,360.99059339)
\curveto(135.82405869,360.84996839)(136.20882431,360.71520276)(136.78694931,360.58629651)
\lineto(137.15608994,360.50426526)
\curveto(137.92171494,360.34020276)(138.46468369,360.10778089)(138.78499619,359.80699964)
\curveto(139.10921494,359.51012464)(139.27132431,359.09410901)(139.27132431,358.55895276)
\curveto(139.27132431,357.94957776)(139.02913681,357.46715589)(138.54476181,357.11168714)
\curveto(138.06429306,356.75621839)(137.40218369,356.57848401)(136.55843369,356.57848401)
\curveto(136.20687119,356.57848401)(135.83968369,356.61364026)(135.45687119,356.68395276)
\curveto(135.07796494,356.75035901)(134.67757431,356.85192151)(134.25569931,356.98864026)
\lineto(134.25569931,358.10192151)
\curveto(134.65413681,357.89489026)(135.04671494,357.73864026)(135.43343369,357.63317151)
\curveto(135.82015244,357.53160901)(136.20296494,357.48082776)(136.58187119,357.48082776)
\curveto(137.08968369,357.48082776)(137.48030869,357.56676526)(137.75374619,357.73864026)
\curveto(138.02718369,357.91442151)(138.16390244,358.16051526)(138.16390244,358.47692151)
\curveto(138.16390244,358.76989026)(138.06429306,358.99449964)(137.86507431,359.15074964)
\curveto(137.66976181,359.30699964)(137.23812119,359.45739026)(136.57015244,359.60192151)
\lineto(136.19515244,359.68981214)
\curveto(135.52718369,359.83043714)(135.04476181,360.04528089)(134.74788681,360.33434339)
\curveto(134.45101181,360.62731214)(134.30257431,361.02770276)(134.30257431,361.53551526)
\curveto(134.30257431,362.15270276)(134.52132431,362.62926526)(134.95882431,362.96520276)
\curveto(135.39632431,363.30114026)(136.01741806,363.46910901)(136.82210556,363.46910901)
\curveto(137.22054306,363.46910901)(137.59554306,363.43981214)(137.94710556,363.38121839)
\curveto(138.29866806,363.32262464)(138.62288681,363.23473401)(138.91976181,363.11754651)
\closepath
}
}
{
\newrgbcolor{curcolor}{0 0 0}
\pscustom[linestyle=none,fillstyle=solid,fillcolor=curcolor]
{
\newpath
\moveto(142.06038681,365.17418714)
\lineto(142.06038681,363.31090589)
\lineto(144.28108994,363.31090589)
\lineto(144.28108994,362.47301526)
\lineto(142.06038681,362.47301526)
\lineto(142.06038681,358.91051526)
\curveto(142.06038681,358.37535901)(142.13265244,358.03160901)(142.27718369,357.87926526)
\curveto(142.42562119,357.72692151)(142.72444931,357.65074964)(143.17366806,357.65074964)
\lineto(144.28108994,357.65074964)
\lineto(144.28108994,356.74840589)
\lineto(143.17366806,356.74840589)
\curveto(142.34163681,356.74840589)(141.76741806,356.90270276)(141.45101181,357.21129651)
\curveto(141.13460556,357.52379651)(140.97640244,358.09020276)(140.97640244,358.91051526)
\lineto(140.97640244,362.47301526)
\lineto(140.18538681,362.47301526)
\lineto(140.18538681,363.31090589)
\lineto(140.97640244,363.31090589)
\lineto(140.97640244,365.17418714)
\lineto(142.06038681,365.17418714)
\closepath
}
}
{
\newrgbcolor{curcolor}{0 0 0}
\pscustom[linestyle=none,fillstyle=solid,fillcolor=curcolor]
{
\newpath
\moveto(151.31819931,360.29918714)
\lineto(151.31819931,359.77184339)
\lineto(146.36116806,359.77184339)
\curveto(146.40804306,359.02965589)(146.63069931,358.46324964)(147.02913681,358.07262464)
\curveto(147.43148056,357.68590589)(147.99007431,357.49254651)(148.70491806,357.49254651)
\curveto(149.11898056,357.49254651)(149.51937119,357.54332776)(149.90608994,357.64489026)
\curveto(150.29671494,357.74645276)(150.68343369,357.89879651)(151.06624619,358.10192151)
\lineto(151.06624619,357.08239026)
\curveto(150.67952744,356.91832776)(150.28304306,356.79332776)(149.87679306,356.70739026)
\curveto(149.47054306,356.62145276)(149.05843369,356.57848401)(148.64046494,356.57848401)
\curveto(147.59358994,356.57848401)(146.76351181,356.88317151)(146.15023056,357.49254651)
\curveto(145.54085556,358.10192151)(145.23616806,358.92614026)(145.23616806,359.96520276)
\curveto(145.23616806,361.03942151)(145.52523056,361.89098401)(146.10335556,362.51989026)
\curveto(146.68538681,363.15270276)(147.46858994,363.46910901)(148.45296494,363.46910901)
\curveto(149.33577744,363.46910901)(150.03304306,363.18395276)(150.54476181,362.61364026)
\curveto(151.06038681,362.04723401)(151.31819931,361.27574964)(151.31819931,360.29918714)
\closepath
\moveto(150.24007431,360.61559339)
\curveto(150.23226181,361.20543714)(150.06624619,361.67614026)(149.74202744,362.02770276)
\curveto(149.42171494,362.37926526)(148.99593369,362.55504651)(148.46468369,362.55504651)
\curveto(147.86312119,362.55504651)(147.38069931,362.38512464)(147.01741806,362.04528089)
\curveto(146.65804306,361.70543714)(146.45101181,361.22692151)(146.39632431,360.60973401)
\lineto(150.24007431,360.61559339)
\closepath
}
}
{
\newrgbcolor{curcolor}{0 0 0}
\pscustom[linestyle=none,fillstyle=solid,fillcolor=curcolor]
{
\newpath
\moveto(158.19710556,362.05114026)
\curveto(158.46663681,362.53551526)(158.78890244,362.89293714)(159.16390244,363.12340589)
\curveto(159.53890244,363.35387464)(159.98030869,363.46910901)(160.48812119,363.46910901)
\curveto(161.17171494,363.46910901)(161.69905869,363.22887464)(162.07015244,362.74840589)
\curveto(162.44124619,362.27184339)(162.62679306,361.59215589)(162.62679306,360.70934339)
\lineto(162.62679306,356.74840589)
\lineto(161.54280869,356.74840589)
\lineto(161.54280869,360.67418714)
\curveto(161.54280869,361.30309339)(161.43148056,361.76989026)(161.20882431,362.07457776)
\curveto(160.98616806,362.37926526)(160.64632431,362.53160901)(160.18929306,362.53160901)
\curveto(159.63069931,362.53160901)(159.18929306,362.34606214)(158.86507431,361.97496839)
\curveto(158.54085556,361.60387464)(158.37874619,361.09801526)(158.37874619,360.45739026)
\lineto(158.37874619,356.74840589)
\lineto(157.29476181,356.74840589)
\lineto(157.29476181,360.67418714)
\curveto(157.29476181,361.30699964)(157.18343369,361.77379651)(156.96077744,362.07457776)
\curveto(156.73812119,362.37926526)(156.39437119,362.53160901)(155.92952744,362.53160901)
\curveto(155.37874619,362.53160901)(154.94124619,362.34410901)(154.61702744,361.96910901)
\curveto(154.29280869,361.59801526)(154.13069931,361.09410901)(154.13069931,360.45739026)
\lineto(154.13069931,356.74840589)
\lineto(153.04671494,356.74840589)
\lineto(153.04671494,363.31090589)
\lineto(154.13069931,363.31090589)
\lineto(154.13069931,362.29137464)
\curveto(154.37679306,362.69371839)(154.67171494,362.99059339)(155.01546494,363.18199964)
\curveto(155.35921494,363.37340589)(155.76741806,363.46910901)(156.24007431,363.46910901)
\curveto(156.71663681,363.46910901)(157.12093369,363.34801526)(157.45296494,363.10582776)
\curveto(157.78890244,362.86364026)(158.03694931,362.51207776)(158.19710556,362.05114026)
\closepath
}
}
{
\newrgbcolor{curcolor}{0 0 0}
\pscustom[linestyle=none,fillstyle=solid,fillcolor=curcolor]
{
\newpath
\moveto(167.76546494,360.04723401)
\curveto(166.89437119,360.04723401)(166.29085556,359.94762464)(165.95491806,359.74840589)
\curveto(165.61898056,359.54918714)(165.45101181,359.20934339)(165.45101181,358.72887464)
\curveto(165.45101181,358.34606214)(165.57601181,358.04137464)(165.82601181,357.81481214)
\curveto(166.07991806,357.59215589)(166.42366806,357.48082776)(166.85726181,357.48082776)
\curveto(167.45491806,357.48082776)(167.93343369,357.69176526)(168.29280869,358.11364026)
\curveto(168.65608994,358.53942151)(168.83773056,359.10387464)(168.83773056,359.80699964)
\lineto(168.83773056,360.04723401)
\lineto(167.76546494,360.04723401)
\closepath
\moveto(169.91585556,360.49254651)
\lineto(169.91585556,356.74840589)
\lineto(168.83773056,356.74840589)
\lineto(168.83773056,357.74449964)
\curveto(168.59163681,357.34606214)(168.28499619,357.05114026)(167.91780869,356.85973401)
\curveto(167.55062119,356.67223401)(167.10140244,356.57848401)(166.57015244,356.57848401)
\curveto(165.89827744,356.57848401)(165.36312119,356.76598401)(164.96468369,357.14098401)
\curveto(164.57015244,357.51989026)(164.37288681,358.02574964)(164.37288681,358.65856214)
\curveto(164.37288681,359.39684339)(164.61898056,359.95348401)(165.11116806,360.32848401)
\curveto(165.60726181,360.70348401)(166.34554306,360.89098401)(167.32601181,360.89098401)
\lineto(168.83773056,360.89098401)
\lineto(168.83773056,360.99645276)
\curveto(168.83773056,361.49254651)(168.67366806,361.87535901)(168.34554306,362.14489026)
\curveto(168.02132431,362.41832776)(167.56429306,362.55504651)(166.97444931,362.55504651)
\curveto(166.59944931,362.55504651)(166.23421494,362.51012464)(165.87874619,362.42028089)
\curveto(165.52327744,362.33043714)(165.18148056,362.19567151)(164.85335556,362.01598401)
\lineto(164.85335556,363.01207776)
\curveto(165.24788681,363.16442151)(165.63069931,363.27770276)(166.00179306,363.35192151)
\curveto(166.37288681,363.43004651)(166.73421494,363.46910901)(167.08577744,363.46910901)
\curveto(168.03499619,363.46910901)(168.74398056,363.22301526)(169.21273056,362.73082776)
\curveto(169.68148056,362.23864026)(169.91585556,361.49254651)(169.91585556,360.49254651)
\closepath
}
}
{
\newrgbcolor{curcolor}{0 0 0}
\pscustom[linestyle=none,fillstyle=solid,fillcolor=curcolor]
{
\newpath
\moveto(181.25374619,365.20934339)
\lineto(181.25374619,364.05504651)
\curveto(180.80452744,364.26989026)(180.38069931,364.43004651)(179.98226181,364.53551526)
\curveto(179.58382431,364.64098401)(179.19905869,364.69371839)(178.82796494,364.69371839)
\curveto(178.18343369,364.69371839)(177.68538681,364.56871839)(177.33382431,364.31871839)
\curveto(176.98616806,364.06871839)(176.81233994,363.71324964)(176.81233994,363.25231214)
\curveto(176.81233994,362.86559339)(176.92757431,362.57262464)(177.15804306,362.37340589)
\curveto(177.39241806,362.17809339)(177.83382431,362.01989026)(178.48226181,361.89879651)
\lineto(179.19710556,361.75231214)
\curveto(180.07991806,361.58434339)(180.73030869,361.28746839)(181.14827744,360.86168714)
\curveto(181.57015244,360.43981214)(181.78108994,359.87340589)(181.78108994,359.16246839)
\curveto(181.78108994,358.31481214)(181.49593369,357.67223401)(180.92562119,357.23473401)
\curveto(180.35921494,356.79723401)(179.52718369,356.57848401)(178.42952744,356.57848401)
\curveto(178.01546494,356.57848401)(177.57405869,356.62535901)(177.10530869,356.71910901)
\curveto(176.64046494,356.81285901)(176.15804306,356.95153089)(175.65804306,357.13512464)
\lineto(175.65804306,358.35387464)
\curveto(176.13851181,358.08434339)(176.60921494,357.88121839)(177.07015244,357.74449964)
\curveto(177.53108994,357.60778089)(177.98421494,357.53942151)(178.42952744,357.53942151)
\curveto(179.10530869,357.53942151)(179.62679306,357.67223401)(179.99398056,357.93785901)
\curveto(180.36116806,358.20348401)(180.54476181,358.58239026)(180.54476181,359.07457776)
\curveto(180.54476181,359.50426526)(180.41194931,359.84020276)(180.14632431,360.08239026)
\curveto(179.88460556,360.32457776)(179.45296494,360.50621839)(178.85140244,360.62731214)
\lineto(178.13069931,360.76793714)
\curveto(177.24788681,360.94371839)(176.60921494,361.21910901)(176.21468369,361.59410901)
\curveto(175.82015244,361.96910901)(175.62288681,362.49059339)(175.62288681,363.15856214)
\curveto(175.62288681,363.93199964)(175.89437119,364.54137464)(176.43733994,364.98668714)
\curveto(176.98421494,365.43199964)(177.73616806,365.65465589)(178.69319931,365.65465589)
\curveto(179.10335556,365.65465589)(179.52132431,365.61754651)(179.94710556,365.54332776)
\curveto(180.37288681,365.46910901)(180.80843369,365.35778089)(181.25374619,365.20934339)
\closepath
}
}
{
\newrgbcolor{curcolor}{0 0 0}
\pscustom[linestyle=none,fillstyle=solid,fillcolor=curcolor]
{
\newpath
\moveto(190.17757431,364.82262464)
\lineto(190.17757431,363.57457776)
\curveto(189.77913681,363.94567151)(189.35335556,364.22301526)(188.90023056,364.40660901)
\curveto(188.45101181,364.59020276)(187.97249619,364.68199964)(187.46468369,364.68199964)
\curveto(186.46468369,364.68199964)(185.69905869,364.37535901)(185.16780869,363.76207776)
\curveto(184.63655869,363.15270276)(184.37093369,362.26989026)(184.37093369,361.11364026)
\curveto(184.37093369,359.96129651)(184.63655869,359.07848401)(185.16780869,358.46520276)
\curveto(185.69905869,357.85582776)(186.46468369,357.55114026)(187.46468369,357.55114026)
\curveto(187.97249619,357.55114026)(188.45101181,357.64293714)(188.90023056,357.82653089)
\curveto(189.35335556,358.01012464)(189.77913681,358.28746839)(190.17757431,358.65856214)
\lineto(190.17757431,357.42223401)
\curveto(189.76351181,357.14098401)(189.32405869,356.93004651)(188.85921494,356.78942151)
\curveto(188.39827744,356.64879651)(187.90999619,356.57848401)(187.39437119,356.57848401)
\curveto(186.07015244,356.57848401)(185.02718369,356.98278089)(184.26546494,357.79137464)
\curveto(183.50374619,358.60387464)(183.12288681,359.71129651)(183.12288681,361.11364026)
\curveto(183.12288681,362.51989026)(183.50374619,363.62731214)(184.26546494,364.43590589)
\curveto(185.02718369,365.24840589)(186.07015244,365.65465589)(187.39437119,365.65465589)
\curveto(187.91780869,365.65465589)(188.40999619,365.58434339)(188.87093369,365.44371839)
\curveto(189.33577744,365.30699964)(189.77132431,365.09996839)(190.17757431,364.82262464)
\closepath
}
}
{
\newrgbcolor{curcolor}{0 0 0}
\pscustom[linestyle=none,fillstyle=solid,fillcolor=curcolor]
{
\newpath
\moveto(194.94124619,364.33043714)
\lineto(193.33577744,359.97692151)
\lineto(196.55257431,359.97692151)
\lineto(194.94124619,364.33043714)
\closepath
\moveto(194.27327744,365.49645276)
\lineto(195.61507431,365.49645276)
\lineto(198.94905869,356.74840589)
\lineto(197.71858994,356.74840589)
\lineto(196.92171494,358.99254651)
\lineto(192.97835556,358.99254651)
\lineto(192.18148056,356.74840589)
\lineto(190.93343369,356.74840589)
\lineto(194.27327744,365.49645276)
\closepath
}
}
{
\newrgbcolor{curcolor}{0 0 0}
\pscustom[linestyle=none,fillstyle=solid,fillcolor=curcolor]
{
\newpath
\moveto(201.40413681,364.52379651)
\lineto(201.40413681,357.72106214)
\lineto(202.83382431,357.72106214)
\curveto(204.04085556,357.72106214)(204.92366806,357.99449964)(205.48226181,358.54137464)
\curveto(206.04476181,359.08824964)(206.32601181,359.95153089)(206.32601181,361.13121839)
\curveto(206.32601181,362.30309339)(206.04476181,363.16051526)(205.48226181,363.70348401)
\curveto(204.92366806,364.25035901)(204.04085556,364.52379651)(202.83382431,364.52379651)
\lineto(201.40413681,364.52379651)
\closepath
\moveto(200.22054306,365.49645276)
\lineto(202.65218369,365.49645276)
\curveto(204.34749619,365.49645276)(205.59163681,365.14293714)(206.38460556,364.43590589)
\curveto(207.17757431,363.73278089)(207.57405869,362.63121839)(207.57405869,361.13121839)
\curveto(207.57405869,359.62340589)(207.17562119,358.51598401)(206.37874619,357.80895276)
\curveto(205.58187119,357.10192151)(204.33968369,356.74840589)(202.65218369,356.74840589)
\lineto(200.22054306,356.74840589)
\lineto(200.22054306,365.49645276)
\closepath
}
}
{
\newrgbcolor{curcolor}{0 0 0}
\pscustom[linestyle=none,fillstyle=solid,fillcolor=curcolor]
{
\newpath
\moveto(212.17952744,364.33043714)
\lineto(210.57405869,359.97692151)
\lineto(213.79085556,359.97692151)
\lineto(212.17952744,364.33043714)
\closepath
\moveto(211.51155869,365.49645276)
\lineto(212.85335556,365.49645276)
\lineto(216.18733994,356.74840589)
\lineto(214.95687119,356.74840589)
\lineto(214.15999619,358.99254651)
\lineto(210.21663681,358.99254651)
\lineto(209.41976181,356.74840589)
\lineto(208.17171494,356.74840589)
\lineto(211.51155869,365.49645276)
\closepath
}
}
{
\newrgbcolor{curcolor}{0 0 0}
\pscustom[linestyle=none,fillstyle=solid,fillcolor=curcolor]
{
\newpath
\moveto(225.96663681,363.05895276)
\lineto(225.96663681,362.05114026)
\curveto(225.66194931,362.21910901)(225.35530869,362.34410901)(225.04671494,362.42614026)
\curveto(224.74202744,362.51207776)(224.43343369,362.55504651)(224.12093369,362.55504651)
\curveto(223.42171494,362.55504651)(222.87874619,362.33239026)(222.49202744,361.88707776)
\curveto(222.10530869,361.44567151)(221.91194931,360.82457776)(221.91194931,360.02379651)
\curveto(221.91194931,359.22301526)(222.10530869,358.59996839)(222.49202744,358.15465589)
\curveto(222.87874619,357.71324964)(223.42171494,357.49254651)(224.12093369,357.49254651)
\curveto(224.43343369,357.49254651)(224.74202744,357.53356214)(225.04671494,357.61559339)
\curveto(225.35530869,357.70153089)(225.66194931,357.82848401)(225.96663681,357.99645276)
\lineto(225.96663681,357.00035901)
\curveto(225.66585556,356.85973401)(225.35335556,356.75426526)(225.02913681,356.68395276)
\curveto(224.70882431,356.61364026)(224.36702744,356.57848401)(224.00374619,356.57848401)
\curveto(223.01546494,356.57848401)(222.23030869,356.88903089)(221.64827744,357.51012464)
\curveto(221.06624619,358.13121839)(220.77523056,358.96910901)(220.77523056,360.02379651)
\curveto(220.77523056,361.09410901)(221.06819931,361.93590589)(221.65413681,362.54918714)
\curveto(222.24398056,363.16246839)(223.05062119,363.46910901)(224.07405869,363.46910901)
\curveto(224.40608994,363.46910901)(224.73030869,363.43395276)(225.04671494,363.36364026)
\curveto(225.36312119,363.29723401)(225.66976181,363.19567151)(225.96663681,363.05895276)
\closepath
}
}
{
\newrgbcolor{curcolor}{0 0 0}
\pscustom[linestyle=none,fillstyle=solid,fillcolor=curcolor]
{
\newpath
\moveto(230.83577744,360.04723401)
\curveto(229.96468369,360.04723401)(229.36116806,359.94762464)(229.02523056,359.74840589)
\curveto(228.68929306,359.54918714)(228.52132431,359.20934339)(228.52132431,358.72887464)
\curveto(228.52132431,358.34606214)(228.64632431,358.04137464)(228.89632431,357.81481214)
\curveto(229.15023056,357.59215589)(229.49398056,357.48082776)(229.92757431,357.48082776)
\curveto(230.52523056,357.48082776)(231.00374619,357.69176526)(231.36312119,358.11364026)
\curveto(231.72640244,358.53942151)(231.90804306,359.10387464)(231.90804306,359.80699964)
\lineto(231.90804306,360.04723401)
\lineto(230.83577744,360.04723401)
\closepath
\moveto(232.98616806,360.49254651)
\lineto(232.98616806,356.74840589)
\lineto(231.90804306,356.74840589)
\lineto(231.90804306,357.74449964)
\curveto(231.66194931,357.34606214)(231.35530869,357.05114026)(230.98812119,356.85973401)
\curveto(230.62093369,356.67223401)(230.17171494,356.57848401)(229.64046494,356.57848401)
\curveto(228.96858994,356.57848401)(228.43343369,356.76598401)(228.03499619,357.14098401)
\curveto(227.64046494,357.51989026)(227.44319931,358.02574964)(227.44319931,358.65856214)
\curveto(227.44319931,359.39684339)(227.68929306,359.95348401)(228.18148056,360.32848401)
\curveto(228.67757431,360.70348401)(229.41585556,360.89098401)(230.39632431,360.89098401)
\lineto(231.90804306,360.89098401)
\lineto(231.90804306,360.99645276)
\curveto(231.90804306,361.49254651)(231.74398056,361.87535901)(231.41585556,362.14489026)
\curveto(231.09163681,362.41832776)(230.63460556,362.55504651)(230.04476181,362.55504651)
\curveto(229.66976181,362.55504651)(229.30452744,362.51012464)(228.94905869,362.42028089)
\curveto(228.59358994,362.33043714)(228.25179306,362.19567151)(227.92366806,362.01598401)
\lineto(227.92366806,363.01207776)
\curveto(228.31819931,363.16442151)(228.70101181,363.27770276)(229.07210556,363.35192151)
\curveto(229.44319931,363.43004651)(229.80452744,363.46910901)(230.15608994,363.46910901)
\curveto(231.10530869,363.46910901)(231.81429306,363.22301526)(232.28304306,362.73082776)
\curveto(232.75179306,362.23864026)(232.98616806,361.49254651)(232.98616806,360.49254651)
\closepath
}
}
{
\newrgbcolor{curcolor}{0 0 0}
\pscustom[linestyle=none,fillstyle=solid,fillcolor=curcolor]
{
\newpath
\moveto(236.25569931,357.73278089)
\lineto(236.25569931,354.25231214)
\lineto(235.17171494,354.25231214)
\lineto(235.17171494,363.31090589)
\lineto(236.25569931,363.31090589)
\lineto(236.25569931,362.31481214)
\curveto(236.48226181,362.70543714)(236.76741806,362.99449964)(237.11116806,363.18199964)
\curveto(237.45882431,363.37340589)(237.87288681,363.46910901)(238.35335556,363.46910901)
\curveto(239.15023056,363.46910901)(239.79671494,363.15270276)(240.29280869,362.51989026)
\curveto(240.79280869,361.88707776)(241.04280869,361.05504651)(241.04280869,360.02379651)
\curveto(241.04280869,358.99254651)(240.79280869,358.16051526)(240.29280869,357.52770276)
\curveto(239.79671494,356.89489026)(239.15023056,356.57848401)(238.35335556,356.57848401)
\curveto(237.87288681,356.57848401)(237.45882431,356.67223401)(237.11116806,356.85973401)
\curveto(236.76741806,357.05114026)(236.48226181,357.34215589)(236.25569931,357.73278089)
\closepath
\moveto(239.92366806,360.02379651)
\curveto(239.92366806,360.81676526)(239.75960556,361.43785901)(239.43148056,361.88707776)
\curveto(239.10726181,362.34020276)(238.65999619,362.56676526)(238.08968369,362.56676526)
\curveto(237.51937119,362.56676526)(237.07015244,362.34020276)(236.74202744,361.88707776)
\curveto(236.41780869,361.43785901)(236.25569931,360.81676526)(236.25569931,360.02379651)
\curveto(236.25569931,359.23082776)(236.41780869,358.60778089)(236.74202744,358.15465589)
\curveto(237.07015244,357.70543714)(237.51937119,357.48082776)(238.08968369,357.48082776)
\curveto(238.65999619,357.48082776)(239.10726181,357.70543714)(239.43148056,358.15465589)
\curveto(239.75960556,358.60778089)(239.92366806,359.23082776)(239.92366806,360.02379651)
\closepath
}
}
{
\newrgbcolor{curcolor}{0 0 0}
\pscustom[linestyle=none,fillstyle=solid,fillcolor=curcolor]
{
\newpath
\moveto(243.89632431,365.17418714)
\lineto(243.89632431,363.31090589)
\lineto(246.11702744,363.31090589)
\lineto(246.11702744,362.47301526)
\lineto(243.89632431,362.47301526)
\lineto(243.89632431,358.91051526)
\curveto(243.89632431,358.37535901)(243.96858994,358.03160901)(244.11312119,357.87926526)
\curveto(244.26155869,357.72692151)(244.56038681,357.65074964)(245.00960556,357.65074964)
\lineto(246.11702744,357.65074964)
\lineto(246.11702744,356.74840589)
\lineto(245.00960556,356.74840589)
\curveto(244.17757431,356.74840589)(243.60335556,356.90270276)(243.28694931,357.21129651)
\curveto(242.97054306,357.52379651)(242.81233994,358.09020276)(242.81233994,358.91051526)
\lineto(242.81233994,362.47301526)
\lineto(242.02132431,362.47301526)
\lineto(242.02132431,363.31090589)
\lineto(242.81233994,363.31090589)
\lineto(242.81233994,365.17418714)
\lineto(243.89632431,365.17418714)
\closepath
}
}
{
\newrgbcolor{curcolor}{0 0 0}
\pscustom[linestyle=none,fillstyle=solid,fillcolor=curcolor]
{
\newpath
\moveto(247.42952744,359.33824964)
\lineto(247.42952744,363.31090589)
\lineto(248.50765244,363.31090589)
\lineto(248.50765244,359.37926526)
\curveto(248.50765244,358.75817151)(248.62874619,358.29137464)(248.87093369,357.97887464)
\curveto(249.11312119,357.67028089)(249.47640244,357.51598401)(249.96077744,357.51598401)
\curveto(250.54280869,357.51598401)(251.00179306,357.70153089)(251.33773056,358.07262464)
\curveto(251.67757431,358.44371839)(251.84749619,358.94957776)(251.84749619,359.59020276)
\lineto(251.84749619,363.31090589)
\lineto(252.92562119,363.31090589)
\lineto(252.92562119,356.74840589)
\lineto(251.84749619,356.74840589)
\lineto(251.84749619,357.75621839)
\curveto(251.58577744,357.35778089)(251.28108994,357.06090589)(250.93343369,356.86559339)
\curveto(250.58968369,356.67418714)(250.18929306,356.57848401)(249.73226181,356.57848401)
\curveto(248.97835556,356.57848401)(248.40608994,356.81285901)(248.01546494,357.28160901)
\curveto(247.62483994,357.75035901)(247.42952744,358.43590589)(247.42952744,359.33824964)
\closepath
}
}
{
\newrgbcolor{curcolor}{0 0 0}
\pscustom[linestyle=none,fillstyle=solid,fillcolor=curcolor]
{
\newpath
\moveto(258.96077744,362.30309339)
\curveto(258.83968369,362.37340589)(258.70687119,362.42418714)(258.56233994,362.45543714)
\curveto(258.42171494,362.49059339)(258.26546494,362.50817151)(258.09358994,362.50817151)
\curveto(257.48421494,362.50817151)(257.01546494,362.30895276)(256.68733994,361.91051526)
\curveto(256.36312119,361.51598401)(256.20101181,360.94762464)(256.20101181,360.20543714)
\lineto(256.20101181,356.74840589)
\lineto(255.11702744,356.74840589)
\lineto(255.11702744,363.31090589)
\lineto(256.20101181,363.31090589)
\lineto(256.20101181,362.29137464)
\curveto(256.42757431,362.68981214)(256.72249619,362.98473401)(257.08577744,363.17614026)
\curveto(257.44905869,363.37145276)(257.89046494,363.46910901)(258.40999619,363.46910901)
\curveto(258.48421494,363.46910901)(258.56624619,363.46324964)(258.65608994,363.45153089)
\curveto(258.74593369,363.44371839)(258.84554306,363.43004651)(258.95491806,363.41051526)
\lineto(258.96077744,362.30309339)
\closepath
}
}
{
\newrgbcolor{curcolor}{0 0 0}
\pscustom[linestyle=none,fillstyle=solid,fillcolor=curcolor]
{
\newpath
\moveto(263.08577744,360.04723401)
\curveto(262.21468369,360.04723401)(261.61116806,359.94762464)(261.27523056,359.74840589)
\curveto(260.93929306,359.54918714)(260.77132431,359.20934339)(260.77132431,358.72887464)
\curveto(260.77132431,358.34606214)(260.89632431,358.04137464)(261.14632431,357.81481214)
\curveto(261.40023056,357.59215589)(261.74398056,357.48082776)(262.17757431,357.48082776)
\curveto(262.77523056,357.48082776)(263.25374619,357.69176526)(263.61312119,358.11364026)
\curveto(263.97640244,358.53942151)(264.15804306,359.10387464)(264.15804306,359.80699964)
\lineto(264.15804306,360.04723401)
\lineto(263.08577744,360.04723401)
\closepath
\moveto(265.23616806,360.49254651)
\lineto(265.23616806,356.74840589)
\lineto(264.15804306,356.74840589)
\lineto(264.15804306,357.74449964)
\curveto(263.91194931,357.34606214)(263.60530869,357.05114026)(263.23812119,356.85973401)
\curveto(262.87093369,356.67223401)(262.42171494,356.57848401)(261.89046494,356.57848401)
\curveto(261.21858994,356.57848401)(260.68343369,356.76598401)(260.28499619,357.14098401)
\curveto(259.89046494,357.51989026)(259.69319931,358.02574964)(259.69319931,358.65856214)
\curveto(259.69319931,359.39684339)(259.93929306,359.95348401)(260.43148056,360.32848401)
\curveto(260.92757431,360.70348401)(261.66585556,360.89098401)(262.64632431,360.89098401)
\lineto(264.15804306,360.89098401)
\lineto(264.15804306,360.99645276)
\curveto(264.15804306,361.49254651)(263.99398056,361.87535901)(263.66585556,362.14489026)
\curveto(263.34163681,362.41832776)(262.88460556,362.55504651)(262.29476181,362.55504651)
\curveto(261.91976181,362.55504651)(261.55452744,362.51012464)(261.19905869,362.42028089)
\curveto(260.84358994,362.33043714)(260.50179306,362.19567151)(260.17366806,362.01598401)
\lineto(260.17366806,363.01207776)
\curveto(260.56819931,363.16442151)(260.95101181,363.27770276)(261.32210556,363.35192151)
\curveto(261.69319931,363.43004651)(262.05452744,363.46910901)(262.40608994,363.46910901)
\curveto(263.35530869,363.46910901)(264.06429306,363.22301526)(264.53304306,362.73082776)
\curveto(265.00179306,362.23864026)(265.23616806,361.49254651)(265.23616806,360.49254651)
\closepath
}
}
{
\newrgbcolor{curcolor}{0 0 0}
\pscustom[linestyle=none,fillstyle=solid,fillcolor=curcolor]
{
\newpath
\moveto(271.28304306,365.86559339)
\lineto(272.36116806,365.86559339)
\lineto(272.36116806,356.74840589)
\lineto(271.28304306,356.74840589)
\lineto(271.28304306,365.86559339)
\closepath
}
}
{
\newrgbcolor{curcolor}{0 0 0}
\pscustom[linestyle=none,fillstyle=solid,fillcolor=curcolor]
{
\newpath
\moveto(277.59358994,360.04723401)
\curveto(276.72249619,360.04723401)(276.11898056,359.94762464)(275.78304306,359.74840589)
\curveto(275.44710556,359.54918714)(275.27913681,359.20934339)(275.27913681,358.72887464)
\curveto(275.27913681,358.34606214)(275.40413681,358.04137464)(275.65413681,357.81481214)
\curveto(275.90804306,357.59215589)(276.25179306,357.48082776)(276.68538681,357.48082776)
\curveto(277.28304306,357.48082776)(277.76155869,357.69176526)(278.12093369,358.11364026)
\curveto(278.48421494,358.53942151)(278.66585556,359.10387464)(278.66585556,359.80699964)
\lineto(278.66585556,360.04723401)
\lineto(277.59358994,360.04723401)
\closepath
\moveto(279.74398056,360.49254651)
\lineto(279.74398056,356.74840589)
\lineto(278.66585556,356.74840589)
\lineto(278.66585556,357.74449964)
\curveto(278.41976181,357.34606214)(278.11312119,357.05114026)(277.74593369,356.85973401)
\curveto(277.37874619,356.67223401)(276.92952744,356.57848401)(276.39827744,356.57848401)
\curveto(275.72640244,356.57848401)(275.19124619,356.76598401)(274.79280869,357.14098401)
\curveto(274.39827744,357.51989026)(274.20101181,358.02574964)(274.20101181,358.65856214)
\curveto(274.20101181,359.39684339)(274.44710556,359.95348401)(274.93929306,360.32848401)
\curveto(275.43538681,360.70348401)(276.17366806,360.89098401)(277.15413681,360.89098401)
\lineto(278.66585556,360.89098401)
\lineto(278.66585556,360.99645276)
\curveto(278.66585556,361.49254651)(278.50179306,361.87535901)(278.17366806,362.14489026)
\curveto(277.84944931,362.41832776)(277.39241806,362.55504651)(276.80257431,362.55504651)
\curveto(276.42757431,362.55504651)(276.06233994,362.51012464)(275.70687119,362.42028089)
\curveto(275.35140244,362.33043714)(275.00960556,362.19567151)(274.68148056,362.01598401)
\lineto(274.68148056,363.01207776)
\curveto(275.07601181,363.16442151)(275.45882431,363.27770276)(275.82991806,363.35192151)
\curveto(276.20101181,363.43004651)(276.56233994,363.46910901)(276.91390244,363.46910901)
\curveto(277.86312119,363.46910901)(278.57210556,363.22301526)(279.04085556,362.73082776)
\curveto(279.50960556,362.23864026)(279.74398056,361.49254651)(279.74398056,360.49254651)
\closepath
}
}
{
\newrgbcolor{curcolor}{0 0 0}
\pscustom[linestyle=none,fillstyle=solid,fillcolor=curcolor]
{
\newpath
\moveto(286.15413681,363.11754651)
\lineto(286.15413681,362.09801526)
\curveto(285.84944931,362.25426526)(285.53304306,362.37145276)(285.20491806,362.44957776)
\curveto(284.87679306,362.52770276)(284.53694931,362.56676526)(284.18538681,362.56676526)
\curveto(283.65023056,362.56676526)(283.24788681,362.48473401)(282.97835556,362.32067151)
\curveto(282.71273056,362.15660901)(282.57991806,361.91051526)(282.57991806,361.58239026)
\curveto(282.57991806,361.33239026)(282.67562119,361.13512464)(282.86702744,360.99059339)
\curveto(283.05843369,360.84996839)(283.44319931,360.71520276)(284.02132431,360.58629651)
\lineto(284.39046494,360.50426526)
\curveto(285.15608994,360.34020276)(285.69905869,360.10778089)(286.01937119,359.80699964)
\curveto(286.34358994,359.51012464)(286.50569931,359.09410901)(286.50569931,358.55895276)
\curveto(286.50569931,357.94957776)(286.26351181,357.46715589)(285.77913681,357.11168714)
\curveto(285.29866806,356.75621839)(284.63655869,356.57848401)(283.79280869,356.57848401)
\curveto(283.44124619,356.57848401)(283.07405869,356.61364026)(282.69124619,356.68395276)
\curveto(282.31233994,356.75035901)(281.91194931,356.85192151)(281.49007431,356.98864026)
\lineto(281.49007431,358.10192151)
\curveto(281.88851181,357.89489026)(282.28108994,357.73864026)(282.66780869,357.63317151)
\curveto(283.05452744,357.53160901)(283.43733994,357.48082776)(283.81624619,357.48082776)
\curveto(284.32405869,357.48082776)(284.71468369,357.56676526)(284.98812119,357.73864026)
\curveto(285.26155869,357.91442151)(285.39827744,358.16051526)(285.39827744,358.47692151)
\curveto(285.39827744,358.76989026)(285.29866806,358.99449964)(285.09944931,359.15074964)
\curveto(284.90413681,359.30699964)(284.47249619,359.45739026)(283.80452744,359.60192151)
\lineto(283.42952744,359.68981214)
\curveto(282.76155869,359.83043714)(282.27913681,360.04528089)(281.98226181,360.33434339)
\curveto(281.68538681,360.62731214)(281.53694931,361.02770276)(281.53694931,361.53551526)
\curveto(281.53694931,362.15270276)(281.75569931,362.62926526)(282.19319931,362.96520276)
\curveto(282.63069931,363.30114026)(283.25179306,363.46910901)(284.05648056,363.46910901)
\curveto(284.45491806,363.46910901)(284.82991806,363.43981214)(285.18148056,363.38121839)
\curveto(285.53304306,363.32262464)(285.85726181,363.23473401)(286.15413681,363.11754651)
\closepath
}
}
{
\newrgbcolor{curcolor}{0 0 0}
\pscustom[linestyle=none,fillstyle=solid,fillcolor=curcolor]
{
\newpath
\moveto(297.15804306,362.05114026)
\curveto(297.42757431,362.53551526)(297.74983994,362.89293714)(298.12483994,363.12340589)
\curveto(298.49983994,363.35387464)(298.94124619,363.46910901)(299.44905869,363.46910901)
\curveto(300.13265244,363.46910901)(300.65999619,363.22887464)(301.03108994,362.74840589)
\curveto(301.40218369,362.27184339)(301.58773056,361.59215589)(301.58773056,360.70934339)
\lineto(301.58773056,356.74840589)
\lineto(300.50374619,356.74840589)
\lineto(300.50374619,360.67418714)
\curveto(300.50374619,361.30309339)(300.39241806,361.76989026)(300.16976181,362.07457776)
\curveto(299.94710556,362.37926526)(299.60726181,362.53160901)(299.15023056,362.53160901)
\curveto(298.59163681,362.53160901)(298.15023056,362.34606214)(297.82601181,361.97496839)
\curveto(297.50179306,361.60387464)(297.33968369,361.09801526)(297.33968369,360.45739026)
\lineto(297.33968369,356.74840589)
\lineto(296.25569931,356.74840589)
\lineto(296.25569931,360.67418714)
\curveto(296.25569931,361.30699964)(296.14437119,361.77379651)(295.92171494,362.07457776)
\curveto(295.69905869,362.37926526)(295.35530869,362.53160901)(294.89046494,362.53160901)
\curveto(294.33968369,362.53160901)(293.90218369,362.34410901)(293.57796494,361.96910901)
\curveto(293.25374619,361.59801526)(293.09163681,361.09410901)(293.09163681,360.45739026)
\lineto(293.09163681,356.74840589)
\lineto(292.00765244,356.74840589)
\lineto(292.00765244,363.31090589)
\lineto(293.09163681,363.31090589)
\lineto(293.09163681,362.29137464)
\curveto(293.33773056,362.69371839)(293.63265244,362.99059339)(293.97640244,363.18199964)
\curveto(294.32015244,363.37340589)(294.72835556,363.46910901)(295.20101181,363.46910901)
\curveto(295.67757431,363.46910901)(296.08187119,363.34801526)(296.41390244,363.10582776)
\curveto(296.74983994,362.86364026)(296.99788681,362.51207776)(297.15804306,362.05114026)
\closepath
}
}
{
\newrgbcolor{curcolor}{0 0 0}
\pscustom[linestyle=none,fillstyle=solid,fillcolor=curcolor]
{
\newpath
\moveto(309.35726181,360.29918714)
\lineto(309.35726181,359.77184339)
\lineto(304.40023056,359.77184339)
\curveto(304.44710556,359.02965589)(304.66976181,358.46324964)(305.06819931,358.07262464)
\curveto(305.47054306,357.68590589)(306.02913681,357.49254651)(306.74398056,357.49254651)
\curveto(307.15804306,357.49254651)(307.55843369,357.54332776)(307.94515244,357.64489026)
\curveto(308.33577744,357.74645276)(308.72249619,357.89879651)(309.10530869,358.10192151)
\lineto(309.10530869,357.08239026)
\curveto(308.71858994,356.91832776)(308.32210556,356.79332776)(307.91585556,356.70739026)
\curveto(307.50960556,356.62145276)(307.09749619,356.57848401)(306.67952744,356.57848401)
\curveto(305.63265244,356.57848401)(304.80257431,356.88317151)(304.18929306,357.49254651)
\curveto(303.57991806,358.10192151)(303.27523056,358.92614026)(303.27523056,359.96520276)
\curveto(303.27523056,361.03942151)(303.56429306,361.89098401)(304.14241806,362.51989026)
\curveto(304.72444931,363.15270276)(305.50765244,363.46910901)(306.49202744,363.46910901)
\curveto(307.37483994,363.46910901)(308.07210556,363.18395276)(308.58382431,362.61364026)
\curveto(309.09944931,362.04723401)(309.35726181,361.27574964)(309.35726181,360.29918714)
\closepath
\moveto(308.27913681,360.61559339)
\curveto(308.27132431,361.20543714)(308.10530869,361.67614026)(307.78108994,362.02770276)
\curveto(307.46077744,362.37926526)(307.03499619,362.55504651)(306.50374619,362.55504651)
\curveto(305.90218369,362.55504651)(305.41976181,362.38512464)(305.05648056,362.04528089)
\curveto(304.69710556,361.70543714)(304.49007431,361.22692151)(304.43538681,360.60973401)
\lineto(308.27913681,360.61559339)
\closepath
}
}
{
\newrgbcolor{curcolor}{0 0 0}
\pscustom[linestyle=none,fillstyle=solid,fillcolor=curcolor]
{
\newpath
\moveto(315.44515244,362.31481214)
\lineto(315.44515244,365.86559339)
\lineto(316.52327744,365.86559339)
\lineto(316.52327744,356.74840589)
\lineto(315.44515244,356.74840589)
\lineto(315.44515244,357.73278089)
\curveto(315.21858994,357.34215589)(314.93148056,357.05114026)(314.58382431,356.85973401)
\curveto(314.24007431,356.67223401)(313.82601181,356.57848401)(313.34163681,356.57848401)
\curveto(312.54866806,356.57848401)(311.90218369,356.89489026)(311.40218369,357.52770276)
\curveto(310.90608994,358.16051526)(310.65804306,358.99254651)(310.65804306,360.02379651)
\curveto(310.65804306,361.05504651)(310.90608994,361.88707776)(311.40218369,362.51989026)
\curveto(311.90218369,363.15270276)(312.54866806,363.46910901)(313.34163681,363.46910901)
\curveto(313.82601181,363.46910901)(314.24007431,363.37340589)(314.58382431,363.18199964)
\curveto(314.93148056,362.99449964)(315.21858994,362.70543714)(315.44515244,362.31481214)
\closepath
\moveto(311.77132431,360.02379651)
\curveto(311.77132431,359.23082776)(311.93343369,358.60778089)(312.25765244,358.15465589)
\curveto(312.58577744,357.70543714)(313.03499619,357.48082776)(313.60530869,357.48082776)
\curveto(314.17562119,357.48082776)(314.62483994,357.70543714)(314.95296494,358.15465589)
\curveto(315.28108994,358.60778089)(315.44515244,359.23082776)(315.44515244,360.02379651)
\curveto(315.44515244,360.81676526)(315.28108994,361.43785901)(314.95296494,361.88707776)
\curveto(314.62483994,362.34020276)(314.17562119,362.56676526)(313.60530869,362.56676526)
\curveto(313.03499619,362.56676526)(312.58577744,362.34020276)(312.25765244,361.88707776)
\curveto(311.93343369,361.43785901)(311.77132431,360.81676526)(311.77132431,360.02379651)
\closepath
}
}
{
\newrgbcolor{curcolor}{0 0 0}
\pscustom[linestyle=none,fillstyle=solid,fillcolor=curcolor]
{
\newpath
\moveto(318.74398056,363.31090589)
\lineto(319.82210556,363.31090589)
\lineto(319.82210556,356.74840589)
\lineto(318.74398056,356.74840589)
\lineto(318.74398056,363.31090589)
\closepath
\moveto(318.74398056,365.86559339)
\lineto(319.82210556,365.86559339)
\lineto(319.82210556,364.50035901)
\lineto(318.74398056,364.50035901)
\lineto(318.74398056,365.86559339)
\closepath
}
}
{
\newrgbcolor{curcolor}{0 0 0}
\pscustom[linestyle=none,fillstyle=solid,fillcolor=curcolor]
{
\newpath
\moveto(326.39046494,362.31481214)
\lineto(326.39046494,365.86559339)
\lineto(327.46858994,365.86559339)
\lineto(327.46858994,356.74840589)
\lineto(326.39046494,356.74840589)
\lineto(326.39046494,357.73278089)
\curveto(326.16390244,357.34215589)(325.87679306,357.05114026)(325.52913681,356.85973401)
\curveto(325.18538681,356.67223401)(324.77132431,356.57848401)(324.28694931,356.57848401)
\curveto(323.49398056,356.57848401)(322.84749619,356.89489026)(322.34749619,357.52770276)
\curveto(321.85140244,358.16051526)(321.60335556,358.99254651)(321.60335556,360.02379651)
\curveto(321.60335556,361.05504651)(321.85140244,361.88707776)(322.34749619,362.51989026)
\curveto(322.84749619,363.15270276)(323.49398056,363.46910901)(324.28694931,363.46910901)
\curveto(324.77132431,363.46910901)(325.18538681,363.37340589)(325.52913681,363.18199964)
\curveto(325.87679306,362.99449964)(326.16390244,362.70543714)(326.39046494,362.31481214)
\closepath
\moveto(322.71663681,360.02379651)
\curveto(322.71663681,359.23082776)(322.87874619,358.60778089)(323.20296494,358.15465589)
\curveto(323.53108994,357.70543714)(323.98030869,357.48082776)(324.55062119,357.48082776)
\curveto(325.12093369,357.48082776)(325.57015244,357.70543714)(325.89827744,358.15465589)
\curveto(326.22640244,358.60778089)(326.39046494,359.23082776)(326.39046494,360.02379651)
\curveto(326.39046494,360.81676526)(326.22640244,361.43785901)(325.89827744,361.88707776)
\curveto(325.57015244,362.34020276)(325.12093369,362.56676526)(324.55062119,362.56676526)
\curveto(323.98030869,362.56676526)(323.53108994,362.34020276)(323.20296494,361.88707776)
\curveto(322.87874619,361.43785901)(322.71663681,360.81676526)(322.71663681,360.02379651)
\closepath
}
}
{
\newrgbcolor{curcolor}{0 0 0}
\pscustom[linestyle=none,fillstyle=solid,fillcolor=curcolor]
{
\newpath
\moveto(332.67171494,360.04723401)
\curveto(331.80062119,360.04723401)(331.19710556,359.94762464)(330.86116806,359.74840589)
\curveto(330.52523056,359.54918714)(330.35726181,359.20934339)(330.35726181,358.72887464)
\curveto(330.35726181,358.34606214)(330.48226181,358.04137464)(330.73226181,357.81481214)
\curveto(330.98616806,357.59215589)(331.32991806,357.48082776)(331.76351181,357.48082776)
\curveto(332.36116806,357.48082776)(332.83968369,357.69176526)(333.19905869,358.11364026)
\curveto(333.56233994,358.53942151)(333.74398056,359.10387464)(333.74398056,359.80699964)
\lineto(333.74398056,360.04723401)
\lineto(332.67171494,360.04723401)
\closepath
\moveto(334.82210556,360.49254651)
\lineto(334.82210556,356.74840589)
\lineto(333.74398056,356.74840589)
\lineto(333.74398056,357.74449964)
\curveto(333.49788681,357.34606214)(333.19124619,357.05114026)(332.82405869,356.85973401)
\curveto(332.45687119,356.67223401)(332.00765244,356.57848401)(331.47640244,356.57848401)
\curveto(330.80452744,356.57848401)(330.26937119,356.76598401)(329.87093369,357.14098401)
\curveto(329.47640244,357.51989026)(329.27913681,358.02574964)(329.27913681,358.65856214)
\curveto(329.27913681,359.39684339)(329.52523056,359.95348401)(330.01741806,360.32848401)
\curveto(330.51351181,360.70348401)(331.25179306,360.89098401)(332.23226181,360.89098401)
\lineto(333.74398056,360.89098401)
\lineto(333.74398056,360.99645276)
\curveto(333.74398056,361.49254651)(333.57991806,361.87535901)(333.25179306,362.14489026)
\curveto(332.92757431,362.41832776)(332.47054306,362.55504651)(331.88069931,362.55504651)
\curveto(331.50569931,362.55504651)(331.14046494,362.51012464)(330.78499619,362.42028089)
\curveto(330.42952744,362.33043714)(330.08773056,362.19567151)(329.75960556,362.01598401)
\lineto(329.75960556,363.01207776)
\curveto(330.15413681,363.16442151)(330.53694931,363.27770276)(330.90804306,363.35192151)
\curveto(331.27913681,363.43004651)(331.64046494,363.46910901)(331.99202744,363.46910901)
\curveto(332.94124619,363.46910901)(333.65023056,363.22301526)(334.11898056,362.73082776)
\curveto(334.58773056,362.23864026)(334.82210556,361.49254651)(334.82210556,360.49254651)
\closepath
}
}
{
\newrgbcolor{curcolor}{0 0 0}
\pscustom[linestyle=none,fillstyle=solid,fillcolor=curcolor]
{
\newpath
\moveto(341.23226181,363.11754651)
\lineto(341.23226181,362.09801526)
\curveto(340.92757431,362.25426526)(340.61116806,362.37145276)(340.28304306,362.44957776)
\curveto(339.95491806,362.52770276)(339.61507431,362.56676526)(339.26351181,362.56676526)
\curveto(338.72835556,362.56676526)(338.32601181,362.48473401)(338.05648056,362.32067151)
\curveto(337.79085556,362.15660901)(337.65804306,361.91051526)(337.65804306,361.58239026)
\curveto(337.65804306,361.33239026)(337.75374619,361.13512464)(337.94515244,360.99059339)
\curveto(338.13655869,360.84996839)(338.52132431,360.71520276)(339.09944931,360.58629651)
\lineto(339.46858994,360.50426526)
\curveto(340.23421494,360.34020276)(340.77718369,360.10778089)(341.09749619,359.80699964)
\curveto(341.42171494,359.51012464)(341.58382431,359.09410901)(341.58382431,358.55895276)
\curveto(341.58382431,357.94957776)(341.34163681,357.46715589)(340.85726181,357.11168714)
\curveto(340.37679306,356.75621839)(339.71468369,356.57848401)(338.87093369,356.57848401)
\curveto(338.51937119,356.57848401)(338.15218369,356.61364026)(337.76937119,356.68395276)
\curveto(337.39046494,356.75035901)(336.99007431,356.85192151)(336.56819931,356.98864026)
\lineto(336.56819931,358.10192151)
\curveto(336.96663681,357.89489026)(337.35921494,357.73864026)(337.74593369,357.63317151)
\curveto(338.13265244,357.53160901)(338.51546494,357.48082776)(338.89437119,357.48082776)
\curveto(339.40218369,357.48082776)(339.79280869,357.56676526)(340.06624619,357.73864026)
\curveto(340.33968369,357.91442151)(340.47640244,358.16051526)(340.47640244,358.47692151)
\curveto(340.47640244,358.76989026)(340.37679306,358.99449964)(340.17757431,359.15074964)
\curveto(339.98226181,359.30699964)(339.55062119,359.45739026)(338.88265244,359.60192151)
\lineto(338.50765244,359.68981214)
\curveto(337.83968369,359.83043714)(337.35726181,360.04528089)(337.06038681,360.33434339)
\curveto(336.76351181,360.62731214)(336.61507431,361.02770276)(336.61507431,361.53551526)
\curveto(336.61507431,362.15270276)(336.83382431,362.62926526)(337.27132431,362.96520276)
\curveto(337.70882431,363.30114026)(338.32991806,363.46910901)(339.13460556,363.46910901)
\curveto(339.53304306,363.46910901)(339.90804306,363.43981214)(340.25960556,363.38121839)
\curveto(340.61116806,363.32262464)(340.93538681,363.23473401)(341.23226181,363.11754651)
\closepath
}
}
{
\newrgbcolor{curcolor}{0 0 0}
\pscustom[linestyle=none,fillstyle=solid,fillcolor=curcolor]
{
\newpath
\moveto(351.44515244,362.31481214)
\lineto(351.44515244,365.86559339)
\lineto(352.52327744,365.86559339)
\lineto(352.52327744,356.74840589)
\lineto(351.44515244,356.74840589)
\lineto(351.44515244,357.73278089)
\curveto(351.21858994,357.34215589)(350.93148056,357.05114026)(350.58382431,356.85973401)
\curveto(350.24007431,356.67223401)(349.82601181,356.57848401)(349.34163681,356.57848401)
\curveto(348.54866806,356.57848401)(347.90218369,356.89489026)(347.40218369,357.52770276)
\curveto(346.90608994,358.16051526)(346.65804306,358.99254651)(346.65804306,360.02379651)
\curveto(346.65804306,361.05504651)(346.90608994,361.88707776)(347.40218369,362.51989026)
\curveto(347.90218369,363.15270276)(348.54866806,363.46910901)(349.34163681,363.46910901)
\curveto(349.82601181,363.46910901)(350.24007431,363.37340589)(350.58382431,363.18199964)
\curveto(350.93148056,362.99449964)(351.21858994,362.70543714)(351.44515244,362.31481214)
\closepath
\moveto(347.77132431,360.02379651)
\curveto(347.77132431,359.23082776)(347.93343369,358.60778089)(348.25765244,358.15465589)
\curveto(348.58577744,357.70543714)(349.03499619,357.48082776)(349.60530869,357.48082776)
\curveto(350.17562119,357.48082776)(350.62483994,357.70543714)(350.95296494,358.15465589)
\curveto(351.28108994,358.60778089)(351.44515244,359.23082776)(351.44515244,360.02379651)
\curveto(351.44515244,360.81676526)(351.28108994,361.43785901)(350.95296494,361.88707776)
\curveto(350.62483994,362.34020276)(350.17562119,362.56676526)(349.60530869,362.56676526)
\curveto(349.03499619,362.56676526)(348.58577744,362.34020276)(348.25765244,361.88707776)
\curveto(347.93343369,361.43785901)(347.77132431,360.81676526)(347.77132431,360.02379651)
\closepath
}
}
{
\newrgbcolor{curcolor}{0 0 0}
\pscustom[linestyle=none,fillstyle=solid,fillcolor=curcolor]
{
\newpath
\moveto(360.35726181,360.29918714)
\lineto(360.35726181,359.77184339)
\lineto(355.40023056,359.77184339)
\curveto(355.44710556,359.02965589)(355.66976181,358.46324964)(356.06819931,358.07262464)
\curveto(356.47054306,357.68590589)(357.02913681,357.49254651)(357.74398056,357.49254651)
\curveto(358.15804306,357.49254651)(358.55843369,357.54332776)(358.94515244,357.64489026)
\curveto(359.33577744,357.74645276)(359.72249619,357.89879651)(360.10530869,358.10192151)
\lineto(360.10530869,357.08239026)
\curveto(359.71858994,356.91832776)(359.32210556,356.79332776)(358.91585556,356.70739026)
\curveto(358.50960556,356.62145276)(358.09749619,356.57848401)(357.67952744,356.57848401)
\curveto(356.63265244,356.57848401)(355.80257431,356.88317151)(355.18929306,357.49254651)
\curveto(354.57991806,358.10192151)(354.27523056,358.92614026)(354.27523056,359.96520276)
\curveto(354.27523056,361.03942151)(354.56429306,361.89098401)(355.14241806,362.51989026)
\curveto(355.72444931,363.15270276)(356.50765244,363.46910901)(357.49202744,363.46910901)
\curveto(358.37483994,363.46910901)(359.07210556,363.18395276)(359.58382431,362.61364026)
\curveto(360.09944931,362.04723401)(360.35726181,361.27574964)(360.35726181,360.29918714)
\closepath
\moveto(359.27913681,360.61559339)
\curveto(359.27132431,361.20543714)(359.10530869,361.67614026)(358.78108994,362.02770276)
\curveto(358.46077744,362.37926526)(358.03499619,362.55504651)(357.50374619,362.55504651)
\curveto(356.90218369,362.55504651)(356.41976181,362.38512464)(356.05648056,362.04528089)
\curveto(355.69710556,361.70543714)(355.49007431,361.22692151)(355.43538681,360.60973401)
\lineto(359.27913681,360.61559339)
\closepath
}
}
{
\newrgbcolor{curcolor}{0 0 0}
\pscustom[linestyle=none,fillstyle=solid,fillcolor=curcolor]
{
\newpath
\moveto(369.26937119,365.86559339)
\lineto(369.26937119,364.96910901)
\lineto(368.23812119,364.96910901)
\curveto(367.85140244,364.96910901)(367.58187119,364.89098401)(367.42952744,364.73473401)
\curveto(367.28108994,364.57848401)(367.20687119,364.29723401)(367.20687119,363.89098401)
\lineto(367.20687119,363.31090589)
\lineto(368.98226181,363.31090589)
\lineto(368.98226181,362.47301526)
\lineto(367.20687119,362.47301526)
\lineto(367.20687119,356.74840589)
\lineto(366.12288681,356.74840589)
\lineto(366.12288681,362.47301526)
\lineto(365.09163681,362.47301526)
\lineto(365.09163681,363.31090589)
\lineto(366.12288681,363.31090589)
\lineto(366.12288681,363.76793714)
\curveto(366.12288681,364.49840589)(366.29280869,365.02965589)(366.63265244,365.36168714)
\curveto(366.97249619,365.69762464)(367.51155869,365.86559339)(368.24983994,365.86559339)
\lineto(369.26937119,365.86559339)
\closepath
}
}
{
\newrgbcolor{curcolor}{0 0 0}
\pscustom[linestyle=none,fillstyle=solid,fillcolor=curcolor]
{
\newpath
\moveto(370.16585556,365.86559339)
\lineto(371.24398056,365.86559339)
\lineto(371.24398056,356.74840589)
\lineto(370.16585556,356.74840589)
\lineto(370.16585556,365.86559339)
\closepath
}
}
{
\newrgbcolor{curcolor}{0 0 0}
\pscustom[linestyle=none,fillstyle=solid,fillcolor=curcolor]
{
\newpath
\moveto(373.38265244,359.33824964)
\lineto(373.38265244,363.31090589)
\lineto(374.46077744,363.31090589)
\lineto(374.46077744,359.37926526)
\curveto(374.46077744,358.75817151)(374.58187119,358.29137464)(374.82405869,357.97887464)
\curveto(375.06624619,357.67028089)(375.42952744,357.51598401)(375.91390244,357.51598401)
\curveto(376.49593369,357.51598401)(376.95491806,357.70153089)(377.29085556,358.07262464)
\curveto(377.63069931,358.44371839)(377.80062119,358.94957776)(377.80062119,359.59020276)
\lineto(377.80062119,363.31090589)
\lineto(378.87874619,363.31090589)
\lineto(378.87874619,356.74840589)
\lineto(377.80062119,356.74840589)
\lineto(377.80062119,357.75621839)
\curveto(377.53890244,357.35778089)(377.23421494,357.06090589)(376.88655869,356.86559339)
\curveto(376.54280869,356.67418714)(376.14241806,356.57848401)(375.68538681,356.57848401)
\curveto(374.93148056,356.57848401)(374.35921494,356.81285901)(373.96858994,357.28160901)
\curveto(373.57796494,357.75035901)(373.38265244,358.43590589)(373.38265244,359.33824964)
\closepath
}
}
{
\newrgbcolor{curcolor}{0 0 0}
\pscustom[linestyle=none,fillstyle=solid,fillcolor=curcolor]
{
\newpath
\moveto(381.11116806,363.31090589)
\lineto(382.18929306,363.31090589)
\lineto(382.18929306,356.63121839)
\curveto(382.18929306,355.79528089)(382.02913681,355.18981214)(381.70882431,354.81481214)
\curveto(381.39241806,354.43981214)(380.88069931,354.25231214)(380.17366806,354.25231214)
\lineto(379.76351181,354.25231214)
\lineto(379.76351181,355.16637464)
\lineto(380.05062119,355.16637464)
\curveto(380.46077744,355.16637464)(380.74007431,355.26207776)(380.88851181,355.45348401)
\curveto(381.03694931,355.64098401)(381.11116806,356.03356214)(381.11116806,356.63121839)
\lineto(381.11116806,363.31090589)
\closepath
\moveto(381.11116806,365.86559339)
\lineto(382.18929306,365.86559339)
\lineto(382.18929306,364.50035901)
\lineto(381.11116806,364.50035901)
\lineto(381.11116806,365.86559339)
\closepath
}
}
{
\newrgbcolor{curcolor}{0 0 0}
\pscustom[linestyle=none,fillstyle=solid,fillcolor=curcolor]
{
\newpath
\moveto(386.98226181,362.55504651)
\curveto(386.40413681,362.55504651)(385.94710556,362.32848401)(385.61116806,361.87535901)
\curveto(385.27523056,361.42614026)(385.10726181,360.80895276)(385.10726181,360.02379651)
\curveto(385.10726181,359.23864026)(385.27327744,358.61949964)(385.60530869,358.16637464)
\curveto(385.94124619,357.71715589)(386.40023056,357.49254651)(386.98226181,357.49254651)
\curveto(387.55648056,357.49254651)(388.01155869,357.71910901)(388.34749619,358.17223401)
\curveto(388.68343369,358.62535901)(388.85140244,359.24254651)(388.85140244,360.02379651)
\curveto(388.85140244,360.80114026)(388.68343369,361.41637464)(388.34749619,361.86949964)
\curveto(388.01155869,362.32653089)(387.55648056,362.55504651)(386.98226181,362.55504651)
\closepath
\moveto(386.98226181,363.46910901)
\curveto(387.91976181,363.46910901)(388.65608994,363.16442151)(389.19124619,362.55504651)
\curveto(389.72640244,361.94567151)(389.99398056,361.10192151)(389.99398056,360.02379651)
\curveto(389.99398056,358.94957776)(389.72640244,358.10582776)(389.19124619,357.49254651)
\curveto(388.65608994,356.88317151)(387.91976181,356.57848401)(386.98226181,356.57848401)
\curveto(386.04085556,356.57848401)(385.30257431,356.88317151)(384.76741806,357.49254651)
\curveto(384.23616806,358.10582776)(383.97054306,358.94957776)(383.97054306,360.02379651)
\curveto(383.97054306,361.10192151)(384.23616806,361.94567151)(384.76741806,362.55504651)
\curveto(385.30257431,363.16442151)(386.04085556,363.46910901)(386.98226181,363.46910901)
\closepath
}
}
{
\newrgbcolor{curcolor}{0 0 0}
\pscustom[linestyle=none,fillstyle=solid,fillcolor=curcolor]
{
\newpath
\moveto(398.32601181,356.13903089)
\curveto(398.02132431,355.35778089)(397.72444931,354.84801526)(397.43538681,354.60973401)
\curveto(397.14632431,354.37145276)(396.75960556,354.25231214)(396.27523056,354.25231214)
\lineto(395.41390244,354.25231214)
\lineto(395.41390244,355.15465589)
\lineto(396.04671494,355.15465589)
\curveto(396.34358994,355.15465589)(396.57405869,355.22496839)(396.73812119,355.36559339)
\curveto(396.90218369,355.50621839)(397.08382431,355.83824964)(397.28304306,356.36168714)
\lineto(397.47640244,356.85387464)
\lineto(394.82210556,363.31090589)
\lineto(395.96468369,363.31090589)
\lineto(398.01546494,358.17809339)
\lineto(400.06624619,363.31090589)
\lineto(401.20882431,363.31090589)
\lineto(398.32601181,356.13903089)
\closepath
}
}
{
\newrgbcolor{curcolor}{0 0 0}
\pscustom[linestyle=none,fillstyle=solid,fillcolor=curcolor]
{
\newpath
\moveto(411.97249619,360.70934339)
\lineto(411.97249619,356.74840589)
\lineto(410.89437119,356.74840589)
\lineto(410.89437119,360.67418714)
\curveto(410.89437119,361.29528089)(410.77327744,361.76012464)(410.53108994,362.06871839)
\curveto(410.28890244,362.37731214)(409.92562119,362.53160901)(409.44124619,362.53160901)
\curveto(408.85921494,362.53160901)(408.40023056,362.34606214)(408.06429306,361.97496839)
\curveto(407.72835556,361.60387464)(407.56038681,361.09801526)(407.56038681,360.45739026)
\lineto(407.56038681,356.74840589)
\lineto(406.47640244,356.74840589)
\lineto(406.47640244,363.31090589)
\lineto(407.56038681,363.31090589)
\lineto(407.56038681,362.29137464)
\curveto(407.81819931,362.68590589)(408.12093369,362.98082776)(408.46858994,363.17614026)
\curveto(408.82015244,363.37145276)(409.22444931,363.46910901)(409.68148056,363.46910901)
\curveto(410.43538681,363.46910901)(411.00569931,363.23473401)(411.39241806,362.76598401)
\curveto(411.77913681,362.30114026)(411.97249619,361.61559339)(411.97249619,360.70934339)
\closepath
}
}
{
\newrgbcolor{curcolor}{0 0 0}
\pscustom[linestyle=none,fillstyle=solid,fillcolor=curcolor]
{
\newpath
\moveto(414.13460556,363.31090589)
\lineto(415.21273056,363.31090589)
\lineto(415.21273056,356.74840589)
\lineto(414.13460556,356.74840589)
\lineto(414.13460556,363.31090589)
\closepath
\moveto(414.13460556,365.86559339)
\lineto(415.21273056,365.86559339)
\lineto(415.21273056,364.50035901)
\lineto(414.13460556,364.50035901)
\lineto(414.13460556,365.86559339)
\closepath
}
}
{
\newrgbcolor{curcolor}{0 0 0}
\pscustom[linestyle=none,fillstyle=solid,fillcolor=curcolor]
{
\newpath
\moveto(416.68929306,363.31090589)
\lineto(417.83187119,363.31090589)
\lineto(419.88265244,357.80309339)
\lineto(421.93343369,363.31090589)
\lineto(423.07601181,363.31090589)
\lineto(420.61507431,356.74840589)
\lineto(419.15023056,356.74840589)
\lineto(416.68929306,363.31090589)
\closepath
}
}
{
\newrgbcolor{curcolor}{0 0 0}
\pscustom[linestyle=none,fillstyle=solid,fillcolor=curcolor]
{
\newpath
\moveto(430.17757431,360.29918714)
\lineto(430.17757431,359.77184339)
\lineto(425.22054306,359.77184339)
\curveto(425.26741806,359.02965589)(425.49007431,358.46324964)(425.88851181,358.07262464)
\curveto(426.29085556,357.68590589)(426.84944931,357.49254651)(427.56429306,357.49254651)
\curveto(427.97835556,357.49254651)(428.37874619,357.54332776)(428.76546494,357.64489026)
\curveto(429.15608994,357.74645276)(429.54280869,357.89879651)(429.92562119,358.10192151)
\lineto(429.92562119,357.08239026)
\curveto(429.53890244,356.91832776)(429.14241806,356.79332776)(428.73616806,356.70739026)
\curveto(428.32991806,356.62145276)(427.91780869,356.57848401)(427.49983994,356.57848401)
\curveto(426.45296494,356.57848401)(425.62288681,356.88317151)(425.00960556,357.49254651)
\curveto(424.40023056,358.10192151)(424.09554306,358.92614026)(424.09554306,359.96520276)
\curveto(424.09554306,361.03942151)(424.38460556,361.89098401)(424.96273056,362.51989026)
\curveto(425.54476181,363.15270276)(426.32796494,363.46910901)(427.31233994,363.46910901)
\curveto(428.19515244,363.46910901)(428.89241806,363.18395276)(429.40413681,362.61364026)
\curveto(429.91976181,362.04723401)(430.17757431,361.27574964)(430.17757431,360.29918714)
\closepath
\moveto(429.09944931,360.61559339)
\curveto(429.09163681,361.20543714)(428.92562119,361.67614026)(428.60140244,362.02770276)
\curveto(428.28108994,362.37926526)(427.85530869,362.55504651)(427.32405869,362.55504651)
\curveto(426.72249619,362.55504651)(426.24007431,362.38512464)(425.87679306,362.04528089)
\curveto(425.51741806,361.70543714)(425.31038681,361.22692151)(425.25569931,360.60973401)
\lineto(429.09944931,360.61559339)
\closepath
}
}
{
\newrgbcolor{curcolor}{0 0 0}
\pscustom[linestyle=none,fillstyle=solid,fillcolor=curcolor]
{
\newpath
\moveto(431.94710556,365.86559339)
\lineto(433.02523056,365.86559339)
\lineto(433.02523056,356.74840589)
\lineto(431.94710556,356.74840589)
\lineto(431.94710556,365.86559339)
\closepath
}
}
{
\newrgbcolor{curcolor}{0 0 0}
\pscustom[linestyle=none,fillstyle=solid,fillcolor=curcolor]
{
\newpath
\moveto(435.55062119,358.23668714)
\lineto(436.78694931,358.23668714)
\lineto(436.78694931,357.22887464)
\lineto(435.82601181,355.35387464)
\lineto(435.07015244,355.35387464)
\lineto(435.55062119,357.22887464)
\lineto(435.55062119,358.23668714)
\closepath
}
}
{
\newrgbcolor{curcolor}{0 0 0}
\pscustom[linestyle=none,fillstyle=solid,fillcolor=curcolor]
{
\newpath
\moveto(175.89241806,341.13903089)
\curveto(175.58773056,340.35778089)(175.29085556,339.84801526)(175.00179306,339.60973401)
\curveto(174.71273056,339.37145276)(174.32601181,339.25231214)(173.84163681,339.25231214)
\lineto(172.98030869,339.25231214)
\lineto(172.98030869,340.15465589)
\lineto(173.61312119,340.15465589)
\curveto(173.90999619,340.15465589)(174.14046494,340.22496839)(174.30452744,340.36559339)
\curveto(174.46858994,340.50621839)(174.65023056,340.83824964)(174.84944931,341.36168714)
\lineto(175.04280869,341.85387464)
\lineto(172.38851181,348.31090589)
\lineto(173.53108994,348.31090589)
\lineto(175.58187119,343.17809339)
\lineto(177.63265244,348.31090589)
\lineto(178.77523056,348.31090589)
\lineto(175.89241806,341.13903089)
\closepath
}
}
{
\newrgbcolor{curcolor}{0 0 0}
\pscustom[linestyle=none,fillstyle=solid,fillcolor=curcolor]
{
\newpath
\moveto(189.19319931,347.05114026)
\curveto(189.46273056,347.53551526)(189.78499619,347.89293714)(190.15999619,348.12340589)
\curveto(190.53499619,348.35387464)(190.97640244,348.46910901)(191.48421494,348.46910901)
\curveto(192.16780869,348.46910901)(192.69515244,348.22887464)(193.06624619,347.74840589)
\curveto(193.43733994,347.27184339)(193.62288681,346.59215589)(193.62288681,345.70934339)
\lineto(193.62288681,341.74840589)
\lineto(192.53890244,341.74840589)
\lineto(192.53890244,345.67418714)
\curveto(192.53890244,346.30309339)(192.42757431,346.76989026)(192.20491806,347.07457776)
\curveto(191.98226181,347.37926526)(191.64241806,347.53160901)(191.18538681,347.53160901)
\curveto(190.62679306,347.53160901)(190.18538681,347.34606214)(189.86116806,346.97496839)
\curveto(189.53694931,346.60387464)(189.37483994,346.09801526)(189.37483994,345.45739026)
\lineto(189.37483994,341.74840589)
\lineto(188.29085556,341.74840589)
\lineto(188.29085556,345.67418714)
\curveto(188.29085556,346.30699964)(188.17952744,346.77379651)(187.95687119,347.07457776)
\curveto(187.73421494,347.37926526)(187.39046494,347.53160901)(186.92562119,347.53160901)
\curveto(186.37483994,347.53160901)(185.93733994,347.34410901)(185.61312119,346.96910901)
\curveto(185.28890244,346.59801526)(185.12679306,346.09410901)(185.12679306,345.45739026)
\lineto(185.12679306,341.74840589)
\lineto(184.04280869,341.74840589)
\lineto(184.04280869,348.31090589)
\lineto(185.12679306,348.31090589)
\lineto(185.12679306,347.29137464)
\curveto(185.37288681,347.69371839)(185.66780869,347.99059339)(186.01155869,348.18199964)
\curveto(186.35530869,348.37340589)(186.76351181,348.46910901)(187.23616806,348.46910901)
\curveto(187.71273056,348.46910901)(188.11702744,348.34801526)(188.44905869,348.10582776)
\curveto(188.78499619,347.86364026)(189.03304306,347.51207776)(189.19319931,347.05114026)
\closepath
}
}
{
\newrgbcolor{curcolor}{0 0 0}
\pscustom[linestyle=none,fillstyle=solid,fillcolor=curcolor]
{
\newpath
\moveto(198.76155869,345.04723401)
\curveto(197.89046494,345.04723401)(197.28694931,344.94762464)(196.95101181,344.74840589)
\curveto(196.61507431,344.54918714)(196.44710556,344.20934339)(196.44710556,343.72887464)
\curveto(196.44710556,343.34606214)(196.57210556,343.04137464)(196.82210556,342.81481214)
\curveto(197.07601181,342.59215589)(197.41976181,342.48082776)(197.85335556,342.48082776)
\curveto(198.45101181,342.48082776)(198.92952744,342.69176526)(199.28890244,343.11364026)
\curveto(199.65218369,343.53942151)(199.83382431,344.10387464)(199.83382431,344.80699964)
\lineto(199.83382431,345.04723401)
\lineto(198.76155869,345.04723401)
\closepath
\moveto(200.91194931,345.49254651)
\lineto(200.91194931,341.74840589)
\lineto(199.83382431,341.74840589)
\lineto(199.83382431,342.74449964)
\curveto(199.58773056,342.34606214)(199.28108994,342.05114026)(198.91390244,341.85973401)
\curveto(198.54671494,341.67223401)(198.09749619,341.57848401)(197.56624619,341.57848401)
\curveto(196.89437119,341.57848401)(196.35921494,341.76598401)(195.96077744,342.14098401)
\curveto(195.56624619,342.51989026)(195.36898056,343.02574964)(195.36898056,343.65856214)
\curveto(195.36898056,344.39684339)(195.61507431,344.95348401)(196.10726181,345.32848401)
\curveto(196.60335556,345.70348401)(197.34163681,345.89098401)(198.32210556,345.89098401)
\lineto(199.83382431,345.89098401)
\lineto(199.83382431,345.99645276)
\curveto(199.83382431,346.49254651)(199.66976181,346.87535901)(199.34163681,347.14489026)
\curveto(199.01741806,347.41832776)(198.56038681,347.55504651)(197.97054306,347.55504651)
\curveto(197.59554306,347.55504651)(197.23030869,347.51012464)(196.87483994,347.42028089)
\curveto(196.51937119,347.33043714)(196.17757431,347.19567151)(195.84944931,347.01598401)
\lineto(195.84944931,348.01207776)
\curveto(196.24398056,348.16442151)(196.62679306,348.27770276)(196.99788681,348.35192151)
\curveto(197.36898056,348.43004651)(197.73030869,348.46910901)(198.08187119,348.46910901)
\curveto(199.03108994,348.46910901)(199.74007431,348.22301526)(200.20882431,347.73082776)
\curveto(200.67757431,347.23864026)(200.91194931,346.49254651)(200.91194931,345.49254651)
\closepath
}
}
{
\newrgbcolor{curcolor}{0 0 0}
\pscustom[linestyle=none,fillstyle=solid,fillcolor=curcolor]
{
\newpath
\moveto(208.59358994,345.70934339)
\lineto(208.59358994,341.74840589)
\lineto(207.51546494,341.74840589)
\lineto(207.51546494,345.67418714)
\curveto(207.51546494,346.29528089)(207.39437119,346.76012464)(207.15218369,347.06871839)
\curveto(206.90999619,347.37731214)(206.54671494,347.53160901)(206.06233994,347.53160901)
\curveto(205.48030869,347.53160901)(205.02132431,347.34606214)(204.68538681,346.97496839)
\curveto(204.34944931,346.60387464)(204.18148056,346.09801526)(204.18148056,345.45739026)
\lineto(204.18148056,341.74840589)
\lineto(203.09749619,341.74840589)
\lineto(203.09749619,348.31090589)
\lineto(204.18148056,348.31090589)
\lineto(204.18148056,347.29137464)
\curveto(204.43929306,347.68590589)(204.74202744,347.98082776)(205.08968369,348.17614026)
\curveto(205.44124619,348.37145276)(205.84554306,348.46910901)(206.30257431,348.46910901)
\curveto(207.05648056,348.46910901)(207.62679306,348.23473401)(208.01351181,347.76598401)
\curveto(208.40023056,347.30114026)(208.59358994,346.61559339)(208.59358994,345.70934339)
\closepath
}
}
{
\newrgbcolor{curcolor}{0 0 0}
\pscustom[linestyle=none,fillstyle=solid,fillcolor=curcolor]
{
\newpath
\moveto(215.07405869,347.31481214)
\lineto(215.07405869,350.86559339)
\lineto(216.15218369,350.86559339)
\lineto(216.15218369,341.74840589)
\lineto(215.07405869,341.74840589)
\lineto(215.07405869,342.73278089)
\curveto(214.84749619,342.34215589)(214.56038681,342.05114026)(214.21273056,341.85973401)
\curveto(213.86898056,341.67223401)(213.45491806,341.57848401)(212.97054306,341.57848401)
\curveto(212.17757431,341.57848401)(211.53108994,341.89489026)(211.03108994,342.52770276)
\curveto(210.53499619,343.16051526)(210.28694931,343.99254651)(210.28694931,345.02379651)
\curveto(210.28694931,346.05504651)(210.53499619,346.88707776)(211.03108994,347.51989026)
\curveto(211.53108994,348.15270276)(212.17757431,348.46910901)(212.97054306,348.46910901)
\curveto(213.45491806,348.46910901)(213.86898056,348.37340589)(214.21273056,348.18199964)
\curveto(214.56038681,347.99449964)(214.84749619,347.70543714)(215.07405869,347.31481214)
\closepath
\moveto(211.40023056,345.02379651)
\curveto(211.40023056,344.23082776)(211.56233994,343.60778089)(211.88655869,343.15465589)
\curveto(212.21468369,342.70543714)(212.66390244,342.48082776)(213.23421494,342.48082776)
\curveto(213.80452744,342.48082776)(214.25374619,342.70543714)(214.58187119,343.15465589)
\curveto(214.90999619,343.60778089)(215.07405869,344.23082776)(215.07405869,345.02379651)
\curveto(215.07405869,345.81676526)(214.90999619,346.43785901)(214.58187119,346.88707776)
\curveto(214.25374619,347.34020276)(213.80452744,347.56676526)(213.23421494,347.56676526)
\curveto(212.66390244,347.56676526)(212.21468369,347.34020276)(211.88655869,346.88707776)
\curveto(211.56233994,346.43785901)(211.40023056,345.81676526)(211.40023056,345.02379651)
\closepath
}
}
{
\newrgbcolor{curcolor}{0 0 0}
\pscustom[linestyle=none,fillstyle=solid,fillcolor=curcolor]
{
\newpath
\moveto(221.35530869,345.04723401)
\curveto(220.48421494,345.04723401)(219.88069931,344.94762464)(219.54476181,344.74840589)
\curveto(219.20882431,344.54918714)(219.04085556,344.20934339)(219.04085556,343.72887464)
\curveto(219.04085556,343.34606214)(219.16585556,343.04137464)(219.41585556,342.81481214)
\curveto(219.66976181,342.59215589)(220.01351181,342.48082776)(220.44710556,342.48082776)
\curveto(221.04476181,342.48082776)(221.52327744,342.69176526)(221.88265244,343.11364026)
\curveto(222.24593369,343.53942151)(222.42757431,344.10387464)(222.42757431,344.80699964)
\lineto(222.42757431,345.04723401)
\lineto(221.35530869,345.04723401)
\closepath
\moveto(223.50569931,345.49254651)
\lineto(223.50569931,341.74840589)
\lineto(222.42757431,341.74840589)
\lineto(222.42757431,342.74449964)
\curveto(222.18148056,342.34606214)(221.87483994,342.05114026)(221.50765244,341.85973401)
\curveto(221.14046494,341.67223401)(220.69124619,341.57848401)(220.15999619,341.57848401)
\curveto(219.48812119,341.57848401)(218.95296494,341.76598401)(218.55452744,342.14098401)
\curveto(218.15999619,342.51989026)(217.96273056,343.02574964)(217.96273056,343.65856214)
\curveto(217.96273056,344.39684339)(218.20882431,344.95348401)(218.70101181,345.32848401)
\curveto(219.19710556,345.70348401)(219.93538681,345.89098401)(220.91585556,345.89098401)
\lineto(222.42757431,345.89098401)
\lineto(222.42757431,345.99645276)
\curveto(222.42757431,346.49254651)(222.26351181,346.87535901)(221.93538681,347.14489026)
\curveto(221.61116806,347.41832776)(221.15413681,347.55504651)(220.56429306,347.55504651)
\curveto(220.18929306,347.55504651)(219.82405869,347.51012464)(219.46858994,347.42028089)
\curveto(219.11312119,347.33043714)(218.77132431,347.19567151)(218.44319931,347.01598401)
\lineto(218.44319931,348.01207776)
\curveto(218.83773056,348.16442151)(219.22054306,348.27770276)(219.59163681,348.35192151)
\curveto(219.96273056,348.43004651)(220.32405869,348.46910901)(220.67562119,348.46910901)
\curveto(221.62483994,348.46910901)(222.33382431,348.22301526)(222.80257431,347.73082776)
\curveto(223.27132431,347.23864026)(223.50569931,346.49254651)(223.50569931,345.49254651)
\closepath
}
}
{
\newrgbcolor{curcolor}{0 0 0}
\pscustom[linestyle=none,fillstyle=solid,fillcolor=curcolor]
{
\newpath
\moveto(229.55257431,350.86559339)
\lineto(230.63069931,350.86559339)
\lineto(230.63069931,341.74840589)
\lineto(229.55257431,341.74840589)
\lineto(229.55257431,350.86559339)
\closepath
}
}
{
\newrgbcolor{curcolor}{0 0 0}
\pscustom[linestyle=none,fillstyle=solid,fillcolor=curcolor]
{
\newpath
\moveto(235.42366806,347.55504651)
\curveto(234.84554306,347.55504651)(234.38851181,347.32848401)(234.05257431,346.87535901)
\curveto(233.71663681,346.42614026)(233.54866806,345.80895276)(233.54866806,345.02379651)
\curveto(233.54866806,344.23864026)(233.71468369,343.61949964)(234.04671494,343.16637464)
\curveto(234.38265244,342.71715589)(234.84163681,342.49254651)(235.42366806,342.49254651)
\curveto(235.99788681,342.49254651)(236.45296494,342.71910901)(236.78890244,343.17223401)
\curveto(237.12483994,343.62535901)(237.29280869,344.24254651)(237.29280869,345.02379651)
\curveto(237.29280869,345.80114026)(237.12483994,346.41637464)(236.78890244,346.86949964)
\curveto(236.45296494,347.32653089)(235.99788681,347.55504651)(235.42366806,347.55504651)
\closepath
\moveto(235.42366806,348.46910901)
\curveto(236.36116806,348.46910901)(237.09749619,348.16442151)(237.63265244,347.55504651)
\curveto(238.16780869,346.94567151)(238.43538681,346.10192151)(238.43538681,345.02379651)
\curveto(238.43538681,343.94957776)(238.16780869,343.10582776)(237.63265244,342.49254651)
\curveto(237.09749619,341.88317151)(236.36116806,341.57848401)(235.42366806,341.57848401)
\curveto(234.48226181,341.57848401)(233.74398056,341.88317151)(233.20882431,342.49254651)
\curveto(232.67757431,343.10582776)(232.41194931,343.94957776)(232.41194931,345.02379651)
\curveto(232.41194931,346.10192151)(232.67757431,346.94567151)(233.20882431,347.55504651)
\curveto(233.74398056,348.16442151)(234.48226181,348.46910901)(235.42366806,348.46910901)
\closepath
}
}
{
\newrgbcolor{curcolor}{0 0 0}
\pscustom[linestyle=none,fillstyle=solid,fillcolor=curcolor]
{
\newpath
\moveto(244.40023056,348.11754651)
\lineto(244.40023056,347.09801526)
\curveto(244.09554306,347.25426526)(243.77913681,347.37145276)(243.45101181,347.44957776)
\curveto(243.12288681,347.52770276)(242.78304306,347.56676526)(242.43148056,347.56676526)
\curveto(241.89632431,347.56676526)(241.49398056,347.48473401)(241.22444931,347.32067151)
\curveto(240.95882431,347.15660901)(240.82601181,346.91051526)(240.82601181,346.58239026)
\curveto(240.82601181,346.33239026)(240.92171494,346.13512464)(241.11312119,345.99059339)
\curveto(241.30452744,345.84996839)(241.68929306,345.71520276)(242.26741806,345.58629651)
\lineto(242.63655869,345.50426526)
\curveto(243.40218369,345.34020276)(243.94515244,345.10778089)(244.26546494,344.80699964)
\curveto(244.58968369,344.51012464)(244.75179306,344.09410901)(244.75179306,343.55895276)
\curveto(244.75179306,342.94957776)(244.50960556,342.46715589)(244.02523056,342.11168714)
\curveto(243.54476181,341.75621839)(242.88265244,341.57848401)(242.03890244,341.57848401)
\curveto(241.68733994,341.57848401)(241.32015244,341.61364026)(240.93733994,341.68395276)
\curveto(240.55843369,341.75035901)(240.15804306,341.85192151)(239.73616806,341.98864026)
\lineto(239.73616806,343.10192151)
\curveto(240.13460556,342.89489026)(240.52718369,342.73864026)(240.91390244,342.63317151)
\curveto(241.30062119,342.53160901)(241.68343369,342.48082776)(242.06233994,342.48082776)
\curveto(242.57015244,342.48082776)(242.96077744,342.56676526)(243.23421494,342.73864026)
\curveto(243.50765244,342.91442151)(243.64437119,343.16051526)(243.64437119,343.47692151)
\curveto(243.64437119,343.76989026)(243.54476181,343.99449964)(243.34554306,344.15074964)
\curveto(243.15023056,344.30699964)(242.71858994,344.45739026)(242.05062119,344.60192151)
\lineto(241.67562119,344.68981214)
\curveto(241.00765244,344.83043714)(240.52523056,345.04528089)(240.22835556,345.33434339)
\curveto(239.93148056,345.62731214)(239.78304306,346.02770276)(239.78304306,346.53551526)
\curveto(239.78304306,347.15270276)(240.00179306,347.62926526)(240.43929306,347.96520276)
\curveto(240.87679306,348.30114026)(241.49788681,348.46910901)(242.30257431,348.46910901)
\curveto(242.70101181,348.46910901)(243.07601181,348.43981214)(243.42757431,348.38121839)
\curveto(243.77913681,348.32262464)(244.10335556,348.23473401)(244.40023056,348.11754651)
\closepath
}
}
{
\newrgbcolor{curcolor}{0 0 0}
\pscustom[linestyle=none,fillstyle=solid,fillcolor=curcolor]
{
\newpath
\moveto(251.33773056,342.73278089)
\lineto(251.33773056,339.25231214)
\lineto(250.25374619,339.25231214)
\lineto(250.25374619,348.31090589)
\lineto(251.33773056,348.31090589)
\lineto(251.33773056,347.31481214)
\curveto(251.56429306,347.70543714)(251.84944931,347.99449964)(252.19319931,348.18199964)
\curveto(252.54085556,348.37340589)(252.95491806,348.46910901)(253.43538681,348.46910901)
\curveto(254.23226181,348.46910901)(254.87874619,348.15270276)(255.37483994,347.51989026)
\curveto(255.87483994,346.88707776)(256.12483994,346.05504651)(256.12483994,345.02379651)
\curveto(256.12483994,343.99254651)(255.87483994,343.16051526)(255.37483994,342.52770276)
\curveto(254.87874619,341.89489026)(254.23226181,341.57848401)(253.43538681,341.57848401)
\curveto(252.95491806,341.57848401)(252.54085556,341.67223401)(252.19319931,341.85973401)
\curveto(251.84944931,342.05114026)(251.56429306,342.34215589)(251.33773056,342.73278089)
\closepath
\moveto(255.00569931,345.02379651)
\curveto(255.00569931,345.81676526)(254.84163681,346.43785901)(254.51351181,346.88707776)
\curveto(254.18929306,347.34020276)(253.74202744,347.56676526)(253.17171494,347.56676526)
\curveto(252.60140244,347.56676526)(252.15218369,347.34020276)(251.82405869,346.88707776)
\curveto(251.49983994,346.43785901)(251.33773056,345.81676526)(251.33773056,345.02379651)
\curveto(251.33773056,344.23082776)(251.49983994,343.60778089)(251.82405869,343.15465589)
\curveto(252.15218369,342.70543714)(252.60140244,342.48082776)(253.17171494,342.48082776)
\curveto(253.74202744,342.48082776)(254.18929306,342.70543714)(254.51351181,343.15465589)
\curveto(254.84163681,343.60778089)(255.00569931,344.23082776)(255.00569931,345.02379651)
\closepath
}
}
{
\newrgbcolor{curcolor}{0 0 0}
\pscustom[linestyle=none,fillstyle=solid,fillcolor=curcolor]
{
\newpath
\moveto(260.89437119,345.04723401)
\curveto(260.02327744,345.04723401)(259.41976181,344.94762464)(259.08382431,344.74840589)
\curveto(258.74788681,344.54918714)(258.57991806,344.20934339)(258.57991806,343.72887464)
\curveto(258.57991806,343.34606214)(258.70491806,343.04137464)(258.95491806,342.81481214)
\curveto(259.20882431,342.59215589)(259.55257431,342.48082776)(259.98616806,342.48082776)
\curveto(260.58382431,342.48082776)(261.06233994,342.69176526)(261.42171494,343.11364026)
\curveto(261.78499619,343.53942151)(261.96663681,344.10387464)(261.96663681,344.80699964)
\lineto(261.96663681,345.04723401)
\lineto(260.89437119,345.04723401)
\closepath
\moveto(263.04476181,345.49254651)
\lineto(263.04476181,341.74840589)
\lineto(261.96663681,341.74840589)
\lineto(261.96663681,342.74449964)
\curveto(261.72054306,342.34606214)(261.41390244,342.05114026)(261.04671494,341.85973401)
\curveto(260.67952744,341.67223401)(260.23030869,341.57848401)(259.69905869,341.57848401)
\curveto(259.02718369,341.57848401)(258.49202744,341.76598401)(258.09358994,342.14098401)
\curveto(257.69905869,342.51989026)(257.50179306,343.02574964)(257.50179306,343.65856214)
\curveto(257.50179306,344.39684339)(257.74788681,344.95348401)(258.24007431,345.32848401)
\curveto(258.73616806,345.70348401)(259.47444931,345.89098401)(260.45491806,345.89098401)
\lineto(261.96663681,345.89098401)
\lineto(261.96663681,345.99645276)
\curveto(261.96663681,346.49254651)(261.80257431,346.87535901)(261.47444931,347.14489026)
\curveto(261.15023056,347.41832776)(260.69319931,347.55504651)(260.10335556,347.55504651)
\curveto(259.72835556,347.55504651)(259.36312119,347.51012464)(259.00765244,347.42028089)
\curveto(258.65218369,347.33043714)(258.31038681,347.19567151)(257.98226181,347.01598401)
\lineto(257.98226181,348.01207776)
\curveto(258.37679306,348.16442151)(258.75960556,348.27770276)(259.13069931,348.35192151)
\curveto(259.50179306,348.43004651)(259.86312119,348.46910901)(260.21468369,348.46910901)
\curveto(261.16390244,348.46910901)(261.87288681,348.22301526)(262.34163681,347.73082776)
\curveto(262.81038681,347.23864026)(263.04476181,346.49254651)(263.04476181,345.49254651)
\closepath
}
}
{
\newrgbcolor{curcolor}{0 0 0}
\pscustom[linestyle=none,fillstyle=solid,fillcolor=curcolor]
{
\newpath
\moveto(269.07405869,347.30309339)
\curveto(268.95296494,347.37340589)(268.82015244,347.42418714)(268.67562119,347.45543714)
\curveto(268.53499619,347.49059339)(268.37874619,347.50817151)(268.20687119,347.50817151)
\curveto(267.59749619,347.50817151)(267.12874619,347.30895276)(266.80062119,346.91051526)
\curveto(266.47640244,346.51598401)(266.31429306,345.94762464)(266.31429306,345.20543714)
\lineto(266.31429306,341.74840589)
\lineto(265.23030869,341.74840589)
\lineto(265.23030869,348.31090589)
\lineto(266.31429306,348.31090589)
\lineto(266.31429306,347.29137464)
\curveto(266.54085556,347.68981214)(266.83577744,347.98473401)(267.19905869,348.17614026)
\curveto(267.56233994,348.37145276)(268.00374619,348.46910901)(268.52327744,348.46910901)
\curveto(268.59749619,348.46910901)(268.67952744,348.46324964)(268.76937119,348.45153089)
\curveto(268.85921494,348.44371839)(268.95882431,348.43004651)(269.06819931,348.41051526)
\lineto(269.07405869,347.30309339)
\closepath
}
}
{
\newrgbcolor{curcolor}{0 0 0}
\pscustom[linestyle=none,fillstyle=solid,fillcolor=curcolor]
{
\newpath
\moveto(273.19905869,345.04723401)
\curveto(272.32796494,345.04723401)(271.72444931,344.94762464)(271.38851181,344.74840589)
\curveto(271.05257431,344.54918714)(270.88460556,344.20934339)(270.88460556,343.72887464)
\curveto(270.88460556,343.34606214)(271.00960556,343.04137464)(271.25960556,342.81481214)
\curveto(271.51351181,342.59215589)(271.85726181,342.48082776)(272.29085556,342.48082776)
\curveto(272.88851181,342.48082776)(273.36702744,342.69176526)(273.72640244,343.11364026)
\curveto(274.08968369,343.53942151)(274.27132431,344.10387464)(274.27132431,344.80699964)
\lineto(274.27132431,345.04723401)
\lineto(273.19905869,345.04723401)
\closepath
\moveto(275.34944931,345.49254651)
\lineto(275.34944931,341.74840589)
\lineto(274.27132431,341.74840589)
\lineto(274.27132431,342.74449964)
\curveto(274.02523056,342.34606214)(273.71858994,342.05114026)(273.35140244,341.85973401)
\curveto(272.98421494,341.67223401)(272.53499619,341.57848401)(272.00374619,341.57848401)
\curveto(271.33187119,341.57848401)(270.79671494,341.76598401)(270.39827744,342.14098401)
\curveto(270.00374619,342.51989026)(269.80648056,343.02574964)(269.80648056,343.65856214)
\curveto(269.80648056,344.39684339)(270.05257431,344.95348401)(270.54476181,345.32848401)
\curveto(271.04085556,345.70348401)(271.77913681,345.89098401)(272.75960556,345.89098401)
\lineto(274.27132431,345.89098401)
\lineto(274.27132431,345.99645276)
\curveto(274.27132431,346.49254651)(274.10726181,346.87535901)(273.77913681,347.14489026)
\curveto(273.45491806,347.41832776)(272.99788681,347.55504651)(272.40804306,347.55504651)
\curveto(272.03304306,347.55504651)(271.66780869,347.51012464)(271.31233994,347.42028089)
\curveto(270.95687119,347.33043714)(270.61507431,347.19567151)(270.28694931,347.01598401)
\lineto(270.28694931,348.01207776)
\curveto(270.68148056,348.16442151)(271.06429306,348.27770276)(271.43538681,348.35192151)
\curveto(271.80648056,348.43004651)(272.16780869,348.46910901)(272.51937119,348.46910901)
\curveto(273.46858994,348.46910901)(274.17757431,348.22301526)(274.64632431,347.73082776)
\curveto(275.11507431,347.23864026)(275.34944931,346.49254651)(275.34944931,345.49254651)
\closepath
}
}
{
\newrgbcolor{curcolor}{0 0 0}
\pscustom[linestyle=none,fillstyle=solid,fillcolor=curcolor]
{
\newpath
\moveto(282.68538681,347.05114026)
\curveto(282.95491806,347.53551526)(283.27718369,347.89293714)(283.65218369,348.12340589)
\curveto(284.02718369,348.35387464)(284.46858994,348.46910901)(284.97640244,348.46910901)
\curveto(285.65999619,348.46910901)(286.18733994,348.22887464)(286.55843369,347.74840589)
\curveto(286.92952744,347.27184339)(287.11507431,346.59215589)(287.11507431,345.70934339)
\lineto(287.11507431,341.74840589)
\lineto(286.03108994,341.74840589)
\lineto(286.03108994,345.67418714)
\curveto(286.03108994,346.30309339)(285.91976181,346.76989026)(285.69710556,347.07457776)
\curveto(285.47444931,347.37926526)(285.13460556,347.53160901)(284.67757431,347.53160901)
\curveto(284.11898056,347.53160901)(283.67757431,347.34606214)(283.35335556,346.97496839)
\curveto(283.02913681,346.60387464)(282.86702744,346.09801526)(282.86702744,345.45739026)
\lineto(282.86702744,341.74840589)
\lineto(281.78304306,341.74840589)
\lineto(281.78304306,345.67418714)
\curveto(281.78304306,346.30699964)(281.67171494,346.77379651)(281.44905869,347.07457776)
\curveto(281.22640244,347.37926526)(280.88265244,347.53160901)(280.41780869,347.53160901)
\curveto(279.86702744,347.53160901)(279.42952744,347.34410901)(279.10530869,346.96910901)
\curveto(278.78108994,346.59801526)(278.61898056,346.09410901)(278.61898056,345.45739026)
\lineto(278.61898056,341.74840589)
\lineto(277.53499619,341.74840589)
\lineto(277.53499619,348.31090589)
\lineto(278.61898056,348.31090589)
\lineto(278.61898056,347.29137464)
\curveto(278.86507431,347.69371839)(279.15999619,347.99059339)(279.50374619,348.18199964)
\curveto(279.84749619,348.37340589)(280.25569931,348.46910901)(280.72835556,348.46910901)
\curveto(281.20491806,348.46910901)(281.60921494,348.34801526)(281.94124619,348.10582776)
\curveto(282.27718369,347.86364026)(282.52523056,347.51207776)(282.68538681,347.05114026)
\closepath
}
}
{
\newrgbcolor{curcolor}{0 0 0}
\pscustom[linestyle=none,fillstyle=solid,fillcolor=curcolor]
{
\newpath
\moveto(294.88460556,345.29918714)
\lineto(294.88460556,344.77184339)
\lineto(289.92757431,344.77184339)
\curveto(289.97444931,344.02965589)(290.19710556,343.46324964)(290.59554306,343.07262464)
\curveto(290.99788681,342.68590589)(291.55648056,342.49254651)(292.27132431,342.49254651)
\curveto(292.68538681,342.49254651)(293.08577744,342.54332776)(293.47249619,342.64489026)
\curveto(293.86312119,342.74645276)(294.24983994,342.89879651)(294.63265244,343.10192151)
\lineto(294.63265244,342.08239026)
\curveto(294.24593369,341.91832776)(293.84944931,341.79332776)(293.44319931,341.70739026)
\curveto(293.03694931,341.62145276)(292.62483994,341.57848401)(292.20687119,341.57848401)
\curveto(291.15999619,341.57848401)(290.32991806,341.88317151)(289.71663681,342.49254651)
\curveto(289.10726181,343.10192151)(288.80257431,343.92614026)(288.80257431,344.96520276)
\curveto(288.80257431,346.03942151)(289.09163681,346.89098401)(289.66976181,347.51989026)
\curveto(290.25179306,348.15270276)(291.03499619,348.46910901)(292.01937119,348.46910901)
\curveto(292.90218369,348.46910901)(293.59944931,348.18395276)(294.11116806,347.61364026)
\curveto(294.62679306,347.04723401)(294.88460556,346.27574964)(294.88460556,345.29918714)
\closepath
\moveto(293.80648056,345.61559339)
\curveto(293.79866806,346.20543714)(293.63265244,346.67614026)(293.30843369,347.02770276)
\curveto(292.98812119,347.37926526)(292.56233994,347.55504651)(292.03108994,347.55504651)
\curveto(291.42952744,347.55504651)(290.94710556,347.38512464)(290.58382431,347.04528089)
\curveto(290.22444931,346.70543714)(290.01741806,346.22692151)(289.96273056,345.60973401)
\lineto(293.80648056,345.61559339)
\closepath
}
}
{
\newrgbcolor{curcolor}{0 0 0}
\pscustom[linestyle=none,fillstyle=solid,fillcolor=curcolor]
{
\newpath
\moveto(297.72054306,350.17418714)
\lineto(297.72054306,348.31090589)
\lineto(299.94124619,348.31090589)
\lineto(299.94124619,347.47301526)
\lineto(297.72054306,347.47301526)
\lineto(297.72054306,343.91051526)
\curveto(297.72054306,343.37535901)(297.79280869,343.03160901)(297.93733994,342.87926526)
\curveto(298.08577744,342.72692151)(298.38460556,342.65074964)(298.83382431,342.65074964)
\lineto(299.94124619,342.65074964)
\lineto(299.94124619,341.74840589)
\lineto(298.83382431,341.74840589)
\curveto(298.00179306,341.74840589)(297.42757431,341.90270276)(297.11116806,342.21129651)
\curveto(296.79476181,342.52379651)(296.63655869,343.09020276)(296.63655869,343.91051526)
\lineto(296.63655869,347.47301526)
\lineto(295.84554306,347.47301526)
\lineto(295.84554306,348.31090589)
\lineto(296.63655869,348.31090589)
\lineto(296.63655869,350.17418714)
\lineto(297.72054306,350.17418714)
\closepath
}
}
{
\newrgbcolor{curcolor}{0 0 0}
\pscustom[linestyle=none,fillstyle=solid,fillcolor=curcolor]
{
\newpath
\moveto(305.16780869,347.30309339)
\curveto(305.04671494,347.37340589)(304.91390244,347.42418714)(304.76937119,347.45543714)
\curveto(304.62874619,347.49059339)(304.47249619,347.50817151)(304.30062119,347.50817151)
\curveto(303.69124619,347.50817151)(303.22249619,347.30895276)(302.89437119,346.91051526)
\curveto(302.57015244,346.51598401)(302.40804306,345.94762464)(302.40804306,345.20543714)
\lineto(302.40804306,341.74840589)
\lineto(301.32405869,341.74840589)
\lineto(301.32405869,348.31090589)
\lineto(302.40804306,348.31090589)
\lineto(302.40804306,347.29137464)
\curveto(302.63460556,347.68981214)(302.92952744,347.98473401)(303.29280869,348.17614026)
\curveto(303.65608994,348.37145276)(304.09749619,348.46910901)(304.61702744,348.46910901)
\curveto(304.69124619,348.46910901)(304.77327744,348.46324964)(304.86312119,348.45153089)
\curveto(304.95296494,348.44371839)(305.05257431,348.43004651)(305.16194931,348.41051526)
\lineto(305.16780869,347.30309339)
\closepath
}
}
{
\newrgbcolor{curcolor}{0 0 0}
\pscustom[linestyle=none,fillstyle=solid,fillcolor=curcolor]
{
\newpath
\moveto(308.60726181,347.55504651)
\curveto(308.02913681,347.55504651)(307.57210556,347.32848401)(307.23616806,346.87535901)
\curveto(306.90023056,346.42614026)(306.73226181,345.80895276)(306.73226181,345.02379651)
\curveto(306.73226181,344.23864026)(306.89827744,343.61949964)(307.23030869,343.16637464)
\curveto(307.56624619,342.71715589)(308.02523056,342.49254651)(308.60726181,342.49254651)
\curveto(309.18148056,342.49254651)(309.63655869,342.71910901)(309.97249619,343.17223401)
\curveto(310.30843369,343.62535901)(310.47640244,344.24254651)(310.47640244,345.02379651)
\curveto(310.47640244,345.80114026)(310.30843369,346.41637464)(309.97249619,346.86949964)
\curveto(309.63655869,347.32653089)(309.18148056,347.55504651)(308.60726181,347.55504651)
\closepath
\moveto(308.60726181,348.46910901)
\curveto(309.54476181,348.46910901)(310.28108994,348.16442151)(310.81624619,347.55504651)
\curveto(311.35140244,346.94567151)(311.61898056,346.10192151)(311.61898056,345.02379651)
\curveto(311.61898056,343.94957776)(311.35140244,343.10582776)(310.81624619,342.49254651)
\curveto(310.28108994,341.88317151)(309.54476181,341.57848401)(308.60726181,341.57848401)
\curveto(307.66585556,341.57848401)(306.92757431,341.88317151)(306.39241806,342.49254651)
\curveto(305.86116806,343.10582776)(305.59554306,343.94957776)(305.59554306,345.02379651)
\curveto(305.59554306,346.10192151)(305.86116806,346.94567151)(306.39241806,347.55504651)
\curveto(306.92757431,348.16442151)(307.66585556,348.46910901)(308.60726181,348.46910901)
\closepath
}
}
{
\newrgbcolor{curcolor}{0 0 0}
\pscustom[linestyle=none,fillstyle=solid,fillcolor=curcolor]
{
\newpath
\moveto(317.59554306,348.11754651)
\lineto(317.59554306,347.09801526)
\curveto(317.29085556,347.25426526)(316.97444931,347.37145276)(316.64632431,347.44957776)
\curveto(316.31819931,347.52770276)(315.97835556,347.56676526)(315.62679306,347.56676526)
\curveto(315.09163681,347.56676526)(314.68929306,347.48473401)(314.41976181,347.32067151)
\curveto(314.15413681,347.15660901)(314.02132431,346.91051526)(314.02132431,346.58239026)
\curveto(314.02132431,346.33239026)(314.11702744,346.13512464)(314.30843369,345.99059339)
\curveto(314.49983994,345.84996839)(314.88460556,345.71520276)(315.46273056,345.58629651)
\lineto(315.83187119,345.50426526)
\curveto(316.59749619,345.34020276)(317.14046494,345.10778089)(317.46077744,344.80699964)
\curveto(317.78499619,344.51012464)(317.94710556,344.09410901)(317.94710556,343.55895276)
\curveto(317.94710556,342.94957776)(317.70491806,342.46715589)(317.22054306,342.11168714)
\curveto(316.74007431,341.75621839)(316.07796494,341.57848401)(315.23421494,341.57848401)
\curveto(314.88265244,341.57848401)(314.51546494,341.61364026)(314.13265244,341.68395276)
\curveto(313.75374619,341.75035901)(313.35335556,341.85192151)(312.93148056,341.98864026)
\lineto(312.93148056,343.10192151)
\curveto(313.32991806,342.89489026)(313.72249619,342.73864026)(314.10921494,342.63317151)
\curveto(314.49593369,342.53160901)(314.87874619,342.48082776)(315.25765244,342.48082776)
\curveto(315.76546494,342.48082776)(316.15608994,342.56676526)(316.42952744,342.73864026)
\curveto(316.70296494,342.91442151)(316.83968369,343.16051526)(316.83968369,343.47692151)
\curveto(316.83968369,343.76989026)(316.74007431,343.99449964)(316.54085556,344.15074964)
\curveto(316.34554306,344.30699964)(315.91390244,344.45739026)(315.24593369,344.60192151)
\lineto(314.87093369,344.68981214)
\curveto(314.20296494,344.83043714)(313.72054306,345.04528089)(313.42366806,345.33434339)
\curveto(313.12679306,345.62731214)(312.97835556,346.02770276)(312.97835556,346.53551526)
\curveto(312.97835556,347.15270276)(313.19710556,347.62926526)(313.63460556,347.96520276)
\curveto(314.07210556,348.30114026)(314.69319931,348.46910901)(315.49788681,348.46910901)
\curveto(315.89632431,348.46910901)(316.27132431,348.43981214)(316.62288681,348.38121839)
\curveto(316.97444931,348.32262464)(317.29866806,348.23473401)(317.59554306,348.11754651)
\closepath
}
}
{
\newrgbcolor{curcolor}{0 0 0}
\pscustom[linestyle=none,fillstyle=solid,fillcolor=curcolor]
{
\newpath
\moveto(326.47249619,345.04723401)
\curveto(325.60140244,345.04723401)(324.99788681,344.94762464)(324.66194931,344.74840589)
\curveto(324.32601181,344.54918714)(324.15804306,344.20934339)(324.15804306,343.72887464)
\curveto(324.15804306,343.34606214)(324.28304306,343.04137464)(324.53304306,342.81481214)
\curveto(324.78694931,342.59215589)(325.13069931,342.48082776)(325.56429306,342.48082776)
\curveto(326.16194931,342.48082776)(326.64046494,342.69176526)(326.99983994,343.11364026)
\curveto(327.36312119,343.53942151)(327.54476181,344.10387464)(327.54476181,344.80699964)
\lineto(327.54476181,345.04723401)
\lineto(326.47249619,345.04723401)
\closepath
\moveto(328.62288681,345.49254651)
\lineto(328.62288681,341.74840589)
\lineto(327.54476181,341.74840589)
\lineto(327.54476181,342.74449964)
\curveto(327.29866806,342.34606214)(326.99202744,342.05114026)(326.62483994,341.85973401)
\curveto(326.25765244,341.67223401)(325.80843369,341.57848401)(325.27718369,341.57848401)
\curveto(324.60530869,341.57848401)(324.07015244,341.76598401)(323.67171494,342.14098401)
\curveto(323.27718369,342.51989026)(323.07991806,343.02574964)(323.07991806,343.65856214)
\curveto(323.07991806,344.39684339)(323.32601181,344.95348401)(323.81819931,345.32848401)
\curveto(324.31429306,345.70348401)(325.05257431,345.89098401)(326.03304306,345.89098401)
\lineto(327.54476181,345.89098401)
\lineto(327.54476181,345.99645276)
\curveto(327.54476181,346.49254651)(327.38069931,346.87535901)(327.05257431,347.14489026)
\curveto(326.72835556,347.41832776)(326.27132431,347.55504651)(325.68148056,347.55504651)
\curveto(325.30648056,347.55504651)(324.94124619,347.51012464)(324.58577744,347.42028089)
\curveto(324.23030869,347.33043714)(323.88851181,347.19567151)(323.56038681,347.01598401)
\lineto(323.56038681,348.01207776)
\curveto(323.95491806,348.16442151)(324.33773056,348.27770276)(324.70882431,348.35192151)
\curveto(325.07991806,348.43004651)(325.44124619,348.46910901)(325.79280869,348.46910901)
\curveto(326.74202744,348.46910901)(327.45101181,348.22301526)(327.91976181,347.73082776)
\curveto(328.38851181,347.23864026)(328.62288681,346.49254651)(328.62288681,345.49254651)
\closepath
}
}
{
\newrgbcolor{curcolor}{0 0 0}
\pscustom[linestyle=none,fillstyle=solid,fillcolor=curcolor]
{
\newpath
\moveto(334.66976181,350.86559339)
\lineto(335.74788681,350.86559339)
\lineto(335.74788681,341.74840589)
\lineto(334.66976181,341.74840589)
\lineto(334.66976181,350.86559339)
\closepath
}
}
{
\newrgbcolor{curcolor}{0 0 0}
\pscustom[linestyle=none,fillstyle=solid,fillcolor=curcolor]
{
\newpath
\moveto(340.54085556,347.55504651)
\curveto(339.96273056,347.55504651)(339.50569931,347.32848401)(339.16976181,346.87535901)
\curveto(338.83382431,346.42614026)(338.66585556,345.80895276)(338.66585556,345.02379651)
\curveto(338.66585556,344.23864026)(338.83187119,343.61949964)(339.16390244,343.16637464)
\curveto(339.49983994,342.71715589)(339.95882431,342.49254651)(340.54085556,342.49254651)
\curveto(341.11507431,342.49254651)(341.57015244,342.71910901)(341.90608994,343.17223401)
\curveto(342.24202744,343.62535901)(342.40999619,344.24254651)(342.40999619,345.02379651)
\curveto(342.40999619,345.80114026)(342.24202744,346.41637464)(341.90608994,346.86949964)
\curveto(341.57015244,347.32653089)(341.11507431,347.55504651)(340.54085556,347.55504651)
\closepath
\moveto(340.54085556,348.46910901)
\curveto(341.47835556,348.46910901)(342.21468369,348.16442151)(342.74983994,347.55504651)
\curveto(343.28499619,346.94567151)(343.55257431,346.10192151)(343.55257431,345.02379651)
\curveto(343.55257431,343.94957776)(343.28499619,343.10582776)(342.74983994,342.49254651)
\curveto(342.21468369,341.88317151)(341.47835556,341.57848401)(340.54085556,341.57848401)
\curveto(339.59944931,341.57848401)(338.86116806,341.88317151)(338.32601181,342.49254651)
\curveto(337.79476181,343.10582776)(337.52913681,343.94957776)(337.52913681,345.02379651)
\curveto(337.52913681,346.10192151)(337.79476181,346.94567151)(338.32601181,347.55504651)
\curveto(338.86116806,348.16442151)(339.59944931,348.46910901)(340.54085556,348.46910901)
\closepath
}
}
{
\newrgbcolor{curcolor}{0 0 0}
\pscustom[linestyle=none,fillstyle=solid,fillcolor=curcolor]
{
\newpath
\moveto(349.51741806,348.11754651)
\lineto(349.51741806,347.09801526)
\curveto(349.21273056,347.25426526)(348.89632431,347.37145276)(348.56819931,347.44957776)
\curveto(348.24007431,347.52770276)(347.90023056,347.56676526)(347.54866806,347.56676526)
\curveto(347.01351181,347.56676526)(346.61116806,347.48473401)(346.34163681,347.32067151)
\curveto(346.07601181,347.15660901)(345.94319931,346.91051526)(345.94319931,346.58239026)
\curveto(345.94319931,346.33239026)(346.03890244,346.13512464)(346.23030869,345.99059339)
\curveto(346.42171494,345.84996839)(346.80648056,345.71520276)(347.38460556,345.58629651)
\lineto(347.75374619,345.50426526)
\curveto(348.51937119,345.34020276)(349.06233994,345.10778089)(349.38265244,344.80699964)
\curveto(349.70687119,344.51012464)(349.86898056,344.09410901)(349.86898056,343.55895276)
\curveto(349.86898056,342.94957776)(349.62679306,342.46715589)(349.14241806,342.11168714)
\curveto(348.66194931,341.75621839)(347.99983994,341.57848401)(347.15608994,341.57848401)
\curveto(346.80452744,341.57848401)(346.43733994,341.61364026)(346.05452744,341.68395276)
\curveto(345.67562119,341.75035901)(345.27523056,341.85192151)(344.85335556,341.98864026)
\lineto(344.85335556,343.10192151)
\curveto(345.25179306,342.89489026)(345.64437119,342.73864026)(346.03108994,342.63317151)
\curveto(346.41780869,342.53160901)(346.80062119,342.48082776)(347.17952744,342.48082776)
\curveto(347.68733994,342.48082776)(348.07796494,342.56676526)(348.35140244,342.73864026)
\curveto(348.62483994,342.91442151)(348.76155869,343.16051526)(348.76155869,343.47692151)
\curveto(348.76155869,343.76989026)(348.66194931,343.99449964)(348.46273056,344.15074964)
\curveto(348.26741806,344.30699964)(347.83577744,344.45739026)(347.16780869,344.60192151)
\lineto(346.79280869,344.68981214)
\curveto(346.12483994,344.83043714)(345.64241806,345.04528089)(345.34554306,345.33434339)
\curveto(345.04866806,345.62731214)(344.90023056,346.02770276)(344.90023056,346.53551526)
\curveto(344.90023056,347.15270276)(345.11898056,347.62926526)(345.55648056,347.96520276)
\curveto(345.99398056,348.30114026)(346.61507431,348.46910901)(347.41976181,348.46910901)
\curveto(347.81819931,348.46910901)(348.19319931,348.43981214)(348.54476181,348.38121839)
\curveto(348.89632431,348.32262464)(349.22054306,348.23473401)(349.51741806,348.11754651)
\closepath
}
}
{
\newrgbcolor{curcolor}{0 0 0}
\pscustom[linestyle=none,fillstyle=solid,fillcolor=curcolor]
{
\newpath
\moveto(356.64241806,349.52379651)
\lineto(356.64241806,346.23668714)
\lineto(358.13069931,346.23668714)
\curveto(358.68148056,346.23668714)(359.10726181,346.37926526)(359.40804306,346.66442151)
\curveto(359.70882431,346.94957776)(359.85921494,347.35582776)(359.85921494,347.88317151)
\curveto(359.85921494,348.40660901)(359.70882431,348.81090589)(359.40804306,349.09606214)
\curveto(359.10726181,349.38121839)(358.68148056,349.52379651)(358.13069931,349.52379651)
\lineto(356.64241806,349.52379651)
\closepath
\moveto(355.45882431,350.49645276)
\lineto(358.13069931,350.49645276)
\curveto(359.11116806,350.49645276)(359.85140244,350.27379651)(360.35140244,349.82848401)
\curveto(360.85530869,349.38707776)(361.10726181,348.73864026)(361.10726181,347.88317151)
\curveto(361.10726181,347.01989026)(360.85530869,346.36754651)(360.35140244,345.92614026)
\curveto(359.85140244,345.48473401)(359.11116806,345.26403089)(358.13069931,345.26403089)
\lineto(356.64241806,345.26403089)
\lineto(356.64241806,341.74840589)
\lineto(355.45882431,341.74840589)
\lineto(355.45882431,350.49645276)
\closepath
}
}
{
\newrgbcolor{curcolor}{0 0 0}
\pscustom[linestyle=none,fillstyle=solid,fillcolor=curcolor]
{
\newpath
\moveto(362.70101181,350.49645276)
\lineto(363.88460556,350.49645276)
\lineto(363.88460556,342.74449964)
\lineto(368.14437119,342.74449964)
\lineto(368.14437119,341.74840589)
\lineto(362.70101181,341.74840589)
\lineto(362.70101181,350.49645276)
\closepath
}
}
{
\newrgbcolor{curcolor}{0 0 0}
\pscustom[linestyle=none,fillstyle=solid,fillcolor=curcolor]
{
\newpath
\moveto(375.93148056,349.82262464)
\lineto(375.93148056,348.57457776)
\curveto(375.53304306,348.94567151)(375.10726181,349.22301526)(374.65413681,349.40660901)
\curveto(374.20491806,349.59020276)(373.72640244,349.68199964)(373.21858994,349.68199964)
\curveto(372.21858994,349.68199964)(371.45296494,349.37535901)(370.92171494,348.76207776)
\curveto(370.39046494,348.15270276)(370.12483994,347.26989026)(370.12483994,346.11364026)
\curveto(370.12483994,344.96129651)(370.39046494,344.07848401)(370.92171494,343.46520276)
\curveto(371.45296494,342.85582776)(372.21858994,342.55114026)(373.21858994,342.55114026)
\curveto(373.72640244,342.55114026)(374.20491806,342.64293714)(374.65413681,342.82653089)
\curveto(375.10726181,343.01012464)(375.53304306,343.28746839)(375.93148056,343.65856214)
\lineto(375.93148056,342.42223401)
\curveto(375.51741806,342.14098401)(375.07796494,341.93004651)(374.61312119,341.78942151)
\curveto(374.15218369,341.64879651)(373.66390244,341.57848401)(373.14827744,341.57848401)
\curveto(371.82405869,341.57848401)(370.78108994,341.98278089)(370.01937119,342.79137464)
\curveto(369.25765244,343.60387464)(368.87679306,344.71129651)(368.87679306,346.11364026)
\curveto(368.87679306,347.51989026)(369.25765244,348.62731214)(370.01937119,349.43590589)
\curveto(370.78108994,350.24840589)(371.82405869,350.65465589)(373.14827744,350.65465589)
\curveto(373.67171494,350.65465589)(374.16390244,350.58434339)(374.62483994,350.44371839)
\curveto(375.08968369,350.30699964)(375.52523056,350.09996839)(375.93148056,349.82262464)
\closepath
}
}
{
\newrgbcolor{curcolor}{0 0 0}
\pscustom[linestyle=none,fillstyle=solid,fillcolor=curcolor]
{
\newpath
\moveto(381.90804306,348.11754651)
\lineto(381.90804306,347.09801526)
\curveto(381.60335556,347.25426526)(381.28694931,347.37145276)(380.95882431,347.44957776)
\curveto(380.63069931,347.52770276)(380.29085556,347.56676526)(379.93929306,347.56676526)
\curveto(379.40413681,347.56676526)(379.00179306,347.48473401)(378.73226181,347.32067151)
\curveto(378.46663681,347.15660901)(378.33382431,346.91051526)(378.33382431,346.58239026)
\curveto(378.33382431,346.33239026)(378.42952744,346.13512464)(378.62093369,345.99059339)
\curveto(378.81233994,345.84996839)(379.19710556,345.71520276)(379.77523056,345.58629651)
\lineto(380.14437119,345.50426526)
\curveto(380.90999619,345.34020276)(381.45296494,345.10778089)(381.77327744,344.80699964)
\curveto(382.09749619,344.51012464)(382.25960556,344.09410901)(382.25960556,343.55895276)
\curveto(382.25960556,342.94957776)(382.01741806,342.46715589)(381.53304306,342.11168714)
\curveto(381.05257431,341.75621839)(380.39046494,341.57848401)(379.54671494,341.57848401)
\curveto(379.19515244,341.57848401)(378.82796494,341.61364026)(378.44515244,341.68395276)
\curveto(378.06624619,341.75035901)(377.66585556,341.85192151)(377.24398056,341.98864026)
\lineto(377.24398056,343.10192151)
\curveto(377.64241806,342.89489026)(378.03499619,342.73864026)(378.42171494,342.63317151)
\curveto(378.80843369,342.53160901)(379.19124619,342.48082776)(379.57015244,342.48082776)
\curveto(380.07796494,342.48082776)(380.46858994,342.56676526)(380.74202744,342.73864026)
\curveto(381.01546494,342.91442151)(381.15218369,343.16051526)(381.15218369,343.47692151)
\curveto(381.15218369,343.76989026)(381.05257431,343.99449964)(380.85335556,344.15074964)
\curveto(380.65804306,344.30699964)(380.22640244,344.45739026)(379.55843369,344.60192151)
\lineto(379.18343369,344.68981214)
\curveto(378.51546494,344.83043714)(378.03304306,345.04528089)(377.73616806,345.33434339)
\curveto(377.43929306,345.62731214)(377.29085556,346.02770276)(377.29085556,346.53551526)
\curveto(377.29085556,347.15270276)(377.50960556,347.62926526)(377.94710556,347.96520276)
\curveto(378.38460556,348.30114026)(379.00569931,348.46910901)(379.81038681,348.46910901)
\curveto(380.20882431,348.46910901)(380.58382431,348.43981214)(380.93538681,348.38121839)
\curveto(381.28694931,348.32262464)(381.61116806,348.23473401)(381.90804306,348.11754651)
\closepath
}
}
{
\newrgbcolor{curcolor}{0 0 0}
\pscustom[linestyle=none,fillstyle=solid,fillcolor=curcolor]
{
\newpath
\moveto(2.03906284,79.07565413)
\lineto(2.03906284,75.78854449)
\lineto(3.52734421,75.78854449)
\curveto(4.0781255,75.78854449)(4.50390679,75.93112263)(4.80468806,76.2162789)
\curveto(5.10546933,76.50143517)(5.25585997,76.9076852)(5.25585997,77.435029)
\curveto(5.25585997,77.95846654)(5.10546933,78.36276344)(4.80468806,78.64791972)
\curveto(4.50390679,78.93307599)(4.0781255,79.07565413)(3.52734421,79.07565413)
\lineto(2.03906284,79.07565413)
\closepath
\moveto(0.855469,80.04831045)
\lineto(3.52734421,80.04831045)
\curveto(4.50781304,80.04831045)(5.24804747,79.82565419)(5.74804751,79.38034165)
\curveto(6.2519538,78.93893536)(6.50390695,78.29049781)(6.50390695,77.435029)
\curveto(6.50390695,76.57174768)(6.2519538,75.91940387)(5.74804751,75.47799759)
\curveto(5.24804747,75.0365913)(4.50781304,74.81588816)(3.52734421,74.81588816)
\lineto(2.03906284,74.81588816)
\lineto(2.03906284,71.30026288)
\lineto(0.855469,71.30026288)
\lineto(0.855469,80.04831045)
\closepath
}
}
{
\newrgbcolor{curcolor}{0 0 0}
\pscustom[linestyle=none,fillstyle=solid,fillcolor=curcolor]
{
\newpath
\moveto(8.09765697,80.04831045)
\lineto(9.28125082,80.04831045)
\lineto(9.28125082,72.29635671)
\lineto(13.54101678,72.29635671)
\lineto(13.54101678,71.30026288)
\lineto(8.09765697,71.30026288)
\lineto(8.09765697,80.04831045)
\closepath
}
}
{
\newrgbcolor{curcolor}{0 0 0}
\pscustom[linestyle=none,fillstyle=solid,fillcolor=curcolor]
{
\newpath
\moveto(21.32812672,79.37448227)
\lineto(21.32812672,78.1264353)
\curveto(20.92968919,78.49752908)(20.50390791,78.77487285)(20.05078287,78.95846662)
\curveto(19.60156408,79.14206038)(19.12304842,79.23385726)(18.61523588,79.23385726)
\curveto(17.6152358,79.23385726)(16.84961074,78.92721661)(16.3183607,78.31393532)
\curveto(15.78711066,77.70456027)(15.52148563,76.8217477)(15.52148563,75.6654976)
\curveto(15.52148563,74.51315376)(15.78711066,73.63034119)(16.3183607,73.01705989)
\curveto(16.84961074,72.40768485)(17.6152358,72.10299732)(18.61523588,72.10299732)
\curveto(19.12304842,72.10299732)(19.60156408,72.1947942)(20.05078287,72.37838797)
\curveto(20.50390791,72.56198173)(20.92968919,72.83932551)(21.32812672,73.21041928)
\lineto(21.32812672,71.97409106)
\curveto(20.91406419,71.69284104)(20.47461103,71.48190352)(20.00976724,71.34127851)
\curveto(19.5488297,71.2006535)(19.06054842,71.13034099)(18.54492337,71.13034099)
\curveto(17.22070452,71.13034099)(16.17773569,71.5346379)(15.41601688,72.34323172)
\curveto(14.65429807,73.15573178)(14.27343866,74.26315374)(14.27343866,75.6654976)
\curveto(14.27343866,77.07174772)(14.65429807,78.17916968)(15.41601688,78.98776349)
\curveto(16.17773569,79.80026356)(17.22070452,80.20651359)(18.54492337,80.20651359)
\curveto(19.06836092,80.20651359)(19.56054846,80.13620109)(20.02148599,79.99557607)
\curveto(20.48632978,79.85885731)(20.92187669,79.65182605)(21.32812672,79.37448227)
\closepath
}
}
{
\newrgbcolor{curcolor}{0 0 0}
\pscustom[linestyle=none,fillstyle=solid,fillcolor=curcolor]
{
\newpath
\moveto(23.4785178,72.29635671)
\lineto(25.41211171,72.29635671)
\lineto(25.41211171,78.97018537)
\lineto(23.30859591,78.54831033)
\lineto(23.30859591,79.62643542)
\lineto(25.40039296,80.04831045)
\lineto(26.5839868,80.04831045)
\lineto(26.5839868,72.29635671)
\lineto(28.5175807,72.29635671)
\lineto(28.5175807,71.30026288)
\lineto(23.4785178,71.30026288)
\lineto(23.4785178,72.29635671)
\closepath
}
}
{
\newrgbcolor{curcolor}{0 0 0}
\pscustom[linestyle=none,fillstyle=solid,fillcolor=curcolor]
{
\newpath
\moveto(39.30469193,77.61081026)
\lineto(39.30469193,76.60299768)
\curveto(39.00000441,76.77096644)(38.69336376,76.89596645)(38.38476998,76.97799771)
\curveto(38.08008246,77.06393522)(37.77148869,77.10690397)(37.45898866,77.10690397)
\curveto(36.75976986,77.10690397)(36.21680106,76.8842477)(35.83008228,76.43893517)
\curveto(35.4433635,75.99752888)(35.25000411,75.37643508)(35.25000411,74.57565377)
\curveto(35.25000411,73.77487245)(35.4433635,73.15182553)(35.83008228,72.70651299)
\curveto(36.21680106,72.26510671)(36.75976986,72.04440357)(37.45898866,72.04440357)
\curveto(37.77148869,72.04440357)(38.08008246,72.0854192)(38.38476998,72.16745045)
\curveto(38.69336376,72.25338796)(39.00000441,72.38034109)(39.30469193,72.54830986)
\lineto(39.30469193,71.55221603)
\curveto(39.00391066,71.41159102)(38.69141063,71.30612226)(38.36719186,71.23580975)
\curveto(38.04687933,71.16549725)(37.70508243,71.13034099)(37.34180115,71.13034099)
\curveto(36.35351982,71.13034099)(35.56836351,71.44088789)(34.98633221,72.06198169)
\curveto(34.40430092,72.68307549)(34.11328527,73.52096618)(34.11328527,74.57565377)
\curveto(34.11328527,75.64596635)(34.40625404,76.4877633)(34.99219159,77.10104459)
\curveto(35.58203539,77.71432589)(36.38867608,78.02096654)(37.41211366,78.02096654)
\curveto(37.74414493,78.02096654)(38.06836371,77.98581029)(38.38476998,77.91549778)
\curveto(38.70117626,77.84909153)(39.00781691,77.74752902)(39.30469193,77.61081026)
\closepath
}
}
{
\newrgbcolor{curcolor}{0 0 0}
\pscustom[linestyle=none,fillstyle=solid,fillcolor=curcolor]
{
\newpath
\moveto(43.73437926,77.10690397)
\curveto(43.15625421,77.10690397)(42.69922293,76.88034145)(42.3632854,76.42721642)
\curveto(42.02734787,75.97799763)(41.85937911,75.36081008)(41.85937911,74.57565377)
\curveto(41.85937911,73.79049746)(42.02539475,73.17135678)(42.35742603,72.71823175)
\curveto(42.69336355,72.26901296)(43.15234796,72.04440357)(43.73437926,72.04440357)
\curveto(44.30859806,72.04440357)(44.76367622,72.27096608)(45.09961374,72.72409112)
\curveto(45.43555127,73.17721616)(45.60352003,73.79440371)(45.60352003,74.57565377)
\curveto(45.60352003,75.35299758)(45.43555127,75.968232)(45.09961374,76.42135704)
\curveto(44.76367622,76.87838833)(44.30859806,77.10690397)(43.73437926,77.10690397)
\closepath
\moveto(43.73437926,78.02096654)
\curveto(44.67187933,78.02096654)(45.40820752,77.71627902)(45.94336381,77.10690397)
\curveto(46.4785201,76.49752892)(46.74609825,75.65377885)(46.74609825,74.57565377)
\curveto(46.74609825,73.50143493)(46.4785201,72.65768487)(45.94336381,72.04440357)
\curveto(45.40820752,71.43502852)(44.67187933,71.13034099)(43.73437926,71.13034099)
\curveto(42.79297293,71.13034099)(42.05469163,71.43502852)(41.51953533,72.04440357)
\curveto(40.98828529,72.65768487)(40.72266027,73.50143493)(40.72266027,74.57565377)
\curveto(40.72266027,75.65377885)(40.98828529,76.49752892)(41.51953533,77.10690397)
\curveto(42.05469163,77.71627902)(42.79297293,78.02096654)(43.73437926,78.02096654)
\closepath
}
}
{
\newrgbcolor{curcolor}{0 0 0}
\pscustom[linestyle=none,fillstyle=solid,fillcolor=curcolor]
{
\newpath
\moveto(53.63672321,76.60299768)
\curveto(53.90625448,77.08737272)(54.22852014,77.44479462)(54.60352017,77.67526339)
\curveto(54.9785202,77.90573216)(55.41992648,78.02096654)(55.92773902,78.02096654)
\curveto(56.61133282,78.02096654)(57.13867662,77.78073215)(57.5097704,77.30026336)
\curveto(57.88086418,76.82370082)(58.06641107,76.14401327)(58.06641107,75.2612007)
\lineto(58.06641107,71.30026288)
\lineto(56.9824266,71.30026288)
\lineto(56.9824266,75.22604444)
\curveto(56.9824266,75.85495074)(56.87109847,76.32174766)(56.6484422,76.62643518)
\curveto(56.42578594,76.93112271)(56.08594216,77.08346647)(55.62891087,77.08346647)
\curveto(55.07031708,77.08346647)(54.62891079,76.89791958)(54.30469202,76.5268258)
\curveto(53.98047324,76.15573202)(53.81836385,75.6498726)(53.81836385,75.00924755)
\lineto(53.81836385,71.30026288)
\lineto(52.73437939,71.30026288)
\lineto(52.73437939,75.22604444)
\curveto(52.73437939,75.858857)(52.62305126,76.32565391)(52.40039499,76.62643518)
\curveto(52.17773872,76.93112271)(51.8339887,77.08346647)(51.36914491,77.08346647)
\curveto(50.81836361,77.08346647)(50.38086358,76.89596645)(50.0566448,76.52096642)
\curveto(49.73242603,76.14987264)(49.57031664,75.64596635)(49.57031664,75.00924755)
\lineto(49.57031664,71.30026288)
\lineto(48.48633218,71.30026288)
\lineto(48.48633218,77.8627634)
\lineto(49.57031664,77.8627634)
\lineto(49.57031664,76.84323207)
\curveto(49.81641041,77.24557586)(50.11133231,77.54245088)(50.45508234,77.73385714)
\curveto(50.79883236,77.92526341)(51.20703552,78.02096654)(51.67969181,78.02096654)
\curveto(52.15625435,78.02096654)(52.56055125,77.89987278)(52.89258253,77.65768526)
\curveto(53.22852006,77.41549774)(53.47656695,77.06393522)(53.63672321,76.60299768)
\closepath
}
}
{
\newrgbcolor{curcolor}{0 0 0}
\pscustom[linestyle=none,fillstyle=solid,fillcolor=curcolor]
{
\newpath
\moveto(61.26562914,72.28463796)
\lineto(61.26562914,68.80416893)
\lineto(60.18164468,68.80416893)
\lineto(60.18164468,77.8627634)
\lineto(61.26562914,77.8627634)
\lineto(61.26562914,76.86666958)
\curveto(61.49219166,77.25729461)(61.77734793,77.54635713)(62.12109796,77.73385714)
\curveto(62.46875424,77.92526341)(62.88281677,78.02096654)(63.36328556,78.02096654)
\curveto(64.16016062,78.02096654)(64.80664505,77.70456027)(65.30273884,77.07174772)
\curveto(65.80273888,76.43893517)(66.0527389,75.60690385)(66.0527389,74.57565377)
\curveto(66.0527389,73.54440369)(65.80273888,72.71237237)(65.30273884,72.07955982)
\curveto(64.80664505,71.44674727)(64.16016062,71.13034099)(63.36328556,71.13034099)
\curveto(62.88281677,71.13034099)(62.46875424,71.224091)(62.12109796,71.41159102)
\curveto(61.77734793,71.60299728)(61.49219166,71.89401293)(61.26562914,72.28463796)
\closepath
\moveto(64.93359818,74.57565377)
\curveto(64.93359818,75.36862258)(64.76953567,75.98971638)(64.44141064,76.43893517)
\curveto(64.11719187,76.8920602)(63.66992621,77.11862272)(63.09961366,77.11862272)
\curveto(62.52930112,77.11862272)(62.08008233,76.8920602)(61.7519573,76.43893517)
\curveto(61.42773853,75.98971638)(61.26562914,75.36862258)(61.26562914,74.57565377)
\curveto(61.26562914,73.78268496)(61.42773853,73.15963803)(61.7519573,72.70651299)
\curveto(62.08008233,72.25729421)(62.52930112,72.03268482)(63.09961366,72.03268482)
\curveto(63.66992621,72.03268482)(64.11719187,72.25729421)(64.44141064,72.70651299)
\curveto(64.76953567,73.15963803)(64.93359818,73.78268496)(64.93359818,74.57565377)
\closepath
}
}
{
\newrgbcolor{curcolor}{0 0 0}
\pscustom[linestyle=none,fillstyle=solid,fillcolor=curcolor]
{
\newpath
\moveto(70.82227373,74.59909127)
\curveto(69.95117991,74.59909127)(69.34766424,74.49948189)(69.01172672,74.30026312)
\curveto(68.67578919,74.10104436)(68.50782043,73.76120058)(68.50782043,73.28073179)
\curveto(68.50782043,72.89791926)(68.63282044,72.59323174)(68.88282045,72.36666922)
\curveto(69.13672673,72.14401295)(69.48047675,72.03268482)(69.91407054,72.03268482)
\curveto(70.51172683,72.03268482)(70.9902425,72.24362233)(71.34961753,72.66549737)
\curveto(71.7128988,73.09127865)(71.89453944,73.65573182)(71.89453944,74.35885688)
\lineto(71.89453944,74.59909127)
\lineto(70.82227373,74.59909127)
\closepath
\moveto(72.97266453,75.04440381)
\lineto(72.97266453,71.30026288)
\lineto(71.89453944,71.30026288)
\lineto(71.89453944,72.29635671)
\curveto(71.64844567,71.89791918)(71.34180503,71.60299728)(70.9746175,71.41159102)
\curveto(70.60742997,71.224091)(70.15821118,71.13034099)(69.62696114,71.13034099)
\curveto(68.95508609,71.13034099)(68.41992979,71.31784101)(68.02149226,71.69284104)
\curveto(67.62696098,72.07174732)(67.42969534,72.57760673)(67.42969534,73.21041928)
\curveto(67.42969534,73.94870059)(67.67578911,74.50534126)(68.16797665,74.88034129)
\curveto(68.66407044,75.25534132)(69.40235175,75.44284134)(70.38282057,75.44284134)
\lineto(71.89453944,75.44284134)
\lineto(71.89453944,75.5483101)
\curveto(71.89453944,76.04440388)(71.73047693,76.42721642)(71.40235191,76.69674769)
\curveto(71.07813313,76.97018521)(70.62110184,77.10690397)(70.03125805,77.10690397)
\curveto(69.65625802,77.10690397)(69.29102361,77.06198209)(68.93555483,76.97213833)
\curveto(68.58008606,76.88229458)(68.23828915,76.74752894)(67.91016413,76.56784143)
\lineto(67.91016413,77.56393526)
\curveto(68.30469541,77.71627902)(68.68750794,77.82956028)(69.05860172,77.90377903)
\curveto(69.4296955,77.98190404)(69.79102365,78.02096654)(70.14258618,78.02096654)
\curveto(71.09180501,78.02096654)(71.80078944,77.77487277)(72.26953947,77.28268523)
\curveto(72.73828951,76.79049769)(72.97266453,76.04440388)(72.97266453,75.04440381)
\closepath
}
}
{
\newrgbcolor{curcolor}{0 0 0}
\pscustom[linestyle=none,fillstyle=solid,fillcolor=curcolor]
{
\newpath
\moveto(79.0019613,76.85495082)
\curveto(78.88086754,76.92526333)(78.74805503,76.97604458)(78.60352377,77.00729459)
\curveto(78.46289876,77.04245084)(78.30664874,77.06002897)(78.13477373,77.06002897)
\curveto(77.52539868,77.06002897)(77.05664864,76.8608102)(76.72852362,76.46237267)
\curveto(76.40430484,76.06784139)(76.24219545,75.49948197)(76.24219545,74.75729441)
\lineto(76.24219545,71.30026288)
\lineto(75.15821099,71.30026288)
\lineto(75.15821099,77.8627634)
\lineto(76.24219545,77.8627634)
\lineto(76.24219545,76.84323207)
\curveto(76.46875797,77.2416696)(76.76367987,77.5365915)(77.12696115,77.72799777)
\curveto(77.49024243,77.92331028)(77.93164871,78.02096654)(78.45118001,78.02096654)
\curveto(78.52539876,78.02096654)(78.60743002,78.01510717)(78.69727378,78.00338842)
\curveto(78.78711753,77.99557591)(78.88672692,77.98190404)(78.99610192,77.96237279)
\lineto(79.0019613,76.85495082)
\closepath
}
}
{
\newrgbcolor{curcolor}{0 0 0}
\pscustom[linestyle=none,fillstyle=solid,fillcolor=curcolor]
{
\newpath
\moveto(83.12696123,74.59909127)
\curveto(82.25586741,74.59909127)(81.65235174,74.49948189)(81.31641422,74.30026312)
\curveto(80.98047669,74.10104436)(80.81250793,73.76120058)(80.81250793,73.28073179)
\curveto(80.81250793,72.89791926)(80.93750794,72.59323174)(81.18750795,72.36666922)
\curveto(81.44141423,72.14401295)(81.78516425,72.03268482)(82.21875804,72.03268482)
\curveto(82.81641433,72.03268482)(83.29493,72.24362233)(83.65430503,72.66549737)
\curveto(84.0175863,73.09127865)(84.19922694,73.65573182)(84.19922694,74.35885688)
\lineto(84.19922694,74.59909127)
\lineto(83.12696123,74.59909127)
\closepath
\moveto(85.27735203,75.04440381)
\lineto(85.27735203,71.30026288)
\lineto(84.19922694,71.30026288)
\lineto(84.19922694,72.29635671)
\curveto(83.95313317,71.89791918)(83.64649253,71.60299728)(83.279305,71.41159102)
\curveto(82.91211747,71.224091)(82.46289868,71.13034099)(81.93164864,71.13034099)
\curveto(81.25977359,71.13034099)(80.72461729,71.31784101)(80.32617976,71.69284104)
\curveto(79.93164848,72.07174732)(79.73438284,72.57760673)(79.73438284,73.21041928)
\curveto(79.73438284,73.94870059)(79.98047661,74.50534126)(80.47266415,74.88034129)
\curveto(80.96875794,75.25534132)(81.70703925,75.44284134)(82.68750807,75.44284134)
\lineto(84.19922694,75.44284134)
\lineto(84.19922694,75.5483101)
\curveto(84.19922694,76.04440388)(84.03516443,76.42721642)(83.70703941,76.69674769)
\curveto(83.38282063,76.97018521)(82.92578934,77.10690397)(82.33594555,77.10690397)
\curveto(81.96094552,77.10690397)(81.59571111,77.06198209)(81.24024233,76.97213833)
\curveto(80.88477356,76.88229458)(80.54297665,76.74752894)(80.21485163,76.56784143)
\lineto(80.21485163,77.56393526)
\curveto(80.60938291,77.71627902)(80.99219544,77.82956028)(81.36328922,77.90377903)
\curveto(81.734383,77.98190404)(82.09571115,78.02096654)(82.44727368,78.02096654)
\curveto(83.39649251,78.02096654)(84.10547694,77.77487277)(84.57422697,77.28268523)
\curveto(85.04297701,76.79049769)(85.27735203,76.04440388)(85.27735203,75.04440381)
\closepath
}
}
{
\newrgbcolor{curcolor}{0 0 0}
\pscustom[linestyle=none,fillstyle=solid,fillcolor=curcolor]
{
\newpath
\moveto(91.32422662,80.41745111)
\lineto(92.40235171,80.41745111)
\lineto(92.40235171,71.30026288)
\lineto(91.32422662,71.30026288)
\lineto(91.32422662,80.41745111)
\closepath
}
}
{
\newrgbcolor{curcolor}{0 0 0}
\pscustom[linestyle=none,fillstyle=solid,fillcolor=curcolor]
{
\newpath
\moveto(97.63477373,74.59909127)
\curveto(96.76367991,74.59909127)(96.16016424,74.49948189)(95.82422672,74.30026312)
\curveto(95.48828919,74.10104436)(95.32032043,73.76120058)(95.32032043,73.28073179)
\curveto(95.32032043,72.89791926)(95.44532044,72.59323174)(95.69532045,72.36666922)
\curveto(95.94922673,72.14401295)(96.29297675,72.03268482)(96.72657054,72.03268482)
\curveto(97.32422683,72.03268482)(97.8027425,72.24362233)(98.16211753,72.66549737)
\curveto(98.5253988,73.09127865)(98.70703944,73.65573182)(98.70703944,74.35885688)
\lineto(98.70703944,74.59909127)
\lineto(97.63477373,74.59909127)
\closepath
\moveto(99.78516453,75.04440381)
\lineto(99.78516453,71.30026288)
\lineto(98.70703944,71.30026288)
\lineto(98.70703944,72.29635671)
\curveto(98.46094567,71.89791918)(98.15430503,71.60299728)(97.7871175,71.41159102)
\curveto(97.41992997,71.224091)(96.97071118,71.13034099)(96.43946114,71.13034099)
\curveto(95.76758609,71.13034099)(95.23242979,71.31784101)(94.83399226,71.69284104)
\curveto(94.43946098,72.07174732)(94.24219534,72.57760673)(94.24219534,73.21041928)
\curveto(94.24219534,73.94870059)(94.48828911,74.50534126)(94.98047665,74.88034129)
\curveto(95.47657044,75.25534132)(96.21485175,75.44284134)(97.19532057,75.44284134)
\lineto(98.70703944,75.44284134)
\lineto(98.70703944,75.5483101)
\curveto(98.70703944,76.04440388)(98.54297693,76.42721642)(98.21485191,76.69674769)
\curveto(97.89063313,76.97018521)(97.43360184,77.10690397)(96.84375805,77.10690397)
\curveto(96.46875802,77.10690397)(96.10352361,77.06198209)(95.74805483,76.97213833)
\curveto(95.39258606,76.88229458)(95.05078915,76.74752894)(94.72266413,76.56784143)
\lineto(94.72266413,77.56393526)
\curveto(95.11719541,77.71627902)(95.50000794,77.82956028)(95.87110172,77.90377903)
\curveto(96.2421955,77.98190404)(96.60352365,78.02096654)(96.95508618,78.02096654)
\curveto(97.90430501,78.02096654)(98.61328944,77.77487277)(99.08203947,77.28268523)
\curveto(99.55078951,76.79049769)(99.78516453,76.04440388)(99.78516453,75.04440381)
\closepath
}
}
{
\newrgbcolor{curcolor}{0 0 0}
\pscustom[linestyle=none,fillstyle=solid,fillcolor=curcolor]
{
\newpath
\moveto(110.94141453,76.60299768)
\curveto(111.2109458,77.08737272)(111.53321145,77.44479462)(111.90821148,77.67526339)
\curveto(112.28321151,77.90573216)(112.72461779,78.02096654)(113.23243034,78.02096654)
\curveto(113.91602414,78.02096654)(114.44336793,77.78073215)(114.81446171,77.30026336)
\curveto(115.18555549,76.82370082)(115.37110238,76.14401327)(115.37110238,75.2612007)
\lineto(115.37110238,71.30026288)
\lineto(114.28711792,71.30026288)
\lineto(114.28711792,75.22604444)
\curveto(114.28711792,75.85495074)(114.17578979,76.32174766)(113.95313352,76.62643518)
\curveto(113.73047725,76.93112271)(113.39063347,77.08346647)(112.93360219,77.08346647)
\curveto(112.37500839,77.08346647)(111.93360211,76.89791958)(111.60938333,76.5268258)
\curveto(111.28516456,76.15573202)(111.12305517,75.6498726)(111.12305517,75.00924755)
\lineto(111.12305517,71.30026288)
\lineto(110.03907071,71.30026288)
\lineto(110.03907071,75.22604444)
\curveto(110.03907071,75.858857)(109.92774257,76.32565391)(109.7050863,76.62643518)
\curveto(109.48243004,76.93112271)(109.13868001,77.08346647)(108.67383622,77.08346647)
\curveto(108.12305493,77.08346647)(107.68555489,76.89596645)(107.36133612,76.52096642)
\curveto(107.03711734,76.14987264)(106.87500795,75.64596635)(106.87500795,75.00924755)
\lineto(106.87500795,71.30026288)
\lineto(105.79102349,71.30026288)
\lineto(105.79102349,77.8627634)
\lineto(106.87500795,77.8627634)
\lineto(106.87500795,76.84323207)
\curveto(107.12110172,77.24557586)(107.41602362,77.54245088)(107.75977365,77.73385714)
\curveto(108.10352368,77.92526341)(108.51172684,78.02096654)(108.98438312,78.02096654)
\curveto(109.46094566,78.02096654)(109.86524257,77.89987278)(110.19727384,77.65768526)
\curveto(110.53321137,77.41549774)(110.78125827,77.06393522)(110.94141453,76.60299768)
\closepath
}
}
{
\newrgbcolor{curcolor}{0 0 0}
\pscustom[linestyle=none,fillstyle=solid,fillcolor=curcolor]
{
\newpath
\moveto(123.14063332,74.85104442)
\lineto(123.14063332,74.32370062)
\lineto(118.18360167,74.32370062)
\curveto(118.23047668,73.58151306)(118.45313295,73.01510677)(118.85157048,72.62448174)
\curveto(119.25391426,72.23776296)(119.81250805,72.04440357)(120.52735186,72.04440357)
\curveto(120.94141439,72.04440357)(121.34180505,72.09518482)(121.72852383,72.19674733)
\curveto(122.11914886,72.29830984)(122.50586764,72.4506536)(122.88868017,72.65377862)
\lineto(122.88868017,71.63424728)
\curveto(122.50196139,71.47018477)(122.10547699,71.34518476)(121.69922695,71.25924725)
\curveto(121.29297692,71.17330975)(120.88086751,71.13034099)(120.46289873,71.13034099)
\curveto(119.41602365,71.13034099)(118.58594546,71.43502852)(117.97266416,72.04440357)
\curveto(117.36328911,72.65377862)(117.05860158,73.47799743)(117.05860158,74.51706001)
\curveto(117.05860158,75.59127885)(117.34766411,76.44284142)(117.92578915,77.07174772)
\curveto(118.50782045,77.70456027)(119.29102364,78.02096654)(120.27539872,78.02096654)
\curveto(121.15821129,78.02096654)(121.85547697,77.73581027)(122.36719576,77.16549772)
\curveto(122.8828208,76.59909143)(123.14063332,75.82760699)(123.14063332,74.85104442)
\closepath
\moveto(122.06250823,75.16745069)
\curveto(122.05469573,75.75729449)(121.88868009,76.22799765)(121.56446132,76.57956018)
\curveto(121.24414879,76.93112271)(120.81836751,77.10690397)(120.28711747,77.10690397)
\curveto(119.68555492,77.10690397)(119.20313301,76.93698208)(118.83985173,76.5971383)
\curveto(118.4804767,76.25729453)(118.27344543,75.77877886)(118.21875793,75.16159131)
\lineto(122.06250823,75.16745069)
\closepath
}
}
{
\newrgbcolor{curcolor}{0 0 0}
\pscustom[linestyle=none,fillstyle=solid,fillcolor=curcolor]
{
\newpath
\moveto(129.22852384,76.86666958)
\lineto(129.22852384,80.41745111)
\lineto(130.30664893,80.41745111)
\lineto(130.30664893,71.30026288)
\lineto(129.22852384,71.30026288)
\lineto(129.22852384,72.28463796)
\curveto(129.00196132,71.89401293)(128.71485192,71.60299728)(128.36719565,71.41159102)
\curveto(128.02344562,71.224091)(127.60938309,71.13034099)(127.12500805,71.13034099)
\curveto(126.33203924,71.13034099)(125.68555481,71.44674727)(125.18555477,72.07955982)
\curveto(124.68946098,72.71237237)(124.44141408,73.54440369)(124.44141408,74.57565377)
\curveto(124.44141408,75.60690385)(124.68946098,76.43893517)(125.18555477,77.07174772)
\curveto(125.68555481,77.70456027)(126.33203924,78.02096654)(127.12500805,78.02096654)
\curveto(127.60938309,78.02096654)(128.02344562,77.92526341)(128.36719565,77.73385714)
\curveto(128.71485192,77.54635713)(129.00196132,77.25729461)(129.22852384,76.86666958)
\closepath
\moveto(125.55469542,74.57565377)
\curveto(125.55469542,73.78268496)(125.71680481,73.15963803)(126.04102359,72.70651299)
\curveto(126.36914861,72.25729421)(126.8183674,72.03268482)(127.38867994,72.03268482)
\curveto(127.95899249,72.03268482)(128.40821128,72.25729421)(128.7363363,72.70651299)
\curveto(129.06446133,73.15963803)(129.22852384,73.78268496)(129.22852384,74.57565377)
\curveto(129.22852384,75.36862258)(129.06446133,75.98971638)(128.7363363,76.43893517)
\curveto(128.40821128,76.8920602)(127.95899249,77.11862272)(127.38867994,77.11862272)
\curveto(126.8183674,77.11862272)(126.36914861,76.8920602)(126.04102359,76.43893517)
\curveto(125.71680481,75.98971638)(125.55469542,75.36862258)(125.55469542,74.57565377)
\closepath
}
}
{
\newrgbcolor{curcolor}{0 0 0}
\pscustom[linestyle=none,fillstyle=solid,fillcolor=curcolor]
{
\newpath
\moveto(132.52735925,77.8627634)
\lineto(133.60548434,77.8627634)
\lineto(133.60548434,71.30026288)
\lineto(132.52735925,71.30026288)
\lineto(132.52735925,77.8627634)
\closepath
\moveto(132.52735925,80.41745111)
\lineto(133.60548434,80.41745111)
\lineto(133.60548434,79.05221662)
\lineto(132.52735925,79.05221662)
\lineto(132.52735925,80.41745111)
\closepath
}
}
{
\newrgbcolor{curcolor}{0 0 0}
\pscustom[linestyle=none,fillstyle=solid,fillcolor=curcolor]
{
\newpath
\moveto(140.17384397,76.86666958)
\lineto(140.17384397,80.41745111)
\lineto(141.25196906,80.41745111)
\lineto(141.25196906,71.30026288)
\lineto(140.17384397,71.30026288)
\lineto(140.17384397,72.28463796)
\curveto(139.94728145,71.89401293)(139.66017205,71.60299728)(139.31251578,71.41159102)
\curveto(138.96876575,71.224091)(138.55470322,71.13034099)(138.07032818,71.13034099)
\curveto(137.27735936,71.13034099)(136.63087494,71.44674727)(136.1308749,72.07955982)
\curveto(135.63478111,72.71237237)(135.38673421,73.54440369)(135.38673421,74.57565377)
\curveto(135.38673421,75.60690385)(135.63478111,76.43893517)(136.1308749,77.07174772)
\curveto(136.63087494,77.70456027)(137.27735936,78.02096654)(138.07032818,78.02096654)
\curveto(138.55470322,78.02096654)(138.96876575,77.92526341)(139.31251578,77.73385714)
\curveto(139.66017205,77.54635713)(139.94728145,77.25729461)(140.17384397,76.86666958)
\closepath
\moveto(136.50001555,74.57565377)
\curveto(136.50001555,73.78268496)(136.66212494,73.15963803)(136.98634372,72.70651299)
\curveto(137.31446874,72.25729421)(137.76368753,72.03268482)(138.33400007,72.03268482)
\curveto(138.90431262,72.03268482)(139.3535314,72.25729421)(139.68165643,72.70651299)
\curveto(140.00978146,73.15963803)(140.17384397,73.78268496)(140.17384397,74.57565377)
\curveto(140.17384397,75.36862258)(140.00978146,75.98971638)(139.68165643,76.43893517)
\curveto(139.3535314,76.8920602)(138.90431262,77.11862272)(138.33400007,77.11862272)
\curveto(137.76368753,77.11862272)(137.31446874,76.8920602)(136.98634372,76.43893517)
\curveto(136.66212494,75.98971638)(136.50001555,75.36862258)(136.50001555,74.57565377)
\closepath
}
}
{
\newrgbcolor{curcolor}{0 0 0}
\pscustom[linestyle=none,fillstyle=solid,fillcolor=curcolor]
{
\newpath
\moveto(146.45509386,74.59909127)
\curveto(145.58400004,74.59909127)(144.98048437,74.49948189)(144.64454684,74.30026312)
\curveto(144.30860932,74.10104436)(144.14064055,73.76120058)(144.14064055,73.28073179)
\curveto(144.14064055,72.89791926)(144.26564056,72.59323174)(144.51564058,72.36666922)
\curveto(144.76954685,72.14401295)(145.11329688,72.03268482)(145.54689067,72.03268482)
\curveto(146.14454696,72.03268482)(146.62306263,72.24362233)(146.98243766,72.66549737)
\curveto(147.34571893,73.09127865)(147.52735957,73.65573182)(147.52735957,74.35885688)
\lineto(147.52735957,74.59909127)
\lineto(146.45509386,74.59909127)
\closepath
\moveto(148.60548466,75.04440381)
\lineto(148.60548466,71.30026288)
\lineto(147.52735957,71.30026288)
\lineto(147.52735957,72.29635671)
\curveto(147.2812658,71.89791918)(146.97462515,71.60299728)(146.60743763,71.41159102)
\curveto(146.2402501,71.224091)(145.79103131,71.13034099)(145.25978127,71.13034099)
\curveto(144.58790622,71.13034099)(144.05274992,71.31784101)(143.65431239,71.69284104)
\curveto(143.25978111,72.07174732)(143.06251547,72.57760673)(143.06251547,73.21041928)
\curveto(143.06251547,73.94870059)(143.30860924,74.50534126)(143.80079678,74.88034129)
\curveto(144.29689057,75.25534132)(145.03517188,75.44284134)(146.0156407,75.44284134)
\lineto(147.52735957,75.44284134)
\lineto(147.52735957,75.5483101)
\curveto(147.52735957,76.04440388)(147.36329706,76.42721642)(147.03517203,76.69674769)
\curveto(146.71095326,76.97018521)(146.25392197,77.10690397)(145.66407818,77.10690397)
\curveto(145.28907815,77.10690397)(144.92384374,77.06198209)(144.56837496,76.97213833)
\curveto(144.21290619,76.88229458)(143.87110928,76.74752894)(143.54298426,76.56784143)
\lineto(143.54298426,77.56393526)
\curveto(143.93751554,77.71627902)(144.32032807,77.82956028)(144.69142185,77.90377903)
\curveto(145.06251563,77.98190404)(145.42384378,78.02096654)(145.77540631,78.02096654)
\curveto(146.72462513,78.02096654)(147.43360957,77.77487277)(147.9023596,77.28268523)
\curveto(148.37110964,76.79049769)(148.60548466,76.04440388)(148.60548466,75.04440381)
\closepath
}
}
{
\newrgbcolor{curcolor}{0 0 0}
\pscustom[linestyle=none,fillstyle=solid,fillcolor=curcolor]
{
\newpath
\moveto(158.97071897,76.86666958)
\lineto(158.97071897,80.41745111)
\lineto(160.04884406,80.41745111)
\lineto(160.04884406,71.30026288)
\lineto(158.97071897,71.30026288)
\lineto(158.97071897,72.28463796)
\curveto(158.74415645,71.89401293)(158.45704705,71.60299728)(158.10939078,71.41159102)
\curveto(157.76564075,71.224091)(157.35157822,71.13034099)(156.86720318,71.13034099)
\curveto(156.07423436,71.13034099)(155.42774994,71.44674727)(154.9277499,72.07955982)
\curveto(154.43165611,72.71237237)(154.18360921,73.54440369)(154.18360921,74.57565377)
\curveto(154.18360921,75.60690385)(154.43165611,76.43893517)(154.9277499,77.07174772)
\curveto(155.42774994,77.70456027)(156.07423436,78.02096654)(156.86720318,78.02096654)
\curveto(157.35157822,78.02096654)(157.76564075,77.92526341)(158.10939078,77.73385714)
\curveto(158.45704705,77.54635713)(158.74415645,77.25729461)(158.97071897,76.86666958)
\closepath
\moveto(155.29689055,74.57565377)
\curveto(155.29689055,73.78268496)(155.45899994,73.15963803)(155.78321872,72.70651299)
\curveto(156.11134374,72.25729421)(156.56056253,72.03268482)(157.13087507,72.03268482)
\curveto(157.70118762,72.03268482)(158.1504064,72.25729421)(158.47853143,72.70651299)
\curveto(158.80665646,73.15963803)(158.97071897,73.78268496)(158.97071897,74.57565377)
\curveto(158.97071897,75.36862258)(158.80665646,75.98971638)(158.47853143,76.43893517)
\curveto(158.1504064,76.8920602)(157.70118762,77.11862272)(157.13087507,77.11862272)
\curveto(156.56056253,77.11862272)(156.11134374,76.8920602)(155.78321872,76.43893517)
\curveto(155.45899994,75.98971638)(155.29689055,75.36862258)(155.29689055,74.57565377)
\closepath
}
}
{
\newrgbcolor{curcolor}{0 0 0}
\pscustom[linestyle=none,fillstyle=solid,fillcolor=curcolor]
{
\newpath
\moveto(167.88282845,74.85104442)
\lineto(167.88282845,74.32370062)
\lineto(162.9257968,74.32370062)
\curveto(162.97267181,73.58151306)(163.19532808,73.01510677)(163.59376561,72.62448174)
\curveto(163.99610939,72.23776296)(164.55470318,72.04440357)(165.26954699,72.04440357)
\curveto(165.68360952,72.04440357)(166.08400018,72.09518482)(166.47071896,72.19674733)
\curveto(166.86134399,72.29830984)(167.24806277,72.4506536)(167.6308753,72.65377862)
\lineto(167.6308753,71.63424728)
\curveto(167.24415652,71.47018477)(166.84767212,71.34518476)(166.44142208,71.25924725)
\curveto(166.03517205,71.17330975)(165.62306264,71.13034099)(165.20509386,71.13034099)
\curveto(164.15821878,71.13034099)(163.32814059,71.43502852)(162.71485929,72.04440357)
\curveto(162.10548424,72.65377862)(161.80079671,73.47799743)(161.80079671,74.51706001)
\curveto(161.80079671,75.59127885)(162.08985924,76.44284142)(162.66798428,77.07174772)
\curveto(163.25001558,77.70456027)(164.03321877,78.02096654)(165.01759384,78.02096654)
\curveto(165.90040642,78.02096654)(166.5976721,77.73581027)(167.10939089,77.16549772)
\curveto(167.62501593,76.59909143)(167.88282845,75.82760699)(167.88282845,74.85104442)
\closepath
\moveto(166.80470336,75.16745069)
\curveto(166.79689086,75.75729449)(166.63087522,76.22799765)(166.30665645,76.57956018)
\curveto(165.98634392,76.93112271)(165.56056264,77.10690397)(165.0293126,77.10690397)
\curveto(164.42775005,77.10690397)(163.94532813,76.93698208)(163.58204686,76.5971383)
\curveto(163.22267183,76.25729453)(163.01564056,75.77877886)(162.96095306,75.16159131)
\lineto(166.80470336,75.16745069)
\closepath
}
}
{
\newrgbcolor{curcolor}{0 0 0}
\pscustom[linestyle=none,fillstyle=solid,fillcolor=curcolor]
{
\newpath
\moveto(176.79493764,80.41745111)
\lineto(176.79493764,79.52096666)
\lineto(175.76368756,79.52096666)
\curveto(175.37696878,79.52096666)(175.10743751,79.44284165)(174.95509374,79.28659164)
\curveto(174.80665623,79.13034163)(174.73243748,78.84909161)(174.73243748,78.44284158)
\lineto(174.73243748,77.8627634)
\lineto(176.50782824,77.8627634)
\lineto(176.50782824,77.02487271)
\lineto(174.73243748,77.02487271)
\lineto(174.73243748,71.30026288)
\lineto(173.64845302,71.30026288)
\lineto(173.64845302,77.02487271)
\lineto(172.61720293,77.02487271)
\lineto(172.61720293,77.8627634)
\lineto(173.64845302,77.8627634)
\lineto(173.64845302,78.31979469)
\curveto(173.64845302,79.0502635)(173.8183749,79.58151354)(174.15821868,79.91354482)
\curveto(174.49806246,80.24948234)(175.037125,80.41745111)(175.77540631,80.41745111)
\lineto(176.79493764,80.41745111)
\closepath
}
}
{
\newrgbcolor{curcolor}{0 0 0}
\pscustom[linestyle=none,fillstyle=solid,fillcolor=curcolor]
{
\newpath
\moveto(177.69142175,80.41745111)
\lineto(178.76954684,80.41745111)
\lineto(178.76954684,71.30026288)
\lineto(177.69142175,71.30026288)
\lineto(177.69142175,80.41745111)
\closepath
}
}
{
\newrgbcolor{curcolor}{0 0 0}
\pscustom[linestyle=none,fillstyle=solid,fillcolor=curcolor]
{
\newpath
\moveto(180.90821862,73.89010684)
\lineto(180.90821862,77.8627634)
\lineto(181.9863437,77.8627634)
\lineto(181.9863437,73.93112247)
\curveto(181.9863437,73.31002867)(182.10743746,72.84323176)(182.34962498,72.53073173)
\curveto(182.5918125,72.22213796)(182.95509378,72.06784107)(183.43946882,72.06784107)
\curveto(184.02150012,72.06784107)(184.48048453,72.25338796)(184.81642205,72.62448174)
\curveto(185.15626583,72.99557552)(185.32618772,73.50143493)(185.32618772,74.14205998)
\lineto(185.32618772,77.8627634)
\lineto(186.4043128,77.8627634)
\lineto(186.4043128,71.30026288)
\lineto(185.32618772,71.30026288)
\lineto(185.32618772,72.30807546)
\curveto(185.06446895,71.90963793)(184.75978142,71.61276291)(184.41212515,71.41745039)
\curveto(184.06837512,71.22604413)(183.66798446,71.13034099)(183.21095318,71.13034099)
\curveto(182.45704687,71.13034099)(181.8847812,71.36471601)(181.49415616,71.83346605)
\curveto(181.10353113,72.30221609)(180.90821862,72.98776302)(180.90821862,73.89010684)
\closepath
}
}
{
\newrgbcolor{curcolor}{0 0 0}
\pscustom[linestyle=none,fillstyle=solid,fillcolor=curcolor]
{
\newpath
\moveto(188.63673425,77.8627634)
\lineto(189.71485934,77.8627634)
\lineto(189.71485934,71.18307537)
\curveto(189.71485934,70.34713781)(189.55470307,69.74166901)(189.23439055,69.36666898)
\curveto(188.91798427,68.99166895)(188.40626548,68.80416893)(187.69923418,68.80416893)
\lineto(187.28907789,68.80416893)
\lineto(187.28907789,69.71823151)
\lineto(187.57618729,69.71823151)
\curveto(187.98634357,69.71823151)(188.26564047,69.81393464)(188.41407798,70.0053409)
\curveto(188.5625155,70.19284092)(188.63673425,70.58541908)(188.63673425,71.18307537)
\lineto(188.63673425,77.8627634)
\closepath
\moveto(188.63673425,80.41745111)
\lineto(189.71485934,80.41745111)
\lineto(189.71485934,79.05221662)
\lineto(188.63673425,79.05221662)
\lineto(188.63673425,80.41745111)
\closepath
}
}
{
\newrgbcolor{curcolor}{0 0 0}
\pscustom[linestyle=none,fillstyle=solid,fillcolor=curcolor]
{
\newpath
\moveto(194.5078282,77.10690397)
\curveto(193.92970316,77.10690397)(193.47267187,76.88034145)(193.13673434,76.42721642)
\curveto(192.80079682,75.97799763)(192.63282805,75.36081008)(192.63282805,74.57565377)
\curveto(192.63282805,73.79049746)(192.79884369,73.17135678)(193.13087497,72.71823175)
\curveto(193.4668125,72.26901296)(193.92579691,72.04440357)(194.5078282,72.04440357)
\curveto(195.082047,72.04440357)(195.53712516,72.27096608)(195.87306269,72.72409112)
\curveto(196.20900021,73.17721616)(196.37696898,73.79440371)(196.37696898,74.57565377)
\curveto(196.37696898,75.35299758)(196.20900021,75.968232)(195.87306269,76.42135704)
\curveto(195.53712516,76.87838833)(195.082047,77.10690397)(194.5078282,77.10690397)
\closepath
\moveto(194.5078282,78.02096654)
\curveto(195.44532828,78.02096654)(196.18165646,77.71627902)(196.71681275,77.10690397)
\curveto(197.25196905,76.49752892)(197.51954719,75.65377885)(197.51954719,74.57565377)
\curveto(197.51954719,73.50143493)(197.25196905,72.65768487)(196.71681275,72.04440357)
\curveto(196.18165646,71.43502852)(195.44532828,71.13034099)(194.5078282,71.13034099)
\curveto(193.56642188,71.13034099)(192.82814057,71.43502852)(192.29298428,72.04440357)
\curveto(191.76173424,72.65768487)(191.49610921,73.50143493)(191.49610921,74.57565377)
\curveto(191.49610921,75.65377885)(191.76173424,76.49752892)(192.29298428,77.10690397)
\curveto(192.82814057,77.71627902)(193.56642188,78.02096654)(194.5078282,78.02096654)
\closepath
}
}
{
\newrgbcolor{curcolor}{0 0 0}
\pscustom[linestyle=none,fillstyle=solid,fillcolor=curcolor]
{
\newpath
\moveto(207.84376588,77.61081026)
\lineto(207.84376588,76.60299768)
\curveto(207.53907835,76.77096644)(207.2324377,76.89596645)(206.92384393,76.97799771)
\curveto(206.6191564,77.06393522)(206.31056263,77.10690397)(205.99806261,77.10690397)
\curveto(205.2988438,77.10690397)(204.75587501,76.8842477)(204.36915623,76.43893517)
\curveto(203.98243744,75.99752888)(203.78907805,75.37643508)(203.78907805,74.57565377)
\curveto(203.78907805,73.77487245)(203.98243744,73.15182553)(204.36915623,72.70651299)
\curveto(204.75587501,72.26510671)(205.2988438,72.04440357)(205.99806261,72.04440357)
\curveto(206.31056263,72.04440357)(206.6191564,72.0854192)(206.92384393,72.16745045)
\curveto(207.2324377,72.25338796)(207.53907835,72.38034109)(207.84376588,72.54830986)
\lineto(207.84376588,71.55221603)
\curveto(207.5429846,71.41159102)(207.23048458,71.30612226)(206.9062658,71.23580975)
\curveto(206.58595328,71.16549725)(206.24415637,71.13034099)(205.8808751,71.13034099)
\curveto(204.89259377,71.13034099)(204.10743745,71.44088789)(203.52540616,72.06198169)
\curveto(202.94337486,72.68307549)(202.65235921,73.52096618)(202.65235921,74.57565377)
\curveto(202.65235921,75.64596635)(202.94532799,76.4877633)(203.53126553,77.10104459)
\curveto(204.12110933,77.71432589)(204.92775002,78.02096654)(205.9511876,78.02096654)
\curveto(206.28321888,78.02096654)(206.60743765,77.98581029)(206.92384393,77.91549778)
\curveto(207.2402502,77.84909153)(207.54689085,77.74752902)(207.84376588,77.61081026)
\closepath
}
}
{
\newrgbcolor{curcolor}{0 0 0}
\pscustom[linestyle=none,fillstyle=solid,fillcolor=curcolor]
{
\newpath
\moveto(212.2734532,77.10690397)
\curveto(211.69532816,77.10690397)(211.23829687,76.88034145)(210.90235934,76.42721642)
\curveto(210.56642182,75.97799763)(210.39845305,75.36081008)(210.39845305,74.57565377)
\curveto(210.39845305,73.79049746)(210.56446869,73.17135678)(210.89649997,72.71823175)
\curveto(211.2324375,72.26901296)(211.69142191,72.04440357)(212.2734532,72.04440357)
\curveto(212.847672,72.04440357)(213.30275016,72.27096608)(213.63868769,72.72409112)
\curveto(213.97462521,73.17721616)(214.14259398,73.79440371)(214.14259398,74.57565377)
\curveto(214.14259398,75.35299758)(213.97462521,75.968232)(213.63868769,76.42135704)
\curveto(213.30275016,76.87838833)(212.847672,77.10690397)(212.2734532,77.10690397)
\closepath
\moveto(212.2734532,78.02096654)
\curveto(213.21095328,78.02096654)(213.94728146,77.71627902)(214.48243775,77.10690397)
\curveto(215.01759405,76.49752892)(215.28517219,75.65377885)(215.28517219,74.57565377)
\curveto(215.28517219,73.50143493)(215.01759405,72.65768487)(214.48243775,72.04440357)
\curveto(213.94728146,71.43502852)(213.21095328,71.13034099)(212.2734532,71.13034099)
\curveto(211.33204688,71.13034099)(210.59376557,71.43502852)(210.05860928,72.04440357)
\curveto(209.52735924,72.65768487)(209.26173421,73.50143493)(209.26173421,74.57565377)
\curveto(209.26173421,75.65377885)(209.52735924,76.49752892)(210.05860928,77.10690397)
\curveto(210.59376557,77.71627902)(211.33204688,78.02096654)(212.2734532,78.02096654)
\closepath
}
}
{
\newrgbcolor{curcolor}{0 0 0}
\pscustom[linestyle=none,fillstyle=solid,fillcolor=curcolor]
{
\newpath
\moveto(222.52150031,75.2612007)
\lineto(222.52150031,71.30026288)
\lineto(221.44337522,71.30026288)
\lineto(221.44337522,75.22604444)
\curveto(221.44337522,75.84713824)(221.32228146,76.31198203)(221.08009395,76.62057581)
\curveto(220.83790643,76.92916958)(220.47462515,77.08346647)(219.99025011,77.08346647)
\curveto(219.40821881,77.08346647)(218.9492344,76.89791958)(218.61329687,76.5268258)
\curveto(218.27735935,76.15573202)(218.10939058,75.6498726)(218.10939058,75.00924755)
\lineto(218.10939058,71.30026288)
\lineto(217.02540612,71.30026288)
\lineto(217.02540612,77.8627634)
\lineto(218.10939058,77.8627634)
\lineto(218.10939058,76.84323207)
\curveto(218.3672031,77.23776335)(218.6699375,77.53268525)(219.01759378,77.72799777)
\curveto(219.36915631,77.92331028)(219.77345322,78.02096654)(220.2304845,78.02096654)
\curveto(220.98439081,78.02096654)(221.55470336,77.78659152)(221.94142214,77.31784149)
\curveto(222.32814092,76.8529977)(222.52150031,76.16745077)(222.52150031,75.2612007)
\closepath
}
}
{
\newrgbcolor{curcolor}{0 0 0}
\pscustom[linestyle=none,fillstyle=solid,fillcolor=curcolor]
{
\newpath
\moveto(10.24218843,59.85104322)
\lineto(10.24218843,59.32369943)
\lineto(5.28515678,59.32369943)
\curveto(5.33203179,58.58151187)(5.55468805,58.01510558)(5.95312559,57.62448055)
\curveto(6.35546937,57.23776177)(6.91406316,57.04440237)(7.62890697,57.04440237)
\curveto(8.0429695,57.04440237)(8.44336016,57.09518363)(8.83007894,57.19674614)
\curveto(9.22070397,57.29830865)(9.60742275,57.45065241)(9.99023528,57.65377742)
\lineto(9.99023528,56.63424609)
\curveto(9.6035165,56.47018358)(9.20703209,56.34518357)(8.80078206,56.25924606)
\curveto(8.39453203,56.17330856)(7.98242262,56.1303398)(7.56445384,56.1303398)
\curveto(6.51757876,56.1303398)(5.68750057,56.43502733)(5.07421927,57.04440237)
\curveto(4.46484422,57.65377742)(4.16015669,58.47799624)(4.16015669,59.51705882)
\curveto(4.16015669,60.59127766)(4.44921922,61.44284022)(5.02734426,62.07174652)
\curveto(5.60937556,62.70455907)(6.39257875,63.02096535)(7.37695382,63.02096535)
\curveto(8.25976639,63.02096535)(8.95703208,62.73580908)(9.46875087,62.16549653)
\curveto(9.98437591,61.59909024)(10.24218843,60.8276058)(10.24218843,59.85104322)
\closepath
\moveto(9.16406334,60.1674495)
\curveto(9.15625084,60.75729329)(8.9902352,61.22799646)(8.66601643,61.57955899)
\curveto(8.3457039,61.93112151)(7.91992262,62.10690278)(7.38867258,62.10690278)
\curveto(6.78711003,62.10690278)(6.30468811,61.93698089)(5.94140684,61.59713711)
\curveto(5.58203181,61.25729333)(5.37500054,60.77877767)(5.32031304,60.16159012)
\lineto(9.16406334,60.1674495)
\closepath
}
}
{
\newrgbcolor{curcolor}{0 0 0}
\pscustom[linestyle=none,fillstyle=solid,fillcolor=curcolor]
{
\newpath
\moveto(12.01171995,65.41744992)
\lineto(13.08984503,65.41744992)
\lineto(13.08984503,56.30026169)
\lineto(12.01171995,56.30026169)
\lineto(12.01171995,65.41744992)
\closepath
}
}
{
\newrgbcolor{curcolor}{0 0 0}
\pscustom[linestyle=none,fillstyle=solid,fillcolor=curcolor]
{
\newpath
\moveto(20.20312723,57.28463677)
\lineto(20.20312723,53.80416774)
\lineto(19.11914277,53.80416774)
\lineto(19.11914277,62.86276221)
\lineto(20.20312723,62.86276221)
\lineto(20.20312723,61.86666838)
\curveto(20.42968975,62.25729341)(20.71484602,62.54635594)(21.05859605,62.73385595)
\curveto(21.40625233,62.92526222)(21.82031486,63.02096535)(22.30078365,63.02096535)
\curveto(23.09765871,63.02096535)(23.74414314,62.70455907)(24.24023693,62.07174652)
\curveto(24.74023697,61.43893397)(24.99023699,60.60690266)(24.99023699,59.57565258)
\curveto(24.99023699,58.54440249)(24.74023697,57.71237118)(24.24023693,57.07955863)
\curveto(23.74414314,56.44674608)(23.09765871,56.1303398)(22.30078365,56.1303398)
\curveto(21.82031486,56.1303398)(21.40625233,56.22408981)(21.05859605,56.41158982)
\curveto(20.71484602,56.60299609)(20.42968975,56.89401174)(20.20312723,57.28463677)
\closepath
\moveto(23.87109627,59.57565258)
\curveto(23.87109627,60.36862139)(23.70703376,60.98971519)(23.37890874,61.43893397)
\curveto(23.05468996,61.89205901)(22.6074243,62.11862153)(22.03711175,62.11862153)
\curveto(21.46679921,62.11862153)(21.01758042,61.89205901)(20.6894554,61.43893397)
\curveto(20.36523662,60.98971519)(20.20312723,60.36862139)(20.20312723,59.57565258)
\curveto(20.20312723,58.78268376)(20.36523662,58.15963684)(20.6894554,57.7065118)
\curveto(21.01758042,57.25729302)(21.46679921,57.03268362)(22.03711175,57.03268362)
\curveto(22.6074243,57.03268362)(23.05468996,57.25729302)(23.37890874,57.7065118)
\curveto(23.70703376,58.15963684)(23.87109627,58.78268376)(23.87109627,59.57565258)
\closepath
}
}
{
\newrgbcolor{curcolor}{0 0 0}
\pscustom[linestyle=none,fillstyle=solid,fillcolor=curcolor]
{
\newpath
\moveto(29.75976801,59.59909008)
\curveto(28.88867419,59.59909008)(28.28515852,59.49948069)(27.94922099,59.30026193)
\curveto(27.61328347,59.10104316)(27.4453147,58.76119939)(27.4453147,58.2807306)
\curveto(27.4453147,57.89791807)(27.57031471,57.59323054)(27.82031473,57.36666803)
\curveto(28.074221,57.14401176)(28.41797103,57.03268362)(28.85156481,57.03268362)
\curveto(29.44922111,57.03268362)(29.92773678,57.24362114)(30.2871118,57.66549617)
\curveto(30.65039308,58.09127746)(30.83203372,58.65573063)(30.83203372,59.35885568)
\lineto(30.83203372,59.59909008)
\lineto(29.75976801,59.59909008)
\closepath
\moveto(31.91015881,60.04440261)
\lineto(31.91015881,56.30026169)
\lineto(30.83203372,56.30026169)
\lineto(30.83203372,57.29635552)
\curveto(30.58593995,56.89791799)(30.2792993,56.60299609)(29.91211177,56.41158982)
\curveto(29.54492425,56.22408981)(29.09570546,56.1303398)(28.56445542,56.1303398)
\curveto(27.89258036,56.1303398)(27.35742407,56.31783982)(26.95898654,56.69283985)
\curveto(26.56445526,57.07174613)(26.36718962,57.57760554)(26.36718962,58.21041809)
\curveto(26.36718962,58.9486994)(26.61328339,59.50534007)(27.10547093,59.8803401)
\curveto(27.60156472,60.25534013)(28.33984602,60.44284014)(29.32031485,60.44284014)
\lineto(30.83203372,60.44284014)
\lineto(30.83203372,60.5483089)
\curveto(30.83203372,61.04440269)(30.66797121,61.42721522)(30.33984618,61.69674649)
\curveto(30.01562741,61.97018402)(29.55859612,62.10690278)(28.96875232,62.10690278)
\curveto(28.59375229,62.10690278)(28.22851789,62.0619809)(27.87304911,61.97213714)
\curveto(27.51758033,61.88229338)(27.17578343,61.74752775)(26.84765841,61.56784023)
\lineto(26.84765841,62.56393406)
\curveto(27.24218969,62.71627783)(27.62500222,62.82955908)(27.996096,62.90377784)
\curveto(28.36718978,62.98190285)(28.72851793,63.02096535)(29.08008046,63.02096535)
\curveto(30.02929928,63.02096535)(30.73828371,62.77487158)(31.20703375,62.28268404)
\curveto(31.67578379,61.7904965)(31.91015881,61.04440269)(31.91015881,60.04440261)
\closepath
}
}
{
\newrgbcolor{curcolor}{0 0 0}
\pscustom[linestyle=none,fillstyle=solid,fillcolor=curcolor]
{
\newpath
\moveto(37.93945748,61.85494963)
\curveto(37.81836372,61.92526214)(37.68555121,61.97604339)(37.54101995,62.00729339)
\curveto(37.40039494,62.04244965)(37.24414493,62.06002777)(37.07226992,62.06002777)
\curveto(36.46289487,62.06002777)(35.99414483,61.86080901)(35.6660198,61.46237148)
\curveto(35.34180103,61.06784019)(35.17969164,60.49948077)(35.17969164,59.75729322)
\lineto(35.17969164,56.30026169)
\lineto(34.09570718,56.30026169)
\lineto(34.09570718,62.86276221)
\lineto(35.17969164,62.86276221)
\lineto(35.17969164,61.84323088)
\curveto(35.40625416,62.24166841)(35.70117606,62.53659031)(36.06445734,62.72799658)
\curveto(36.42773861,62.92330909)(36.8691449,63.02096535)(37.38867619,63.02096535)
\curveto(37.46289495,63.02096535)(37.5449262,63.01510597)(37.63476996,63.00338722)
\curveto(37.72461372,62.99557472)(37.8242231,62.98190285)(37.93359811,62.9623716)
\lineto(37.93945748,61.85494963)
\closepath
}
}
{
\newrgbcolor{curcolor}{0 0 0}
\pscustom[linestyle=none,fillstyle=solid,fillcolor=curcolor]
{
\newpath
\moveto(42.06445742,59.59909008)
\curveto(41.1933636,59.59909008)(40.58984793,59.49948069)(40.2539104,59.30026193)
\curveto(39.91797287,59.10104316)(39.75000411,58.76119939)(39.75000411,58.2807306)
\curveto(39.75000411,57.89791807)(39.87500412,57.59323054)(40.12500414,57.36666803)
\curveto(40.37891041,57.14401176)(40.72266044,57.03268362)(41.15625422,57.03268362)
\curveto(41.75391052,57.03268362)(42.23242618,57.24362114)(42.59180121,57.66549617)
\curveto(42.95508249,58.09127746)(43.13672313,58.65573063)(43.13672313,59.35885568)
\lineto(43.13672313,59.59909008)
\lineto(42.06445742,59.59909008)
\closepath
\moveto(44.21484822,60.04440261)
\lineto(44.21484822,56.30026169)
\lineto(43.13672313,56.30026169)
\lineto(43.13672313,57.29635552)
\curveto(42.89062936,56.89791799)(42.58398871,56.60299609)(42.21680118,56.41158982)
\curveto(41.84961365,56.22408981)(41.40039487,56.1303398)(40.86914482,56.1303398)
\curveto(40.19726977,56.1303398)(39.66211348,56.31783982)(39.26367595,56.69283985)
\curveto(38.86914467,57.07174613)(38.67187902,57.57760554)(38.67187902,58.21041809)
\curveto(38.67187902,58.9486994)(38.91797279,59.50534007)(39.41016033,59.8803401)
\curveto(39.90625412,60.25534013)(40.64453543,60.44284014)(41.62500426,60.44284014)
\lineto(43.13672313,60.44284014)
\lineto(43.13672313,60.5483089)
\curveto(43.13672313,61.04440269)(42.97266062,61.42721522)(42.64453559,61.69674649)
\curveto(42.32031681,61.97018402)(41.86328553,62.10690278)(41.27344173,62.10690278)
\curveto(40.8984417,62.10690278)(40.5332073,62.0619809)(40.17773852,61.97213714)
\curveto(39.82226974,61.88229338)(39.48047284,61.74752775)(39.15234781,61.56784023)
\lineto(39.15234781,62.56393406)
\curveto(39.54687909,62.71627783)(39.92969162,62.82955908)(40.3007854,62.90377784)
\curveto(40.67187918,62.98190285)(41.03320734,63.02096535)(41.38476987,63.02096535)
\curveto(42.33398869,63.02096535)(43.04297312,62.77487158)(43.51172316,62.28268404)
\curveto(43.9804732,61.7904965)(44.21484822,61.04440269)(44.21484822,60.04440261)
\closepath
\moveto(42.24609806,65.8979187)
\lineto(43.41211378,65.8979187)
\lineto(41.50195737,63.69479353)
\lineto(40.60547293,63.69479353)
\lineto(42.24609806,65.8979187)
\closepath
}
}
{
\newrgbcolor{curcolor}{0 0 0}
\pscustom[linestyle=none,fillstyle=solid,fillcolor=curcolor]
{
\newpath
\moveto(51.55078571,61.60299649)
\curveto(51.82031698,62.08737153)(52.14258264,62.44479343)(52.51758267,62.6752622)
\curveto(52.8925827,62.90573097)(53.33398898,63.02096535)(53.84180152,63.02096535)
\curveto(54.52539532,63.02096535)(55.05273912,62.78073096)(55.4238329,62.30026217)
\curveto(55.79492668,61.82369963)(55.98047357,61.14401208)(55.98047357,60.26119951)
\lineto(55.98047357,56.30026169)
\lineto(54.8964891,56.30026169)
\lineto(54.8964891,60.22604325)
\curveto(54.8964891,60.85494955)(54.78516097,61.32174646)(54.5625047,61.62643399)
\curveto(54.33984844,61.93112151)(54.00000466,62.08346528)(53.54297337,62.08346528)
\curveto(52.98437958,62.08346528)(52.54297329,61.89791839)(52.21875452,61.52682461)
\curveto(51.89453574,61.15573083)(51.73242635,60.64987141)(51.73242635,60.00924636)
\lineto(51.73242635,56.30026169)
\lineto(50.64844189,56.30026169)
\lineto(50.64844189,60.22604325)
\curveto(50.64844189,60.8588558)(50.53711376,61.32565272)(50.31445749,61.62643399)
\curveto(50.09180122,61.93112151)(49.7480512,62.08346528)(49.28320741,62.08346528)
\curveto(48.73242611,62.08346528)(48.29492608,61.89596526)(47.9707073,61.52096523)
\curveto(47.64648853,61.14987145)(47.48437914,60.64596516)(47.48437914,60.00924636)
\lineto(47.48437914,56.30026169)
\lineto(46.40039468,56.30026169)
\lineto(46.40039468,62.86276221)
\lineto(47.48437914,62.86276221)
\lineto(47.48437914,61.84323088)
\curveto(47.73047291,62.24557466)(48.02539481,62.54244969)(48.36914484,62.73385595)
\curveto(48.71289486,62.92526222)(49.12109802,63.02096535)(49.59375431,63.02096535)
\curveto(50.07031685,63.02096535)(50.47461375,62.89987159)(50.80664503,62.65768407)
\curveto(51.14258256,62.41549655)(51.39062945,62.06393402)(51.55078571,61.60299649)
\closepath
}
}
{
\newrgbcolor{curcolor}{0 0 0}
\pscustom[linestyle=none,fillstyle=solid,fillcolor=curcolor]
{
\newpath
\moveto(63.7500045,59.85104322)
\lineto(63.7500045,59.32369943)
\lineto(58.79297286,59.32369943)
\curveto(58.83984786,58.58151187)(59.06250413,58.01510558)(59.46094166,57.62448055)
\curveto(59.86328544,57.23776177)(60.42187924,57.04440237)(61.13672305,57.04440237)
\curveto(61.55078558,57.04440237)(61.95117624,57.09518363)(62.33789502,57.19674614)
\curveto(62.72852005,57.29830865)(63.11523883,57.45065241)(63.49805136,57.65377742)
\lineto(63.49805136,56.63424609)
\curveto(63.11133258,56.47018358)(62.71484817,56.34518357)(62.30859814,56.25924606)
\curveto(61.90234811,56.17330856)(61.4902387,56.1303398)(61.07226992,56.1303398)
\curveto(60.02539483,56.1303398)(59.19531664,56.43502733)(58.58203534,57.04440237)
\curveto(57.97266029,57.65377742)(57.66797277,58.47799624)(57.66797277,59.51705882)
\curveto(57.66797277,60.59127766)(57.95703529,61.44284022)(58.53516034,62.07174652)
\curveto(59.11719164,62.70455907)(59.90039482,63.02096535)(60.8847699,63.02096535)
\curveto(61.76758247,63.02096535)(62.46484815,62.73580908)(62.97656694,62.16549653)
\curveto(63.49219198,61.59909024)(63.7500045,60.8276058)(63.7500045,59.85104322)
\closepath
\moveto(62.67187942,60.1674495)
\curveto(62.66406692,60.75729329)(62.49805128,61.22799646)(62.1738325,61.57955899)
\curveto(61.85351998,61.93112151)(61.42773869,62.10690278)(60.89648865,62.10690278)
\curveto(60.2949261,62.10690278)(59.81250419,61.93698089)(59.44922291,61.59713711)
\curveto(59.08984788,61.25729333)(58.88281662,60.77877767)(58.82812911,60.16159012)
\lineto(62.67187942,60.1674495)
\closepath
}
}
{
\newrgbcolor{curcolor}{0 0 0}
\pscustom[linestyle=none,fillstyle=solid,fillcolor=curcolor]
{
\newpath
\moveto(66.58594546,64.72604361)
\lineto(66.58594546,62.86276221)
\lineto(68.80664876,62.86276221)
\lineto(68.80664876,62.02487152)
\lineto(66.58594546,62.02487152)
\lineto(66.58594546,58.46237124)
\curveto(66.58594546,57.92721495)(66.65821109,57.58346492)(66.80274235,57.43112116)
\curveto(66.95117986,57.27877739)(67.25000801,57.20260551)(67.6992268,57.20260551)
\lineto(68.80664876,57.20260551)
\lineto(68.80664876,56.30026169)
\lineto(67.6992268,56.30026169)
\curveto(66.86719548,56.30026169)(66.29297668,56.45455858)(65.97657041,56.76315235)
\curveto(65.66016413,57.07565238)(65.501961,57.64205867)(65.501961,58.46237124)
\lineto(65.501961,62.02487152)
\lineto(64.71094531,62.02487152)
\lineto(64.71094531,62.86276221)
\lineto(65.501961,62.86276221)
\lineto(65.501961,64.72604361)
\lineto(66.58594546,64.72604361)
\closepath
}
}
{
\newrgbcolor{curcolor}{0 0 0}
\pscustom[linestyle=none,fillstyle=solid,fillcolor=curcolor]
{
\newpath
\moveto(74.0332113,61.85494963)
\curveto(73.91211754,61.92526214)(73.77930503,61.97604339)(73.63477377,62.00729339)
\curveto(73.49414876,62.04244965)(73.33789874,62.06002777)(73.16602373,62.06002777)
\curveto(72.55664868,62.06002777)(72.08789864,61.86080901)(71.75977362,61.46237148)
\curveto(71.43555484,61.06784019)(71.27344545,60.49948077)(71.27344545,59.75729322)
\lineto(71.27344545,56.30026169)
\lineto(70.18946099,56.30026169)
\lineto(70.18946099,62.86276221)
\lineto(71.27344545,62.86276221)
\lineto(71.27344545,61.84323088)
\curveto(71.50000797,62.24166841)(71.79492987,62.53659031)(72.15821115,62.72799658)
\curveto(72.52149243,62.92330909)(72.96289871,63.02096535)(73.48243001,63.02096535)
\curveto(73.55664876,63.02096535)(73.63868002,63.01510597)(73.72852378,63.00338722)
\curveto(73.81836753,62.99557472)(73.91797692,62.98190285)(74.02735192,62.9623716)
\lineto(74.0332113,61.85494963)
\closepath
}
}
{
\newrgbcolor{curcolor}{0 0 0}
\pscustom[linestyle=none,fillstyle=solid,fillcolor=curcolor]
{
\newpath
\moveto(77.47266432,62.10690278)
\curveto(76.89453928,62.10690278)(76.43750799,61.88034026)(76.10157047,61.42721522)
\curveto(75.76563294,60.97799644)(75.59766418,60.36080889)(75.59766418,59.57565258)
\curveto(75.59766418,58.79049626)(75.76367981,58.17135559)(76.09571109,57.71823055)
\curveto(76.43164862,57.26901177)(76.89063303,57.04440237)(77.47266432,57.04440237)
\curveto(78.04688312,57.04440237)(78.50196128,57.27096489)(78.83789881,57.72408993)
\curveto(79.17383633,58.17721496)(79.3418051,58.79440251)(79.3418051,59.57565258)
\curveto(79.3418051,60.35299639)(79.17383633,60.96823081)(78.83789881,61.42135585)
\curveto(78.50196128,61.87838713)(78.04688312,62.10690278)(77.47266432,62.10690278)
\closepath
\moveto(77.47266432,63.02096535)
\curveto(78.4101644,63.02096535)(79.14649258,62.71627783)(79.68164887,62.10690278)
\curveto(80.21680517,61.49752773)(80.48438331,60.65377766)(80.48438331,59.57565258)
\curveto(80.48438331,58.50143374)(80.21680517,57.65768367)(79.68164887,57.04440237)
\curveto(79.14649258,56.43502733)(78.4101644,56.1303398)(77.47266432,56.1303398)
\curveto(76.531258,56.1303398)(75.79297669,56.43502733)(75.2578204,57.04440237)
\curveto(74.72657036,57.65768367)(74.46094533,58.50143374)(74.46094533,59.57565258)
\curveto(74.46094533,60.65377766)(74.72657036,61.49752773)(75.2578204,62.10690278)
\curveto(75.79297669,62.71627783)(76.531258,63.02096535)(77.47266432,63.02096535)
\closepath
}
}
{
\newrgbcolor{curcolor}{0 0 0}
\pscustom[linestyle=none,fillstyle=solid,fillcolor=curcolor]
{
\newpath
\moveto(82.55274227,57.78854306)
\lineto(83.78907049,57.78854306)
\lineto(83.78907049,56.78073048)
\lineto(82.82813292,54.90573033)
\lineto(82.07227348,54.90573033)
\lineto(82.55274227,56.78073048)
\lineto(82.55274227,57.78854306)
\closepath
}
}
{
\newrgbcolor{curcolor}{0 0 0}
\pscustom[linestyle=none,fillstyle=solid,fillcolor=curcolor]
{
\newpath
\moveto(94.64063325,62.61080907)
\lineto(94.64063325,61.60299649)
\curveto(94.33594572,61.77096525)(94.02930507,61.89596526)(93.7207113,61.97799652)
\curveto(93.41602378,62.06393402)(93.10743,62.10690278)(92.79492998,62.10690278)
\curveto(92.09571117,62.10690278)(91.55274238,61.88424651)(91.1660236,61.43893397)
\curveto(90.77930482,60.99752769)(90.58594543,60.37643389)(90.58594543,59.57565258)
\curveto(90.58594543,58.77487126)(90.77930482,58.15182434)(91.1660236,57.7065118)
\curveto(91.55274238,57.26510552)(92.09571117,57.04440237)(92.79492998,57.04440237)
\curveto(93.10743,57.04440237)(93.41602378,57.085418)(93.7207113,57.16744926)
\curveto(94.02930507,57.25338677)(94.33594572,57.3803399)(94.64063325,57.54830866)
\lineto(94.64063325,56.55221484)
\curveto(94.33985197,56.41158982)(94.02735195,56.30612107)(93.70313317,56.23580856)
\curveto(93.38282065,56.16549606)(93.04102375,56.1303398)(92.67774247,56.1303398)
\curveto(91.68946114,56.1303398)(90.90430483,56.4408867)(90.32227353,57.0619805)
\curveto(89.74024223,57.6830743)(89.44922658,58.52096499)(89.44922658,59.57565258)
\curveto(89.44922658,60.64596516)(89.74219536,61.4877621)(90.3281329,62.1010434)
\curveto(90.9179767,62.7143247)(91.72461739,63.02096535)(92.74805497,63.02096535)
\curveto(93.08008625,63.02096535)(93.40430502,62.9858091)(93.7207113,62.91549659)
\curveto(94.03711757,62.84909034)(94.34375822,62.74752783)(94.64063325,62.61080907)
\closepath
}
}
{
\newrgbcolor{curcolor}{0 0 0}
\pscustom[linestyle=none,fillstyle=solid,fillcolor=curcolor]
{
\newpath
\moveto(99.07032057,62.10690278)
\curveto(98.49219553,62.10690278)(98.03516424,61.88034026)(97.69922672,61.42721522)
\curveto(97.36328919,60.97799644)(97.19532043,60.36080889)(97.19532043,59.57565258)
\curveto(97.19532043,58.79049626)(97.36133606,58.17135559)(97.69336734,57.71823055)
\curveto(98.02930487,57.26901177)(98.48828928,57.04440237)(99.07032057,57.04440237)
\curveto(99.64453937,57.04440237)(100.09961753,57.27096489)(100.43555506,57.72408993)
\curveto(100.77149258,58.17721496)(100.93946135,58.79440251)(100.93946135,59.57565258)
\curveto(100.93946135,60.35299639)(100.77149258,60.96823081)(100.43555506,61.42135585)
\curveto(100.09961753,61.87838713)(99.64453937,62.10690278)(99.07032057,62.10690278)
\closepath
\moveto(99.07032057,63.02096535)
\curveto(100.00782065,63.02096535)(100.74414883,62.71627783)(101.27930512,62.10690278)
\curveto(101.81446142,61.49752773)(102.08203956,60.65377766)(102.08203956,59.57565258)
\curveto(102.08203956,58.50143374)(101.81446142,57.65768367)(101.27930512,57.04440237)
\curveto(100.74414883,56.43502733)(100.00782065,56.1303398)(99.07032057,56.1303398)
\curveto(98.12891425,56.1303398)(97.39063294,56.43502733)(96.85547665,57.04440237)
\curveto(96.32422661,57.65768367)(96.05860158,58.50143374)(96.05860158,59.57565258)
\curveto(96.05860158,60.65377766)(96.32422661,61.49752773)(96.85547665,62.10690278)
\curveto(97.39063294,62.71627783)(98.12891425,63.02096535)(99.07032057,63.02096535)
\closepath
}
}
{
\newrgbcolor{curcolor}{0 0 0}
\pscustom[linestyle=none,fillstyle=solid,fillcolor=curcolor]
{
\newpath
\moveto(109.31836768,60.26119951)
\lineto(109.31836768,56.30026169)
\lineto(108.24024259,56.30026169)
\lineto(108.24024259,60.22604325)
\curveto(108.24024259,60.84713705)(108.11914884,61.31198084)(107.87696132,61.62057461)
\curveto(107.6347738,61.92916839)(107.27149252,62.08346528)(106.78711748,62.08346528)
\curveto(106.20508618,62.08346528)(105.74610177,61.89791839)(105.41016425,61.52682461)
\curveto(105.07422672,61.15573083)(104.90625795,60.64987141)(104.90625795,60.00924636)
\lineto(104.90625795,56.30026169)
\lineto(103.82227349,56.30026169)
\lineto(103.82227349,62.86276221)
\lineto(104.90625795,62.86276221)
\lineto(104.90625795,61.84323088)
\curveto(105.16407048,62.23776216)(105.46680487,62.53268406)(105.81446115,62.72799658)
\curveto(106.16602368,62.92330909)(106.57032059,63.02096535)(107.02735187,63.02096535)
\curveto(107.78125818,63.02096535)(108.35157073,62.78659033)(108.73828951,62.31784029)
\curveto(109.12500829,61.85299651)(109.31836768,61.16744958)(109.31836768,60.26119951)
\closepath
}
}
{
\newrgbcolor{curcolor}{0 0 0}
\pscustom[linestyle=none,fillstyle=solid,fillcolor=curcolor]
{
\newpath
\moveto(112.54688296,64.72604361)
\lineto(112.54688296,62.86276221)
\lineto(114.76758626,62.86276221)
\lineto(114.76758626,62.02487152)
\lineto(112.54688296,62.02487152)
\lineto(112.54688296,58.46237124)
\curveto(112.54688296,57.92721495)(112.61914859,57.58346492)(112.76367985,57.43112116)
\curveto(112.91211736,57.27877739)(113.21094551,57.20260551)(113.6601643,57.20260551)
\lineto(114.76758626,57.20260551)
\lineto(114.76758626,56.30026169)
\lineto(113.6601643,56.30026169)
\curveto(112.82813298,56.30026169)(112.25391418,56.45455858)(111.93750791,56.76315235)
\curveto(111.62110163,57.07565238)(111.4628985,57.64205867)(111.4628985,58.46237124)
\lineto(111.4628985,62.02487152)
\lineto(110.67188281,62.02487152)
\lineto(110.67188281,62.86276221)
\lineto(111.4628985,62.86276221)
\lineto(111.4628985,64.72604361)
\lineto(112.54688296,64.72604361)
\closepath
}
}
{
\newrgbcolor{curcolor}{0 0 0}
\pscustom[linestyle=none,fillstyle=solid,fillcolor=curcolor]
{
\newpath
\moveto(119.9941488,61.85494963)
\curveto(119.87305504,61.92526214)(119.74024253,61.97604339)(119.59571127,62.00729339)
\curveto(119.45508626,62.04244965)(119.29883624,62.06002777)(119.12696123,62.06002777)
\curveto(118.51758618,62.06002777)(118.04883614,61.86080901)(117.72071112,61.46237148)
\curveto(117.39649234,61.06784019)(117.23438295,60.49948077)(117.23438295,59.75729322)
\lineto(117.23438295,56.30026169)
\lineto(116.15039849,56.30026169)
\lineto(116.15039849,62.86276221)
\lineto(117.23438295,62.86276221)
\lineto(117.23438295,61.84323088)
\curveto(117.46094547,62.24166841)(117.75586737,62.53659031)(118.11914865,62.72799658)
\curveto(118.48242993,62.92330909)(118.92383621,63.02096535)(119.44336751,63.02096535)
\curveto(119.51758626,63.02096535)(119.59961752,63.01510597)(119.68946128,63.00338722)
\curveto(119.77930503,62.99557472)(119.87891442,62.98190285)(119.98828942,62.9623716)
\lineto(119.9941488,61.85494963)
\closepath
}
}
{
\newrgbcolor{curcolor}{0 0 0}
\pscustom[linestyle=none,fillstyle=solid,fillcolor=curcolor]
{
\newpath
\moveto(123.43360182,62.10690278)
\curveto(122.85547678,62.10690278)(122.39844549,61.88034026)(122.06250797,61.42721522)
\curveto(121.72657044,60.97799644)(121.55860168,60.36080889)(121.55860168,59.57565258)
\curveto(121.55860168,58.79049626)(121.72461731,58.17135559)(122.05664859,57.71823055)
\curveto(122.39258612,57.26901177)(122.85157053,57.04440237)(123.43360182,57.04440237)
\curveto(124.00782062,57.04440237)(124.46289878,57.27096489)(124.79883631,57.72408993)
\curveto(125.13477383,58.17721496)(125.3027426,58.79440251)(125.3027426,59.57565258)
\curveto(125.3027426,60.35299639)(125.13477383,60.96823081)(124.79883631,61.42135585)
\curveto(124.46289878,61.87838713)(124.00782062,62.10690278)(123.43360182,62.10690278)
\closepath
\moveto(123.43360182,63.02096535)
\curveto(124.3711019,63.02096535)(125.10743008,62.71627783)(125.64258637,62.10690278)
\curveto(126.17774267,61.49752773)(126.44532081,60.65377766)(126.44532081,59.57565258)
\curveto(126.44532081,58.50143374)(126.17774267,57.65768367)(125.64258637,57.04440237)
\curveto(125.10743008,56.43502733)(124.3711019,56.1303398)(123.43360182,56.1303398)
\curveto(122.4921955,56.1303398)(121.75391419,56.43502733)(121.2187579,57.04440237)
\curveto(120.68750786,57.65768367)(120.42188283,58.50143374)(120.42188283,59.57565258)
\curveto(120.42188283,60.65377766)(120.68750786,61.49752773)(121.2187579,62.10690278)
\curveto(121.75391419,62.71627783)(122.4921955,63.02096535)(123.43360182,63.02096535)
\closepath
}
}
{
\newrgbcolor{curcolor}{0 0 0}
\pscustom[linestyle=none,fillstyle=solid,fillcolor=curcolor]
{
\newpath
\moveto(128.23828912,65.41744992)
\lineto(129.31641421,65.41744992)
\lineto(129.31641421,56.30026169)
\lineto(128.23828912,56.30026169)
\lineto(128.23828912,65.41744992)
\closepath
}
}
{
\newrgbcolor{curcolor}{0 0 0}
\pscustom[linestyle=none,fillstyle=solid,fillcolor=curcolor]
{
\newpath
\moveto(134.54884386,59.59909008)
\curveto(133.67775004,59.59909008)(133.07423437,59.49948069)(132.73829684,59.30026193)
\curveto(132.40235932,59.10104316)(132.23439055,58.76119939)(132.23439055,58.2807306)
\curveto(132.23439055,57.89791807)(132.35939056,57.59323054)(132.60939058,57.36666803)
\curveto(132.86329685,57.14401176)(133.20704688,57.03268362)(133.64064067,57.03268362)
\curveto(134.23829696,57.03268362)(134.71681263,57.24362114)(135.07618766,57.66549617)
\curveto(135.43946893,58.09127746)(135.62110957,58.65573063)(135.62110957,59.35885568)
\lineto(135.62110957,59.59909008)
\lineto(134.54884386,59.59909008)
\closepath
\moveto(136.69923466,60.04440261)
\lineto(136.69923466,56.30026169)
\lineto(135.62110957,56.30026169)
\lineto(135.62110957,57.29635552)
\curveto(135.3750158,56.89791799)(135.06837515,56.60299609)(134.70118763,56.41158982)
\curveto(134.3340001,56.22408981)(133.88478131,56.1303398)(133.35353127,56.1303398)
\curveto(132.68165622,56.1303398)(132.14649992,56.31783982)(131.74806239,56.69283985)
\curveto(131.35353111,57.07174613)(131.15626547,57.57760554)(131.15626547,58.21041809)
\curveto(131.15626547,58.9486994)(131.40235924,59.50534007)(131.89454678,59.8803401)
\curveto(132.39064057,60.25534013)(133.12892188,60.44284014)(134.1093907,60.44284014)
\lineto(135.62110957,60.44284014)
\lineto(135.62110957,60.5483089)
\curveto(135.62110957,61.04440269)(135.45704706,61.42721522)(135.12892203,61.69674649)
\curveto(134.80470326,61.97018402)(134.34767197,62.10690278)(133.75782818,62.10690278)
\curveto(133.38282815,62.10690278)(133.01759374,62.0619809)(132.66212496,61.97213714)
\curveto(132.30665619,61.88229338)(131.96485928,61.74752775)(131.63673426,61.56784023)
\lineto(131.63673426,62.56393406)
\curveto(132.03126554,62.71627783)(132.41407807,62.82955908)(132.78517185,62.90377784)
\curveto(133.15626563,62.98190285)(133.51759378,63.02096535)(133.86915631,63.02096535)
\curveto(134.81837513,63.02096535)(135.52735957,62.77487158)(135.9961096,62.28268404)
\curveto(136.46485964,61.7904965)(136.69923466,61.04440269)(136.69923466,60.04440261)
\closepath
}
}
{
\newrgbcolor{curcolor}{0 0 0}
\pscustom[linestyle=none,fillstyle=solid,fillcolor=curcolor]
{
\newpath
\moveto(142.74610925,65.41744992)
\lineto(143.82423434,65.41744992)
\lineto(143.82423434,56.30026169)
\lineto(142.74610925,56.30026169)
\lineto(142.74610925,65.41744992)
\closepath
}
}
{
\newrgbcolor{curcolor}{0 0 0}
\pscustom[linestyle=none,fillstyle=solid,fillcolor=curcolor]
{
\newpath
\moveto(149.05665636,59.59909008)
\curveto(148.18556254,59.59909008)(147.58204687,59.49948069)(147.24610934,59.30026193)
\curveto(146.91017182,59.10104316)(146.74220305,58.76119939)(146.74220305,58.2807306)
\curveto(146.74220305,57.89791807)(146.86720306,57.59323054)(147.11720308,57.36666803)
\curveto(147.37110935,57.14401176)(147.71485938,57.03268362)(148.14845317,57.03268362)
\curveto(148.74610946,57.03268362)(149.22462513,57.24362114)(149.58400016,57.66549617)
\curveto(149.94728143,58.09127746)(150.12892207,58.65573063)(150.12892207,59.35885568)
\lineto(150.12892207,59.59909008)
\lineto(149.05665636,59.59909008)
\closepath
\moveto(151.20704716,60.04440261)
\lineto(151.20704716,56.30026169)
\lineto(150.12892207,56.30026169)
\lineto(150.12892207,57.29635552)
\curveto(149.8828283,56.89791799)(149.57618765,56.60299609)(149.20900013,56.41158982)
\curveto(148.8418126,56.22408981)(148.39259381,56.1303398)(147.86134377,56.1303398)
\curveto(147.18946872,56.1303398)(146.65431242,56.31783982)(146.25587489,56.69283985)
\curveto(145.86134361,57.07174613)(145.66407797,57.57760554)(145.66407797,58.21041809)
\curveto(145.66407797,58.9486994)(145.91017174,59.50534007)(146.40235928,59.8803401)
\curveto(146.89845307,60.25534013)(147.63673438,60.44284014)(148.6172032,60.44284014)
\lineto(150.12892207,60.44284014)
\lineto(150.12892207,60.5483089)
\curveto(150.12892207,61.04440269)(149.96485956,61.42721522)(149.63673453,61.69674649)
\curveto(149.31251576,61.97018402)(148.85548447,62.10690278)(148.26564068,62.10690278)
\curveto(147.89064065,62.10690278)(147.52540624,62.0619809)(147.16993746,61.97213714)
\curveto(146.81446869,61.88229338)(146.47267178,61.74752775)(146.14454676,61.56784023)
\lineto(146.14454676,62.56393406)
\curveto(146.53907804,62.71627783)(146.92189057,62.82955908)(147.29298435,62.90377784)
\curveto(147.66407813,62.98190285)(148.02540628,63.02096535)(148.37696881,63.02096535)
\curveto(149.32618763,63.02096535)(150.03517207,62.77487158)(150.5039221,62.28268404)
\curveto(150.97267214,61.7904965)(151.20704716,61.04440269)(151.20704716,60.04440261)
\closepath
}
}
{
\newrgbcolor{curcolor}{0 0 0}
\pscustom[linestyle=none,fillstyle=solid,fillcolor=curcolor]
{
\newpath
\moveto(156.48048419,62.86276221)
\lineto(157.62306241,62.86276221)
\lineto(159.67384382,57.35494927)
\lineto(161.72462523,62.86276221)
\lineto(162.86720345,62.86276221)
\lineto(160.40626575,56.30026169)
\lineto(158.94142189,56.30026169)
\lineto(156.48048419,62.86276221)
\closepath
}
}
{
\newrgbcolor{curcolor}{0 0 0}
\pscustom[linestyle=none,fillstyle=solid,fillcolor=curcolor]
{
\newpath
\moveto(169.96876595,59.85104322)
\lineto(169.96876595,59.32369943)
\lineto(165.0117343,59.32369943)
\curveto(165.05860931,58.58151187)(165.28126558,58.01510558)(165.67970311,57.62448055)
\curveto(166.08204689,57.23776177)(166.64064068,57.04440237)(167.35548449,57.04440237)
\curveto(167.76954702,57.04440237)(168.16993768,57.09518363)(168.55665646,57.19674614)
\curveto(168.94728149,57.29830865)(169.33400027,57.45065241)(169.7168128,57.65377742)
\lineto(169.7168128,56.63424609)
\curveto(169.33009402,56.47018358)(168.93360962,56.34518357)(168.52735958,56.25924606)
\curveto(168.12110955,56.17330856)(167.70900014,56.1303398)(167.29103136,56.1303398)
\curveto(166.24415628,56.1303398)(165.41407809,56.43502733)(164.80079679,57.04440237)
\curveto(164.19142174,57.65377742)(163.88673421,58.47799624)(163.88673421,59.51705882)
\curveto(163.88673421,60.59127766)(164.17579674,61.44284022)(164.75392178,62.07174652)
\curveto(165.33595308,62.70455907)(166.11915627,63.02096535)(167.10353134,63.02096535)
\curveto(167.98634392,63.02096535)(168.6836096,62.73580908)(169.19532839,62.16549653)
\curveto(169.71095343,61.59909024)(169.96876595,60.8276058)(169.96876595,59.85104322)
\closepath
\moveto(168.89064086,60.1674495)
\curveto(168.88282836,60.75729329)(168.71681272,61.22799646)(168.39259395,61.57955899)
\curveto(168.07228142,61.93112151)(167.64650014,62.10690278)(167.1152501,62.10690278)
\curveto(166.51368755,62.10690278)(166.03126563,61.93698089)(165.66798436,61.59713711)
\curveto(165.30860933,61.25729333)(165.10157806,60.77877767)(165.04689056,60.16159012)
\lineto(168.89064086,60.1674495)
\closepath
}
}
{
\newrgbcolor{curcolor}{0 0 0}
\pscustom[linestyle=none,fillstyle=solid,fillcolor=curcolor]
{
\newpath
\moveto(171.73829675,65.41744992)
\lineto(172.81642184,65.41744992)
\lineto(172.81642184,56.30026169)
\lineto(171.73829675,56.30026169)
\lineto(171.73829675,65.41744992)
\closepath
}
}
{
\newrgbcolor{curcolor}{0 0 0}
\pscustom[linestyle=none,fillstyle=solid,fillcolor=curcolor]
{
\newpath
\moveto(177.6093907,62.10690278)
\curveto(177.03126566,62.10690278)(176.57423437,61.88034026)(176.23829684,61.42721522)
\curveto(175.90235932,60.97799644)(175.73439055,60.36080889)(175.73439055,59.57565258)
\curveto(175.73439055,58.79049626)(175.90040619,58.17135559)(176.23243747,57.71823055)
\curveto(176.568375,57.26901177)(177.02735941,57.04440237)(177.6093907,57.04440237)
\curveto(178.1836095,57.04440237)(178.63868766,57.27096489)(178.97462519,57.72408993)
\curveto(179.31056271,58.17721496)(179.47853148,58.79440251)(179.47853148,59.57565258)
\curveto(179.47853148,60.35299639)(179.31056271,60.96823081)(178.97462519,61.42135585)
\curveto(178.63868766,61.87838713)(178.1836095,62.10690278)(177.6093907,62.10690278)
\closepath
\moveto(177.6093907,63.02096535)
\curveto(178.54689078,63.02096535)(179.28321896,62.71627783)(179.81837525,62.10690278)
\curveto(180.35353155,61.49752773)(180.62110969,60.65377766)(180.62110969,59.57565258)
\curveto(180.62110969,58.50143374)(180.35353155,57.65768367)(179.81837525,57.04440237)
\curveto(179.28321896,56.43502733)(178.54689078,56.1303398)(177.6093907,56.1303398)
\curveto(176.66798438,56.1303398)(175.92970307,56.43502733)(175.39454678,57.04440237)
\curveto(174.86329674,57.65768367)(174.59767171,58.50143374)(174.59767171,59.57565258)
\curveto(174.59767171,60.65377766)(174.86329674,61.49752773)(175.39454678,62.10690278)
\curveto(175.92970307,62.71627783)(176.66798438,63.02096535)(177.6093907,63.02096535)
\closepath
}
}
{
\newrgbcolor{curcolor}{0 0 0}
\pscustom[linestyle=none,fillstyle=solid,fillcolor=curcolor]
{
\newpath
\moveto(187.12501588,62.61080907)
\lineto(187.12501588,61.60299649)
\curveto(186.82032835,61.77096525)(186.5136877,61.89596526)(186.20509393,61.97799652)
\curveto(185.9004064,62.06393402)(185.59181263,62.10690278)(185.27931261,62.10690278)
\curveto(184.5800938,62.10690278)(184.03712501,61.88424651)(183.65040623,61.43893397)
\curveto(183.26368744,60.99752769)(183.07032805,60.37643389)(183.07032805,59.57565258)
\curveto(183.07032805,58.77487126)(183.26368744,58.15182434)(183.65040623,57.7065118)
\curveto(184.03712501,57.26510552)(184.5800938,57.04440237)(185.27931261,57.04440237)
\curveto(185.59181263,57.04440237)(185.9004064,57.085418)(186.20509393,57.16744926)
\curveto(186.5136877,57.25338677)(186.82032835,57.3803399)(187.12501588,57.54830866)
\lineto(187.12501588,56.55221484)
\curveto(186.8242346,56.41158982)(186.51173458,56.30612107)(186.1875158,56.23580856)
\curveto(185.86720328,56.16549606)(185.52540637,56.1303398)(185.1621251,56.1303398)
\curveto(184.17384377,56.1303398)(183.38868745,56.4408867)(182.80665616,57.0619805)
\curveto(182.22462486,57.6830743)(181.93360921,58.52096499)(181.93360921,59.57565258)
\curveto(181.93360921,60.64596516)(182.22657799,61.4877621)(182.81251553,62.1010434)
\curveto(183.40235933,62.7143247)(184.20900002,63.02096535)(185.2324376,63.02096535)
\curveto(185.56446888,63.02096535)(185.88868765,62.9858091)(186.20509393,62.91549659)
\curveto(186.5215002,62.84909034)(186.82814085,62.74752783)(187.12501588,62.61080907)
\closepath
}
}
{
\newrgbcolor{curcolor}{0 0 0}
\pscustom[linestyle=none,fillstyle=solid,fillcolor=curcolor]
{
\newpath
\moveto(189.01173425,62.86276221)
\lineto(190.08985934,62.86276221)
\lineto(190.08985934,56.30026169)
\lineto(189.01173425,56.30026169)
\lineto(189.01173425,62.86276221)
\closepath
\moveto(189.01173425,65.41744992)
\lineto(190.08985934,65.41744992)
\lineto(190.08985934,64.05221543)
\lineto(189.01173425,64.05221543)
\lineto(189.01173425,65.41744992)
\closepath
}
}
{
\newrgbcolor{curcolor}{0 0 0}
\pscustom[linestyle=none,fillstyle=solid,fillcolor=curcolor]
{
\newpath
\moveto(196.65821897,61.86666838)
\lineto(196.65821897,65.41744992)
\lineto(197.73634406,65.41744992)
\lineto(197.73634406,56.30026169)
\lineto(196.65821897,56.30026169)
\lineto(196.65821897,57.28463677)
\curveto(196.43165645,56.89401174)(196.14454705,56.60299609)(195.79689078,56.41158982)
\curveto(195.45314075,56.22408981)(195.03907822,56.1303398)(194.55470318,56.1303398)
\curveto(193.76173436,56.1303398)(193.11524994,56.44674608)(192.6152499,57.07955863)
\curveto(192.11915611,57.71237118)(191.87110921,58.54440249)(191.87110921,59.57565258)
\curveto(191.87110921,60.60690266)(192.11915611,61.43893397)(192.6152499,62.07174652)
\curveto(193.11524994,62.70455907)(193.76173436,63.02096535)(194.55470318,63.02096535)
\curveto(195.03907822,63.02096535)(195.45314075,62.92526222)(195.79689078,62.73385595)
\curveto(196.14454705,62.54635594)(196.43165645,62.25729341)(196.65821897,61.86666838)
\closepath
\moveto(192.98439055,59.57565258)
\curveto(192.98439055,58.78268376)(193.14649994,58.15963684)(193.47071872,57.7065118)
\curveto(193.79884374,57.25729302)(194.24806253,57.03268362)(194.81837507,57.03268362)
\curveto(195.38868762,57.03268362)(195.8379064,57.25729302)(196.16603143,57.7065118)
\curveto(196.49415646,58.15963684)(196.65821897,58.78268376)(196.65821897,59.57565258)
\curveto(196.65821897,60.36862139)(196.49415646,60.98971519)(196.16603143,61.43893397)
\curveto(195.8379064,61.89205901)(195.38868762,62.11862153)(194.81837507,62.11862153)
\curveto(194.24806253,62.11862153)(193.79884374,61.89205901)(193.47071872,61.43893397)
\curveto(193.14649994,60.98971519)(192.98439055,60.36862139)(192.98439055,59.57565258)
\closepath
}
}
{
\newrgbcolor{curcolor}{0 0 0}
\pscustom[linestyle=none,fillstyle=solid,fillcolor=curcolor]
{
\newpath
\moveto(202.93946886,59.59909008)
\curveto(202.06837504,59.59909008)(201.46485937,59.49948069)(201.12892184,59.30026193)
\curveto(200.79298432,59.10104316)(200.62501555,58.76119939)(200.62501555,58.2807306)
\curveto(200.62501555,57.89791807)(200.75001556,57.59323054)(201.00001558,57.36666803)
\curveto(201.25392185,57.14401176)(201.59767188,57.03268362)(202.03126567,57.03268362)
\curveto(202.62892196,57.03268362)(203.10743763,57.24362114)(203.46681266,57.66549617)
\curveto(203.83009393,58.09127746)(204.01173457,58.65573063)(204.01173457,59.35885568)
\lineto(204.01173457,59.59909008)
\lineto(202.93946886,59.59909008)
\closepath
\moveto(205.08985966,60.04440261)
\lineto(205.08985966,56.30026169)
\lineto(204.01173457,56.30026169)
\lineto(204.01173457,57.29635552)
\curveto(203.7656408,56.89791799)(203.45900015,56.60299609)(203.09181263,56.41158982)
\curveto(202.7246251,56.22408981)(202.27540631,56.1303398)(201.74415627,56.1303398)
\curveto(201.07228122,56.1303398)(200.53712492,56.31783982)(200.13868739,56.69283985)
\curveto(199.74415611,57.07174613)(199.54689047,57.57760554)(199.54689047,58.21041809)
\curveto(199.54689047,58.9486994)(199.79298424,59.50534007)(200.28517178,59.8803401)
\curveto(200.78126557,60.25534013)(201.51954688,60.44284014)(202.5000157,60.44284014)
\lineto(204.01173457,60.44284014)
\lineto(204.01173457,60.5483089)
\curveto(204.01173457,61.04440269)(203.84767206,61.42721522)(203.51954703,61.69674649)
\curveto(203.19532826,61.97018402)(202.73829697,62.10690278)(202.14845318,62.10690278)
\curveto(201.77345315,62.10690278)(201.40821874,62.0619809)(201.05274996,61.97213714)
\curveto(200.69728119,61.88229338)(200.35548428,61.74752775)(200.02735926,61.56784023)
\lineto(200.02735926,62.56393406)
\curveto(200.42189054,62.71627783)(200.80470307,62.82955908)(201.17579685,62.90377784)
\curveto(201.54689063,62.98190285)(201.90821878,63.02096535)(202.25978131,63.02096535)
\curveto(203.20900013,63.02096535)(203.91798457,62.77487158)(204.3867346,62.28268404)
\curveto(204.85548464,61.7904965)(205.08985966,61.04440269)(205.08985966,60.04440261)
\closepath
}
}
{
\newrgbcolor{curcolor}{0 0 0}
\pscustom[linestyle=none,fillstyle=solid,fillcolor=curcolor]
{
\newpath
\moveto(211.63478147,61.86666838)
\lineto(211.63478147,65.41744992)
\lineto(212.71290656,65.41744992)
\lineto(212.71290656,56.30026169)
\lineto(211.63478147,56.30026169)
\lineto(211.63478147,57.28463677)
\curveto(211.40821895,56.89401174)(211.12110955,56.60299609)(210.77345328,56.41158982)
\curveto(210.42970325,56.22408981)(210.01564072,56.1303398)(209.53126568,56.1303398)
\curveto(208.73829686,56.1303398)(208.09181244,56.44674608)(207.5918124,57.07955863)
\curveto(207.09571861,57.71237118)(206.84767171,58.54440249)(206.84767171,59.57565258)
\curveto(206.84767171,60.60690266)(207.09571861,61.43893397)(207.5918124,62.07174652)
\curveto(208.09181244,62.70455907)(208.73829686,63.02096535)(209.53126568,63.02096535)
\curveto(210.01564072,63.02096535)(210.42970325,62.92526222)(210.77345328,62.73385595)
\curveto(211.12110955,62.54635594)(211.40821895,62.25729341)(211.63478147,61.86666838)
\closepath
\moveto(207.96095305,59.57565258)
\curveto(207.96095305,58.78268376)(208.12306244,58.15963684)(208.44728122,57.7065118)
\curveto(208.77540624,57.25729302)(209.22462503,57.03268362)(209.79493757,57.03268362)
\curveto(210.36525012,57.03268362)(210.8144689,57.25729302)(211.14259393,57.7065118)
\curveto(211.47071896,58.15963684)(211.63478147,58.78268376)(211.63478147,59.57565258)
\curveto(211.63478147,60.36862139)(211.47071896,60.98971519)(211.14259393,61.43893397)
\curveto(210.8144689,61.89205901)(210.36525012,62.11862153)(209.79493757,62.11862153)
\curveto(209.22462503,62.11862153)(208.77540624,61.89205901)(208.44728122,61.43893397)
\curveto(208.12306244,60.98971519)(207.96095305,60.36862139)(207.96095305,59.57565258)
\closepath
}
}
{
\newrgbcolor{curcolor}{0 0 0}
\pscustom[linestyle=none,fillstyle=solid,fillcolor=curcolor]
{
\newpath
\moveto(223.07228147,61.86666838)
\lineto(223.07228147,65.41744992)
\lineto(224.15040656,65.41744992)
\lineto(224.15040656,56.30026169)
\lineto(223.07228147,56.30026169)
\lineto(223.07228147,57.28463677)
\curveto(222.84571895,56.89401174)(222.55860955,56.60299609)(222.21095328,56.41158982)
\curveto(221.86720325,56.22408981)(221.45314072,56.1303398)(220.96876568,56.1303398)
\curveto(220.17579686,56.1303398)(219.52931244,56.44674608)(219.0293124,57.07955863)
\curveto(218.53321861,57.71237118)(218.28517171,58.54440249)(218.28517171,59.57565258)
\curveto(218.28517171,60.60690266)(218.53321861,61.43893397)(219.0293124,62.07174652)
\curveto(219.52931244,62.70455907)(220.17579686,63.02096535)(220.96876568,63.02096535)
\curveto(221.45314072,63.02096535)(221.86720325,62.92526222)(222.21095328,62.73385595)
\curveto(222.55860955,62.54635594)(222.84571895,62.25729341)(223.07228147,61.86666838)
\closepath
\moveto(219.39845305,59.57565258)
\curveto(219.39845305,58.78268376)(219.56056244,58.15963684)(219.88478122,57.7065118)
\curveto(220.21290624,57.25729302)(220.66212503,57.03268362)(221.23243757,57.03268362)
\curveto(221.80275012,57.03268362)(222.2519689,57.25729302)(222.58009393,57.7065118)
\curveto(222.90821896,58.15963684)(223.07228147,58.78268376)(223.07228147,59.57565258)
\curveto(223.07228147,60.36862139)(222.90821896,60.98971519)(222.58009393,61.43893397)
\curveto(222.2519689,61.89205901)(221.80275012,62.11862153)(221.23243757,62.11862153)
\curveto(220.66212503,62.11862153)(220.21290624,61.89205901)(219.88478122,61.43893397)
\curveto(219.56056244,60.98971519)(219.39845305,60.36862139)(219.39845305,59.57565258)
\closepath
}
}
{
\newrgbcolor{curcolor}{0 0 0}
\pscustom[linestyle=none,fillstyle=solid,fillcolor=curcolor]
{
\newpath
\moveto(231.98439095,59.85104322)
\lineto(231.98439095,59.32369943)
\lineto(227.0273593,59.32369943)
\curveto(227.07423431,58.58151187)(227.29689058,58.01510558)(227.69532811,57.62448055)
\curveto(228.09767189,57.23776177)(228.65626568,57.04440237)(229.37110949,57.04440237)
\curveto(229.78517202,57.04440237)(230.18556268,57.09518363)(230.57228146,57.19674614)
\curveto(230.96290649,57.29830865)(231.34962527,57.45065241)(231.7324378,57.65377742)
\lineto(231.7324378,56.63424609)
\curveto(231.34571902,56.47018358)(230.94923462,56.34518357)(230.54298458,56.25924606)
\curveto(230.13673455,56.17330856)(229.72462514,56.1303398)(229.30665636,56.1303398)
\curveto(228.25978128,56.1303398)(227.42970309,56.43502733)(226.81642179,57.04440237)
\curveto(226.20704674,57.65377742)(225.90235921,58.47799624)(225.90235921,59.51705882)
\curveto(225.90235921,60.59127766)(226.19142174,61.44284022)(226.76954678,62.07174652)
\curveto(227.35157808,62.70455907)(228.13478127,63.02096535)(229.11915634,63.02096535)
\curveto(230.00196892,63.02096535)(230.6992346,62.73580908)(231.21095339,62.16549653)
\curveto(231.72657843,61.59909024)(231.98439095,60.8276058)(231.98439095,59.85104322)
\closepath
\moveto(230.90626586,60.1674495)
\curveto(230.89845336,60.75729329)(230.73243772,61.22799646)(230.40821895,61.57955899)
\curveto(230.08790642,61.93112151)(229.66212514,62.10690278)(229.1308751,62.10690278)
\curveto(228.52931255,62.10690278)(228.04689063,61.93698089)(227.68360936,61.59713711)
\curveto(227.32423433,61.25729333)(227.11720306,60.77877767)(227.06251556,60.16159012)
\lineto(230.90626586,60.1674495)
\closepath
}
}
{
\newrgbcolor{curcolor}{0 0 0}
\pscustom[linestyle=none,fillstyle=solid,fillcolor=curcolor]
{
\newpath
\moveto(4.62890673,50.41744872)
\lineto(5.70703182,50.41744872)
\lineto(5.70703182,41.3002605)
\lineto(4.62890673,41.3002605)
\lineto(4.62890673,50.41744872)
\closepath
}
}
{
\newrgbcolor{curcolor}{0 0 0}
\pscustom[linestyle=none,fillstyle=solid,fillcolor=curcolor]
{
\newpath
\moveto(10.93945408,44.59908889)
\curveto(10.06836026,44.59908889)(9.46484459,44.4994795)(9.12890706,44.30026074)
\curveto(8.79296954,44.10104197)(8.62500077,43.76119819)(8.62500077,43.28072941)
\curveto(8.62500077,42.89791688)(8.75000078,42.59322935)(9.0000008,42.36666683)
\curveto(9.25390707,42.14401057)(9.5976571,42.03268243)(10.03125088,42.03268243)
\curveto(10.62890718,42.03268243)(11.10742284,42.24361995)(11.46679787,42.66549498)
\curveto(11.83007915,43.09127627)(12.01171979,43.65572944)(12.01171979,44.35885449)
\lineto(12.01171979,44.59908889)
\lineto(10.93945408,44.59908889)
\closepath
\moveto(13.08984488,45.04440142)
\lineto(13.08984488,41.3002605)
\lineto(12.01171979,41.3002605)
\lineto(12.01171979,42.29635433)
\curveto(11.76562602,41.8979168)(11.45898537,41.6029949)(11.09179784,41.41158863)
\curveto(10.72461031,41.22408862)(10.27539153,41.13033861)(9.74414149,41.13033861)
\curveto(9.07226643,41.13033861)(8.53711014,41.31783863)(8.13867261,41.69283865)
\curveto(7.74414133,42.07174493)(7.54687569,42.57760435)(7.54687569,43.2104169)
\curveto(7.54687569,43.94869821)(7.79296946,44.50533888)(8.285157,44.88033891)
\curveto(8.78125079,45.25533894)(9.51953209,45.44283895)(10.50000092,45.44283895)
\lineto(12.01171979,45.44283895)
\lineto(12.01171979,45.54830771)
\curveto(12.01171979,46.0444015)(11.84765728,46.42721403)(11.51953225,46.6967453)
\curveto(11.19531348,46.97018282)(10.73828219,47.10690159)(10.14843839,47.10690159)
\curveto(9.77343836,47.10690159)(9.40820396,47.06197971)(9.05273518,46.97213595)
\curveto(8.6972664,46.88229219)(8.3554695,46.74752656)(8.02734448,46.56783904)
\lineto(8.02734448,47.56393287)
\curveto(8.42187576,47.71627663)(8.80468829,47.82955789)(9.17578207,47.90377665)
\curveto(9.54687585,47.98190165)(9.908204,48.02096416)(10.25976653,48.02096416)
\curveto(11.20898535,48.02096416)(11.91796978,47.77487039)(12.38671982,47.28268285)
\curveto(12.85546986,46.79049531)(13.08984488,46.0444015)(13.08984488,45.04440142)
\closepath
}
}
{
\newrgbcolor{curcolor}{0 0 0}
\pscustom[linestyle=none,fillstyle=solid,fillcolor=curcolor]
{
\newpath
\moveto(23.84765877,44.57565138)
\curveto(23.84765877,45.3686202)(23.68359626,45.989714)(23.35547124,46.43893278)
\curveto(23.03125246,46.89205782)(22.5839868,47.11862034)(22.01367425,47.11862034)
\curveto(21.44336171,47.11862034)(20.99414292,46.89205782)(20.6660179,46.43893278)
\curveto(20.34179912,45.989714)(20.17968973,45.3686202)(20.17968973,44.57565138)
\curveto(20.17968973,43.78268257)(20.34179912,43.15963565)(20.6660179,42.70651061)
\curveto(20.99414292,42.25729182)(21.44336171,42.03268243)(22.01367425,42.03268243)
\curveto(22.5839868,42.03268243)(23.03125246,42.25729182)(23.35547124,42.70651061)
\curveto(23.68359626,43.15963565)(23.84765877,43.78268257)(23.84765877,44.57565138)
\closepath
\moveto(20.17968973,46.86666719)
\curveto(20.40625225,47.25729222)(20.69140852,47.54635475)(21.03515855,47.73385476)
\curveto(21.38281483,47.92526103)(21.79687736,48.02096416)(22.27734615,48.02096416)
\curveto(23.07422121,48.02096416)(23.72070564,47.70455788)(24.21679943,47.07174533)
\curveto(24.71679947,46.43893278)(24.96679949,45.60690147)(24.96679949,44.57565138)
\curveto(24.96679949,43.5444013)(24.71679947,42.71236999)(24.21679943,42.07955744)
\curveto(23.72070564,41.44674489)(23.07422121,41.13033861)(22.27734615,41.13033861)
\curveto(21.79687736,41.13033861)(21.38281483,41.22408862)(21.03515855,41.41158863)
\curveto(20.69140852,41.6029949)(20.40625225,41.89401055)(20.17968973,42.28463558)
\lineto(20.17968973,41.3002605)
\lineto(19.09570527,41.3002605)
\lineto(19.09570527,50.41744872)
\lineto(20.17968973,50.41744872)
\lineto(20.17968973,46.86666719)
\closepath
}
}
{
\newrgbcolor{curcolor}{0 0 0}
\pscustom[linestyle=none,fillstyle=solid,fillcolor=curcolor]
{
\newpath
\moveto(29.29687735,47.10690159)
\curveto(28.71875231,47.10690159)(28.26172102,46.88033907)(27.92578349,46.42721403)
\curveto(27.58984597,45.97799525)(27.4218772,45.3608077)(27.4218772,44.57565138)
\curveto(27.4218772,43.79049507)(27.58789284,43.1713544)(27.91992412,42.71822936)
\curveto(28.25586164,42.26901058)(28.71484606,42.04440118)(29.29687735,42.04440118)
\curveto(29.87109615,42.04440118)(30.32617431,42.2709637)(30.66211184,42.72408874)
\curveto(30.99804936,43.17721377)(31.16601813,43.79440132)(31.16601813,44.57565138)
\curveto(31.16601813,45.3529952)(30.99804936,45.96822962)(30.66211184,46.42135466)
\curveto(30.32617431,46.87838594)(29.87109615,47.10690159)(29.29687735,47.10690159)
\closepath
\moveto(29.29687735,48.02096416)
\curveto(30.23437743,48.02096416)(30.97070561,47.71627663)(31.5058619,47.10690159)
\curveto(32.0410182,46.49752654)(32.30859634,45.65377647)(32.30859634,44.57565138)
\curveto(32.30859634,43.50143255)(32.0410182,42.65768248)(31.5058619,42.04440118)
\curveto(30.97070561,41.43502613)(30.23437743,41.13033861)(29.29687735,41.13033861)
\curveto(28.35547103,41.13033861)(27.61718972,41.43502613)(27.08203343,42.04440118)
\curveto(26.55078338,42.65768248)(26.28515836,43.50143255)(26.28515836,44.57565138)
\curveto(26.28515836,45.65377647)(26.55078338,46.49752654)(27.08203343,47.10690159)
\curveto(27.61718972,47.71627663)(28.35547103,48.02096416)(29.29687735,48.02096416)
\closepath
}
}
{
\newrgbcolor{curcolor}{0 0 0}
\pscustom[linestyle=none,fillstyle=solid,fillcolor=curcolor]
{
\newpath
\moveto(39.19922321,46.6029953)
\curveto(39.46875448,47.08737033)(39.79102014,47.44479224)(40.16602017,47.67526101)
\curveto(40.5410202,47.90572977)(40.98242648,48.02096416)(41.49023902,48.02096416)
\curveto(42.17383282,48.02096416)(42.70117662,47.78072976)(43.0722704,47.30026098)
\curveto(43.44336418,46.82369844)(43.62891107,46.14401088)(43.62891107,45.26119831)
\lineto(43.62891107,41.3002605)
\lineto(42.5449266,41.3002605)
\lineto(42.5449266,45.22604206)
\curveto(42.5449266,45.85494836)(42.43359847,46.32174527)(42.2109422,46.6264328)
\curveto(41.98828594,46.93112032)(41.64844216,47.08346408)(41.19141087,47.08346408)
\curveto(40.63281708,47.08346408)(40.19141079,46.89791719)(39.86719202,46.52682341)
\curveto(39.54297324,46.15572963)(39.38086385,45.64987022)(39.38086385,45.00924517)
\lineto(39.38086385,41.3002605)
\lineto(38.29687939,41.3002605)
\lineto(38.29687939,45.22604206)
\curveto(38.29687939,45.85885461)(38.18555126,46.32565152)(37.96289499,46.6264328)
\curveto(37.74023872,46.93112032)(37.3964887,47.08346408)(36.93164491,47.08346408)
\curveto(36.38086361,47.08346408)(35.94336358,46.89596407)(35.6191448,46.52096404)
\curveto(35.29492603,46.14987026)(35.13281664,45.64596397)(35.13281664,45.00924517)
\lineto(35.13281664,41.3002605)
\lineto(34.04883218,41.3002605)
\lineto(34.04883218,47.86276102)
\lineto(35.13281664,47.86276102)
\lineto(35.13281664,46.84322969)
\curveto(35.37891041,47.24557347)(35.67383231,47.54244849)(36.01758234,47.73385476)
\curveto(36.36133236,47.92526103)(36.76953552,48.02096416)(37.24219181,48.02096416)
\curveto(37.71875435,48.02096416)(38.12305125,47.8998704)(38.45508253,47.65768288)
\curveto(38.79102006,47.41549536)(39.03906695,47.06393283)(39.19922321,46.6029953)
\closepath
}
}
{
\newrgbcolor{curcolor}{0 0 0}
\pscustom[linestyle=none,fillstyle=solid,fillcolor=curcolor]
{
\newpath
\moveto(50.49609818,44.57565138)
\curveto(50.49609818,45.3686202)(50.33203567,45.989714)(50.00391064,46.43893278)
\curveto(49.67969187,46.89205782)(49.23242621,47.11862034)(48.66211366,47.11862034)
\curveto(48.09180112,47.11862034)(47.64258233,46.89205782)(47.3144573,46.43893278)
\curveto(46.99023853,45.989714)(46.82812914,45.3686202)(46.82812914,44.57565138)
\curveto(46.82812914,43.78268257)(46.99023853,43.15963565)(47.3144573,42.70651061)
\curveto(47.64258233,42.25729182)(48.09180112,42.03268243)(48.66211366,42.03268243)
\curveto(49.23242621,42.03268243)(49.67969187,42.25729182)(50.00391064,42.70651061)
\curveto(50.33203567,43.15963565)(50.49609818,43.78268257)(50.49609818,44.57565138)
\closepath
\moveto(46.82812914,46.86666719)
\curveto(47.05469166,47.25729222)(47.33984793,47.54635475)(47.68359796,47.73385476)
\curveto(48.03125424,47.92526103)(48.44531677,48.02096416)(48.92578556,48.02096416)
\curveto(49.72266062,48.02096416)(50.36914505,47.70455788)(50.86523884,47.07174533)
\curveto(51.36523888,46.43893278)(51.6152389,45.60690147)(51.6152389,44.57565138)
\curveto(51.6152389,43.5444013)(51.36523888,42.71236999)(50.86523884,42.07955744)
\curveto(50.36914505,41.44674489)(49.72266062,41.13033861)(48.92578556,41.13033861)
\curveto(48.44531677,41.13033861)(48.03125424,41.22408862)(47.68359796,41.41158863)
\curveto(47.33984793,41.6029949)(47.05469166,41.89401055)(46.82812914,42.28463558)
\lineto(46.82812914,41.3002605)
\lineto(45.74414468,41.3002605)
\lineto(45.74414468,50.41744872)
\lineto(46.82812914,50.41744872)
\lineto(46.82812914,46.86666719)
\closepath
}
}
{
\newrgbcolor{curcolor}{0 0 0}
\pscustom[linestyle=none,fillstyle=solid,fillcolor=curcolor]
{
\newpath
\moveto(56.38476992,44.59908889)
\curveto(55.5136761,44.59908889)(54.91016043,44.4994795)(54.5742229,44.30026074)
\curveto(54.23828537,44.10104197)(54.07031661,43.76119819)(54.07031661,43.28072941)
\curveto(54.07031661,42.89791688)(54.19531662,42.59322935)(54.44531664,42.36666683)
\curveto(54.69922291,42.14401057)(55.04297294,42.03268243)(55.47656672,42.03268243)
\curveto(56.07422302,42.03268243)(56.55273868,42.24361995)(56.91211371,42.66549498)
\curveto(57.27539499,43.09127627)(57.45703563,43.65572944)(57.45703563,44.35885449)
\lineto(57.45703563,44.59908889)
\lineto(56.38476992,44.59908889)
\closepath
\moveto(58.53516072,45.04440142)
\lineto(58.53516072,41.3002605)
\lineto(57.45703563,41.3002605)
\lineto(57.45703563,42.29635433)
\curveto(57.21094186,41.8979168)(56.90430121,41.6029949)(56.53711368,41.41158863)
\curveto(56.16992615,41.22408862)(55.72070737,41.13033861)(55.18945732,41.13033861)
\curveto(54.51758227,41.13033861)(53.98242598,41.31783863)(53.58398845,41.69283865)
\curveto(53.18945717,42.07174493)(52.99219152,42.57760435)(52.99219152,43.2104169)
\curveto(52.99219152,43.94869821)(53.23828529,44.50533888)(53.73047283,44.88033891)
\curveto(54.22656662,45.25533894)(54.96484793,45.44283895)(55.94531676,45.44283895)
\lineto(57.45703563,45.44283895)
\lineto(57.45703563,45.54830771)
\curveto(57.45703563,46.0444015)(57.29297312,46.42721403)(56.96484809,46.6967453)
\curveto(56.64062931,46.97018282)(56.18359803,47.10690159)(55.59375423,47.10690159)
\curveto(55.2187542,47.10690159)(54.8535198,47.06197971)(54.49805102,46.97213595)
\curveto(54.14258224,46.88229219)(53.80078534,46.74752656)(53.47266031,46.56783904)
\lineto(53.47266031,47.56393287)
\curveto(53.86719159,47.71627663)(54.25000412,47.82955789)(54.6210979,47.90377665)
\curveto(54.99219168,47.98190165)(55.35351984,48.02096416)(55.70508237,48.02096416)
\curveto(56.65430119,48.02096416)(57.36328562,47.77487039)(57.83203566,47.28268285)
\curveto(58.3007857,46.79049531)(58.53516072,46.0444015)(58.53516072,45.04440142)
\closepath
}
}
{
\newrgbcolor{curcolor}{0 0 0}
\pscustom[linestyle=none,fillstyle=solid,fillcolor=curcolor]
{
\newpath
\moveto(65.62500414,42.28463558)
\lineto(65.62500414,38.80416655)
\lineto(64.54101968,38.80416655)
\lineto(64.54101968,47.86276102)
\lineto(65.62500414,47.86276102)
\lineto(65.62500414,46.86666719)
\curveto(65.85156666,47.25729222)(66.13672293,47.54635475)(66.48047296,47.73385476)
\curveto(66.82812924,47.92526103)(67.24219177,48.02096416)(67.72266056,48.02096416)
\curveto(68.51953562,48.02096416)(69.16602005,47.70455788)(69.66211384,47.07174533)
\curveto(70.16211388,46.43893278)(70.4121139,45.60690147)(70.4121139,44.57565138)
\curveto(70.4121139,43.5444013)(70.16211388,42.71236999)(69.66211384,42.07955744)
\curveto(69.16602005,41.44674489)(68.51953562,41.13033861)(67.72266056,41.13033861)
\curveto(67.24219177,41.13033861)(66.82812924,41.22408862)(66.48047296,41.41158863)
\curveto(66.13672293,41.6029949)(65.85156666,41.89401055)(65.62500414,42.28463558)
\closepath
\moveto(69.29297318,44.57565138)
\curveto(69.29297318,45.3686202)(69.12891067,45.989714)(68.80078564,46.43893278)
\curveto(68.47656687,46.89205782)(68.02930121,47.11862034)(67.45898866,47.11862034)
\curveto(66.88867612,47.11862034)(66.43945733,46.89205782)(66.1113323,46.43893278)
\curveto(65.78711353,45.989714)(65.62500414,45.3686202)(65.62500414,44.57565138)
\curveto(65.62500414,43.78268257)(65.78711353,43.15963565)(66.1113323,42.70651061)
\curveto(66.43945733,42.25729182)(66.88867612,42.03268243)(67.45898866,42.03268243)
\curveto(68.02930121,42.03268243)(68.47656687,42.25729182)(68.80078564,42.70651061)
\curveto(69.12891067,43.15963565)(69.29297318,43.78268257)(69.29297318,44.57565138)
\closepath
}
}
{
\newrgbcolor{curcolor}{0 0 0}
\pscustom[linestyle=none,fillstyle=solid,fillcolor=curcolor]
{
\newpath
\moveto(75.18164873,44.59908889)
\curveto(74.31055491,44.59908889)(73.70703924,44.4994795)(73.37110172,44.30026074)
\curveto(73.03516419,44.10104197)(72.86719543,43.76119819)(72.86719543,43.28072941)
\curveto(72.86719543,42.89791688)(72.99219544,42.59322935)(73.24219545,42.36666683)
\curveto(73.49610173,42.14401057)(73.83985175,42.03268243)(74.27344554,42.03268243)
\curveto(74.87110183,42.03268243)(75.3496175,42.24361995)(75.70899253,42.66549498)
\curveto(76.0722738,43.09127627)(76.25391444,43.65572944)(76.25391444,44.35885449)
\lineto(76.25391444,44.59908889)
\lineto(75.18164873,44.59908889)
\closepath
\moveto(77.33203953,45.04440142)
\lineto(77.33203953,41.3002605)
\lineto(76.25391444,41.3002605)
\lineto(76.25391444,42.29635433)
\curveto(76.00782067,41.8979168)(75.70118003,41.6029949)(75.3339925,41.41158863)
\curveto(74.96680497,41.22408862)(74.51758618,41.13033861)(73.98633614,41.13033861)
\curveto(73.31446109,41.13033861)(72.77930479,41.31783863)(72.38086726,41.69283865)
\curveto(71.98633598,42.07174493)(71.78907034,42.57760435)(71.78907034,43.2104169)
\curveto(71.78907034,43.94869821)(72.03516411,44.50533888)(72.52735165,44.88033891)
\curveto(73.02344544,45.25533894)(73.76172675,45.44283895)(74.74219557,45.44283895)
\lineto(76.25391444,45.44283895)
\lineto(76.25391444,45.54830771)
\curveto(76.25391444,46.0444015)(76.08985193,46.42721403)(75.76172691,46.6967453)
\curveto(75.43750813,46.97018282)(74.98047684,47.10690159)(74.39063305,47.10690159)
\curveto(74.01563302,47.10690159)(73.65039861,47.06197971)(73.29492983,46.97213595)
\curveto(72.93946106,46.88229219)(72.59766415,46.74752656)(72.26953913,46.56783904)
\lineto(72.26953913,47.56393287)
\curveto(72.66407041,47.71627663)(73.04688294,47.82955789)(73.41797672,47.90377665)
\curveto(73.7890705,47.98190165)(74.15039865,48.02096416)(74.50196118,48.02096416)
\curveto(75.45118001,48.02096416)(76.16016444,47.77487039)(76.62891447,47.28268285)
\curveto(77.09766451,46.79049531)(77.33203953,46.0444015)(77.33203953,45.04440142)
\closepath
}
}
{
\newrgbcolor{curcolor}{0 0 0}
\pscustom[linestyle=none,fillstyle=solid,fillcolor=curcolor]
{
\newpath
\moveto(83.3613363,46.85494844)
\curveto(83.24024254,46.92526095)(83.10743003,46.9760422)(82.96289877,47.0072922)
\curveto(82.82227376,47.04244845)(82.66602374,47.06002658)(82.49414873,47.06002658)
\curveto(81.88477368,47.06002658)(81.41602364,46.86080782)(81.08789862,46.46237028)
\curveto(80.76367984,46.067839)(80.60157045,45.49947958)(80.60157045,44.75729202)
\lineto(80.60157045,41.3002605)
\lineto(79.51758599,41.3002605)
\lineto(79.51758599,47.86276102)
\lineto(80.60157045,47.86276102)
\lineto(80.60157045,46.84322969)
\curveto(80.82813297,47.24166722)(81.12305487,47.53658912)(81.48633615,47.72799538)
\curveto(81.84961743,47.9233079)(82.29102371,48.02096416)(82.81055501,48.02096416)
\curveto(82.88477376,48.02096416)(82.96680502,48.01510478)(83.05664878,48.00338603)
\curveto(83.14649253,47.99557353)(83.24610192,47.98190165)(83.35547692,47.9623704)
\lineto(83.3613363,46.85494844)
\closepath
}
}
{
\newrgbcolor{curcolor}{0 0 0}
\pscustom[linestyle=none,fillstyle=solid,fillcolor=curcolor]
{
\newpath
\moveto(87.48633623,44.59908889)
\curveto(86.61524241,44.59908889)(86.01172674,44.4994795)(85.67578922,44.30026074)
\curveto(85.33985169,44.10104197)(85.17188293,43.76119819)(85.17188293,43.28072941)
\curveto(85.17188293,42.89791688)(85.29688294,42.59322935)(85.54688295,42.36666683)
\curveto(85.80078923,42.14401057)(86.14453925,42.03268243)(86.57813304,42.03268243)
\curveto(87.17578933,42.03268243)(87.654305,42.24361995)(88.01368003,42.66549498)
\curveto(88.3769613,43.09127627)(88.55860194,43.65572944)(88.55860194,44.35885449)
\lineto(88.55860194,44.59908889)
\lineto(87.48633623,44.59908889)
\closepath
\moveto(89.63672703,45.04440142)
\lineto(89.63672703,41.3002605)
\lineto(88.55860194,41.3002605)
\lineto(88.55860194,42.29635433)
\curveto(88.31250817,41.8979168)(88.00586753,41.6029949)(87.63868,41.41158863)
\curveto(87.27149247,41.22408862)(86.82227368,41.13033861)(86.29102364,41.13033861)
\curveto(85.61914859,41.13033861)(85.08399229,41.31783863)(84.68555476,41.69283865)
\curveto(84.29102348,42.07174493)(84.09375784,42.57760435)(84.09375784,43.2104169)
\curveto(84.09375784,43.94869821)(84.33985161,44.50533888)(84.83203915,44.88033891)
\curveto(85.32813294,45.25533894)(86.06641425,45.44283895)(87.04688307,45.44283895)
\lineto(88.55860194,45.44283895)
\lineto(88.55860194,45.54830771)
\curveto(88.55860194,46.0444015)(88.39453943,46.42721403)(88.06641441,46.6967453)
\curveto(87.74219563,46.97018282)(87.28516434,47.10690159)(86.69532055,47.10690159)
\curveto(86.32032052,47.10690159)(85.95508611,47.06197971)(85.59961733,46.97213595)
\curveto(85.24414856,46.88229219)(84.90235165,46.74752656)(84.57422663,46.56783904)
\lineto(84.57422663,47.56393287)
\curveto(84.96875791,47.71627663)(85.35157044,47.82955789)(85.72266422,47.90377665)
\curveto(86.093758,47.98190165)(86.45508615,48.02096416)(86.80664868,48.02096416)
\curveto(87.75586751,48.02096416)(88.46485194,47.77487039)(88.93360197,47.28268285)
\curveto(89.40235201,46.79049531)(89.63672703,46.0444015)(89.63672703,45.04440142)
\closepath
}
}
{
\newrgbcolor{curcolor}{0 0 0}
\pscustom[linestyle=none,fillstyle=solid,fillcolor=curcolor]
{
\newpath
\moveto(95.68360162,50.41744872)
\lineto(96.76172671,50.41744872)
\lineto(96.76172671,41.3002605)
\lineto(95.68360162,41.3002605)
\lineto(95.68360162,50.41744872)
\closepath
}
}
{
\newrgbcolor{curcolor}{0 0 0}
\pscustom[linestyle=none,fillstyle=solid,fillcolor=curcolor]
{
\newpath
\moveto(99.01172662,50.41744872)
\lineto(100.08985171,50.41744872)
\lineto(100.08985171,41.3002605)
\lineto(99.01172662,41.3002605)
\lineto(99.01172662,50.41744872)
\closepath
}
}
{
\newrgbcolor{curcolor}{0 0 0}
\pscustom[linestyle=none,fillstyle=solid,fillcolor=curcolor]
{
\newpath
\moveto(107.95313332,44.85104203)
\lineto(107.95313332,44.32369824)
\lineto(102.99610167,44.32369824)
\curveto(103.04297668,43.58151068)(103.26563295,43.01510438)(103.66407048,42.62447935)
\curveto(104.06641426,42.23776057)(104.62500805,42.04440118)(105.33985186,42.04440118)
\curveto(105.75391439,42.04440118)(106.15430505,42.09518244)(106.54102383,42.19674494)
\curveto(106.93164886,42.29830745)(107.31836764,42.45065122)(107.70118017,42.65377623)
\lineto(107.70118017,41.6342449)
\curveto(107.31446139,41.47018239)(106.91797699,41.34518238)(106.51172695,41.25924487)
\curveto(106.10547692,41.17330736)(105.69336751,41.13033861)(105.27539873,41.13033861)
\curveto(104.22852365,41.13033861)(103.39844546,41.43502613)(102.78516416,42.04440118)
\curveto(102.17578911,42.65377623)(101.87110158,43.47799505)(101.87110158,44.51705763)
\curveto(101.87110158,45.59127646)(102.16016411,46.44283903)(102.73828915,47.07174533)
\curveto(103.32032045,47.70455788)(104.10352364,48.02096416)(105.08789872,48.02096416)
\curveto(105.97071129,48.02096416)(106.66797697,47.73580789)(107.17969576,47.16549534)
\curveto(107.6953208,46.59908904)(107.95313332,45.82760461)(107.95313332,44.85104203)
\closepath
\moveto(106.87500823,45.16744831)
\curveto(106.86719573,45.7572921)(106.70118009,46.22799527)(106.37696132,46.57955779)
\curveto(106.05664879,46.93112032)(105.63086751,47.10690159)(105.09961747,47.10690159)
\curveto(104.49805492,47.10690159)(104.01563301,46.9369797)(103.65235173,46.59713592)
\curveto(103.2929767,46.25729214)(103.08594543,45.77877648)(103.03125793,45.16158893)
\lineto(106.87500823,45.16744831)
\closepath
}
}
{
\newrgbcolor{curcolor}{0 0 0}
\pscustom[linestyle=none,fillstyle=solid,fillcolor=curcolor]
{
\newpath
\moveto(114.04102384,44.65768264)
\curveto(114.04102384,45.4389327)(113.87891445,46.0444015)(113.55469568,46.47408903)
\curveto(113.23438315,46.90377657)(112.78321124,47.11862034)(112.20117994,47.11862034)
\curveto(111.6230549,47.11862034)(111.17188299,46.90377657)(110.84766421,46.47408903)
\curveto(110.52735169,46.0444015)(110.36719542,45.4389327)(110.36719542,44.65768264)
\curveto(110.36719542,43.88033883)(110.52735169,43.27682316)(110.84766421,42.84713562)
\curveto(111.17188299,42.41744809)(111.6230549,42.20260432)(112.20117994,42.20260432)
\curveto(112.78321124,42.20260432)(113.23438315,42.41744809)(113.55469568,42.84713562)
\curveto(113.87891445,43.27682316)(114.04102384,43.88033883)(114.04102384,44.65768264)
\closepath
\moveto(115.11914893,42.11471369)
\curveto(115.11914893,40.9975261)(114.87110203,40.16744791)(114.37500824,39.62447912)
\curveto(113.87891445,39.07760407)(113.11914877,38.80416655)(112.09571119,38.80416655)
\curveto(111.71680491,38.80416655)(111.359383,38.83346343)(111.02344548,38.89205718)
\curveto(110.68750795,38.94674469)(110.36133605,39.03268219)(110.04492977,39.1498697)
\lineto(110.04492977,40.19869791)
\curveto(110.36133605,40.0268229)(110.67383607,39.89986976)(110.98242985,39.81783851)
\curveto(111.29102362,39.73580725)(111.60547677,39.69479162)(111.9257893,39.69479162)
\curveto(112.6328206,39.69479162)(113.16211752,39.88033851)(113.51368005,40.25143229)
\curveto(113.86524258,40.61861982)(114.04102384,41.17526049)(114.04102384,41.9213543)
\lineto(114.04102384,42.45455747)
\curveto(113.81836757,42.06783868)(113.5332113,41.77877616)(113.18555502,41.5873699)
\curveto(112.83789874,41.39596363)(112.42188309,41.3002605)(111.93750805,41.3002605)
\curveto(111.13282048,41.3002605)(110.48438293,41.60690115)(109.99219539,42.22018245)
\curveto(109.50000785,42.83346375)(109.25391408,43.64596381)(109.25391408,44.65768264)
\curveto(109.25391408,45.67330772)(109.50000785,46.48776091)(109.99219539,47.10104221)
\curveto(110.48438293,47.71432351)(111.13282048,48.02096416)(111.93750805,48.02096416)
\curveto(112.42188309,48.02096416)(112.83789874,47.92526103)(113.18555502,47.73385476)
\curveto(113.5332113,47.54244849)(113.81836757,47.25338597)(114.04102384,46.86666719)
\lineto(114.04102384,47.86276102)
\lineto(115.11914893,47.86276102)
\lineto(115.11914893,42.11471369)
\closepath
}
}
{
\newrgbcolor{curcolor}{0 0 0}
\pscustom[linestyle=none,fillstyle=solid,fillcolor=curcolor]
{
\newpath
\moveto(120.32227373,44.59908889)
\curveto(119.45117991,44.59908889)(118.84766424,44.4994795)(118.51172672,44.30026074)
\curveto(118.17578919,44.10104197)(118.00782043,43.76119819)(118.00782043,43.28072941)
\curveto(118.00782043,42.89791688)(118.13282044,42.59322935)(118.38282045,42.36666683)
\curveto(118.63672673,42.14401057)(118.98047675,42.03268243)(119.41407054,42.03268243)
\curveto(120.01172683,42.03268243)(120.4902425,42.24361995)(120.84961753,42.66549498)
\curveto(121.2128988,43.09127627)(121.39453944,43.65572944)(121.39453944,44.35885449)
\lineto(121.39453944,44.59908889)
\lineto(120.32227373,44.59908889)
\closepath
\moveto(122.47266453,45.04440142)
\lineto(122.47266453,41.3002605)
\lineto(121.39453944,41.3002605)
\lineto(121.39453944,42.29635433)
\curveto(121.14844567,41.8979168)(120.84180503,41.6029949)(120.4746175,41.41158863)
\curveto(120.10742997,41.22408862)(119.65821118,41.13033861)(119.12696114,41.13033861)
\curveto(118.45508609,41.13033861)(117.91992979,41.31783863)(117.52149226,41.69283865)
\curveto(117.12696098,42.07174493)(116.92969534,42.57760435)(116.92969534,43.2104169)
\curveto(116.92969534,43.94869821)(117.17578911,44.50533888)(117.66797665,44.88033891)
\curveto(118.16407044,45.25533894)(118.90235175,45.44283895)(119.88282057,45.44283895)
\lineto(121.39453944,45.44283895)
\lineto(121.39453944,45.54830771)
\curveto(121.39453944,46.0444015)(121.23047693,46.42721403)(120.90235191,46.6967453)
\curveto(120.57813313,46.97018282)(120.12110184,47.10690159)(119.53125805,47.10690159)
\curveto(119.15625802,47.10690159)(118.79102361,47.06197971)(118.43555483,46.97213595)
\curveto(118.08008606,46.88229219)(117.73828915,46.74752656)(117.41016413,46.56783904)
\lineto(117.41016413,47.56393287)
\curveto(117.80469541,47.71627663)(118.18750794,47.82955789)(118.55860172,47.90377665)
\curveto(118.9296955,47.98190165)(119.29102365,48.02096416)(119.64258618,48.02096416)
\curveto(120.59180501,48.02096416)(121.30078944,47.77487039)(121.76953947,47.28268285)
\curveto(122.23828951,46.79049531)(122.47266453,46.0444015)(122.47266453,45.04440142)
\closepath
}
}
{
\newrgbcolor{curcolor}{0 0 0}
\pscustom[linestyle=none,fillstyle=solid,fillcolor=curcolor]
{
\newpath
\moveto(128.5019613,46.85494844)
\curveto(128.38086754,46.92526095)(128.24805503,46.9760422)(128.10352377,47.0072922)
\curveto(127.96289876,47.04244845)(127.80664874,47.06002658)(127.63477373,47.06002658)
\curveto(127.02539868,47.06002658)(126.55664864,46.86080782)(126.22852362,46.46237028)
\curveto(125.90430484,46.067839)(125.74219545,45.49947958)(125.74219545,44.75729202)
\lineto(125.74219545,41.3002605)
\lineto(124.65821099,41.3002605)
\lineto(124.65821099,47.86276102)
\lineto(125.74219545,47.86276102)
\lineto(125.74219545,46.84322969)
\curveto(125.96875797,47.24166722)(126.26367987,47.53658912)(126.62696115,47.72799538)
\curveto(126.99024243,47.9233079)(127.43164871,48.02096416)(127.95118001,48.02096416)
\curveto(128.02539876,48.02096416)(128.10743002,48.01510478)(128.19727378,48.00338603)
\curveto(128.28711753,47.99557353)(128.38672692,47.98190165)(128.49610192,47.9623704)
\lineto(128.5019613,46.85494844)
\closepath
}
}
{
\newrgbcolor{curcolor}{0 0 0}
\pscustom[linestyle=none,fillstyle=solid,fillcolor=curcolor]
{
\newpath
\moveto(136.44728136,44.59908889)
\curveto(135.57618754,44.59908889)(134.97267187,44.4994795)(134.63673434,44.30026074)
\curveto(134.30079682,44.10104197)(134.13282805,43.76119819)(134.13282805,43.28072941)
\curveto(134.13282805,42.89791688)(134.25782806,42.59322935)(134.50782808,42.36666683)
\curveto(134.76173435,42.14401057)(135.10548438,42.03268243)(135.53907817,42.03268243)
\curveto(136.13673446,42.03268243)(136.61525013,42.24361995)(136.97462516,42.66549498)
\curveto(137.33790643,43.09127627)(137.51954707,43.65572944)(137.51954707,44.35885449)
\lineto(137.51954707,44.59908889)
\lineto(136.44728136,44.59908889)
\closepath
\moveto(138.59767216,45.04440142)
\lineto(138.59767216,41.3002605)
\lineto(137.51954707,41.3002605)
\lineto(137.51954707,42.29635433)
\curveto(137.2734533,41.8979168)(136.96681265,41.6029949)(136.59962513,41.41158863)
\curveto(136.2324376,41.22408862)(135.78321881,41.13033861)(135.25196877,41.13033861)
\curveto(134.58009372,41.13033861)(134.04493742,41.31783863)(133.64649989,41.69283865)
\curveto(133.25196861,42.07174493)(133.05470297,42.57760435)(133.05470297,43.2104169)
\curveto(133.05470297,43.94869821)(133.30079674,44.50533888)(133.79298428,44.88033891)
\curveto(134.28907807,45.25533894)(135.02735938,45.44283895)(136.0078282,45.44283895)
\lineto(137.51954707,45.44283895)
\lineto(137.51954707,45.54830771)
\curveto(137.51954707,46.0444015)(137.35548456,46.42721403)(137.02735953,46.6967453)
\curveto(136.70314076,46.97018282)(136.24610947,47.10690159)(135.65626568,47.10690159)
\curveto(135.28126565,47.10690159)(134.91603124,47.06197971)(134.56056246,46.97213595)
\curveto(134.20509369,46.88229219)(133.86329678,46.74752656)(133.53517176,46.56783904)
\lineto(133.53517176,47.56393287)
\curveto(133.92970304,47.71627663)(134.31251557,47.82955789)(134.68360935,47.90377665)
\curveto(135.05470313,47.98190165)(135.41603128,48.02096416)(135.76759381,48.02096416)
\curveto(136.71681263,48.02096416)(137.42579707,47.77487039)(137.8945471,47.28268285)
\curveto(138.36329714,46.79049531)(138.59767216,46.0444015)(138.59767216,45.04440142)
\closepath
}
}
{
\newrgbcolor{curcolor}{0 0 0}
\pscustom[linestyle=none,fillstyle=solid,fillcolor=curcolor]
{
\newpath
\moveto(140.82423425,50.41744872)
\lineto(141.90235934,50.41744872)
\lineto(141.90235934,41.3002605)
\lineto(140.82423425,41.3002605)
\lineto(140.82423425,50.41744872)
\closepath
}
}
{
\newrgbcolor{curcolor}{0 0 0}
\pscustom[linestyle=none,fillstyle=solid,fillcolor=curcolor]
{
\newpath
\moveto(147.19923419,47.86276102)
\lineto(148.34181241,47.86276102)
\lineto(150.39259382,42.35494808)
\lineto(152.44337523,47.86276102)
\lineto(153.58595345,47.86276102)
\lineto(151.12501575,41.3002605)
\lineto(149.66017189,41.3002605)
\lineto(147.19923419,47.86276102)
\closepath
}
}
{
\newrgbcolor{curcolor}{0 0 0}
\pscustom[linestyle=none,fillstyle=solid,fillcolor=curcolor]
{
\newpath
\moveto(158.05665636,44.59908889)
\curveto(157.18556254,44.59908889)(156.58204687,44.4994795)(156.24610934,44.30026074)
\curveto(155.91017182,44.10104197)(155.74220305,43.76119819)(155.74220305,43.28072941)
\curveto(155.74220305,42.89791688)(155.86720306,42.59322935)(156.11720308,42.36666683)
\curveto(156.37110935,42.14401057)(156.71485938,42.03268243)(157.14845317,42.03268243)
\curveto(157.74610946,42.03268243)(158.22462513,42.24361995)(158.58400016,42.66549498)
\curveto(158.94728143,43.09127627)(159.12892207,43.65572944)(159.12892207,44.35885449)
\lineto(159.12892207,44.59908889)
\lineto(158.05665636,44.59908889)
\closepath
\moveto(160.20704716,45.04440142)
\lineto(160.20704716,41.3002605)
\lineto(159.12892207,41.3002605)
\lineto(159.12892207,42.29635433)
\curveto(158.8828283,41.8979168)(158.57618765,41.6029949)(158.20900013,41.41158863)
\curveto(157.8418126,41.22408862)(157.39259381,41.13033861)(156.86134377,41.13033861)
\curveto(156.18946872,41.13033861)(155.65431242,41.31783863)(155.25587489,41.69283865)
\curveto(154.86134361,42.07174493)(154.66407797,42.57760435)(154.66407797,43.2104169)
\curveto(154.66407797,43.94869821)(154.91017174,44.50533888)(155.40235928,44.88033891)
\curveto(155.89845307,45.25533894)(156.63673438,45.44283895)(157.6172032,45.44283895)
\lineto(159.12892207,45.44283895)
\lineto(159.12892207,45.54830771)
\curveto(159.12892207,46.0444015)(158.96485956,46.42721403)(158.63673453,46.6967453)
\curveto(158.31251576,46.97018282)(157.85548447,47.10690159)(157.26564068,47.10690159)
\curveto(156.89064065,47.10690159)(156.52540624,47.06197971)(156.16993746,46.97213595)
\curveto(155.81446869,46.88229219)(155.47267178,46.74752656)(155.14454676,46.56783904)
\lineto(155.14454676,47.56393287)
\curveto(155.53907804,47.71627663)(155.92189057,47.82955789)(156.29298435,47.90377665)
\curveto(156.66407813,47.98190165)(157.02540628,48.02096416)(157.37696881,48.02096416)
\curveto(158.32618763,48.02096416)(159.03517207,47.77487039)(159.5039221,47.28268285)
\curveto(159.97267214,46.79049531)(160.20704716,46.0444015)(160.20704716,45.04440142)
\closepath
}
}
{
\newrgbcolor{curcolor}{0 0 0}
\pscustom[linestyle=none,fillstyle=solid,fillcolor=curcolor]
{
\newpath
\moveto(162.43360925,50.41744872)
\lineto(163.51173434,50.41744872)
\lineto(163.51173434,41.3002605)
\lineto(162.43360925,41.3002605)
\lineto(162.43360925,50.41744872)
\closepath
}
}
{
\newrgbcolor{curcolor}{0 0 0}
\pscustom[linestyle=none,fillstyle=solid,fillcolor=curcolor]
{
\newpath
\moveto(168.3047032,47.10690159)
\curveto(167.72657816,47.10690159)(167.26954687,46.88033907)(166.93360934,46.42721403)
\curveto(166.59767182,45.97799525)(166.42970305,45.3608077)(166.42970305,44.57565138)
\curveto(166.42970305,43.79049507)(166.59571869,43.1713544)(166.92774997,42.71822936)
\curveto(167.2636875,42.26901058)(167.72267191,42.04440118)(168.3047032,42.04440118)
\curveto(168.878922,42.04440118)(169.33400016,42.2709637)(169.66993769,42.72408874)
\curveto(170.00587521,43.17721377)(170.17384398,43.79440132)(170.17384398,44.57565138)
\curveto(170.17384398,45.3529952)(170.00587521,45.96822962)(169.66993769,46.42135466)
\curveto(169.33400016,46.87838594)(168.878922,47.10690159)(168.3047032,47.10690159)
\closepath
\moveto(168.3047032,48.02096416)
\curveto(169.24220328,48.02096416)(169.97853146,47.71627663)(170.51368775,47.10690159)
\curveto(171.04884405,46.49752654)(171.31642219,45.65377647)(171.31642219,44.57565138)
\curveto(171.31642219,43.50143255)(171.04884405,42.65768248)(170.51368775,42.04440118)
\curveto(169.97853146,41.43502613)(169.24220328,41.13033861)(168.3047032,41.13033861)
\curveto(167.36329688,41.13033861)(166.62501557,41.43502613)(166.08985928,42.04440118)
\curveto(165.55860924,42.65768248)(165.29298421,43.50143255)(165.29298421,44.57565138)
\curveto(165.29298421,45.65377647)(165.55860924,46.49752654)(166.08985928,47.10690159)
\curveto(166.62501557,47.71627663)(167.36329688,48.02096416)(168.3047032,48.02096416)
\closepath
}
}
{
\newrgbcolor{curcolor}{0 0 0}
\pscustom[linestyle=none,fillstyle=solid,fillcolor=curcolor]
{
\newpath
\moveto(176.90040643,46.85494844)
\curveto(176.77931267,46.92526095)(176.64650016,46.9760422)(176.5019689,47.0072922)
\curveto(176.36134389,47.04244845)(176.20509387,47.06002658)(176.03321886,47.06002658)
\curveto(175.42384381,47.06002658)(174.95509377,46.86080782)(174.62696875,46.46237028)
\curveto(174.30274997,46.067839)(174.14064058,45.49947958)(174.14064058,44.75729202)
\lineto(174.14064058,41.3002605)
\lineto(173.05665612,41.3002605)
\lineto(173.05665612,47.86276102)
\lineto(174.14064058,47.86276102)
\lineto(174.14064058,46.84322969)
\curveto(174.3672031,47.24166722)(174.662125,47.53658912)(175.02540628,47.72799538)
\curveto(175.38868756,47.9233079)(175.83009384,48.02096416)(176.34962513,48.02096416)
\curveto(176.42384389,48.02096416)(176.50587515,48.01510478)(176.5957189,48.00338603)
\curveto(176.68556266,47.99557353)(176.78517204,47.98190165)(176.89454705,47.9623704)
\lineto(176.90040643,46.85494844)
\closepath
}
}
{
\newrgbcolor{curcolor}{0 0 0}
\pscustom[linestyle=none,fillstyle=solid,fillcolor=curcolor]
{
\newpath
\moveto(185.66603143,46.85494844)
\curveto(185.54493767,46.92526095)(185.41212516,46.9760422)(185.2675939,47.0072922)
\curveto(185.12696889,47.04244845)(184.97071887,47.06002658)(184.79884386,47.06002658)
\curveto(184.18946881,47.06002658)(183.72071877,46.86080782)(183.39259375,46.46237028)
\curveto(183.06837497,46.067839)(182.90626558,45.49947958)(182.90626558,44.75729202)
\lineto(182.90626558,41.3002605)
\lineto(181.82228112,41.3002605)
\lineto(181.82228112,47.86276102)
\lineto(182.90626558,47.86276102)
\lineto(182.90626558,46.84322969)
\curveto(183.1328281,47.24166722)(183.42775,47.53658912)(183.79103128,47.72799538)
\curveto(184.15431256,47.9233079)(184.59571884,48.02096416)(185.11525013,48.02096416)
\curveto(185.18946889,48.02096416)(185.27150015,48.01510478)(185.3613439,48.00338603)
\curveto(185.45118766,47.99557353)(185.55079704,47.98190165)(185.66017205,47.9623704)
\lineto(185.66603143,46.85494844)
\closepath
}
}
{
\newrgbcolor{curcolor}{0 0 0}
\pscustom[linestyle=none,fillstyle=solid,fillcolor=curcolor]
{
\newpath
\moveto(192.1757972,44.85104203)
\lineto(192.1757972,44.32369824)
\lineto(187.21876555,44.32369824)
\curveto(187.26564056,43.58151068)(187.48829683,43.01510438)(187.88673436,42.62447935)
\curveto(188.28907814,42.23776057)(188.84767193,42.04440118)(189.56251574,42.04440118)
\curveto(189.97657827,42.04440118)(190.37696893,42.09518244)(190.76368771,42.19674494)
\curveto(191.15431274,42.29830745)(191.54103152,42.45065122)(191.92384405,42.65377623)
\lineto(191.92384405,41.6342449)
\curveto(191.53712527,41.47018239)(191.14064087,41.34518238)(190.73439083,41.25924487)
\curveto(190.3281408,41.17330736)(189.91603139,41.13033861)(189.49806261,41.13033861)
\curveto(188.45118753,41.13033861)(187.62110934,41.43502613)(187.00782804,42.04440118)
\curveto(186.39845299,42.65377623)(186.09376546,43.47799505)(186.09376546,44.51705763)
\curveto(186.09376546,45.59127646)(186.38282799,46.44283903)(186.96095303,47.07174533)
\curveto(187.54298433,47.70455788)(188.32618752,48.02096416)(189.31056259,48.02096416)
\curveto(190.19337517,48.02096416)(190.89064085,47.73580789)(191.40235964,47.16549534)
\curveto(191.91798468,46.59908904)(192.1757972,45.82760461)(192.1757972,44.85104203)
\closepath
\moveto(191.09767211,45.16744831)
\curveto(191.08985961,45.7572921)(190.92384397,46.22799527)(190.5996252,46.57955779)
\curveto(190.27931267,46.93112032)(189.85353139,47.10690159)(189.32228135,47.10690159)
\curveto(188.7207188,47.10690159)(188.23829688,46.9369797)(187.87501561,46.59713592)
\curveto(187.51564058,46.25729214)(187.30860931,45.77877648)(187.25392181,45.16158893)
\lineto(191.09767211,45.16744831)
\closepath
}
}
{
\newrgbcolor{curcolor}{0 0 0}
\pscustom[linestyle=none,fillstyle=solid,fillcolor=curcolor]
{
\newpath
\moveto(194.60157805,44.57565138)
\curveto(194.60157805,43.78268257)(194.76368744,43.15963565)(195.08790622,42.70651061)
\curveto(195.41603124,42.25729182)(195.86525003,42.03268243)(196.43556257,42.03268243)
\curveto(197.00587512,42.03268243)(197.4550939,42.25729182)(197.78321893,42.70651061)
\curveto(198.11134396,43.15963565)(198.27540647,43.78268257)(198.27540647,44.57565138)
\curveto(198.27540647,45.3686202)(198.11134396,45.989714)(197.78321893,46.43893278)
\curveto(197.4550939,46.89205782)(197.00587512,47.11862034)(196.43556257,47.11862034)
\curveto(195.86525003,47.11862034)(195.41603124,46.89205782)(195.08790622,46.43893278)
\curveto(194.76368744,45.989714)(194.60157805,45.3686202)(194.60157805,44.57565138)
\closepath
\moveto(198.27540647,42.28463558)
\curveto(198.04884395,41.89401055)(197.76173455,41.6029949)(197.41407828,41.41158863)
\curveto(197.07032825,41.22408862)(196.65626572,41.13033861)(196.17189068,41.13033861)
\curveto(195.37892186,41.13033861)(194.73243744,41.44674489)(194.2324374,42.07955744)
\curveto(193.73634361,42.71236999)(193.48829671,43.5444013)(193.48829671,44.57565138)
\curveto(193.48829671,45.60690147)(193.73634361,46.43893278)(194.2324374,47.07174533)
\curveto(194.73243744,47.70455788)(195.37892186,48.02096416)(196.17189068,48.02096416)
\curveto(196.65626572,48.02096416)(197.07032825,47.92526103)(197.41407828,47.73385476)
\curveto(197.76173455,47.54635475)(198.04884395,47.25729222)(198.27540647,46.86666719)
\lineto(198.27540647,47.86276102)
\lineto(199.35353156,47.86276102)
\lineto(199.35353156,38.80416655)
\lineto(198.27540647,38.80416655)
\lineto(198.27540647,42.28463558)
\closepath
}
}
{
\newrgbcolor{curcolor}{0 0 0}
\pscustom[linestyle=none,fillstyle=solid,fillcolor=curcolor]
{
\newpath
\moveto(201.46290612,43.89010445)
\lineto(201.46290612,47.86276102)
\lineto(202.5410312,47.86276102)
\lineto(202.5410312,43.93112008)
\curveto(202.5410312,43.31002628)(202.66212496,42.84322937)(202.90431248,42.53072935)
\curveto(203.1465,42.22213557)(203.50978128,42.06783868)(203.99415632,42.06783868)
\curveto(204.57618762,42.06783868)(205.03517203,42.25338557)(205.37110955,42.62447935)
\curveto(205.71095333,42.99557313)(205.88087522,43.50143255)(205.88087522,44.1420576)
\lineto(205.88087522,47.86276102)
\lineto(206.9590003,47.86276102)
\lineto(206.9590003,41.3002605)
\lineto(205.88087522,41.3002605)
\lineto(205.88087522,42.30807308)
\curveto(205.61915645,41.90963555)(205.31446892,41.61276052)(204.96681265,41.41744801)
\curveto(204.62306262,41.22604174)(204.22267196,41.13033861)(203.76564068,41.13033861)
\curveto(203.01173437,41.13033861)(202.4394687,41.36471363)(202.04884366,41.83346367)
\curveto(201.65821863,42.3022137)(201.46290612,42.98776063)(201.46290612,43.89010445)
\closepath
}
}
{
\newrgbcolor{curcolor}{0 0 0}
\pscustom[linestyle=none,fillstyle=solid,fillcolor=curcolor]
{
\newpath
\moveto(214.80470345,44.85104203)
\lineto(214.80470345,44.32369824)
\lineto(209.8476718,44.32369824)
\curveto(209.89454681,43.58151068)(210.11720308,43.01510438)(210.51564061,42.62447935)
\curveto(210.91798439,42.23776057)(211.47657818,42.04440118)(212.19142199,42.04440118)
\curveto(212.60548452,42.04440118)(213.00587518,42.09518244)(213.39259396,42.19674494)
\curveto(213.78321899,42.29830745)(214.16993777,42.45065122)(214.5527503,42.65377623)
\lineto(214.5527503,41.6342449)
\curveto(214.16603152,41.47018239)(213.76954712,41.34518238)(213.36329708,41.25924487)
\curveto(212.95704705,41.17330736)(212.54493764,41.13033861)(212.12696886,41.13033861)
\curveto(211.08009378,41.13033861)(210.25001559,41.43502613)(209.63673429,42.04440118)
\curveto(209.02735924,42.65377623)(208.72267171,43.47799505)(208.72267171,44.51705763)
\curveto(208.72267171,45.59127646)(209.01173424,46.44283903)(209.58985928,47.07174533)
\curveto(210.17189058,47.70455788)(210.95509377,48.02096416)(211.93946884,48.02096416)
\curveto(212.82228142,48.02096416)(213.5195471,47.73580789)(214.03126589,47.16549534)
\curveto(214.54689093,46.59908904)(214.80470345,45.82760461)(214.80470345,44.85104203)
\closepath
\moveto(213.72657836,45.16744831)
\curveto(213.71876586,45.7572921)(213.55275022,46.22799527)(213.22853145,46.57955779)
\curveto(212.90821892,46.93112032)(212.48243764,47.10690159)(211.9511876,47.10690159)
\curveto(211.34962505,47.10690159)(210.86720313,46.9369797)(210.50392186,46.59713592)
\curveto(210.14454683,46.25729214)(209.93751556,45.77877648)(209.88282806,45.16158893)
\lineto(213.72657836,45.16744831)
\closepath
}
}
{
\newrgbcolor{curcolor}{0 0 0}
\pscustom[linestyle=none,fillstyle=solid,fillcolor=curcolor]
{
\newpath
\moveto(220.37696893,46.85494844)
\curveto(220.25587517,46.92526095)(220.12306266,46.9760422)(219.9785314,47.0072922)
\curveto(219.83790639,47.04244845)(219.68165637,47.06002658)(219.50978136,47.06002658)
\curveto(218.90040631,47.06002658)(218.43165627,46.86080782)(218.10353125,46.46237028)
\curveto(217.77931247,46.067839)(217.61720308,45.49947958)(217.61720308,44.75729202)
\lineto(217.61720308,41.3002605)
\lineto(216.53321862,41.3002605)
\lineto(216.53321862,47.86276102)
\lineto(217.61720308,47.86276102)
\lineto(217.61720308,46.84322969)
\curveto(217.8437656,47.24166722)(218.1386875,47.53658912)(218.50196878,47.72799538)
\curveto(218.86525006,47.9233079)(219.30665634,48.02096416)(219.82618763,48.02096416)
\curveto(219.90040639,48.02096416)(219.98243765,48.01510478)(220.0722814,48.00338603)
\curveto(220.16212516,47.99557353)(220.26173454,47.98190165)(220.37110955,47.9623704)
\lineto(220.37696893,46.85494844)
\closepath
}
}
{
\newrgbcolor{curcolor}{0 0 0}
\pscustom[linestyle=none,fillstyle=solid,fillcolor=curcolor]
{
\newpath
\moveto(221.51954675,47.86276102)
\lineto(222.59767184,47.86276102)
\lineto(222.59767184,41.3002605)
\lineto(221.51954675,41.3002605)
\lineto(221.51954675,47.86276102)
\closepath
\moveto(221.51954675,50.41744872)
\lineto(222.59767184,50.41744872)
\lineto(222.59767184,49.05221424)
\lineto(221.51954675,49.05221424)
\lineto(221.51954675,50.41744872)
\closepath
}
}
{
\newrgbcolor{curcolor}{0 0 0}
\pscustom[linestyle=none,fillstyle=solid,fillcolor=curcolor]
{
\newpath
\moveto(229.16603147,46.86666719)
\lineto(229.16603147,50.41744872)
\lineto(230.24415656,50.41744872)
\lineto(230.24415656,41.3002605)
\lineto(229.16603147,41.3002605)
\lineto(229.16603147,42.28463558)
\curveto(228.93946895,41.89401055)(228.65235955,41.6029949)(228.30470328,41.41158863)
\curveto(227.96095325,41.22408862)(227.54689072,41.13033861)(227.06251568,41.13033861)
\curveto(226.26954686,41.13033861)(225.62306244,41.44674489)(225.1230624,42.07955744)
\curveto(224.62696861,42.71236999)(224.37892171,43.5444013)(224.37892171,44.57565138)
\curveto(224.37892171,45.60690147)(224.62696861,46.43893278)(225.1230624,47.07174533)
\curveto(225.62306244,47.70455788)(226.26954686,48.02096416)(227.06251568,48.02096416)
\curveto(227.54689072,48.02096416)(227.96095325,47.92526103)(228.30470328,47.73385476)
\curveto(228.65235955,47.54635475)(228.93946895,47.25729222)(229.16603147,46.86666719)
\closepath
\moveto(225.49220305,44.57565138)
\curveto(225.49220305,43.78268257)(225.65431244,43.15963565)(225.97853122,42.70651061)
\curveto(226.30665624,42.25729182)(226.75587503,42.03268243)(227.32618757,42.03268243)
\curveto(227.89650012,42.03268243)(228.3457189,42.25729182)(228.67384393,42.70651061)
\curveto(229.00196896,43.15963565)(229.16603147,43.78268257)(229.16603147,44.57565138)
\curveto(229.16603147,45.3686202)(229.00196896,45.989714)(228.67384393,46.43893278)
\curveto(228.3457189,46.89205782)(227.89650012,47.11862034)(227.32618757,47.11862034)
\curveto(226.75587503,47.11862034)(226.30665624,46.89205782)(225.97853122,46.43893278)
\curveto(225.65431244,45.989714)(225.49220305,45.3686202)(225.49220305,44.57565138)
\closepath
}
}
{
\newrgbcolor{curcolor}{0 0 0}
\pscustom[linestyle=none,fillstyle=solid,fillcolor=curcolor]
{
\newpath
\moveto(235.0078282,47.10690159)
\curveto(234.42970316,47.10690159)(233.97267187,46.88033907)(233.63673434,46.42721403)
\curveto(233.30079682,45.97799525)(233.13282805,45.3608077)(233.13282805,44.57565138)
\curveto(233.13282805,43.79049507)(233.29884369,43.1713544)(233.63087497,42.71822936)
\curveto(233.9668125,42.26901058)(234.42579691,42.04440118)(235.0078282,42.04440118)
\curveto(235.582047,42.04440118)(236.03712516,42.2709637)(236.37306269,42.72408874)
\curveto(236.70900021,43.17721377)(236.87696898,43.79440132)(236.87696898,44.57565138)
\curveto(236.87696898,45.3529952)(236.70900021,45.96822962)(236.37306269,46.42135466)
\curveto(236.03712516,46.87838594)(235.582047,47.10690159)(235.0078282,47.10690159)
\closepath
\moveto(235.0078282,48.02096416)
\curveto(235.94532828,48.02096416)(236.68165646,47.71627663)(237.21681275,47.10690159)
\curveto(237.75196905,46.49752654)(238.01954719,45.65377647)(238.01954719,44.57565138)
\curveto(238.01954719,43.50143255)(237.75196905,42.65768248)(237.21681275,42.04440118)
\curveto(236.68165646,41.43502613)(235.94532828,41.13033861)(235.0078282,41.13033861)
\curveto(234.06642188,41.13033861)(233.32814057,41.43502613)(232.79298428,42.04440118)
\curveto(232.26173424,42.65768248)(231.99610921,43.50143255)(231.99610921,44.57565138)
\curveto(231.99610921,45.65377647)(232.26173424,46.49752654)(232.79298428,47.10690159)
\curveto(233.32814057,47.71627663)(234.06642188,48.02096416)(235.0078282,48.02096416)
\closepath
}
}
{
\newrgbcolor{curcolor}{0 0 0}
\pscustom[linestyle=none,fillstyle=solid,fillcolor=curcolor]
{
\newpath
\moveto(254.55590091,53.93814192)
\lineto(254.55590091,50.65103228)
\lineto(256.04418228,50.65103228)
\curveto(256.59496358,50.65103228)(257.02074486,50.79361042)(257.32152613,51.07876669)
\curveto(257.62230741,51.36392296)(257.77269804,51.770173)(257.77269804,52.29751679)
\curveto(257.77269804,52.82095433)(257.62230741,53.22525124)(257.32152613,53.51040751)
\curveto(257.02074486,53.79556378)(256.59496358,53.93814192)(256.04418228,53.93814192)
\lineto(254.55590091,53.93814192)
\closepath
\moveto(253.37230707,54.91079825)
\lineto(256.04418228,54.91079825)
\curveto(257.02465111,54.91079825)(257.76488554,54.68814198)(258.26488558,54.24282944)
\curveto(258.76879187,53.80142316)(259.02074502,53.15298561)(259.02074502,52.29751679)
\curveto(259.02074502,51.43423547)(258.76879187,50.78189167)(258.26488558,50.34048538)
\curveto(257.76488554,49.8990791)(257.02465111,49.67837596)(256.04418228,49.67837596)
\lineto(254.55590091,49.67837596)
\lineto(254.55590091,46.16275068)
\lineto(253.37230707,46.16275068)
\lineto(253.37230707,54.91079825)
\closepath
}
}
{
\newrgbcolor{curcolor}{0 0 0}
\pscustom[linestyle=none,fillstyle=solid,fillcolor=curcolor]
{
\newpath
\moveto(260.61449505,54.91079825)
\lineto(261.79808889,54.91079825)
\lineto(261.79808889,47.1588445)
\lineto(266.05785485,47.1588445)
\lineto(266.05785485,46.16275068)
\lineto(260.61449505,46.16275068)
\lineto(260.61449505,54.91079825)
\closepath
}
}
{
\newrgbcolor{curcolor}{0 0 0}
\pscustom[linestyle=none,fillstyle=solid,fillcolor=curcolor]
{
\newpath
\moveto(273.84496479,54.23697007)
\lineto(273.84496479,52.98892309)
\curveto(273.44652726,53.36001687)(273.02074598,53.63736064)(272.56762094,53.82095441)
\curveto(272.11840216,54.00454817)(271.63988649,54.09634506)(271.13207395,54.09634506)
\curveto(270.13207387,54.09634506)(269.36644881,53.78970441)(268.83519877,53.17642311)
\curveto(268.30394873,52.56704806)(268.03832371,51.68423549)(268.03832371,50.5279854)
\curveto(268.03832371,49.37564156)(268.30394873,48.49282899)(268.83519877,47.87954769)
\curveto(269.36644881,47.27017264)(270.13207387,46.96548511)(271.13207395,46.96548511)
\curveto(271.63988649,46.96548511)(272.11840216,47.057282)(272.56762094,47.24087576)
\curveto(273.02074598,47.42446953)(273.44652726,47.7018133)(273.84496479,48.07290708)
\lineto(273.84496479,46.83657885)
\curveto(273.43090226,46.55532883)(272.9914491,46.34439132)(272.52660531,46.2037663)
\curveto(272.06566778,46.06314129)(271.57738649,45.99282879)(271.06176145,45.99282879)
\curveto(269.73754259,45.99282879)(268.69457376,46.39712569)(267.93285495,47.20571951)
\curveto(267.17113614,48.01821957)(266.79027673,49.12564154)(266.79027673,50.5279854)
\curveto(266.79027673,51.93423551)(267.17113614,53.04165747)(267.93285495,53.85025129)
\curveto(268.69457376,54.66275135)(269.73754259,55.06900138)(271.06176145,55.06900138)
\curveto(271.58519899,55.06900138)(272.07738653,54.99868888)(272.53832407,54.85806387)
\curveto(273.00316785,54.72134511)(273.43871476,54.51431384)(273.84496479,54.23697007)
\closepath
}
}
{
\newrgbcolor{curcolor}{0 0 0}
\pscustom[linestyle=none,fillstyle=solid,fillcolor=curcolor]
{
\newpath
\moveto(276.80980907,47.1588445)
\lineto(280.94066877,47.1588445)
\lineto(280.94066877,46.16275068)
\lineto(275.38598083,46.16275068)
\lineto(275.38598083,47.1588445)
\curveto(275.83519961,47.62368829)(276.44652779,48.24673522)(277.21996535,49.02798528)
\curveto(277.99730916,49.81314159)(278.48559045,50.31900101)(278.68480922,50.54556352)
\curveto(279.0637155,50.97134481)(279.32738739,51.33071984)(279.4758249,51.62368861)
\curveto(279.62816867,51.92056363)(279.70434055,52.21157928)(279.70434055,52.49673555)
\curveto(279.70434055,52.96157934)(279.54027803,53.34048562)(279.21215301,53.63345439)
\curveto(278.88793423,53.92642317)(278.46410607,54.07290755)(277.94066853,54.07290755)
\curveto(277.56957475,54.07290755)(277.1769966,54.00845442)(276.76293406,53.87954816)
\curveto(276.35277778,53.7506419)(275.91332462,53.55532939)(275.44457458,53.29361062)
\lineto(275.44457458,54.48892321)
\curveto(275.92113712,54.68032948)(276.36644966,54.82486074)(276.78051219,54.922517)
\curveto(277.19457472,55.02017325)(277.573481,55.06900138)(277.91723103,55.06900138)
\curveto(278.8234811,55.06900138)(279.54613741,54.84243887)(280.08519995,54.38931383)
\curveto(280.6242625,53.93618879)(280.89379377,53.33072)(280.89379377,52.57290744)
\curveto(280.89379377,52.21353241)(280.82543439,51.8717355)(280.68871563,51.54751673)
\curveto(280.55590311,51.2272042)(280.31176247,50.84829792)(279.95629369,50.41079789)
\curveto(279.85863743,50.29751663)(279.54809053,49.9693916)(279.02465299,49.42642281)
\curveto(278.50121545,48.88736027)(277.76293414,48.13150083)(276.80980907,47.1588445)
\closepath
}
}
{
\newrgbcolor{curcolor}{0 0 0}
\pscustom[linestyle=none,fillstyle=solid,fillcolor=curcolor]
{
\newpath
\moveto(291.82153001,52.47329805)
\lineto(291.82153001,51.46548547)
\curveto(291.51684248,51.63345424)(291.21020183,51.75845425)(290.90160806,51.8404855)
\curveto(290.59692053,51.92642301)(290.28832676,51.96939176)(289.97582673,51.96939176)
\curveto(289.27660793,51.96939176)(288.73363914,51.74673549)(288.34692036,51.30142296)
\curveto(287.96020157,50.86001667)(287.76684218,50.23892287)(287.76684218,49.43814156)
\curveto(287.76684218,48.63736025)(287.96020157,48.01431332)(288.34692036,47.56900079)
\curveto(288.73363914,47.1275945)(289.27660793,46.90689136)(289.97582673,46.90689136)
\curveto(290.28832676,46.90689136)(290.59692053,46.94790699)(290.90160806,47.02993824)
\curveto(291.21020183,47.11587575)(291.51684248,47.24282889)(291.82153001,47.41079765)
\lineto(291.82153001,46.41470382)
\curveto(291.52074873,46.27407881)(291.20824871,46.16861005)(290.88402993,46.09829755)
\curveto(290.56371741,46.02798504)(290.2219205,45.99282879)(289.85863923,45.99282879)
\curveto(288.8703579,45.99282879)(288.08520158,46.30337569)(287.50317029,46.92446949)
\curveto(286.92113899,47.54556329)(286.63012334,48.38345398)(286.63012334,49.43814156)
\curveto(286.63012334,50.50845415)(286.92309212,51.35025109)(287.50902966,51.96353239)
\curveto(288.09887346,52.57681369)(288.90551415,52.88345433)(289.92895173,52.88345433)
\curveto(290.26098301,52.88345433)(290.58520178,52.84829808)(290.90160806,52.77798558)
\curveto(291.21801433,52.71157932)(291.52465498,52.61001681)(291.82153001,52.47329805)
\closepath
}
}
{
\newrgbcolor{curcolor}{0 0 0}
\pscustom[linestyle=none,fillstyle=solid,fillcolor=curcolor]
{
\newpath
\moveto(296.25121733,51.96939176)
\curveto(295.67309229,51.96939176)(295.216061,51.74282924)(294.88012347,51.28970421)
\curveto(294.54418595,50.84048542)(294.37621718,50.22329787)(294.37621718,49.43814156)
\curveto(294.37621718,48.65298525)(294.54223282,48.03384457)(294.8742641,47.58071954)
\curveto(295.21020163,47.13150075)(295.66918604,46.90689136)(296.25121733,46.90689136)
\curveto(296.82543613,46.90689136)(297.28051429,47.13345388)(297.61645182,47.58657891)
\curveto(297.95238934,48.03970395)(298.12035811,48.6568915)(298.12035811,49.43814156)
\curveto(298.12035811,50.21548537)(297.95238934,50.8307198)(297.61645182,51.28384483)
\curveto(297.28051429,51.74087612)(296.82543613,51.96939176)(296.25121733,51.96939176)
\closepath
\moveto(296.25121733,52.88345433)
\curveto(297.18871741,52.88345433)(297.92504559,52.57876681)(298.46020188,51.96939176)
\curveto(298.99535818,51.36001671)(299.26293632,50.51626665)(299.26293632,49.43814156)
\curveto(299.26293632,48.36392273)(298.99535818,47.52017266)(298.46020188,46.90689136)
\curveto(297.92504559,46.29751631)(297.18871741,45.99282879)(296.25121733,45.99282879)
\curveto(295.30981101,45.99282879)(294.5715297,46.29751631)(294.03637341,46.90689136)
\curveto(293.50512336,47.52017266)(293.23949834,48.36392273)(293.23949834,49.43814156)
\curveto(293.23949834,50.51626665)(293.50512336,51.36001671)(294.03637341,51.96939176)
\curveto(294.5715297,52.57876681)(295.30981101,52.88345433)(296.25121733,52.88345433)
\closepath
}
}
{
\newrgbcolor{curcolor}{0 0 0}
\pscustom[linestyle=none,fillstyle=solid,fillcolor=curcolor]
{
\newpath
\moveto(306.15356129,51.46548547)
\curveto(306.42309256,51.94986051)(306.74535821,52.30728241)(307.12035824,52.53775118)
\curveto(307.49535827,52.76821995)(307.93676455,52.88345433)(308.44457709,52.88345433)
\curveto(309.1281709,52.88345433)(309.65551469,52.64321994)(310.02660847,52.16275115)
\curveto(310.39770225,51.68618861)(310.58324914,51.00650106)(310.58324914,50.12368849)
\lineto(310.58324914,46.16275068)
\lineto(309.49926468,46.16275068)
\lineto(309.49926468,50.08853224)
\curveto(309.49926468,50.71743854)(309.38793654,51.18423545)(309.16528028,51.48892297)
\curveto(308.94262401,51.7936105)(308.60278023,51.94595426)(308.14574895,51.94595426)
\curveto(307.58715515,51.94595426)(307.14574887,51.76040737)(306.82153009,51.38931359)
\curveto(306.49731131,51.01821981)(306.33520193,50.5123604)(306.33520193,49.87173535)
\lineto(306.33520193,46.16275068)
\lineto(305.25121747,46.16275068)
\lineto(305.25121747,50.08853224)
\curveto(305.25121747,50.72134479)(305.13988933,51.1881417)(304.91723306,51.48892297)
\curveto(304.6945768,51.7936105)(304.35082677,51.94595426)(303.88598298,51.94595426)
\curveto(303.33520169,51.94595426)(302.89770165,51.75845425)(302.57348288,51.38345422)
\curveto(302.2492641,51.01236044)(302.08715471,50.50845415)(302.08715471,49.87173535)
\lineto(302.08715471,46.16275068)
\lineto(301.00317025,46.16275068)
\lineto(301.00317025,52.7252512)
\lineto(302.08715471,52.7252512)
\lineto(302.08715471,51.70571987)
\curveto(302.33324848,52.10806365)(302.62817038,52.40493867)(302.97192041,52.59634494)
\curveto(303.31567044,52.7877512)(303.72387359,52.88345433)(304.19652988,52.88345433)
\curveto(304.67309242,52.88345433)(305.07738933,52.76236058)(305.4094206,52.52017306)
\curveto(305.74535813,52.27798554)(305.99340502,51.92642301)(306.15356129,51.46548547)
\closepath
}
}
{
\newrgbcolor{curcolor}{0 0 0}
\pscustom[linestyle=none,fillstyle=solid,fillcolor=curcolor]
{
\newpath
\moveto(313.78246721,47.14712575)
\lineto(313.78246721,43.66665673)
\lineto(312.69848275,43.66665673)
\lineto(312.69848275,52.7252512)
\lineto(313.78246721,52.7252512)
\lineto(313.78246721,51.72915737)
\curveto(314.00902973,52.1197824)(314.294186,52.40884492)(314.63793603,52.59634494)
\curveto(314.98559231,52.7877512)(315.39965484,52.88345433)(315.88012363,52.88345433)
\curveto(316.67699869,52.88345433)(317.32348312,52.56704806)(317.81957691,51.93423551)
\curveto(318.31957695,51.30142296)(318.56957697,50.46939164)(318.56957697,49.43814156)
\curveto(318.56957697,48.40689148)(318.31957695,47.57486016)(317.81957691,46.94204761)
\curveto(317.32348312,46.30923506)(316.67699869,45.99282879)(315.88012363,45.99282879)
\curveto(315.39965484,45.99282879)(314.98559231,46.08657879)(314.63793603,46.27407881)
\curveto(314.294186,46.46548507)(314.00902973,46.75650072)(313.78246721,47.14712575)
\closepath
\moveto(317.45043626,49.43814156)
\curveto(317.45043626,50.23111037)(317.28637374,50.85220417)(316.95824872,51.30142296)
\curveto(316.63402994,51.754548)(316.18676428,51.98111051)(315.61645173,51.98111051)
\curveto(315.04613919,51.98111051)(314.5969204,51.754548)(314.26879538,51.30142296)
\curveto(313.9445766,50.85220417)(313.78246721,50.23111037)(313.78246721,49.43814156)
\curveto(313.78246721,48.64517275)(313.9445766,48.02212582)(314.26879538,47.56900079)
\curveto(314.5969204,47.119782)(315.04613919,46.89517261)(315.61645173,46.89517261)
\curveto(316.18676428,46.89517261)(316.63402994,47.119782)(316.95824872,47.56900079)
\curveto(317.28637374,48.02212582)(317.45043626,48.64517275)(317.45043626,49.43814156)
\closepath
}
}
{
\newrgbcolor{curcolor}{0 0 0}
\pscustom[linestyle=none,fillstyle=solid,fillcolor=curcolor]
{
\newpath
\moveto(323.33911181,49.46157906)
\curveto(322.46801799,49.46157906)(321.86450232,49.36196968)(321.52856479,49.16275091)
\curveto(321.19262726,48.96353215)(321.0246585,48.62368837)(321.0246585,48.14321958)
\curveto(321.0246585,47.76040705)(321.14965851,47.45571953)(321.39965853,47.22915701)
\curveto(321.6535648,47.00650074)(321.99731483,46.89517261)(322.43090861,46.89517261)
\curveto(323.02856491,46.89517261)(323.50708057,47.10611013)(323.8664556,47.52798516)
\curveto(324.22973688,47.95376644)(324.41137752,48.51821961)(324.41137752,49.22134467)
\lineto(324.41137752,49.46157906)
\lineto(323.33911181,49.46157906)
\closepath
\moveto(325.4895026,49.9068916)
\lineto(325.4895026,46.16275068)
\lineto(324.41137752,46.16275068)
\lineto(324.41137752,47.1588445)
\curveto(324.16528375,46.76040697)(323.8586431,46.46548507)(323.49145557,46.27407881)
\curveto(323.12426804,46.08657879)(322.67504926,45.99282879)(322.14379921,45.99282879)
\curveto(321.47192416,45.99282879)(320.93676787,46.1803288)(320.53833034,46.55532883)
\curveto(320.14379905,46.93423511)(319.94653341,47.44009453)(319.94653341,48.07290708)
\curveto(319.94653341,48.81118839)(320.19262718,49.36782906)(320.68481472,49.74282909)
\curveto(321.18090851,50.11782912)(321.91918982,50.30532913)(322.89965865,50.30532913)
\lineto(324.41137752,50.30532913)
\lineto(324.41137752,50.41079789)
\curveto(324.41137752,50.90689168)(324.24731501,51.28970421)(323.91918998,51.55923548)
\curveto(323.5949712,51.832673)(323.13793992,51.96939176)(322.54809612,51.96939176)
\curveto(322.17309609,51.96939176)(321.80786169,51.92446988)(321.45239291,51.83462613)
\curveto(321.09692413,51.74478237)(320.75512723,51.61001673)(320.4270022,51.43032922)
\lineto(320.4270022,52.42642305)
\curveto(320.82153348,52.57876681)(321.20434601,52.69204807)(321.57543979,52.76626683)
\curveto(321.94653357,52.84439183)(322.30786173,52.88345433)(322.65942425,52.88345433)
\curveto(323.60864308,52.88345433)(324.31762751,52.63736057)(324.78637755,52.14517303)
\curveto(325.25512759,51.65298549)(325.4895026,50.90689168)(325.4895026,49.9068916)
\closepath
}
}
{
\newrgbcolor{curcolor}{0 0 0}
\pscustom[linestyle=none,fillstyle=solid,fillcolor=curcolor]
{
\newpath
\moveto(331.51879937,51.71743862)
\curveto(331.39770561,51.78775112)(331.2648931,51.83853238)(331.12036184,51.86978238)
\curveto(330.97973683,51.90493863)(330.82348682,51.92251676)(330.6516118,51.92251676)
\curveto(330.04223676,51.92251676)(329.57348672,51.72329799)(329.24536169,51.32486046)
\curveto(328.92114292,50.93032918)(328.75903353,50.36196976)(328.75903353,49.6197822)
\lineto(328.75903353,46.16275068)
\lineto(327.67504907,46.16275068)
\lineto(327.67504907,52.7252512)
\lineto(328.75903353,52.7252512)
\lineto(328.75903353,51.70571987)
\curveto(328.98559605,52.1041574)(329.28051795,52.3990793)(329.64379922,52.59048556)
\curveto(330.0070805,52.78579808)(330.44848679,52.88345433)(330.96801808,52.88345433)
\curveto(331.04223684,52.88345433)(331.12426809,52.87759496)(331.21411185,52.86587621)
\curveto(331.30395561,52.85806371)(331.40356499,52.84439183)(331.51294,52.82486058)
\lineto(331.51879937,51.71743862)
\closepath
}
}
{
\newrgbcolor{curcolor}{0 0 0}
\pscustom[linestyle=none,fillstyle=solid,fillcolor=curcolor]
{
\newpath
\moveto(335.64379931,49.46157906)
\curveto(334.77270549,49.46157906)(334.16918982,49.36196968)(333.83325229,49.16275091)
\curveto(333.49731476,48.96353215)(333.329346,48.62368837)(333.329346,48.14321958)
\curveto(333.329346,47.76040705)(333.45434601,47.45571953)(333.70434603,47.22915701)
\curveto(333.9582523,47.00650074)(334.30200233,46.89517261)(334.73559611,46.89517261)
\curveto(335.33325241,46.89517261)(335.81176807,47.10611013)(336.1711431,47.52798516)
\curveto(336.53442438,47.95376644)(336.71606502,48.51821961)(336.71606502,49.22134467)
\lineto(336.71606502,49.46157906)
\lineto(335.64379931,49.46157906)
\closepath
\moveto(337.7941901,49.9068916)
\lineto(337.7941901,46.16275068)
\lineto(336.71606502,46.16275068)
\lineto(336.71606502,47.1588445)
\curveto(336.46997125,46.76040697)(336.1633306,46.46548507)(335.79614307,46.27407881)
\curveto(335.42895554,46.08657879)(334.97973676,45.99282879)(334.44848671,45.99282879)
\curveto(333.77661166,45.99282879)(333.24145537,46.1803288)(332.84301784,46.55532883)
\curveto(332.44848655,46.93423511)(332.25122091,47.44009453)(332.25122091,48.07290708)
\curveto(332.25122091,48.81118839)(332.49731468,49.36782906)(332.98950222,49.74282909)
\curveto(333.48559601,50.11782912)(334.22387732,50.30532913)(335.20434615,50.30532913)
\lineto(336.71606502,50.30532913)
\lineto(336.71606502,50.41079789)
\curveto(336.71606502,50.90689168)(336.55200251,51.28970421)(336.22387748,51.55923548)
\curveto(335.8996587,51.832673)(335.44262742,51.96939176)(334.85278362,51.96939176)
\curveto(334.47778359,51.96939176)(334.11254919,51.92446988)(333.75708041,51.83462613)
\curveto(333.40161163,51.74478237)(333.05981473,51.61001673)(332.7316897,51.43032922)
\lineto(332.7316897,52.42642305)
\curveto(333.12622098,52.57876681)(333.50903351,52.69204807)(333.88012729,52.76626683)
\curveto(334.25122107,52.84439183)(334.61254923,52.88345433)(334.96411175,52.88345433)
\curveto(335.91333058,52.88345433)(336.62231501,52.63736057)(337.09106505,52.14517303)
\curveto(337.55981509,51.65298549)(337.7941901,50.90689168)(337.7941901,49.9068916)
\closepath
}
}
{
\newrgbcolor{curcolor}{0 0 0}
\pscustom[linestyle=none,fillstyle=solid,fillcolor=curcolor]
{
\newpath
\moveto(349.45434639,49.71353221)
\lineto(349.45434639,49.18618842)
\lineto(344.49731475,49.18618842)
\curveto(344.54418975,48.44400086)(344.76684602,47.87759456)(345.16528355,47.48696953)
\curveto(345.56762733,47.10025075)(346.12622113,46.90689136)(346.84106493,46.90689136)
\curveto(347.25512747,46.90689136)(347.65551812,46.95767261)(348.0422369,47.05923512)
\curveto(348.43286194,47.16079763)(348.81958072,47.31314139)(349.20239325,47.51626641)
\lineto(349.20239325,46.49673508)
\curveto(348.81567447,46.33267256)(348.41919006,46.20767255)(348.01294003,46.12173505)
\curveto(347.60669,46.03579754)(347.19458059,45.99282879)(346.7766118,45.99282879)
\curveto(345.72973672,45.99282879)(344.89965853,46.29751631)(344.28637723,46.90689136)
\curveto(343.67700218,47.51626641)(343.37231466,48.34048522)(343.37231466,49.37954781)
\curveto(343.37231466,50.45376664)(343.66137718,51.30532921)(344.23950223,51.93423551)
\curveto(344.82153352,52.56704806)(345.60473671,52.88345433)(346.58911179,52.88345433)
\curveto(347.47192436,52.88345433)(348.16919004,52.59829806)(348.68090883,52.02798552)
\curveto(349.19653387,51.46157922)(349.45434639,50.69009479)(349.45434639,49.71353221)
\closepath
\moveto(348.37622131,50.02993848)
\curveto(348.36840881,50.61978228)(348.20239317,51.09048544)(347.87817439,51.44204797)
\curveto(347.55786187,51.7936105)(347.13208058,51.96939176)(346.60083054,51.96939176)
\curveto(345.99926799,51.96939176)(345.51684608,51.79946987)(345.1535648,51.4596261)
\curveto(344.79418977,51.11978232)(344.58715851,50.64126666)(344.532471,50.02407911)
\lineto(348.37622131,50.02993848)
\closepath
}
}
{
\newrgbcolor{curcolor}{0 0 0}
\pscustom[linestyle=none,fillstyle=solid,fillcolor=curcolor]
{
\newpath
\moveto(351.2238772,55.2799389)
\lineto(352.30200228,55.2799389)
\lineto(352.30200228,46.16275068)
\lineto(351.2238772,46.16275068)
\lineto(351.2238772,55.2799389)
\closepath
}
}
{
\newrgbcolor{curcolor}{0 0 0}
\pscustom[linestyle=none,fillstyle=solid,fillcolor=curcolor]
{
\newpath
\moveto(363.82739325,50.12368849)
\lineto(363.82739325,46.16275068)
\lineto(362.74926817,46.16275068)
\lineto(362.74926817,50.08853224)
\curveto(362.74926817,50.70962604)(362.62817441,51.17446982)(362.38598689,51.4830636)
\curveto(362.14379937,51.79165737)(361.78051809,51.94595426)(361.29614305,51.94595426)
\curveto(360.71411176,51.94595426)(360.25512735,51.76040737)(359.91918982,51.38931359)
\curveto(359.58325229,51.01821981)(359.41528353,50.5123604)(359.41528353,49.87173535)
\lineto(359.41528353,46.16275068)
\lineto(358.33129907,46.16275068)
\lineto(358.33129907,52.7252512)
\lineto(359.41528353,52.7252512)
\lineto(359.41528353,51.70571987)
\curveto(359.67309605,52.10025115)(359.97583045,52.39517305)(360.32348673,52.59048556)
\curveto(360.67504925,52.78579808)(361.07934616,52.88345433)(361.53637745,52.88345433)
\curveto(362.29028376,52.88345433)(362.8605963,52.64907932)(363.24731508,52.18032928)
\curveto(363.63403386,51.71548549)(363.82739325,51.02993856)(363.82739325,50.12368849)
\closepath
}
}
{
\newrgbcolor{curcolor}{0 0 0}
\pscustom[linestyle=none,fillstyle=solid,fillcolor=curcolor]
{
\newpath
\moveto(365.9895022,52.7252512)
\lineto(367.06762728,52.7252512)
\lineto(367.06762728,46.16275068)
\lineto(365.9895022,46.16275068)
\lineto(365.9895022,52.7252512)
\closepath
\moveto(365.9895022,55.2799389)
\lineto(367.06762728,55.2799389)
\lineto(367.06762728,53.91470442)
\lineto(365.9895022,53.91470442)
\lineto(365.9895022,55.2799389)
\closepath
}
}
{
\newrgbcolor{curcolor}{0 0 0}
\pscustom[linestyle=none,fillstyle=solid,fillcolor=curcolor]
{
\newpath
\moveto(368.54418963,52.7252512)
\lineto(369.68676785,52.7252512)
\lineto(371.73754926,47.21743826)
\lineto(373.78833068,52.7252512)
\lineto(374.93090889,52.7252512)
\lineto(372.4699712,46.16275068)
\lineto(371.00512733,46.16275068)
\lineto(368.54418963,52.7252512)
\closepath
}
}
{
\newrgbcolor{curcolor}{0 0 0}
\pscustom[linestyle=none,fillstyle=solid,fillcolor=curcolor]
{
\newpath
\moveto(382.03247139,49.71353221)
\lineto(382.03247139,49.18618842)
\lineto(377.07543975,49.18618842)
\curveto(377.12231475,48.44400086)(377.34497102,47.87759456)(377.74340855,47.48696953)
\curveto(378.14575233,47.10025075)(378.70434613,46.90689136)(379.41918993,46.90689136)
\curveto(379.83325247,46.90689136)(380.23364312,46.95767261)(380.6203619,47.05923512)
\curveto(381.01098694,47.16079763)(381.39770572,47.31314139)(381.78051825,47.51626641)
\lineto(381.78051825,46.49673508)
\curveto(381.39379947,46.33267256)(380.99731506,46.20767255)(380.59106503,46.12173505)
\curveto(380.184815,46.03579754)(379.77270559,45.99282879)(379.3547368,45.99282879)
\curveto(378.30786172,45.99282879)(377.47778353,46.29751631)(376.86450223,46.90689136)
\curveto(376.25512718,47.51626641)(375.95043966,48.34048522)(375.95043966,49.37954781)
\curveto(375.95043966,50.45376664)(376.23950218,51.30532921)(376.81762723,51.93423551)
\curveto(377.39965852,52.56704806)(378.18286171,52.88345433)(379.16723679,52.88345433)
\curveto(380.05004936,52.88345433)(380.74731504,52.59829806)(381.25903383,52.02798552)
\curveto(381.77465887,51.46157922)(382.03247139,50.69009479)(382.03247139,49.71353221)
\closepath
\moveto(380.95434631,50.02993848)
\curveto(380.94653381,50.61978228)(380.78051817,51.09048544)(380.45629939,51.44204797)
\curveto(380.13598687,51.7936105)(379.71020558,51.96939176)(379.17895554,51.96939176)
\curveto(378.57739299,51.96939176)(378.09497108,51.79946987)(377.7316898,51.4596261)
\curveto(377.37231477,51.11978232)(377.16528351,50.64126666)(377.110596,50.02407911)
\lineto(380.95434631,50.02993848)
\closepath
}
}
{
\newrgbcolor{curcolor}{0 0 0}
\pscustom[linestyle=none,fillstyle=solid,fillcolor=curcolor]
{
\newpath
\moveto(383.80200983,55.2799389)
\lineto(384.88013491,55.2799389)
\lineto(384.88013491,46.16275068)
\lineto(383.80200983,46.16275068)
\lineto(383.80200983,55.2799389)
\closepath
}
}
{
\newrgbcolor{curcolor}{0 0 0}
\pscustom[linestyle=none,fillstyle=solid,fillcolor=curcolor]
{
\newpath
\moveto(393.49341628,51.96939176)
\curveto(392.91529123,51.96939176)(392.45825995,51.74282924)(392.12232242,51.28970421)
\curveto(391.78638489,50.84048542)(391.61841613,50.22329787)(391.61841613,49.43814156)
\curveto(391.61841613,48.65298525)(391.78443177,48.03384457)(392.11646304,47.58071954)
\curveto(392.45240057,47.13150075)(392.91138498,46.90689136)(393.49341628,46.90689136)
\curveto(394.06763507,46.90689136)(394.52271323,47.13345388)(394.85865076,47.58657891)
\curveto(395.19458829,48.03970395)(395.36255705,48.6568915)(395.36255705,49.43814156)
\curveto(395.36255705,50.21548537)(395.19458829,50.8307198)(394.85865076,51.28384483)
\curveto(394.52271323,51.74087612)(394.06763507,51.96939176)(393.49341628,51.96939176)
\closepath
\moveto(393.49341628,52.88345433)
\curveto(394.43091635,52.88345433)(395.16724454,52.57876681)(395.70240083,51.96939176)
\curveto(396.23755712,51.36001671)(396.50513527,50.51626665)(396.50513527,49.43814156)
\curveto(396.50513527,48.36392273)(396.23755712,47.52017266)(395.70240083,46.90689136)
\curveto(395.16724454,46.29751631)(394.43091635,45.99282879)(393.49341628,45.99282879)
\curveto(392.55200995,45.99282879)(391.81372864,46.29751631)(391.27857235,46.90689136)
\curveto(390.74732231,47.52017266)(390.48169729,48.36392273)(390.48169729,49.43814156)
\curveto(390.48169729,50.51626665)(390.74732231,51.36001671)(391.27857235,51.96939176)
\curveto(391.81372864,52.57876681)(392.55200995,52.88345433)(393.49341628,52.88345433)
\closepath
}
}
{
\newrgbcolor{curcolor}{0 0 0}
\pscustom[linestyle=none,fillstyle=solid,fillcolor=curcolor]
{
\newpath
\moveto(402.10669733,55.2799389)
\lineto(403.18482241,55.2799389)
\lineto(403.18482241,46.16275068)
\lineto(402.10669733,46.16275068)
\lineto(402.10669733,55.2799389)
\closepath
}
}
{
\newrgbcolor{curcolor}{0 0 0}
\pscustom[linestyle=none,fillstyle=solid,fillcolor=curcolor]
{
\newpath
\moveto(408.41724444,49.46157906)
\curveto(407.54615062,49.46157906)(406.94263495,49.36196968)(406.60669742,49.16275091)
\curveto(406.27075989,48.96353215)(406.10279113,48.62368837)(406.10279113,48.14321958)
\curveto(406.10279113,47.76040705)(406.22779114,47.45571953)(406.47779116,47.22915701)
\curveto(406.73169743,47.00650074)(407.07544746,46.89517261)(407.50904124,46.89517261)
\curveto(408.10669754,46.89517261)(408.5852132,47.10611013)(408.94458823,47.52798516)
\curveto(409.30786951,47.95376644)(409.48951015,48.51821961)(409.48951015,49.22134467)
\lineto(409.48951015,49.46157906)
\lineto(408.41724444,49.46157906)
\closepath
\moveto(410.56763523,49.9068916)
\lineto(410.56763523,46.16275068)
\lineto(409.48951015,46.16275068)
\lineto(409.48951015,47.1588445)
\curveto(409.24341638,46.76040697)(408.93677573,46.46548507)(408.5695882,46.27407881)
\curveto(408.20240067,46.08657879)(407.75318188,45.99282879)(407.22193184,45.99282879)
\curveto(406.55005679,45.99282879)(406.0149005,46.1803288)(405.61646296,46.55532883)
\curveto(405.22193168,46.93423511)(405.02466604,47.44009453)(405.02466604,48.07290708)
\curveto(405.02466604,48.81118839)(405.27075981,49.36782906)(405.76294735,49.74282909)
\curveto(406.25904114,50.11782912)(406.99732245,50.30532913)(407.97779128,50.30532913)
\lineto(409.48951015,50.30532913)
\lineto(409.48951015,50.41079789)
\curveto(409.48951015,50.90689168)(409.32544763,51.28970421)(408.99732261,51.55923548)
\curveto(408.67310383,51.832673)(408.21607255,51.96939176)(407.62622875,51.96939176)
\curveto(407.25122872,51.96939176)(406.88599432,51.92446988)(406.53052554,51.83462613)
\curveto(406.17505676,51.74478237)(405.83325986,51.61001673)(405.50513483,51.43032922)
\lineto(405.50513483,52.42642305)
\curveto(405.89966611,52.57876681)(406.28247864,52.69204807)(406.65357242,52.76626683)
\curveto(407.0246662,52.84439183)(407.38599436,52.88345433)(407.73755688,52.88345433)
\curveto(408.68677571,52.88345433)(409.39576014,52.63736057)(409.86451018,52.14517303)
\curveto(410.33326021,51.65298549)(410.56763523,50.90689168)(410.56763523,49.9068916)
\closepath
}
}
{
\newrgbcolor{curcolor}{0 0 0}
\pscustom[linestyle=none,fillstyle=solid,fillcolor=curcolor]
{
\newpath
\moveto(421.33716645,52.47329805)
\lineto(421.33716645,51.46548547)
\curveto(421.03247893,51.63345424)(420.72583828,51.75845425)(420.4172445,51.8404855)
\curveto(420.11255698,51.92642301)(419.8039632,51.96939176)(419.49146318,51.96939176)
\curveto(418.79224437,51.96939176)(418.24927558,51.74673549)(417.8625568,51.30142296)
\curveto(417.47583802,50.86001667)(417.28247863,50.23892287)(417.28247863,49.43814156)
\curveto(417.28247863,48.63736025)(417.47583802,48.01431332)(417.8625568,47.56900079)
\curveto(418.24927558,47.1275945)(418.79224437,46.90689136)(419.49146318,46.90689136)
\curveto(419.8039632,46.90689136)(420.11255698,46.94790699)(420.4172445,47.02993824)
\curveto(420.72583828,47.11587575)(421.03247893,47.24282889)(421.33716645,47.41079765)
\lineto(421.33716645,46.41470382)
\curveto(421.03638518,46.27407881)(420.72388515,46.16861005)(420.39966638,46.09829755)
\curveto(420.07935385,46.02798504)(419.73755695,45.99282879)(419.37427567,45.99282879)
\curveto(418.38599434,45.99282879)(417.60083803,46.30337569)(417.01880673,46.92446949)
\curveto(416.43677544,47.54556329)(416.14575979,48.38345398)(416.14575979,49.43814156)
\curveto(416.14575979,50.50845415)(416.43872856,51.35025109)(417.02466611,51.96353239)
\curveto(417.6145099,52.57681369)(418.42115059,52.88345433)(419.44458818,52.88345433)
\curveto(419.77661945,52.88345433)(420.10083823,52.84829808)(420.4172445,52.77798558)
\curveto(420.73365078,52.71157932)(421.04029143,52.61001681)(421.33716645,52.47329805)
\closepath
}
}
{
\newrgbcolor{curcolor}{0 0 0}
\pscustom[linestyle=none,fillstyle=solid,fillcolor=curcolor]
{
\newpath
\moveto(426.20630694,49.46157906)
\curveto(425.33521312,49.46157906)(424.73169745,49.36196968)(424.39575992,49.16275091)
\curveto(424.05982239,48.96353215)(423.89185363,48.62368837)(423.89185363,48.14321958)
\curveto(423.89185363,47.76040705)(424.01685364,47.45571953)(424.26685366,47.22915701)
\curveto(424.52075993,47.00650074)(424.86450996,46.89517261)(425.29810374,46.89517261)
\curveto(425.89576004,46.89517261)(426.3742757,47.10611013)(426.73365073,47.52798516)
\curveto(427.09693201,47.95376644)(427.27857265,48.51821961)(427.27857265,49.22134467)
\lineto(427.27857265,49.46157906)
\lineto(426.20630694,49.46157906)
\closepath
\moveto(428.35669773,49.9068916)
\lineto(428.35669773,46.16275068)
\lineto(427.27857265,46.16275068)
\lineto(427.27857265,47.1588445)
\curveto(427.03247888,46.76040697)(426.72583823,46.46548507)(426.3586507,46.27407881)
\curveto(425.99146317,46.08657879)(425.54224438,45.99282879)(425.01099434,45.99282879)
\curveto(424.33911929,45.99282879)(423.803963,46.1803288)(423.40552546,46.55532883)
\curveto(423.01099418,46.93423511)(422.81372854,47.44009453)(422.81372854,48.07290708)
\curveto(422.81372854,48.81118839)(423.05982231,49.36782906)(423.55200985,49.74282909)
\curveto(424.04810364,50.11782912)(424.78638495,50.30532913)(425.76685378,50.30532913)
\lineto(427.27857265,50.30532913)
\lineto(427.27857265,50.41079789)
\curveto(427.27857265,50.90689168)(427.11451013,51.28970421)(426.78638511,51.55923548)
\curveto(426.46216633,51.832673)(426.00513505,51.96939176)(425.41529125,51.96939176)
\curveto(425.04029122,51.96939176)(424.67505682,51.92446988)(424.31958804,51.83462613)
\curveto(423.96411926,51.74478237)(423.62232236,51.61001673)(423.29419733,51.43032922)
\lineto(423.29419733,52.42642305)
\curveto(423.68872861,52.57876681)(424.07154114,52.69204807)(424.44263492,52.76626683)
\curveto(424.8137287,52.84439183)(425.17505686,52.88345433)(425.52661938,52.88345433)
\curveto(426.47583821,52.88345433)(427.18482264,52.63736057)(427.65357268,52.14517303)
\curveto(428.12232271,51.65298549)(428.35669773,50.90689168)(428.35669773,49.9068916)
\closepath
}
}
{
\newrgbcolor{curcolor}{0 0 0}
\pscustom[linestyle=none,fillstyle=solid,fillcolor=curcolor]
{
\newpath
\moveto(436.03833838,50.12368849)
\lineto(436.03833838,46.16275068)
\lineto(434.9602133,46.16275068)
\lineto(434.9602133,50.08853224)
\curveto(434.9602133,50.70962604)(434.83911954,51.17446982)(434.59693202,51.4830636)
\curveto(434.3547445,51.79165737)(433.99146322,51.94595426)(433.50708818,51.94595426)
\curveto(432.92505689,51.94595426)(432.46607247,51.76040737)(432.13013495,51.38931359)
\curveto(431.79419742,51.01821981)(431.62622866,50.5123604)(431.62622866,49.87173535)
\lineto(431.62622866,46.16275068)
\lineto(430.5422442,46.16275068)
\lineto(430.5422442,52.7252512)
\lineto(431.62622866,52.7252512)
\lineto(431.62622866,51.70571987)
\curveto(431.88404118,52.10025115)(432.18677558,52.39517305)(432.53443186,52.59048556)
\curveto(432.88599438,52.78579808)(433.29029129,52.88345433)(433.74732258,52.88345433)
\curveto(434.50122889,52.88345433)(435.07154143,52.64907932)(435.45826021,52.18032928)
\curveto(435.84497899,51.71548549)(436.03833838,51.02993856)(436.03833838,50.12368849)
\closepath
}
}
{
\newrgbcolor{curcolor}{0 0 0}
\pscustom[linestyle=none,fillstyle=solid,fillcolor=curcolor]
{
\newpath
\moveto(439.26685366,54.5885326)
\lineto(439.26685366,52.7252512)
\lineto(441.48755696,52.7252512)
\lineto(441.48755696,51.88736051)
\lineto(439.26685366,51.88736051)
\lineto(439.26685366,48.32486022)
\curveto(439.26685366,47.78970393)(439.33911929,47.4459539)(439.48365055,47.29361014)
\curveto(439.63208806,47.14126638)(439.93091621,47.0650945)(440.380135,47.0650945)
\lineto(441.48755696,47.0650945)
\lineto(441.48755696,46.16275068)
\lineto(440.380135,46.16275068)
\curveto(439.54810368,46.16275068)(438.97388489,46.31704756)(438.65747861,46.62564134)
\curveto(438.34107234,46.93814136)(438.1828692,47.50454766)(438.1828692,48.32486022)
\lineto(438.1828692,51.88736051)
\lineto(437.39185351,51.88736051)
\lineto(437.39185351,52.7252512)
\lineto(438.1828692,52.7252512)
\lineto(438.1828692,54.5885326)
\lineto(439.26685366,54.5885326)
\closepath
}
}
{
\newrgbcolor{curcolor}{0 0 0}
\pscustom[linestyle=none,fillstyle=solid,fillcolor=curcolor]
{
\newpath
\moveto(442.91138483,52.7252512)
\lineto(443.98950991,52.7252512)
\lineto(443.98950991,46.16275068)
\lineto(442.91138483,46.16275068)
\lineto(442.91138483,52.7252512)
\closepath
\moveto(442.91138483,55.2799389)
\lineto(443.98950991,55.2799389)
\lineto(443.98950991,53.91470442)
\lineto(442.91138483,53.91470442)
\lineto(442.91138483,55.2799389)
\closepath
}
}
{
\newrgbcolor{curcolor}{0 0 0}
\pscustom[linestyle=none,fillstyle=solid,fillcolor=curcolor]
{
\newpath
\moveto(450.55786954,51.72915737)
\lineto(450.55786954,55.2799389)
\lineto(451.63599463,55.2799389)
\lineto(451.63599463,46.16275068)
\lineto(450.55786954,46.16275068)
\lineto(450.55786954,47.14712575)
\curveto(450.33130703,46.75650072)(450.04419763,46.46548507)(449.69654135,46.27407881)
\curveto(449.35279132,46.08657879)(448.93872879,45.99282879)(448.45435375,45.99282879)
\curveto(447.66138494,45.99282879)(447.01490051,46.30923506)(446.51490047,46.94204761)
\curveto(446.01880668,47.57486016)(445.77075979,48.40689148)(445.77075979,49.43814156)
\curveto(445.77075979,50.46939164)(446.01880668,51.30142296)(446.51490047,51.93423551)
\curveto(447.01490051,52.56704806)(447.66138494,52.88345433)(448.45435375,52.88345433)
\curveto(448.93872879,52.88345433)(449.35279132,52.7877512)(449.69654135,52.59634494)
\curveto(450.04419763,52.40884492)(450.33130703,52.1197824)(450.55786954,51.72915737)
\closepath
\moveto(446.88404113,49.43814156)
\curveto(446.88404113,48.64517275)(447.04615051,48.02212582)(447.37036929,47.56900079)
\curveto(447.69849432,47.119782)(448.1477131,46.89517261)(448.71802565,46.89517261)
\curveto(449.28833819,46.89517261)(449.73755698,47.119782)(450.065682,47.56900079)
\curveto(450.39380703,48.02212582)(450.55786954,48.64517275)(450.55786954,49.43814156)
\curveto(450.55786954,50.23111037)(450.39380703,50.85220417)(450.065682,51.30142296)
\curveto(449.73755698,51.754548)(449.28833819,51.98111051)(448.71802565,51.98111051)
\curveto(448.1477131,51.98111051)(447.69849432,51.754548)(447.37036929,51.30142296)
\curveto(447.04615051,50.85220417)(446.88404113,50.23111037)(446.88404113,49.43814156)
\closepath
}
}
{
\newrgbcolor{curcolor}{0 0 0}
\pscustom[linestyle=none,fillstyle=solid,fillcolor=curcolor]
{
\newpath
\moveto(456.83911944,49.46157906)
\curveto(455.96802562,49.46157906)(455.36450995,49.36196968)(455.02857242,49.16275091)
\curveto(454.69263489,48.96353215)(454.52466613,48.62368837)(454.52466613,48.14321958)
\curveto(454.52466613,47.76040705)(454.64966614,47.45571953)(454.89966616,47.22915701)
\curveto(455.15357243,47.00650074)(455.49732246,46.89517261)(455.93091624,46.89517261)
\curveto(456.52857254,46.89517261)(457.0070882,47.10611013)(457.36646323,47.52798516)
\curveto(457.72974451,47.95376644)(457.91138515,48.51821961)(457.91138515,49.22134467)
\lineto(457.91138515,49.46157906)
\lineto(456.83911944,49.46157906)
\closepath
\moveto(458.98951023,49.9068916)
\lineto(458.98951023,46.16275068)
\lineto(457.91138515,46.16275068)
\lineto(457.91138515,47.1588445)
\curveto(457.66529138,46.76040697)(457.35865073,46.46548507)(456.9914632,46.27407881)
\curveto(456.62427567,46.08657879)(456.17505688,45.99282879)(455.64380684,45.99282879)
\curveto(454.97193179,45.99282879)(454.4367755,46.1803288)(454.03833796,46.55532883)
\curveto(453.64380668,46.93423511)(453.44654104,47.44009453)(453.44654104,48.07290708)
\curveto(453.44654104,48.81118839)(453.69263481,49.36782906)(454.18482235,49.74282909)
\curveto(454.68091614,50.11782912)(455.41919745,50.30532913)(456.39966628,50.30532913)
\lineto(457.91138515,50.30532913)
\lineto(457.91138515,50.41079789)
\curveto(457.91138515,50.90689168)(457.74732263,51.28970421)(457.41919761,51.55923548)
\curveto(457.09497883,51.832673)(456.63794755,51.96939176)(456.04810375,51.96939176)
\curveto(455.67310372,51.96939176)(455.30786932,51.92446988)(454.95240054,51.83462613)
\curveto(454.59693176,51.74478237)(454.25513486,51.61001673)(453.92700983,51.43032922)
\lineto(453.92700983,52.42642305)
\curveto(454.32154111,52.57876681)(454.70435364,52.69204807)(455.07544742,52.76626683)
\curveto(455.4465412,52.84439183)(455.80786936,52.88345433)(456.15943188,52.88345433)
\curveto(457.10865071,52.88345433)(457.81763514,52.63736057)(458.28638518,52.14517303)
\curveto(458.75513521,51.65298549)(458.98951023,50.90689168)(458.98951023,49.9068916)
\closepath
}
}
{
\newrgbcolor{curcolor}{0 0 0}
\pscustom[linestyle=none,fillstyle=solid,fillcolor=curcolor]
{
\newpath
\moveto(465.53443204,51.72915737)
\lineto(465.53443204,55.2799389)
\lineto(466.61255713,55.2799389)
\lineto(466.61255713,46.16275068)
\lineto(465.53443204,46.16275068)
\lineto(465.53443204,47.14712575)
\curveto(465.30786953,46.75650072)(465.02076013,46.46548507)(464.67310385,46.27407881)
\curveto(464.32935382,46.08657879)(463.91529129,45.99282879)(463.43091625,45.99282879)
\curveto(462.63794744,45.99282879)(461.99146301,46.30923506)(461.49146297,46.94204761)
\curveto(460.99536918,47.57486016)(460.74732229,48.40689148)(460.74732229,49.43814156)
\curveto(460.74732229,50.46939164)(460.99536918,51.30142296)(461.49146297,51.93423551)
\curveto(461.99146301,52.56704806)(462.63794744,52.88345433)(463.43091625,52.88345433)
\curveto(463.91529129,52.88345433)(464.32935382,52.7877512)(464.67310385,52.59634494)
\curveto(465.02076013,52.40884492)(465.30786953,52.1197824)(465.53443204,51.72915737)
\closepath
\moveto(461.86060363,49.43814156)
\curveto(461.86060363,48.64517275)(462.02271301,48.02212582)(462.34693179,47.56900079)
\curveto(462.67505682,47.119782)(463.1242756,46.89517261)(463.69458815,46.89517261)
\curveto(464.26490069,46.89517261)(464.71411948,47.119782)(465.0422445,47.56900079)
\curveto(465.37036953,48.02212582)(465.53443204,48.64517275)(465.53443204,49.43814156)
\curveto(465.53443204,50.23111037)(465.37036953,50.85220417)(465.0422445,51.30142296)
\curveto(464.71411948,51.754548)(464.26490069,51.98111051)(463.69458815,51.98111051)
\curveto(463.1242756,51.98111051)(462.67505682,51.754548)(462.34693179,51.30142296)
\curveto(462.02271301,50.85220417)(461.86060363,50.23111037)(461.86060363,49.43814156)
\closepath
}
}
{
\newrgbcolor{curcolor}{0 0 0}
\pscustom[linestyle=none,fillstyle=solid,fillcolor=curcolor]
{
\newpath
\moveto(257.64379178,36.72915618)
\lineto(257.64379178,40.27993771)
\lineto(258.72191687,40.27993771)
\lineto(258.72191687,31.16274948)
\lineto(257.64379178,31.16274948)
\lineto(257.64379178,32.14712456)
\curveto(257.41722927,31.75649953)(257.13011987,31.46548388)(256.78246359,31.27407762)
\curveto(256.43871356,31.0865776)(256.02465103,30.9928276)(255.54027599,30.9928276)
\curveto(254.74730718,30.9928276)(254.10082275,31.30923387)(253.60082271,31.94204642)
\curveto(253.10472892,32.57485897)(252.85668203,33.40689029)(252.85668203,34.43814037)
\curveto(252.85668203,35.46939045)(253.10472892,36.30142177)(253.60082271,36.93423432)
\curveto(254.10082275,37.56704687)(254.74730718,37.88345314)(255.54027599,37.88345314)
\curveto(256.02465103,37.88345314)(256.43871356,37.78775001)(256.78246359,37.59634374)
\curveto(257.13011987,37.40884373)(257.41722927,37.11978121)(257.64379178,36.72915618)
\closepath
\moveto(253.96996337,34.43814037)
\curveto(253.96996337,33.64517156)(254.13207276,33.02212463)(254.45629153,32.5689996)
\curveto(254.78441656,32.11978081)(255.23363534,31.89517142)(255.80394789,31.89517142)
\curveto(256.37426043,31.89517142)(256.82347922,32.11978081)(257.15160425,32.5689996)
\curveto(257.47972927,33.02212463)(257.64379178,33.64517156)(257.64379178,34.43814037)
\curveto(257.64379178,35.23110918)(257.47972927,35.85220298)(257.15160425,36.30142177)
\curveto(256.82347922,36.7545468)(256.37426043,36.98110932)(255.80394789,36.98110932)
\curveto(255.23363534,36.98110932)(254.78441656,36.7545468)(254.45629153,36.30142177)
\curveto(254.13207276,35.85220298)(253.96996337,35.23110918)(253.96996337,34.43814037)
\closepath
}
}
{
\newrgbcolor{curcolor}{0 0 0}
\pscustom[linestyle=none,fillstyle=solid,fillcolor=curcolor]
{
\newpath
\moveto(266.55590174,34.71353102)
\lineto(266.55590174,34.18618722)
\lineto(261.5988701,34.18618722)
\curveto(261.6457451,33.44399966)(261.86840137,32.87759337)(262.2668389,32.48696834)
\curveto(262.66918268,32.10024956)(263.22777647,31.90689017)(263.94262028,31.90689017)
\curveto(264.35668281,31.90689017)(264.75707347,31.95767142)(265.14379225,32.05923393)
\curveto(265.53441728,32.16079644)(265.92113606,32.3131402)(266.30394859,32.51626522)
\lineto(266.30394859,31.49673389)
\curveto(265.91722981,31.33267137)(265.52074541,31.20767136)(265.11449537,31.12173386)
\curveto(264.70824534,31.03579635)(264.29613593,30.9928276)(263.87816715,30.9928276)
\curveto(262.83129207,30.9928276)(262.00121388,31.29751512)(261.38793258,31.90689017)
\curveto(260.77855753,32.51626522)(260.47387001,33.34048403)(260.47387001,34.37954661)
\curveto(260.47387001,35.45376545)(260.76293253,36.30532802)(261.34105757,36.93423432)
\curveto(261.92308887,37.56704687)(262.70629206,37.88345314)(263.69066714,37.88345314)
\curveto(264.57347971,37.88345314)(265.27074539,37.59829687)(265.78246418,37.02798432)
\curveto(266.29808922,36.46157803)(266.55590174,35.69009359)(266.55590174,34.71353102)
\closepath
\moveto(265.47777665,35.02993729)
\curveto(265.46996415,35.61978109)(265.30394851,36.09048425)(264.97972974,36.44204678)
\curveto(264.65941721,36.79360931)(264.23363593,36.96939057)(263.70238589,36.96939057)
\curveto(263.10082334,36.96939057)(262.61840143,36.79946868)(262.25512015,36.4596249)
\curveto(261.89574512,36.11978113)(261.68871385,35.64126546)(261.63402635,35.02407792)
\lineto(265.47777665,35.02993729)
\closepath
}
}
{
\newrgbcolor{curcolor}{0 0 0}
\pscustom[linestyle=none,fillstyle=solid,fillcolor=curcolor]
{
\newpath
\moveto(275.46801236,40.27993771)
\lineto(275.46801236,39.38345326)
\lineto(274.43676228,39.38345326)
\curveto(274.0500435,39.38345326)(273.78051223,39.30532826)(273.62816847,39.14907824)
\curveto(273.47973095,38.99282823)(273.4055122,38.71157821)(273.4055122,38.30532818)
\lineto(273.4055122,37.72525001)
\lineto(275.18090296,37.72525001)
\lineto(275.18090296,36.88735931)
\lineto(273.4055122,36.88735931)
\lineto(273.4055122,31.16274948)
\lineto(272.32152774,31.16274948)
\lineto(272.32152774,36.88735931)
\lineto(271.29027766,36.88735931)
\lineto(271.29027766,37.72525001)
\lineto(272.32152774,37.72525001)
\lineto(272.32152774,38.18228129)
\curveto(272.32152774,38.9127501)(272.49144963,39.44400014)(272.8312934,39.77603142)
\curveto(273.17113718,40.11196894)(273.71019972,40.27993771)(274.44848103,40.27993771)
\lineto(275.46801236,40.27993771)
\closepath
}
}
{
\newrgbcolor{curcolor}{0 0 0}
\pscustom[linestyle=none,fillstyle=solid,fillcolor=curcolor]
{
\newpath
\moveto(276.36449647,40.27993771)
\lineto(277.44262156,40.27993771)
\lineto(277.44262156,31.16274948)
\lineto(276.36449647,31.16274948)
\lineto(276.36449647,40.27993771)
\closepath
}
}
{
\newrgbcolor{curcolor}{0 0 0}
\pscustom[linestyle=none,fillstyle=solid,fillcolor=curcolor]
{
\newpath
\moveto(279.58129334,33.75259344)
\lineto(279.58129334,37.72525001)
\lineto(280.65941843,37.72525001)
\lineto(280.65941843,33.79360907)
\curveto(280.65941843,33.17251527)(280.78051219,32.70571836)(281.0226997,32.39321833)
\curveto(281.26488722,32.08462456)(281.6281685,31.93032767)(282.11254354,31.93032767)
\curveto(282.69457484,31.93032767)(283.15355925,32.11587456)(283.48949678,32.48696834)
\curveto(283.82934055,32.85806212)(283.99926244,33.36392153)(283.99926244,34.00454658)
\lineto(283.99926244,37.72525001)
\lineto(285.07738753,37.72525001)
\lineto(285.07738753,31.16274948)
\lineto(283.99926244,31.16274948)
\lineto(283.99926244,32.17056206)
\curveto(283.73754367,31.77212453)(283.43285615,31.47524951)(283.08519987,31.27993699)
\curveto(282.74144984,31.08853073)(282.34105918,30.9928276)(281.8840279,30.9928276)
\curveto(281.13012159,30.9928276)(280.55785592,31.22720261)(280.16723089,31.69595265)
\curveto(279.77660586,32.16470269)(279.58129334,32.85024962)(279.58129334,33.75259344)
\closepath
}
}
{
\newrgbcolor{curcolor}{0 0 0}
\pscustom[linestyle=none,fillstyle=solid,fillcolor=curcolor]
{
\newpath
\moveto(287.30981088,37.72525001)
\lineto(288.38793597,37.72525001)
\lineto(288.38793597,31.16274948)
\lineto(287.30981088,31.16274948)
\lineto(287.30981088,37.72525001)
\closepath
\moveto(287.30981088,40.27993771)
\lineto(288.38793597,40.27993771)
\lineto(288.38793597,38.91470322)
\lineto(287.30981088,38.91470322)
\lineto(287.30981088,40.27993771)
\closepath
}
}
{
\newrgbcolor{curcolor}{0 0 0}
\pscustom[linestyle=none,fillstyle=solid,fillcolor=curcolor]
{
\newpath
\moveto(294.9562956,36.72915618)
\lineto(294.9562956,40.27993771)
\lineto(296.03442069,40.27993771)
\lineto(296.03442069,31.16274948)
\lineto(294.9562956,31.16274948)
\lineto(294.9562956,32.14712456)
\curveto(294.72973308,31.75649953)(294.44262368,31.46548388)(294.09496741,31.27407762)
\curveto(293.75121738,31.0865776)(293.33715485,30.9928276)(292.85277981,30.9928276)
\curveto(292.05981099,30.9928276)(291.41332657,31.30923387)(290.91332653,31.94204642)
\curveto(290.41723274,32.57485897)(290.16918584,33.40689029)(290.16918584,34.43814037)
\curveto(290.16918584,35.46939045)(290.41723274,36.30142177)(290.91332653,36.93423432)
\curveto(291.41332657,37.56704687)(292.05981099,37.88345314)(292.85277981,37.88345314)
\curveto(293.33715485,37.88345314)(293.75121738,37.78775001)(294.09496741,37.59634374)
\curveto(294.44262368,37.40884373)(294.72973308,37.11978121)(294.9562956,36.72915618)
\closepath
\moveto(291.28246718,34.43814037)
\curveto(291.28246718,33.64517156)(291.44457657,33.02212463)(291.76879535,32.5689996)
\curveto(292.09692037,32.11978081)(292.54613916,31.89517142)(293.1164517,31.89517142)
\curveto(293.68676425,31.89517142)(294.13598303,32.11978081)(294.46410806,32.5689996)
\curveto(294.79223309,33.02212463)(294.9562956,33.64517156)(294.9562956,34.43814037)
\curveto(294.9562956,35.23110918)(294.79223309,35.85220298)(294.46410806,36.30142177)
\curveto(294.13598303,36.7545468)(293.68676425,36.98110932)(293.1164517,36.98110932)
\curveto(292.54613916,36.98110932)(292.09692037,36.7545468)(291.76879535,36.30142177)
\curveto(291.44457657,35.85220298)(291.28246718,35.23110918)(291.28246718,34.43814037)
\closepath
}
}
{
\newrgbcolor{curcolor}{0 0 0}
\pscustom[linestyle=none,fillstyle=solid,fillcolor=curcolor]
{
\newpath
\moveto(300.79809233,36.96939057)
\curveto(300.21996729,36.96939057)(299.762936,36.74282805)(299.42699847,36.28970302)
\curveto(299.09106095,35.84048423)(298.92309218,35.22329668)(298.92309218,34.43814037)
\curveto(298.92309218,33.65298406)(299.08910782,33.03384338)(299.4211391,32.58071835)
\curveto(299.75707663,32.13149956)(300.21606104,31.90689017)(300.79809233,31.90689017)
\curveto(301.37231113,31.90689017)(301.82738929,32.13345269)(302.16332682,32.58657772)
\curveto(302.49926434,33.03970276)(302.66723311,33.65689031)(302.66723311,34.43814037)
\curveto(302.66723311,35.21548418)(302.49926434,35.8307186)(302.16332682,36.28384364)
\curveto(301.82738929,36.74087493)(301.37231113,36.96939057)(300.79809233,36.96939057)
\closepath
\moveto(300.79809233,37.88345314)
\curveto(301.73559241,37.88345314)(302.47192059,37.57876562)(303.00707688,36.96939057)
\curveto(303.54223318,36.36001552)(303.80981132,35.51626545)(303.80981132,34.43814037)
\curveto(303.80981132,33.36392153)(303.54223318,32.52017147)(303.00707688,31.90689017)
\curveto(302.47192059,31.29751512)(301.73559241,30.9928276)(300.79809233,30.9928276)
\curveto(299.85668601,30.9928276)(299.1184047,31.29751512)(298.58324841,31.90689017)
\curveto(298.05199836,32.52017147)(297.78637334,33.36392153)(297.78637334,34.43814037)
\curveto(297.78637334,35.51626545)(298.05199836,36.36001552)(298.58324841,36.96939057)
\curveto(299.1184047,37.57876562)(299.85668601,37.88345314)(300.79809233,37.88345314)
\closepath
}
}
{
\newrgbcolor{curcolor}{0 0 0}
\pscustom[linestyle=none,fillstyle=solid,fillcolor=curcolor]
{
\newpath
\moveto(314.13403001,37.47329686)
\lineto(314.13403001,36.46548428)
\curveto(313.82934248,36.63345304)(313.52270183,36.75845305)(313.21410806,36.84048431)
\curveto(312.90942053,36.92642182)(312.60082676,36.96939057)(312.28832673,36.96939057)
\curveto(311.58910793,36.96939057)(311.04613914,36.7467343)(310.65942036,36.30142177)
\curveto(310.27270157,35.86001548)(310.07934218,35.23892168)(310.07934218,34.43814037)
\curveto(310.07934218,33.63735906)(310.27270157,33.01431213)(310.65942036,32.5689996)
\curveto(311.04613914,32.12759331)(311.58910793,31.90689017)(312.28832673,31.90689017)
\curveto(312.60082676,31.90689017)(312.90942053,31.9479058)(313.21410806,32.02993705)
\curveto(313.52270183,32.11587456)(313.82934248,32.24282769)(314.13403001,32.41079646)
\lineto(314.13403001,31.41470263)
\curveto(313.83324873,31.27407762)(313.52074871,31.16860886)(313.19652993,31.09829635)
\curveto(312.87621741,31.02798385)(312.5344205,30.9928276)(312.17113923,30.9928276)
\curveto(311.1828579,30.9928276)(310.39770158,31.30337449)(309.81567029,31.92446829)
\curveto(309.23363899,32.54556209)(308.94262334,33.38345279)(308.94262334,34.43814037)
\curveto(308.94262334,35.50845295)(309.23559212,36.3502499)(309.82152966,36.96353119)
\curveto(310.41137346,37.57681249)(311.21801415,37.88345314)(312.24145173,37.88345314)
\curveto(312.57348301,37.88345314)(312.89770178,37.84829689)(313.21410806,37.77798438)
\curveto(313.53051433,37.71157813)(313.83715498,37.61001562)(314.13403001,37.47329686)
\closepath
}
}
{
\newrgbcolor{curcolor}{0 0 0}
\pscustom[linestyle=none,fillstyle=solid,fillcolor=curcolor]
{
\newpath
\moveto(318.56371733,36.96939057)
\curveto(317.98559229,36.96939057)(317.528561,36.74282805)(317.19262347,36.28970302)
\curveto(316.85668595,35.84048423)(316.68871718,35.22329668)(316.68871718,34.43814037)
\curveto(316.68871718,33.65298406)(316.85473282,33.03384338)(317.1867641,32.58071835)
\curveto(317.52270163,32.13149956)(317.98168604,31.90689017)(318.56371733,31.90689017)
\curveto(319.13793613,31.90689017)(319.59301429,32.13345269)(319.92895182,32.58657772)
\curveto(320.26488934,33.03970276)(320.43285811,33.65689031)(320.43285811,34.43814037)
\curveto(320.43285811,35.21548418)(320.26488934,35.8307186)(319.92895182,36.28384364)
\curveto(319.59301429,36.74087493)(319.13793613,36.96939057)(318.56371733,36.96939057)
\closepath
\moveto(318.56371733,37.88345314)
\curveto(319.50121741,37.88345314)(320.23754559,37.57876562)(320.77270188,36.96939057)
\curveto(321.30785818,36.36001552)(321.57543632,35.51626545)(321.57543632,34.43814037)
\curveto(321.57543632,33.36392153)(321.30785818,32.52017147)(320.77270188,31.90689017)
\curveto(320.23754559,31.29751512)(319.50121741,30.9928276)(318.56371733,30.9928276)
\curveto(317.62231101,30.9928276)(316.8840297,31.29751512)(316.34887341,31.90689017)
\curveto(315.81762336,32.52017147)(315.55199834,33.36392153)(315.55199834,34.43814037)
\curveto(315.55199834,35.51626545)(315.81762336,36.36001552)(316.34887341,36.96939057)
\curveto(316.8840297,37.57876562)(317.62231101,37.88345314)(318.56371733,37.88345314)
\closepath
}
}
{
\newrgbcolor{curcolor}{0 0 0}
\pscustom[linestyle=none,fillstyle=solid,fillcolor=curcolor]
{
\newpath
\moveto(328.81176825,35.1236873)
\lineto(328.81176825,31.16274948)
\lineto(327.73364317,31.16274948)
\lineto(327.73364317,35.08853105)
\curveto(327.73364317,35.70962485)(327.61254941,36.17446863)(327.37036189,36.48306241)
\curveto(327.12817437,36.79165618)(326.76489309,36.94595307)(326.28051805,36.94595307)
\curveto(325.69848676,36.94595307)(325.23950235,36.76040618)(324.90356482,36.3893124)
\curveto(324.56762729,36.01821862)(324.39965853,35.5123592)(324.39965853,34.87173415)
\lineto(324.39965853,31.16274948)
\lineto(323.31567407,31.16274948)
\lineto(323.31567407,37.72525001)
\lineto(324.39965853,37.72525001)
\lineto(324.39965853,36.70571867)
\curveto(324.65747105,37.10024996)(324.96020545,37.39517185)(325.30786173,37.59048437)
\curveto(325.65942425,37.78579689)(326.06372116,37.88345314)(326.52075245,37.88345314)
\curveto(327.27465876,37.88345314)(327.8449713,37.64907812)(328.23169008,37.18032809)
\curveto(328.61840886,36.7154843)(328.81176825,36.02993737)(328.81176825,35.1236873)
\closepath
}
}
{
\newrgbcolor{curcolor}{0 0 0}
\pscustom[linestyle=none,fillstyle=solid,fillcolor=curcolor]
{
\newpath
\moveto(340.40747139,34.71353102)
\lineto(340.40747139,34.18618722)
\lineto(335.45043975,34.18618722)
\curveto(335.49731475,33.44399966)(335.71997102,32.87759337)(336.11840855,32.48696834)
\curveto(336.52075233,32.10024956)(337.07934613,31.90689017)(337.79418993,31.90689017)
\curveto(338.20825247,31.90689017)(338.60864312,31.95767142)(338.9953619,32.05923393)
\curveto(339.38598694,32.16079644)(339.77270572,32.3131402)(340.15551825,32.51626522)
\lineto(340.15551825,31.49673389)
\curveto(339.76879947,31.33267137)(339.37231506,31.20767136)(338.96606503,31.12173386)
\curveto(338.559815,31.03579635)(338.14770559,30.9928276)(337.7297368,30.9928276)
\curveto(336.68286172,30.9928276)(335.85278353,31.29751512)(335.23950223,31.90689017)
\curveto(334.63012718,32.51626522)(334.32543966,33.34048403)(334.32543966,34.37954661)
\curveto(334.32543966,35.45376545)(334.61450218,36.30532802)(335.19262723,36.93423432)
\curveto(335.77465852,37.56704687)(336.55786171,37.88345314)(337.54223679,37.88345314)
\curveto(338.42504936,37.88345314)(339.12231504,37.59829687)(339.63403383,37.02798432)
\curveto(340.14965887,36.46157803)(340.40747139,35.69009359)(340.40747139,34.71353102)
\closepath
\moveto(339.32934631,35.02993729)
\curveto(339.32153381,35.61978109)(339.15551817,36.09048425)(338.83129939,36.44204678)
\curveto(338.51098687,36.79360931)(338.08520558,36.96939057)(337.55395554,36.96939057)
\curveto(336.95239299,36.96939057)(336.46997108,36.79946868)(336.1066898,36.4596249)
\curveto(335.74731477,36.11978113)(335.54028351,35.64126546)(335.485596,35.02407792)
\lineto(339.32934631,35.02993729)
\closepath
}
}
{
\newrgbcolor{curcolor}{0 0 0}
\pscustom[linestyle=none,fillstyle=solid,fillcolor=curcolor]
{
\newpath
\moveto(342.1770022,40.27993771)
\lineto(343.25512728,40.27993771)
\lineto(343.25512728,31.16274948)
\lineto(342.1770022,31.16274948)
\lineto(342.1770022,40.27993771)
\closepath
}
}
{
\newrgbcolor{curcolor}{0 0 0}
\pscustom[linestyle=none,fillstyle=solid,fillcolor=curcolor]
{
\newpath
\moveto(350.36840853,32.14712456)
\lineto(350.36840853,28.66665554)
\lineto(349.28442407,28.66665554)
\lineto(349.28442407,37.72525001)
\lineto(350.36840853,37.72525001)
\lineto(350.36840853,36.72915618)
\curveto(350.59497105,37.11978121)(350.88012732,37.40884373)(351.22387735,37.59634374)
\curveto(351.57153362,37.78775001)(351.98559616,37.88345314)(352.46606495,37.88345314)
\curveto(353.26294001,37.88345314)(353.90942444,37.56704687)(354.40551822,36.93423432)
\curveto(354.90551826,36.30142177)(355.15551828,35.46939045)(355.15551828,34.43814037)
\curveto(355.15551828,33.40689029)(354.90551826,32.57485897)(354.40551822,31.94204642)
\curveto(353.90942444,31.30923387)(353.26294001,30.9928276)(352.46606495,30.9928276)
\curveto(351.98559616,30.9928276)(351.57153362,31.0865776)(351.22387735,31.27407762)
\curveto(350.88012732,31.46548388)(350.59497105,31.75649953)(350.36840853,32.14712456)
\closepath
\moveto(354.03637757,34.43814037)
\curveto(354.03637757,35.23110918)(353.87231506,35.85220298)(353.54419003,36.30142177)
\curveto(353.21997126,36.7545468)(352.77270559,36.98110932)(352.20239305,36.98110932)
\curveto(351.6320805,36.98110932)(351.18286172,36.7545468)(350.85473669,36.30142177)
\curveto(350.53051792,35.85220298)(350.36840853,35.23110918)(350.36840853,34.43814037)
\curveto(350.36840853,33.64517156)(350.53051792,33.02212463)(350.85473669,32.5689996)
\curveto(351.18286172,32.11978081)(351.6320805,31.89517142)(352.20239305,31.89517142)
\curveto(352.77270559,31.89517142)(353.21997126,32.11978081)(353.54419003,32.5689996)
\curveto(353.87231506,33.02212463)(354.03637757,33.64517156)(354.03637757,34.43814037)
\closepath
}
}
{
\newrgbcolor{curcolor}{0 0 0}
\pscustom[linestyle=none,fillstyle=solid,fillcolor=curcolor]
{
\newpath
\moveto(359.92504931,34.46157787)
\curveto(359.05395549,34.46157787)(358.45043982,34.36196849)(358.11450229,34.16274972)
\curveto(357.77856476,33.96353096)(357.610596,33.62368718)(357.610596,33.14321839)
\curveto(357.610596,32.76040586)(357.73559601,32.45571834)(357.98559603,32.22915582)
\curveto(358.2395023,32.00649955)(358.58325233,31.89517142)(359.01684611,31.89517142)
\curveto(359.61450241,31.89517142)(360.09301807,32.10610893)(360.4523931,32.52798397)
\curveto(360.81567438,32.95376525)(360.99731502,33.51821842)(360.99731502,34.22134348)
\lineto(360.99731502,34.46157787)
\lineto(359.92504931,34.46157787)
\closepath
\moveto(362.0754401,34.90689041)
\lineto(362.0754401,31.16274948)
\lineto(360.99731502,31.16274948)
\lineto(360.99731502,32.15884331)
\curveto(360.75122125,31.76040578)(360.4445806,31.46548388)(360.07739307,31.27407762)
\curveto(359.71020554,31.0865776)(359.26098676,30.9928276)(358.72973671,30.9928276)
\curveto(358.05786166,30.9928276)(357.52270537,31.18032761)(357.12426784,31.55532764)
\curveto(356.72973655,31.93423392)(356.53247091,32.44009334)(356.53247091,33.07290589)
\curveto(356.53247091,33.81118719)(356.77856468,34.36782786)(357.27075222,34.74282789)
\curveto(357.76684601,35.11782792)(358.50512732,35.30532794)(359.48559615,35.30532794)
\lineto(360.99731502,35.30532794)
\lineto(360.99731502,35.4107967)
\curveto(360.99731502,35.90689049)(360.83325251,36.28970302)(360.50512748,36.55923429)
\curveto(360.1809087,36.83267181)(359.72387742,36.96939057)(359.13403362,36.96939057)
\curveto(358.75903359,36.96939057)(358.39379919,36.92446869)(358.03833041,36.83462493)
\curveto(357.68286163,36.74478118)(357.34106473,36.61001554)(357.0129397,36.43032803)
\lineto(357.0129397,37.42642186)
\curveto(357.40747098,37.57876562)(357.79028351,37.69204688)(358.16137729,37.76626563)
\curveto(358.53247107,37.84439064)(358.89379923,37.88345314)(359.24536175,37.88345314)
\curveto(360.19458058,37.88345314)(360.90356501,37.63735937)(361.37231505,37.14517183)
\curveto(361.84106509,36.6529843)(362.0754401,35.90689049)(362.0754401,34.90689041)
\closepath
}
}
{
\newrgbcolor{curcolor}{0 0 0}
\pscustom[linestyle=none,fillstyle=solid,fillcolor=curcolor]
{
\newpath
\moveto(368.10473687,36.71743743)
\curveto(367.98364311,36.78774993)(367.8508306,36.83853118)(367.70629934,36.86978119)
\curveto(367.56567433,36.90493744)(367.40942432,36.92251557)(367.2375493,36.92251557)
\curveto(366.62817426,36.92251557)(366.15942422,36.7232968)(365.83129919,36.32485927)
\curveto(365.50708042,35.93032799)(365.34497103,35.36196857)(365.34497103,34.61978101)
\lineto(365.34497103,31.16274948)
\lineto(364.26098657,31.16274948)
\lineto(364.26098657,37.72525001)
\lineto(365.34497103,37.72525001)
\lineto(365.34497103,36.70571867)
\curveto(365.57153355,37.10415621)(365.86645545,37.3990781)(366.22973672,37.59048437)
\curveto(366.593018,37.78579689)(367.03442429,37.88345314)(367.55395558,37.88345314)
\curveto(367.62817434,37.88345314)(367.71020559,37.87759377)(367.80004935,37.86587502)
\curveto(367.88989311,37.85806252)(367.98950249,37.84439064)(368.0988775,37.82485939)
\lineto(368.10473687,36.71743743)
\closepath
}
}
{
\newrgbcolor{curcolor}{0 0 0}
\pscustom[linestyle=none,fillstyle=solid,fillcolor=curcolor]
{
\newpath
\moveto(372.22973681,34.46157787)
\curveto(371.35864299,34.46157787)(370.75512732,34.36196849)(370.41918979,34.16274972)
\curveto(370.08325226,33.96353096)(369.9152835,33.62368718)(369.9152835,33.14321839)
\curveto(369.9152835,32.76040586)(370.04028351,32.45571834)(370.29028353,32.22915582)
\curveto(370.5441898,32.00649955)(370.88793983,31.89517142)(371.32153361,31.89517142)
\curveto(371.91918991,31.89517142)(372.39770557,32.10610893)(372.7570806,32.52798397)
\curveto(373.12036188,32.95376525)(373.30200252,33.51821842)(373.30200252,34.22134348)
\lineto(373.30200252,34.46157787)
\lineto(372.22973681,34.46157787)
\closepath
\moveto(374.3801276,34.90689041)
\lineto(374.3801276,31.16274948)
\lineto(373.30200252,31.16274948)
\lineto(373.30200252,32.15884331)
\curveto(373.05590875,31.76040578)(372.7492681,31.46548388)(372.38208057,31.27407762)
\curveto(372.01489304,31.0865776)(371.56567426,30.9928276)(371.03442421,30.9928276)
\curveto(370.36254916,30.9928276)(369.82739287,31.18032761)(369.42895534,31.55532764)
\curveto(369.03442405,31.93423392)(368.83715841,32.44009334)(368.83715841,33.07290589)
\curveto(368.83715841,33.81118719)(369.08325218,34.36782786)(369.57543972,34.74282789)
\curveto(370.07153351,35.11782792)(370.80981482,35.30532794)(371.79028365,35.30532794)
\lineto(373.30200252,35.30532794)
\lineto(373.30200252,35.4107967)
\curveto(373.30200252,35.90689049)(373.13794001,36.28970302)(372.80981498,36.55923429)
\curveto(372.4855962,36.83267181)(372.02856492,36.96939057)(371.43872112,36.96939057)
\curveto(371.06372109,36.96939057)(370.69848669,36.92446869)(370.34301791,36.83462493)
\curveto(369.98754913,36.74478118)(369.64575223,36.61001554)(369.3176272,36.43032803)
\lineto(369.3176272,37.42642186)
\curveto(369.71215848,37.57876562)(370.09497101,37.69204688)(370.46606479,37.76626563)
\curveto(370.83715857,37.84439064)(371.19848673,37.88345314)(371.55004925,37.88345314)
\curveto(372.49926808,37.88345314)(373.20825251,37.63735937)(373.67700255,37.14517183)
\curveto(374.14575259,36.6529843)(374.3801276,35.90689049)(374.3801276,34.90689041)
\closepath
\moveto(372.41137745,40.7604065)
\lineto(373.57739316,40.7604065)
\lineto(371.66723676,38.55728132)
\lineto(370.77075232,38.55728132)
\lineto(372.41137745,40.7604065)
\closepath
}
}
{
\newrgbcolor{curcolor}{0 0 0}
\pscustom[linestyle=none,fillstyle=solid,fillcolor=curcolor]
{
\newpath
\moveto(381.7160651,36.46548428)
\curveto(381.98559637,36.94985932)(382.30786202,37.30728122)(382.68286205,37.53774999)
\curveto(383.05786208,37.76821876)(383.49926837,37.88345314)(384.00708091,37.88345314)
\curveto(384.69067471,37.88345314)(385.21801851,37.64321875)(385.58911228,37.16274996)
\curveto(385.96020606,36.68618742)(386.14575295,36.00649987)(386.14575295,35.1236873)
\lineto(386.14575295,31.16274948)
\lineto(385.06176849,31.16274948)
\lineto(385.06176849,35.08853105)
\curveto(385.06176849,35.71743735)(384.95044036,36.18423426)(384.72778409,36.48892178)
\curveto(384.50512782,36.79360931)(384.16528405,36.94595307)(383.70825276,36.94595307)
\curveto(383.14965897,36.94595307)(382.70825268,36.76040618)(382.3840339,36.3893124)
\curveto(382.05981513,36.01821862)(381.89770574,35.5123592)(381.89770574,34.87173415)
\lineto(381.89770574,31.16274948)
\lineto(380.81372128,31.16274948)
\lineto(380.81372128,35.08853105)
\curveto(380.81372128,35.7213436)(380.70239315,36.18814051)(380.47973688,36.48892178)
\curveto(380.25708061,36.79360931)(379.91333058,36.94595307)(379.4484868,36.94595307)
\curveto(378.8977055,36.94595307)(378.46020547,36.75845305)(378.13598669,36.38345302)
\curveto(377.81176792,36.01235924)(377.64965853,35.50845295)(377.64965853,34.87173415)
\lineto(377.64965853,31.16274948)
\lineto(376.56567407,31.16274948)
\lineto(376.56567407,37.72525001)
\lineto(377.64965853,37.72525001)
\lineto(377.64965853,36.70571867)
\curveto(377.8957523,37.10806246)(378.1906742,37.40493748)(378.53442422,37.59634374)
\curveto(378.87817425,37.78775001)(379.28637741,37.88345314)(379.7590337,37.88345314)
\curveto(380.23559623,37.88345314)(380.63989314,37.76235938)(380.97192442,37.52017186)
\curveto(381.30786194,37.27798434)(381.55590884,36.92642182)(381.7160651,36.46548428)
\closepath
}
}
{
\newrgbcolor{curcolor}{0 0 0}
\pscustom[linestyle=none,fillstyle=solid,fillcolor=curcolor]
{
\newpath
\moveto(393.91529152,34.71353102)
\lineto(393.91529152,34.18618722)
\lineto(388.95825988,34.18618722)
\curveto(389.00513488,33.44399966)(389.22779115,32.87759337)(389.62622868,32.48696834)
\curveto(390.02857246,32.10024956)(390.58716626,31.90689017)(391.30201006,31.90689017)
\curveto(391.7160726,31.90689017)(392.11646325,31.95767142)(392.50318203,32.05923393)
\curveto(392.89380707,32.16079644)(393.28052585,32.3131402)(393.66333838,32.51626522)
\lineto(393.66333838,31.49673389)
\curveto(393.2766196,31.33267137)(392.88013519,31.20767136)(392.47388516,31.12173386)
\curveto(392.06763512,31.03579635)(391.65552572,30.9928276)(391.23755693,30.9928276)
\curveto(390.19068185,30.9928276)(389.36060366,31.29751512)(388.74732236,31.90689017)
\curveto(388.13794731,32.51626522)(387.83325979,33.34048403)(387.83325979,34.37954661)
\curveto(387.83325979,35.45376545)(388.12232231,36.30532802)(388.70044736,36.93423432)
\curveto(389.28247865,37.56704687)(390.06568184,37.88345314)(391.05005692,37.88345314)
\curveto(391.93286949,37.88345314)(392.63013517,37.59829687)(393.14185396,37.02798432)
\curveto(393.657479,36.46157803)(393.91529152,35.69009359)(393.91529152,34.71353102)
\closepath
\moveto(392.83716644,35.02993729)
\curveto(392.82935394,35.61978109)(392.6633383,36.09048425)(392.33911952,36.44204678)
\curveto(392.018807,36.79360931)(391.59302571,36.96939057)(391.06177567,36.96939057)
\curveto(390.46021312,36.96939057)(389.97779121,36.79946868)(389.61450993,36.4596249)
\curveto(389.2551349,36.11978113)(389.04810363,35.64126546)(388.99341613,35.02407792)
\lineto(392.83716644,35.02993729)
\closepath
}
}
{
\newrgbcolor{curcolor}{0 0 0}
\pscustom[linestyle=none,fillstyle=solid,fillcolor=curcolor]
{
\newpath
\moveto(396.75122866,39.5885314)
\lineto(396.75122866,37.72525001)
\lineto(398.97193196,37.72525001)
\lineto(398.97193196,36.88735931)
\lineto(396.75122866,36.88735931)
\lineto(396.75122866,33.32485903)
\curveto(396.75122866,32.78970274)(396.82349429,32.44595271)(396.96802555,32.29360895)
\curveto(397.11646306,32.14126519)(397.41529121,32.06509331)(397.86451,32.06509331)
\lineto(398.97193196,32.06509331)
\lineto(398.97193196,31.16274948)
\lineto(397.86451,31.16274948)
\curveto(397.03247868,31.16274948)(396.45825989,31.31704637)(396.14185361,31.62564015)
\curveto(395.82544734,31.93814017)(395.6672442,32.50454647)(395.6672442,33.32485903)
\lineto(395.6672442,36.88735931)
\lineto(394.87622851,36.88735931)
\lineto(394.87622851,37.72525001)
\lineto(395.6672442,37.72525001)
\lineto(395.6672442,39.5885314)
\lineto(396.75122866,39.5885314)
\closepath
}
}
{
\newrgbcolor{curcolor}{0 0 0}
\pscustom[linestyle=none,fillstyle=solid,fillcolor=curcolor]
{
\newpath
\moveto(404.1984945,36.71743743)
\curveto(404.07740074,36.78774993)(403.94458823,36.83853118)(403.80005697,36.86978119)
\curveto(403.65943196,36.90493744)(403.50318195,36.92251557)(403.33130693,36.92251557)
\curveto(402.72193189,36.92251557)(402.25318185,36.7232968)(401.92505682,36.32485927)
\curveto(401.60083805,35.93032799)(401.43872866,35.36196857)(401.43872866,34.61978101)
\lineto(401.43872866,31.16274948)
\lineto(400.3547442,31.16274948)
\lineto(400.3547442,37.72525001)
\lineto(401.43872866,37.72525001)
\lineto(401.43872866,36.70571867)
\curveto(401.66529118,37.10415621)(401.96021307,37.3990781)(402.32349435,37.59048437)
\curveto(402.68677563,37.78579689)(403.12818192,37.88345314)(403.64771321,37.88345314)
\curveto(403.72193196,37.88345314)(403.80396322,37.87759377)(403.89380698,37.86587502)
\curveto(403.98365074,37.85806252)(404.08326012,37.84439064)(404.19263513,37.82485939)
\lineto(404.1984945,36.71743743)
\closepath
}
}
{
\newrgbcolor{curcolor}{0 0 0}
\pscustom[linestyle=none,fillstyle=solid,fillcolor=curcolor]
{
\newpath
\moveto(407.63794753,36.96939057)
\curveto(407.05982248,36.96939057)(406.6027912,36.74282805)(406.26685367,36.28970302)
\curveto(405.93091614,35.84048423)(405.76294738,35.22329668)(405.76294738,34.43814037)
\curveto(405.76294738,33.65298406)(405.92896302,33.03384338)(406.26099429,32.58071835)
\curveto(406.59693182,32.13149956)(407.05591623,31.90689017)(407.63794753,31.90689017)
\curveto(408.21216632,31.90689017)(408.66724448,32.13345269)(409.00318201,32.58657772)
\curveto(409.33911954,33.03970276)(409.5070883,33.65689031)(409.5070883,34.43814037)
\curveto(409.5070883,35.21548418)(409.33911954,35.8307186)(409.00318201,36.28384364)
\curveto(408.66724448,36.74087493)(408.21216632,36.96939057)(407.63794753,36.96939057)
\closepath
\moveto(407.63794753,37.88345314)
\curveto(408.5754476,37.88345314)(409.31177579,37.57876562)(409.84693208,36.96939057)
\curveto(410.38208837,36.36001552)(410.64966652,35.51626545)(410.64966652,34.43814037)
\curveto(410.64966652,33.36392153)(410.38208837,32.52017147)(409.84693208,31.90689017)
\curveto(409.31177579,31.29751512)(408.5754476,30.9928276)(407.63794753,30.9928276)
\curveto(406.6965412,30.9928276)(405.95825989,31.29751512)(405.4231036,31.90689017)
\curveto(404.89185356,32.52017147)(404.62622854,33.36392153)(404.62622854,34.43814037)
\curveto(404.62622854,35.51626545)(404.89185356,36.36001552)(405.4231036,36.96939057)
\curveto(405.95825989,37.57876562)(406.6965412,37.88345314)(407.63794753,37.88345314)
\closepath
}
}
{
\newrgbcolor{curcolor}{0 0 0}
\pscustom[linestyle=none,fillstyle=solid,fillcolor=curcolor]
{
\newpath
\moveto(412.71802547,32.65103085)
\lineto(413.9543537,32.65103085)
\lineto(413.9543537,31.64321827)
\lineto(412.99341612,29.76821812)
\lineto(412.23755668,29.76821812)
\lineto(412.71802547,31.64321827)
\lineto(412.71802547,32.65103085)
\closepath
}
}
{
\newrgbcolor{curcolor}{0 0 0}
\pscustom[linestyle=none,fillstyle=solid,fillcolor=curcolor]
{
\newpath
\moveto(422.81372879,30.55337444)
\curveto(422.50904127,29.77212437)(422.21216624,29.26235871)(421.92310372,29.02407744)
\curveto(421.6340412,28.78579617)(421.24732242,28.66665554)(420.76294738,28.66665554)
\lineto(419.90161919,28.66665554)
\lineto(419.90161919,29.56899936)
\lineto(420.53443174,29.56899936)
\curveto(420.83130676,29.56899936)(421.06177553,29.63931186)(421.22583804,29.77993687)
\curveto(421.38990055,29.92056188)(421.57154119,30.25259316)(421.77075996,30.7760307)
\lineto(421.96411935,31.26821824)
\lineto(419.30982226,37.72525001)
\lineto(420.45240048,37.72525001)
\lineto(422.50318189,32.5924371)
\lineto(424.55396331,37.72525001)
\lineto(425.69654152,37.72525001)
\lineto(422.81372879,30.55337444)
\closepath
}
}
{
\newrgbcolor{curcolor}{0 0 0}
\pscustom[linestyle=none,fillstyle=solid,fillcolor=curcolor]
{
\newpath
\moveto(435.72779145,37.47329686)
\lineto(435.72779145,36.46548428)
\curveto(435.42310393,36.63345304)(435.11646328,36.75845305)(434.8078695,36.84048431)
\curveto(434.50318198,36.92642182)(434.1945882,36.96939057)(433.88208818,36.96939057)
\curveto(433.18286937,36.96939057)(432.63990058,36.7467343)(432.2531818,36.30142177)
\curveto(431.86646302,35.86001548)(431.67310363,35.23892168)(431.67310363,34.43814037)
\curveto(431.67310363,33.63735906)(431.86646302,33.01431213)(432.2531818,32.5689996)
\curveto(432.63990058,32.12759331)(433.18286937,31.90689017)(433.88208818,31.90689017)
\curveto(434.1945882,31.90689017)(434.50318198,31.9479058)(434.8078695,32.02993705)
\curveto(435.11646328,32.11587456)(435.42310393,32.24282769)(435.72779145,32.41079646)
\lineto(435.72779145,31.41470263)
\curveto(435.42701018,31.27407762)(435.11451015,31.16860886)(434.79029138,31.09829635)
\curveto(434.46997885,31.02798385)(434.12818195,30.9928276)(433.76490067,30.9928276)
\curveto(432.77661934,30.9928276)(431.99146303,31.30337449)(431.40943173,31.92446829)
\curveto(430.82740044,32.54556209)(430.53638479,33.38345279)(430.53638479,34.43814037)
\curveto(430.53638479,35.50845295)(430.82935356,36.3502499)(431.41529111,36.96353119)
\curveto(432.0051349,37.57681249)(432.81177559,37.88345314)(433.83521318,37.88345314)
\curveto(434.16724445,37.88345314)(434.49146323,37.84829689)(434.8078695,37.77798438)
\curveto(435.12427578,37.71157813)(435.43091643,37.61001562)(435.72779145,37.47329686)
\closepath
}
}
{
\newrgbcolor{curcolor}{0 0 0}
\pscustom[linestyle=none,fillstyle=solid,fillcolor=curcolor]
{
\newpath
\moveto(440.15747878,36.96939057)
\curveto(439.57935373,36.96939057)(439.12232245,36.74282805)(438.78638492,36.28970302)
\curveto(438.45044739,35.84048423)(438.28247863,35.22329668)(438.28247863,34.43814037)
\curveto(438.28247863,33.65298406)(438.44849427,33.03384338)(438.78052554,32.58071835)
\curveto(439.11646307,32.13149956)(439.57544748,31.90689017)(440.15747878,31.90689017)
\curveto(440.73169757,31.90689017)(441.18677573,32.13345269)(441.52271326,32.58657772)
\curveto(441.85865079,33.03970276)(442.02661955,33.65689031)(442.02661955,34.43814037)
\curveto(442.02661955,35.21548418)(441.85865079,35.8307186)(441.52271326,36.28384364)
\curveto(441.18677573,36.74087493)(440.73169757,36.96939057)(440.15747878,36.96939057)
\closepath
\moveto(440.15747878,37.88345314)
\curveto(441.09497885,37.88345314)(441.83130704,37.57876562)(442.36646333,36.96939057)
\curveto(442.90161962,36.36001552)(443.16919777,35.51626545)(443.16919777,34.43814037)
\curveto(443.16919777,33.36392153)(442.90161962,32.52017147)(442.36646333,31.90689017)
\curveto(441.83130704,31.29751512)(441.09497885,30.9928276)(440.15747878,30.9928276)
\curveto(439.21607245,30.9928276)(438.47779114,31.29751512)(437.94263485,31.90689017)
\curveto(437.41138481,32.52017147)(437.14575979,33.36392153)(437.14575979,34.43814037)
\curveto(437.14575979,35.51626545)(437.41138481,36.36001552)(437.94263485,36.96939057)
\curveto(438.47779114,37.57876562)(439.21607245,37.88345314)(440.15747878,37.88345314)
\closepath
}
}
{
\newrgbcolor{curcolor}{0 0 0}
\pscustom[linestyle=none,fillstyle=solid,fillcolor=curcolor]
{
\newpath
\moveto(450.40552588,35.1236873)
\lineto(450.40552588,31.16274948)
\lineto(449.3274008,31.16274948)
\lineto(449.3274008,35.08853105)
\curveto(449.3274008,35.70962485)(449.20630704,36.17446863)(448.96411952,36.48306241)
\curveto(448.721932,36.79165618)(448.35865072,36.94595307)(447.87427568,36.94595307)
\curveto(447.29224439,36.94595307)(446.83325997,36.76040618)(446.49732245,36.3893124)
\curveto(446.16138492,36.01821862)(445.99341616,35.5123592)(445.99341616,34.87173415)
\lineto(445.99341616,31.16274948)
\lineto(444.9094317,31.16274948)
\lineto(444.9094317,37.72525001)
\lineto(445.99341616,37.72525001)
\lineto(445.99341616,36.70571867)
\curveto(446.25122868,37.10024996)(446.55396308,37.39517185)(446.90161936,37.59048437)
\curveto(447.25318188,37.78579689)(447.65747879,37.88345314)(448.11451008,37.88345314)
\curveto(448.86841639,37.88345314)(449.43872893,37.64907812)(449.82544771,37.18032809)
\curveto(450.21216649,36.7154843)(450.40552588,36.02993737)(450.40552588,35.1236873)
\closepath
}
}
{
\newrgbcolor{curcolor}{0 0 0}
\pscustom[linestyle=none,fillstyle=solid,fillcolor=curcolor]
{
\newpath
\moveto(453.63404116,39.5885314)
\lineto(453.63404116,37.72525001)
\lineto(455.85474446,37.72525001)
\lineto(455.85474446,36.88735931)
\lineto(453.63404116,36.88735931)
\lineto(453.63404116,33.32485903)
\curveto(453.63404116,32.78970274)(453.70630679,32.44595271)(453.85083805,32.29360895)
\curveto(453.99927556,32.14126519)(454.29810371,32.06509331)(454.7473225,32.06509331)
\lineto(455.85474446,32.06509331)
\lineto(455.85474446,31.16274948)
\lineto(454.7473225,31.16274948)
\curveto(453.91529118,31.16274948)(453.34107239,31.31704637)(453.02466611,31.62564015)
\curveto(452.70825984,31.93814017)(452.5500567,32.50454647)(452.5500567,33.32485903)
\lineto(452.5500567,36.88735931)
\lineto(451.75904101,36.88735931)
\lineto(451.75904101,37.72525001)
\lineto(452.5500567,37.72525001)
\lineto(452.5500567,39.5885314)
\lineto(453.63404116,39.5885314)
\closepath
}
}
{
\newrgbcolor{curcolor}{0 0 0}
\pscustom[linestyle=none,fillstyle=solid,fillcolor=curcolor]
{
\newpath
\moveto(461.081307,36.71743743)
\curveto(460.96021324,36.78774993)(460.82740073,36.83853118)(460.68286947,36.86978119)
\curveto(460.54224446,36.90493744)(460.38599445,36.92251557)(460.21411943,36.92251557)
\curveto(459.60474439,36.92251557)(459.13599435,36.7232968)(458.80786932,36.32485927)
\curveto(458.48365055,35.93032799)(458.32154116,35.36196857)(458.32154116,34.61978101)
\lineto(458.32154116,31.16274948)
\lineto(457.2375567,31.16274948)
\lineto(457.2375567,37.72525001)
\lineto(458.32154116,37.72525001)
\lineto(458.32154116,36.70571867)
\curveto(458.54810368,37.10415621)(458.84302557,37.3990781)(459.20630685,37.59048437)
\curveto(459.56958813,37.78579689)(460.01099442,37.88345314)(460.53052571,37.88345314)
\curveto(460.60474446,37.88345314)(460.68677572,37.87759377)(460.77661948,37.86587502)
\curveto(460.86646324,37.85806252)(460.96607262,37.84439064)(461.07544763,37.82485939)
\lineto(461.081307,36.71743743)
\closepath
}
}
{
\newrgbcolor{curcolor}{0 0 0}
\pscustom[linestyle=none,fillstyle=solid,fillcolor=curcolor]
{
\newpath
\moveto(464.52076003,36.96939057)
\curveto(463.94263498,36.96939057)(463.4856037,36.74282805)(463.14966617,36.28970302)
\curveto(462.81372864,35.84048423)(462.64575988,35.22329668)(462.64575988,34.43814037)
\curveto(462.64575988,33.65298406)(462.81177552,33.03384338)(463.14380679,32.58071835)
\curveto(463.47974432,32.13149956)(463.93872873,31.90689017)(464.52076003,31.90689017)
\curveto(465.09497882,31.90689017)(465.55005698,32.13345269)(465.88599451,32.58657772)
\curveto(466.22193204,33.03970276)(466.3899008,33.65689031)(466.3899008,34.43814037)
\curveto(466.3899008,35.21548418)(466.22193204,35.8307186)(465.88599451,36.28384364)
\curveto(465.55005698,36.74087493)(465.09497882,36.96939057)(464.52076003,36.96939057)
\closepath
\moveto(464.52076003,37.88345314)
\curveto(465.4582601,37.88345314)(466.19458829,37.57876562)(466.72974458,36.96939057)
\curveto(467.26490087,36.36001552)(467.53247902,35.51626545)(467.53247902,34.43814037)
\curveto(467.53247902,33.36392153)(467.26490087,32.52017147)(466.72974458,31.90689017)
\curveto(466.19458829,31.29751512)(465.4582601,30.9928276)(464.52076003,30.9928276)
\curveto(463.5793537,30.9928276)(462.84107239,31.29751512)(462.3059161,31.90689017)
\curveto(461.77466606,32.52017147)(461.50904104,33.36392153)(461.50904104,34.43814037)
\curveto(461.50904104,35.51626545)(461.77466606,36.36001552)(462.3059161,36.96939057)
\curveto(462.84107239,37.57876562)(463.5793537,37.88345314)(464.52076003,37.88345314)
\closepath
}
}
{
\newrgbcolor{curcolor}{0 0 0}
\pscustom[linestyle=none,fillstyle=solid,fillcolor=curcolor]
{
\newpath
\moveto(469.32544733,40.27993771)
\lineto(470.40357241,40.27993771)
\lineto(470.40357241,31.16274948)
\lineto(469.32544733,31.16274948)
\lineto(469.32544733,40.27993771)
\closepath
}
}
{
\newrgbcolor{curcolor}{0 0 0}
\pscustom[linestyle=none,fillstyle=solid,fillcolor=curcolor]
{
\newpath
\moveto(475.63599444,34.46157787)
\curveto(474.76490062,34.46157787)(474.16138495,34.36196849)(473.82544742,34.16274972)
\curveto(473.48950989,33.96353096)(473.32154113,33.62368718)(473.32154113,33.14321839)
\curveto(473.32154113,32.76040586)(473.44654114,32.45571834)(473.69654116,32.22915582)
\curveto(473.95044743,32.00649955)(474.29419746,31.89517142)(474.72779124,31.89517142)
\curveto(475.32544754,31.89517142)(475.8039632,32.10610893)(476.16333823,32.52798397)
\curveto(476.52661951,32.95376525)(476.70826015,33.51821842)(476.70826015,34.22134348)
\lineto(476.70826015,34.46157787)
\lineto(475.63599444,34.46157787)
\closepath
\moveto(477.78638523,34.90689041)
\lineto(477.78638523,31.16274948)
\lineto(476.70826015,31.16274948)
\lineto(476.70826015,32.15884331)
\curveto(476.46216638,31.76040578)(476.15552573,31.46548388)(475.7883382,31.27407762)
\curveto(475.42115067,31.0865776)(474.97193188,30.9928276)(474.44068184,30.9928276)
\curveto(473.76880679,30.9928276)(473.2336505,31.18032761)(472.83521296,31.55532764)
\curveto(472.44068168,31.93423392)(472.24341604,32.44009334)(472.24341604,33.07290589)
\curveto(472.24341604,33.81118719)(472.48950981,34.36782786)(472.98169735,34.74282789)
\curveto(473.47779114,35.11782792)(474.21607245,35.30532794)(475.19654128,35.30532794)
\lineto(476.70826015,35.30532794)
\lineto(476.70826015,35.4107967)
\curveto(476.70826015,35.90689049)(476.54419763,36.28970302)(476.21607261,36.55923429)
\curveto(475.89185383,36.83267181)(475.43482255,36.96939057)(474.84497875,36.96939057)
\curveto(474.46997872,36.96939057)(474.10474432,36.92446869)(473.74927554,36.83462493)
\curveto(473.39380676,36.74478118)(473.05200986,36.61001554)(472.72388483,36.43032803)
\lineto(472.72388483,37.42642186)
\curveto(473.11841611,37.57876562)(473.50122864,37.69204688)(473.87232242,37.76626563)
\curveto(474.2434162,37.84439064)(474.60474436,37.88345314)(474.95630688,37.88345314)
\curveto(475.90552571,37.88345314)(476.61451014,37.63735937)(477.08326018,37.14517183)
\curveto(477.55201021,36.6529843)(477.78638523,35.90689049)(477.78638523,34.90689041)
\closepath
}
}
{
\newrgbcolor{curcolor}{0 0 0}
\pscustom[linestyle=none,fillstyle=solid,fillcolor=curcolor]
{
\newpath
\moveto(480.28833797,32.65103085)
\lineto(481.5246662,32.65103085)
\lineto(481.5246662,31.64321827)
\lineto(480.56372862,29.76821812)
\lineto(479.80786918,29.76821812)
\lineto(480.28833797,31.64321827)
\lineto(480.28833797,32.65103085)
\closepath
}
}
{
\newrgbcolor{curcolor}{0 0 0}
\pscustom[linestyle=none,fillstyle=solid,fillcolor=curcolor]
{
\newpath
\moveto(256.30785418,19.46157668)
\curveto(255.43676036,19.46157668)(254.83324469,19.3619673)(254.49730716,19.16274853)
\curveto(254.16136963,18.96352976)(253.99340087,18.62368599)(253.99340087,18.1432172)
\curveto(253.99340087,17.76040467)(254.11840088,17.45571714)(254.3684009,17.22915463)
\curveto(254.62230717,17.00649836)(254.9660572,16.89517022)(255.39965098,16.89517022)
\curveto(255.99730728,16.89517022)(256.47582294,17.10610774)(256.83519797,17.52798278)
\curveto(257.19847925,17.95376406)(257.38011989,18.51821723)(257.38011989,19.22134228)
\lineto(257.38011989,19.46157668)
\lineto(256.30785418,19.46157668)
\closepath
\moveto(258.45824497,19.90688921)
\lineto(258.45824497,16.16274829)
\lineto(257.38011989,16.16274829)
\lineto(257.38011989,17.15884212)
\curveto(257.13402612,16.76040459)(256.82738547,16.46548269)(256.46019794,16.27407643)
\curveto(256.09301041,16.08657641)(255.64379163,15.9928264)(255.11254158,15.9928264)
\curveto(254.44066653,15.9928264)(253.90551024,16.18032642)(253.50707271,16.55532645)
\curveto(253.11254142,16.93423273)(252.91527578,17.44009214)(252.91527578,18.07290469)
\curveto(252.91527578,18.811186)(253.16136955,19.36782667)(253.65355709,19.7428267)
\curveto(254.14965088,20.11782673)(254.88793219,20.30532675)(255.86840102,20.30532675)
\lineto(257.38011989,20.30532675)
\lineto(257.38011989,20.4107955)
\curveto(257.38011989,20.90688929)(257.21605738,21.28970182)(256.88793235,21.5592331)
\curveto(256.56371357,21.83267062)(256.10668229,21.96938938)(255.51683849,21.96938938)
\curveto(255.14183846,21.96938938)(254.77660406,21.9244675)(254.42113528,21.83462374)
\curveto(254.0656665,21.74477999)(253.7238696,21.61001435)(253.39574457,21.43032684)
\lineto(253.39574457,22.42642066)
\curveto(253.79027585,22.57876443)(254.17308838,22.69204569)(254.54418216,22.76626444)
\curveto(254.91527594,22.84438945)(255.2766041,22.88345195)(255.62816662,22.88345195)
\curveto(256.57738545,22.88345195)(257.28636988,22.63735818)(257.75511992,22.14517064)
\curveto(258.22386996,21.6529831)(258.45824497,20.90688929)(258.45824497,19.90688921)
\closepath
}
}
{
\newrgbcolor{curcolor}{0 0 0}
\pscustom[linestyle=none,fillstyle=solid,fillcolor=curcolor]
{
\newpath
\moveto(265.57152685,24.58853021)
\lineto(265.57152685,22.72524881)
\lineto(267.79223016,22.72524881)
\lineto(267.79223016,21.88735812)
\lineto(265.57152685,21.88735812)
\lineto(265.57152685,18.32485784)
\curveto(265.57152685,17.78970155)(265.64379249,17.44595152)(265.78832375,17.29360776)
\curveto(265.93676126,17.14126399)(266.23558941,17.06509211)(266.68480819,17.06509211)
\lineto(267.79223016,17.06509211)
\lineto(267.79223016,16.16274829)
\lineto(266.68480819,16.16274829)
\curveto(265.85277688,16.16274829)(265.27855808,16.31704518)(264.96215181,16.62563895)
\curveto(264.64574553,16.93813898)(264.48754239,17.50454527)(264.48754239,18.32485784)
\lineto(264.48754239,21.88735812)
\lineto(263.69652671,21.88735812)
\lineto(263.69652671,22.72524881)
\lineto(264.48754239,22.72524881)
\lineto(264.48754239,24.58853021)
\lineto(265.57152685,24.58853021)
\closepath
}
}
{
\newrgbcolor{curcolor}{0 0 0}
\pscustom[linestyle=none,fillstyle=solid,fillcolor=curcolor]
{
\newpath
\moveto(273.0187927,21.71743623)
\curveto(272.89769894,21.78774874)(272.76488643,21.83852999)(272.62035517,21.86978)
\curveto(272.47973015,21.90493625)(272.32348014,21.92251437)(272.15160513,21.92251437)
\curveto(271.54223008,21.92251437)(271.07348004,21.72329561)(270.74535502,21.32485808)
\curveto(270.42113624,20.9303268)(270.25902685,20.36196738)(270.25902685,19.61977982)
\lineto(270.25902685,16.16274829)
\lineto(269.17504239,16.16274829)
\lineto(269.17504239,22.72524881)
\lineto(270.25902685,22.72524881)
\lineto(270.25902685,21.70571748)
\curveto(270.48558937,22.10415501)(270.78051127,22.39907691)(271.14379255,22.59048318)
\curveto(271.50707383,22.78579569)(271.94848011,22.88345195)(272.4680114,22.88345195)
\curveto(272.54223016,22.88345195)(272.62426142,22.87759258)(272.71410517,22.86587382)
\curveto(272.80394893,22.85806132)(272.90355831,22.84438945)(273.01293332,22.8248582)
\lineto(273.0187927,21.71743623)
\closepath
}
}
{
\newrgbcolor{curcolor}{0 0 0}
\pscustom[linestyle=none,fillstyle=solid,fillcolor=curcolor]
{
\newpath
\moveto(277.14379359,19.46157668)
\curveto(276.27269977,19.46157668)(275.66918409,19.3619673)(275.33324657,19.16274853)
\curveto(274.99730904,18.96352976)(274.82934028,18.62368599)(274.82934028,18.1432172)
\curveto(274.82934028,17.76040467)(274.95434029,17.45571714)(275.20434031,17.22915463)
\curveto(275.45824658,17.00649836)(275.8019966,16.89517022)(276.23559039,16.89517022)
\curveto(276.83324669,16.89517022)(277.31176235,17.10610774)(277.67113738,17.52798278)
\curveto(278.03441866,17.95376406)(278.2160593,18.51821723)(278.2160593,19.22134228)
\lineto(278.2160593,19.46157668)
\lineto(277.14379359,19.46157668)
\closepath
\moveto(279.29418438,19.90688921)
\lineto(279.29418438,16.16274829)
\lineto(278.2160593,16.16274829)
\lineto(278.2160593,17.15884212)
\curveto(277.96996553,16.76040459)(277.66332488,16.46548269)(277.29613735,16.27407643)
\curveto(276.92894982,16.08657641)(276.47973103,15.9928264)(275.94848099,15.9928264)
\curveto(275.27660594,15.9928264)(274.74144964,16.18032642)(274.34301211,16.55532645)
\curveto(273.94848083,16.93423273)(273.75121519,17.44009214)(273.75121519,18.07290469)
\curveto(273.75121519,18.811186)(273.99730896,19.36782667)(274.4894965,19.7428267)
\curveto(274.98559029,20.11782673)(275.7238716,20.30532675)(276.70434043,20.30532675)
\lineto(278.2160593,20.30532675)
\lineto(278.2160593,20.4107955)
\curveto(278.2160593,20.90688929)(278.05199678,21.28970182)(277.72387176,21.5592331)
\curveto(277.39965298,21.83267062)(276.94262169,21.96938938)(276.3527779,21.96938938)
\curveto(275.97777787,21.96938938)(275.61254346,21.9244675)(275.25707469,21.83462374)
\curveto(274.90160591,21.74477999)(274.55980901,21.61001435)(274.23168398,21.43032684)
\lineto(274.23168398,22.42642066)
\curveto(274.62621526,22.57876443)(275.00902779,22.69204569)(275.38012157,22.76626444)
\curveto(275.75121535,22.84438945)(276.1125435,22.88345195)(276.46410603,22.88345195)
\curveto(277.41332486,22.88345195)(278.12230929,22.63735818)(278.59105933,22.14517064)
\curveto(279.05980936,21.6529831)(279.29418438,20.90688929)(279.29418438,19.90688921)
\closepath
}
}
{
\newrgbcolor{curcolor}{0 0 0}
\pscustom[linestyle=none,fillstyle=solid,fillcolor=curcolor]
{
\newpath
\moveto(280.74730891,22.72524881)
\lineto(281.88988713,22.72524881)
\lineto(283.94066854,17.21743588)
\lineto(285.99144995,22.72524881)
\lineto(287.13402817,22.72524881)
\lineto(284.67309047,16.16274829)
\lineto(283.20824661,16.16274829)
\lineto(280.74730891,22.72524881)
\closepath
}
}
{
\newrgbcolor{curcolor}{0 0 0}
\pscustom[linestyle=none,fillstyle=solid,fillcolor=curcolor]
{
\newpath
\moveto(294.23559258,19.71352982)
\lineto(294.23559258,19.18618603)
\lineto(289.27856093,19.18618603)
\curveto(289.32543594,18.44399847)(289.5480922,17.87759218)(289.94652974,17.48696715)
\curveto(290.34887352,17.10024837)(290.90746731,16.90688898)(291.62231112,16.90688898)
\curveto(292.03637365,16.90688898)(292.43676431,16.95767023)(292.82348309,17.05923274)
\curveto(293.21410812,17.16079525)(293.6008269,17.31313901)(293.98363943,17.51626402)
\lineto(293.98363943,16.49673269)
\curveto(293.59692065,16.33267018)(293.20043624,16.20767017)(292.79418621,16.12173266)
\curveto(292.38793618,16.03579516)(291.97582677,15.9928264)(291.55785799,15.9928264)
\curveto(290.51098291,15.9928264)(289.68090472,16.29751393)(289.06762342,16.90688898)
\curveto(288.45824837,17.51626402)(288.15356084,18.34048284)(288.15356084,19.37954542)
\curveto(288.15356084,20.45376426)(288.44262337,21.30532683)(289.02074841,21.93423313)
\curveto(289.60277971,22.56704568)(290.3859829,22.88345195)(291.37035797,22.88345195)
\curveto(292.25317054,22.88345195)(292.95043623,22.59829568)(293.46215502,22.02798313)
\curveto(293.97778006,21.46157684)(294.23559258,20.6900924)(294.23559258,19.71352982)
\closepath
\moveto(293.15746749,20.0299361)
\curveto(293.14965499,20.6197799)(292.98363935,21.09048306)(292.65942058,21.44204559)
\curveto(292.33910805,21.79360811)(291.91332677,21.96938938)(291.38207673,21.96938938)
\curveto(290.78051418,21.96938938)(290.29809226,21.79946749)(289.93481099,21.45962371)
\curveto(289.57543596,21.11977994)(289.36840469,20.64126427)(289.31371719,20.02407672)
\lineto(293.15746749,20.0299361)
\closepath
\moveto(292.12035803,25.7604053)
\lineto(293.28637375,25.7604053)
\lineto(291.37621735,23.55728013)
\lineto(290.4797329,23.55728013)
\lineto(292.12035803,25.7604053)
\closepath
}
}
{
\newrgbcolor{curcolor}{0 0 0}
\pscustom[linestyle=none,fillstyle=solid,fillcolor=curcolor]
{
\newpath
\moveto(300.18871746,22.53188942)
\lineto(300.18871746,21.51235809)
\curveto(299.88402994,21.6686081)(299.56762366,21.78579561)(299.23949864,21.86392062)
\curveto(298.91137361,21.94204563)(298.57152984,21.98110813)(298.21996731,21.98110813)
\curveto(297.68481101,21.98110813)(297.28246723,21.89907687)(297.01293596,21.73501436)
\curveto(296.74731094,21.57095185)(296.61449843,21.32485808)(296.61449843,20.99673305)
\curveto(296.61449843,20.74673303)(296.71020156,20.54946739)(296.90160783,20.40493613)
\curveto(297.09301409,20.26431112)(297.47777975,20.12954548)(298.05590479,20.00063922)
\lineto(298.42504545,19.91860797)
\curveto(299.19067051,19.75454545)(299.7336393,19.52212356)(300.05395183,19.22134228)
\curveto(300.3781706,18.92446726)(300.54027999,18.5084516)(300.54027999,17.97329531)
\curveto(300.54027999,17.36392026)(300.29809247,16.88149835)(299.81371743,16.52602957)
\curveto(299.33324865,16.17056079)(298.67113922,15.9928264)(297.82738915,15.9928264)
\curveto(297.47582662,15.9928264)(297.10863909,16.02798266)(296.72582656,16.09829516)
\curveto(296.34692028,16.16470142)(295.94652963,16.26626392)(295.52465459,16.40298269)
\lineto(295.52465459,17.51626402)
\curveto(295.92309212,17.30923276)(296.31567028,17.15298275)(296.70238906,17.04751399)
\curveto(297.08910784,16.94595148)(297.47192037,16.89517022)(297.85082665,16.89517022)
\curveto(298.35863919,16.89517022)(298.74926422,16.98110773)(299.02270175,17.15298275)
\curveto(299.29613927,17.32876401)(299.43285803,17.57485778)(299.43285803,17.89126405)
\curveto(299.43285803,18.18423283)(299.33324865,18.40884222)(299.13402988,18.56509223)
\curveto(298.93871736,18.72134224)(298.5070767,18.87173288)(297.8391079,19.01626414)
\lineto(297.46410787,19.10415478)
\curveto(296.79613907,19.24477979)(296.31371716,19.45962355)(296.01684213,19.74868608)
\curveto(295.71996711,20.04165485)(295.5715296,20.44204551)(295.5715296,20.94985805)
\curveto(295.5715296,21.5670456)(295.79027961,22.04360813)(296.22777965,22.37954566)
\curveto(296.66527968,22.71548319)(297.28637348,22.88345195)(298.09106105,22.88345195)
\curveto(298.48949858,22.88345195)(298.86449861,22.85415507)(299.21606114,22.79556132)
\curveto(299.56762366,22.73696756)(299.89184244,22.64907693)(300.18871746,22.53188942)
\closepath
}
}
{
\newrgbcolor{curcolor}{0 0 0}
\pscustom[linestyle=none,fillstyle=solid,fillcolor=curcolor]
{
\newpath
\moveto(310.4016081,21.72915498)
\lineto(310.4016081,25.27993652)
\lineto(311.47973319,25.27993652)
\lineto(311.47973319,16.16274829)
\lineto(310.4016081,16.16274829)
\lineto(310.4016081,17.14712337)
\curveto(310.17504558,16.75649834)(309.88793618,16.46548269)(309.54027991,16.27407643)
\curveto(309.19652988,16.08657641)(308.78246735,15.9928264)(308.29809231,15.9928264)
\curveto(307.50512349,15.9928264)(306.85863907,16.30923268)(306.35863903,16.94204523)
\curveto(305.86254524,17.57485778)(305.61449834,18.40688909)(305.61449834,19.43813918)
\curveto(305.61449834,20.46938926)(305.86254524,21.30142057)(306.35863903,21.93423313)
\curveto(306.85863907,22.56704568)(307.50512349,22.88345195)(308.29809231,22.88345195)
\curveto(308.78246735,22.88345195)(309.19652988,22.78774882)(309.54027991,22.59634255)
\curveto(309.88793618,22.40884254)(310.17504558,22.11978002)(310.4016081,21.72915498)
\closepath
\moveto(306.72777968,19.43813918)
\curveto(306.72777968,18.64517036)(306.88988907,18.02212344)(307.21410785,17.5689984)
\curveto(307.54223287,17.11977962)(307.99145166,16.89517022)(308.5617642,16.89517022)
\curveto(309.13207675,16.89517022)(309.58129553,17.11977962)(309.90942056,17.5689984)
\curveto(310.23754559,18.02212344)(310.4016081,18.64517036)(310.4016081,19.43813918)
\curveto(310.4016081,20.23110799)(310.23754559,20.85220179)(309.90942056,21.30142057)
\curveto(309.58129553,21.75454561)(309.13207675,21.98110813)(308.5617642,21.98110813)
\curveto(307.99145166,21.98110813)(307.54223287,21.75454561)(307.21410785,21.30142057)
\curveto(306.88988907,20.85220179)(306.72777968,20.23110799)(306.72777968,19.43813918)
\closepath
}
}
{
\newrgbcolor{curcolor}{0 0 0}
\pscustom[linestyle=none,fillstyle=solid,fillcolor=curcolor]
{
\newpath
\moveto(319.31371758,19.71352982)
\lineto(319.31371758,19.18618603)
\lineto(314.35668593,19.18618603)
\curveto(314.40356094,18.44399847)(314.6262172,17.87759218)(315.02465474,17.48696715)
\curveto(315.42699852,17.10024837)(315.98559231,16.90688898)(316.70043612,16.90688898)
\curveto(317.11449865,16.90688898)(317.51488931,16.95767023)(317.90160809,17.05923274)
\curveto(318.29223312,17.16079525)(318.6789519,17.31313901)(319.06176443,17.51626402)
\lineto(319.06176443,16.49673269)
\curveto(318.67504565,16.33267018)(318.27856124,16.20767017)(317.87231121,16.12173266)
\curveto(317.46606118,16.03579516)(317.05395177,15.9928264)(316.63598299,15.9928264)
\curveto(315.58910791,15.9928264)(314.75902972,16.29751393)(314.14574842,16.90688898)
\curveto(313.53637337,17.51626402)(313.23168584,18.34048284)(313.23168584,19.37954542)
\curveto(313.23168584,20.45376426)(313.52074837,21.30532683)(314.09887341,21.93423313)
\curveto(314.68090471,22.56704568)(315.4641079,22.88345195)(316.44848297,22.88345195)
\curveto(317.33129554,22.88345195)(318.02856123,22.59829568)(318.54028002,22.02798313)
\curveto(319.05590506,21.46157684)(319.31371758,20.6900924)(319.31371758,19.71352982)
\closepath
\moveto(318.23559249,20.0299361)
\curveto(318.22777999,20.6197799)(318.06176435,21.09048306)(317.73754558,21.44204559)
\curveto(317.41723305,21.79360811)(316.99145177,21.96938938)(316.46020173,21.96938938)
\curveto(315.85863918,21.96938938)(315.37621726,21.79946749)(315.01293599,21.45962371)
\curveto(314.65356096,21.11977994)(314.44652969,20.64126427)(314.39184219,20.02407672)
\lineto(318.23559249,20.0299361)
\closepath
}
}
{
\newrgbcolor{curcolor}{0 0 0}
\pscustom[linestyle=none,fillstyle=solid,fillcolor=curcolor]
{
\newpath
\moveto(324.9035647,25.27993652)
\lineto(325.98168978,25.27993652)
\lineto(325.98168978,16.16274829)
\lineto(324.9035647,16.16274829)
\lineto(324.9035647,25.27993652)
\closepath
}
}
{
\newrgbcolor{curcolor}{0 0 0}
\pscustom[linestyle=none,fillstyle=solid,fillcolor=curcolor]
{
\newpath
\moveto(331.21411181,19.46157668)
\curveto(330.34301799,19.46157668)(329.73950232,19.3619673)(329.40356479,19.16274853)
\curveto(329.06762726,18.96352976)(328.8996585,18.62368599)(328.8996585,18.1432172)
\curveto(328.8996585,17.76040467)(329.02465851,17.45571714)(329.27465853,17.22915463)
\curveto(329.5285648,17.00649836)(329.87231483,16.89517022)(330.30590861,16.89517022)
\curveto(330.90356491,16.89517022)(331.38208057,17.10610774)(331.7414556,17.52798278)
\curveto(332.10473688,17.95376406)(332.28637752,18.51821723)(332.28637752,19.22134228)
\lineto(332.28637752,19.46157668)
\lineto(331.21411181,19.46157668)
\closepath
\moveto(333.3645026,19.90688921)
\lineto(333.3645026,16.16274829)
\lineto(332.28637752,16.16274829)
\lineto(332.28637752,17.15884212)
\curveto(332.04028375,16.76040459)(331.7336431,16.46548269)(331.36645557,16.27407643)
\curveto(330.99926804,16.08657641)(330.55004926,15.9928264)(330.01879921,15.9928264)
\curveto(329.34692416,15.9928264)(328.81176787,16.18032642)(328.41333034,16.55532645)
\curveto(328.01879905,16.93423273)(327.82153341,17.44009214)(327.82153341,18.07290469)
\curveto(327.82153341,18.811186)(328.06762718,19.36782667)(328.55981472,19.7428267)
\curveto(329.05590851,20.11782673)(329.79418982,20.30532675)(330.77465865,20.30532675)
\lineto(332.28637752,20.30532675)
\lineto(332.28637752,20.4107955)
\curveto(332.28637752,20.90688929)(332.12231501,21.28970182)(331.79418998,21.5592331)
\curveto(331.4699712,21.83267062)(331.01293992,21.96938938)(330.42309612,21.96938938)
\curveto(330.04809609,21.96938938)(329.68286169,21.9244675)(329.32739291,21.83462374)
\curveto(328.97192413,21.74477999)(328.63012723,21.61001435)(328.3020022,21.43032684)
\lineto(328.3020022,22.42642066)
\curveto(328.69653348,22.57876443)(329.07934601,22.69204569)(329.45043979,22.76626444)
\curveto(329.82153357,22.84438945)(330.18286173,22.88345195)(330.53442425,22.88345195)
\curveto(331.48364308,22.88345195)(332.19262751,22.63735818)(332.66137755,22.14517064)
\curveto(333.13012759,21.6529831)(333.3645026,20.90688929)(333.3645026,19.90688921)
\closepath
}
}
{
\newrgbcolor{curcolor}{0 0 0}
\pscustom[linestyle=none,fillstyle=solid,fillcolor=curcolor]
{
\newpath
\moveto(338.63793963,22.72524881)
\lineto(339.78051785,22.72524881)
\lineto(341.83129926,17.21743588)
\lineto(343.88208068,22.72524881)
\lineto(345.02465889,22.72524881)
\lineto(342.5637212,16.16274829)
\lineto(341.09887733,16.16274829)
\lineto(338.63793963,22.72524881)
\closepath
}
}
{
\newrgbcolor{curcolor}{0 0 0}
\pscustom[linestyle=none,fillstyle=solid,fillcolor=curcolor]
{
\newpath
\moveto(349.49536181,19.46157668)
\curveto(348.62426799,19.46157668)(348.02075232,19.3619673)(347.68481479,19.16274853)
\curveto(347.34887726,18.96352976)(347.1809085,18.62368599)(347.1809085,18.1432172)
\curveto(347.1809085,17.76040467)(347.30590851,17.45571714)(347.55590853,17.22915463)
\curveto(347.8098148,17.00649836)(348.15356483,16.89517022)(348.58715861,16.89517022)
\curveto(349.18481491,16.89517022)(349.66333057,17.10610774)(350.0227056,17.52798278)
\curveto(350.38598688,17.95376406)(350.56762752,18.51821723)(350.56762752,19.22134228)
\lineto(350.56762752,19.46157668)
\lineto(349.49536181,19.46157668)
\closepath
\moveto(351.6457526,19.90688921)
\lineto(351.6457526,16.16274829)
\lineto(350.56762752,16.16274829)
\lineto(350.56762752,17.15884212)
\curveto(350.32153375,16.76040459)(350.0148931,16.46548269)(349.64770557,16.27407643)
\curveto(349.28051804,16.08657641)(348.83129926,15.9928264)(348.30004921,15.9928264)
\curveto(347.62817416,15.9928264)(347.09301787,16.18032642)(346.69458034,16.55532645)
\curveto(346.30004905,16.93423273)(346.10278341,17.44009214)(346.10278341,18.07290469)
\curveto(346.10278341,18.811186)(346.34887718,19.36782667)(346.84106472,19.7428267)
\curveto(347.33715851,20.11782673)(348.07543982,20.30532675)(349.05590865,20.30532675)
\lineto(350.56762752,20.30532675)
\lineto(350.56762752,20.4107955)
\curveto(350.56762752,20.90688929)(350.40356501,21.28970182)(350.07543998,21.5592331)
\curveto(349.7512212,21.83267062)(349.29418992,21.96938938)(348.70434612,21.96938938)
\curveto(348.32934609,21.96938938)(347.96411169,21.9244675)(347.60864291,21.83462374)
\curveto(347.25317413,21.74477999)(346.91137723,21.61001435)(346.5832522,21.43032684)
\lineto(346.5832522,22.42642066)
\curveto(346.97778348,22.57876443)(347.36059601,22.69204569)(347.73168979,22.76626444)
\curveto(348.10278357,22.84438945)(348.46411173,22.88345195)(348.81567425,22.88345195)
\curveto(349.76489308,22.88345195)(350.47387751,22.63735818)(350.94262755,22.14517064)
\curveto(351.41137759,21.6529831)(351.6457526,20.90688929)(351.6457526,19.90688921)
\closepath
\moveto(349.67700245,25.7604053)
\lineto(350.84301816,25.7604053)
\lineto(348.93286176,23.55728013)
\lineto(348.03637732,23.55728013)
\lineto(349.67700245,25.7604053)
\closepath
}
}
{
\newrgbcolor{curcolor}{0 0 0}
\pscustom[linestyle=none,fillstyle=solid,fillcolor=curcolor]
{
\newpath
\moveto(353.8723147,25.27993652)
\lineto(354.95043978,25.27993652)
\lineto(354.95043978,16.16274829)
\lineto(353.8723147,16.16274829)
\lineto(353.8723147,25.27993652)
\closepath
}
}
{
\newrgbcolor{curcolor}{0 0 0}
\pscustom[linestyle=none,fillstyle=solid,fillcolor=curcolor]
{
\newpath
\moveto(356.42700213,22.72524881)
\lineto(357.56958035,22.72524881)
\lineto(359.62036176,17.21743588)
\lineto(361.67114318,22.72524881)
\lineto(362.81372139,22.72524881)
\lineto(360.3527837,16.16274829)
\lineto(358.88793983,16.16274829)
\lineto(356.42700213,22.72524881)
\closepath
}
}
{
\newrgbcolor{curcolor}{0 0 0}
\pscustom[linestyle=none,fillstyle=solid,fillcolor=curcolor]
{
\newpath
\moveto(364.19067406,18.75259225)
\lineto(364.19067406,22.72524881)
\lineto(365.26879915,22.72524881)
\lineto(365.26879915,18.79360788)
\curveto(365.26879915,18.17251408)(365.38989291,17.70571716)(365.63208043,17.39321714)
\curveto(365.87426795,17.08462336)(366.23754922,16.93032648)(366.72192426,16.93032648)
\curveto(367.30395556,16.93032648)(367.76293997,17.11587337)(368.0988775,17.48696715)
\curveto(368.43872127,17.85806093)(368.60864316,18.36392034)(368.60864316,19.00454539)
\lineto(368.60864316,22.72524881)
\lineto(369.68676825,22.72524881)
\lineto(369.68676825,16.16274829)
\lineto(368.60864316,16.16274829)
\lineto(368.60864316,17.17056087)
\curveto(368.34692439,16.77212334)(368.04223687,16.47524832)(367.69458059,16.2799358)
\curveto(367.35083056,16.08852954)(366.95043991,15.9928264)(366.49340862,15.9928264)
\curveto(365.73950231,15.9928264)(365.16723664,16.22720142)(364.77661161,16.69595146)
\curveto(364.38598658,17.1647015)(364.19067406,17.85024843)(364.19067406,18.75259225)
\closepath
}
}
{
\newrgbcolor{curcolor}{0 0 0}
\pscustom[linestyle=none,fillstyle=solid,fillcolor=curcolor]
{
\newpath
\moveto(371.9191897,25.27993652)
\lineto(372.99731478,25.27993652)
\lineto(372.99731478,16.16274829)
\lineto(371.9191897,16.16274829)
\lineto(371.9191897,25.27993652)
\closepath
}
}
{
\newrgbcolor{curcolor}{0 0 0}
\pscustom[linestyle=none,fillstyle=solid,fillcolor=curcolor]
{
\newpath
\moveto(378.22973681,19.46157668)
\curveto(377.35864299,19.46157668)(376.75512732,19.3619673)(376.41918979,19.16274853)
\curveto(376.08325226,18.96352976)(375.9152835,18.62368599)(375.9152835,18.1432172)
\curveto(375.9152835,17.76040467)(376.04028351,17.45571714)(376.29028353,17.22915463)
\curveto(376.5441898,17.00649836)(376.88793983,16.89517022)(377.32153361,16.89517022)
\curveto(377.91918991,16.89517022)(378.39770557,17.10610774)(378.7570806,17.52798278)
\curveto(379.12036188,17.95376406)(379.30200252,18.51821723)(379.30200252,19.22134228)
\lineto(379.30200252,19.46157668)
\lineto(378.22973681,19.46157668)
\closepath
\moveto(380.3801276,19.90688921)
\lineto(380.3801276,16.16274829)
\lineto(379.30200252,16.16274829)
\lineto(379.30200252,17.15884212)
\curveto(379.05590875,16.76040459)(378.7492681,16.46548269)(378.38208057,16.27407643)
\curveto(378.01489304,16.08657641)(377.56567426,15.9928264)(377.03442421,15.9928264)
\curveto(376.36254916,15.9928264)(375.82739287,16.18032642)(375.42895534,16.55532645)
\curveto(375.03442405,16.93423273)(374.83715841,17.44009214)(374.83715841,18.07290469)
\curveto(374.83715841,18.811186)(375.08325218,19.36782667)(375.57543972,19.7428267)
\curveto(376.07153351,20.11782673)(376.80981482,20.30532675)(377.79028365,20.30532675)
\lineto(379.30200252,20.30532675)
\lineto(379.30200252,20.4107955)
\curveto(379.30200252,20.90688929)(379.13794001,21.28970182)(378.80981498,21.5592331)
\curveto(378.4855962,21.83267062)(378.02856492,21.96938938)(377.43872112,21.96938938)
\curveto(377.06372109,21.96938938)(376.69848669,21.9244675)(376.34301791,21.83462374)
\curveto(375.98754913,21.74477999)(375.64575223,21.61001435)(375.3176272,21.43032684)
\lineto(375.3176272,22.42642066)
\curveto(375.71215848,22.57876443)(376.09497101,22.69204569)(376.46606479,22.76626444)
\curveto(376.83715857,22.84438945)(377.19848673,22.88345195)(377.55004925,22.88345195)
\curveto(378.49926808,22.88345195)(379.20825251,22.63735818)(379.67700255,22.14517064)
\curveto(380.14575259,21.6529831)(380.3801276,20.90688929)(380.3801276,19.90688921)
\closepath
}
}
{
\newrgbcolor{curcolor}{0 0 0}
\pscustom[linestyle=none,fillstyle=solid,fillcolor=curcolor]
{
\newpath
\moveto(225.59814259,292.30388306)
\lineto(225.59814259,290.61846995)
\curveto(224.94222886,290.93216782)(224.32338852,291.16601533)(223.74162156,291.32001246)
\curveto(223.1598546,291.4740096)(222.59805023,291.55100817)(222.05620846,291.55100817)
\curveto(221.11511485,291.55100817)(220.38790615,291.36849304)(219.87458236,291.00346279)
\curveto(219.36696217,290.63843255)(219.11315207,290.11940516)(219.11315207,289.44638064)
\curveto(219.11315207,288.88172447)(219.2814082,288.45395465)(219.61792046,288.16307117)
\curveto(219.96013632,287.87789129)(220.60464286,287.64689558)(221.55144006,287.47008406)
\lineto(222.59519843,287.25619915)
\curveto(223.8842115,287.01094445)(224.83386051,286.57747103)(225.44414545,285.95577888)
\curveto(226.060134,285.33979034)(226.36812827,284.51276868)(226.36812827,283.47471391)
\curveto(226.36812827,282.23703322)(225.95176564,281.29879141)(225.11904039,280.65998847)
\curveto(224.29201873,280.02118554)(223.07715243,279.70178407)(221.4744415,279.70178407)
\curveto(220.86986015,279.70178407)(220.22535361,279.77022724)(219.5409219,279.90711359)
\curveto(218.86219378,280.04399993)(218.15779947,280.24647765)(217.42773897,280.51454673)
\lineto(217.42773897,282.2940692)
\curveto(218.12928148,281.90052096)(218.816565,281.60393388)(219.48958952,281.40430797)
\curveto(220.16261404,281.20468205)(220.82423137,281.10486909)(221.4744415,281.10486909)
\curveto(222.46116389,281.10486909)(223.22259417,281.29879141)(223.75873235,281.68663605)
\curveto(224.29487053,282.07448069)(224.56293962,282.62772966)(224.56293962,283.34638296)
\curveto(224.56293962,283.9737787)(224.3690173,284.4642881)(223.98117266,284.81791115)
\curveto(223.59903162,285.17153421)(222.96878408,285.4367515)(222.09043004,285.61356302)
\lineto(221.03811628,285.81889254)
\curveto(219.74910321,286.07555443)(218.816565,286.47765807)(218.24050163,287.02520344)
\curveto(217.66443827,287.57274881)(217.37640659,288.3341791)(217.37640659,289.3094943)
\curveto(217.37640659,290.43880663)(217.77280663,291.32856786)(218.5656067,291.97877799)
\curveto(219.36411037,292.62898812)(220.46205292,292.95409319)(221.85943434,292.95409319)
\curveto(222.45831209,292.95409319)(223.06859704,292.89990901)(223.69028918,292.79154065)
\curveto(224.31198132,292.6831723)(224.94793246,292.52061977)(225.59814259,292.30388306)
\closepath
}
}
{
\newrgbcolor{curcolor}{0 0 0}
\pscustom[linestyle=none,fillstyle=solid,fillcolor=curcolor]
{
\newpath
\moveto(238.628011,291.73922689)
\lineto(238.628011,289.91692744)
\curveto(238.04624404,290.45876922)(237.4245519,290.86372465)(236.76293457,291.13179374)
\curveto(236.10702084,291.39986283)(235.40833013,291.53389738)(234.66686244,291.53389738)
\curveto(233.20674144,291.53389738)(232.08883631,291.08616496)(231.31314703,290.19070013)
\curveto(230.53745775,289.3009389)(230.14961311,288.01192583)(230.14961311,286.32366093)
\curveto(230.14961311,284.64109963)(230.53745775,283.35208656)(231.31314703,282.45662173)
\curveto(232.08883631,281.5668605)(233.20674144,281.12197988)(234.66686244,281.12197988)
\curveto(235.40833013,281.12197988)(236.10702084,281.25601443)(236.76293457,281.52408352)
\curveto(237.4245519,281.79215261)(238.04624404,282.19710804)(238.628011,282.73894981)
\lineto(238.628011,280.93376116)
\curveto(238.02342965,280.52310213)(237.38177492,280.21510786)(236.7030468,280.00977834)
\curveto(236.03002228,279.80444883)(235.31707257,279.70178407)(234.56419768,279.70178407)
\curveto(232.63067808,279.70178407)(231.10781751,280.29210643)(229.99561597,281.47275114)
\curveto(228.88341443,282.65909945)(228.32731366,284.27606938)(228.32731366,286.32366093)
\curveto(228.32731366,288.37695608)(228.88341443,289.99392601)(229.99561597,291.17457072)
\curveto(231.10781751,292.36091903)(232.63067808,292.95409319)(234.56419768,292.95409319)
\curveto(235.32847977,292.95409319)(236.04713307,292.85142843)(236.72015759,292.64609891)
\curveto(237.39888571,292.446473)(238.03483685,292.14418232)(238.628011,291.73922689)
\closepath
}
}
{
\newrgbcolor{curcolor}{0 0 0}
\pscustom[linestyle=none,fillstyle=solid,fillcolor=curcolor]
{
\newpath
\moveto(245.58354811,291.02057359)
\lineto(243.23936948,284.66391402)
\lineto(247.93628213,284.66391402)
\lineto(245.58354811,291.02057359)
\closepath
\moveto(244.60823291,292.72309748)
\lineto(246.5674187,292.72309748)
\lineto(251.43543928,279.94989057)
\lineto(249.63880603,279.94989057)
\lineto(248.47527211,283.22660741)
\lineto(242.71749029,283.22660741)
\lineto(241.55395637,279.94989057)
\lineto(239.73165693,279.94989057)
\lineto(244.60823291,292.72309748)
\closepath
}
}
{
\newrgbcolor{curcolor}{0 0 0}
\pscustom[linestyle=none,fillstyle=solid,fillcolor=curcolor]
{
\newpath
\moveto(255.02015159,291.30290167)
\lineto(255.02015159,281.37008638)
\lineto(257.10766832,281.37008638)
\curveto(258.87007999,281.37008638)(260.15909306,281.76933822)(260.97470752,282.56784189)
\curveto(261.79602558,283.36634555)(262.20668461,284.62684063)(262.20668461,286.34932712)
\curveto(262.20668461,288.06040641)(261.79602558,289.31234609)(260.97470752,290.10514617)
\curveto(260.15909306,290.90364984)(258.87007999,291.30290167)(257.10766832,291.30290167)
\lineto(255.02015159,291.30290167)
\closepath
\moveto(253.2919615,292.72309748)
\lineto(256.84245103,292.72309748)
\curveto(259.31781241,292.72309748)(261.13440826,292.2069219)(262.29223858,291.17457072)
\curveto(263.4500689,290.14792315)(264.02898406,288.53950861)(264.02898406,286.34932712)
\curveto(264.02898406,284.14773843)(263.4472171,282.5307685)(262.28368318,281.49841733)
\curveto(261.12014926,280.46606615)(259.30640521,279.94989057)(256.84245103,279.94989057)
\lineto(253.2919615,279.94989057)
\lineto(253.2919615,292.72309748)
\closepath
}
}
{
\newrgbcolor{curcolor}{0 0 0}
\pscustom[linestyle=none,fillstyle=solid,fillcolor=curcolor]
{
\newpath
\moveto(270.75352529,291.02057359)
\lineto(268.40934666,284.66391402)
\lineto(273.10625932,284.66391402)
\lineto(270.75352529,291.02057359)
\closepath
\moveto(269.7782101,292.72309748)
\lineto(271.73739589,292.72309748)
\lineto(276.60541647,279.94989057)
\lineto(274.80878322,279.94989057)
\lineto(273.6452493,283.22660741)
\lineto(267.88746748,283.22660741)
\lineto(266.72393356,279.94989057)
\lineto(264.90163412,279.94989057)
\lineto(269.7782101,292.72309748)
\closepath
}
}
\end{pspicture}

   \caption{SCADA}
   \label{fig:SCADA}
 \end{figure}
 La realimentación, también denominada retroalimentación o feedback es, en una
 organización, el proceso de compartir observaciones, preocupaciones y
 sugerencias, con la intención de recabar información, a nivel individual o
 colectivo, para mejorar o modificar diversos aspectos del funcionamiento de una
 organización. La realimentación tiene que ser bidireccional de modo que la
 mejora continua sea posible, en el escalafón jerárquico, de arriba para abajo y
 de abajo para arriba.

 En la teoría de control, la realimentación es un proceso por el que una cierta
 proporción de la señal de salida de un sistema se redirige de nuevo a la
 entrada. Esto es de uso frecuente para controlar el comportamiento dinámico del
 sistema. Los ejemplos de la realimentación se pueden encontrar en la mayoría de
 los sistemas complejos, tales como ingeniería, arquitectura, economía,
 sociología y biología.

\section{Conclusiones}

Se utilizara los conceptos de SCADA, pero se substituira los Controladores
Lógicos Programables(PLCs); por micro controladores(µC), tanto para reducir
costos, como para tener mayor capacidad de cambios y mejoras.
