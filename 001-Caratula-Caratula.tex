\documentclass[11pt,twoside,letterpaper]{book}
\usepackage{longtable}
\usepackage{multirow}
\usepackage{xltxtra}


\usepackage{pdflscape}
\usepackage{etoolbox}
\usepackage{tocloft}

\usepackage[bookmarksnumbered, bookmarksopen=true]{hyperref}
\hypersetup{colorlinks,
  linkcolor=blue,
  linktoc=page}

\usepackage{amsmath}
\usepackage{microtype}
%para codigo fuente
\usepackage{color}
\usepackage{xcolor}
\usepackage{listings}

\usepackage{bookmark}
\bookmarksetup{numbered}

\usepackage{titlesec} %reformatting the chapter headings
\titleformat{\chapter}[block]
{\normalfont\LARGE\bfseries\centering}
{CAPITULO \thechapter: }{0em}{}

\makeatletter
\bookmarksetup{%
  addtohook={%
    \ifnum\toclevel@chapter=\bookmarkget{level}\relax
    \renewcommand*{\numberline}[1]{Capítulo #1: }%
    \fi
  },
}
\makeatother

\renewcommand{\lstlistingname}{Listado}

\lstset{
  basicstyle=\footnotesize\ttfamily,
  language=C++
}

%margenes

%segun guia
\usepackage[top=2cm, bottom=2cm, inner=3cm, outer=2cm]{geometry}

%segun CD de carrera
%\usepackage[top=2.5cm, bottom=2.5cm, inner=3.5cm, outer=2.5cm]{geometry}

% quita la sangria a los parrafos (sin sangria se ve feo :-)
%\setlength{\parindent}{0cm}

% Para tener cabecera y pie de página con un estilo personalizado
\usepackage{fancyhdr}

% Espacio parrafos
\setlength{\parskip}{6pt}
\usepackage{adjustbox}

\usepackage{graphicx}
\usepackage{pstricks}
% Todas las imágenes están en el directorio tp-img:
\newcommand{\imgdir}{img}
\graphicspath{{\imgdir/}}

\usepackage{fontspec}
\usepackage{polyglossia}
\setmainlanguage{spanish}

\setromanfont[Mapping=tex-text]{Linux Libertine O}
\setsansfont[Mapping=tex-text]{DejaVu Sans}
\setmonofont[Mapping=tex-text]{DejaVu Sans Mono}

\title{Mejoramiento del Proceso de deshidratación mediante la construcción de un
  sistema de control automático}
\author{Daniel Rodrigo Saguez Tezanos Pinto}
\date{Diciembre, 2015}

\begin{document}
  %
  % caratula
  %
  %para poner bookmark del CD de requisitos
  \bookmark[page=1,level=-2]{INICIO}

  \pagenumbering{Roman} % para comenzar la numeracion de paginas en numeros romanos

  % caratula --------------------------------------------------------------
  \newcommand{\umsslogo}{%
    \adjustbox{valign=t}{\includegraphics[scale=0.04]{umss}}%
  }
  \newcommand{\fcytlogo}{%
    \adjustbox{valign=t}{\includegraphics[scale=0.1]{fcyt}}%
  }

  % Carátula:
  \makeatletter
  \begin{titlepage}
    \thispagestyle{empty}

    \begin{tabular}[t]{c p{10cm} c}
      \umsslogo &
      \begin{center}
        \large{\textsc{Universidad Mayor de San Simón }} \\
        \large{\textsc{Facultad de Ciencias y Tecnología }} \\
        \large{\textsc{Carrera de Ingeniería en Informática}}
      \end{center}
      &
      \fcytlogo \\
    \end{tabular}
    \vfill

    \begin{center}
      \huge{\textsc{\@title}}
    \end{center}
    \vspace{0.5cm}

    %\begin{flushright}
    \begin{center}
      \textsc{
        Proyecto de grado, presentado para optar\\
        al Diploma Académico de Licenciatura \\
        en Ingeniería en Informática.
      }
      %\end{flushright}
    \end{center}

    \vfill
    \begin{tabbing}
      \hspace{2cm}\=\+
      \textsc{Presentado por:} \@author    \\
      \\
      \textsc{Tutor:} MA Leticia Blanco Coca    \\
      \\
      %\textsc{Cochabamba - Bolivia}\\
      \\
    \end{tabbing}

    \begin{center}
      \textsc{Cochabamba - Bolivia}\\
      \textsc{\@date}
    \end{center}

    \vfill

    %\hrule
    %\vspace{0.2cm}
    %\noindent\small{Trabajo de Grado \hfill}

  \end{titlepage}

  % caratula --------------------------------------------------------------

  %Dedicatoria y agradecimientos
\addcontentsline{toc}{chapter}{Dedicatoria}
\chapter*{}
\begin{flushright}
\textit{Dedicado a \\
    Valery Tamara Saguez Lamas \\
    Cristal Amalia Tezanos Pinto Solares}
    \end{flushright}
\newpage


  \addcontentsline{toc}{chapter}{Agradecimientos}
\chapter*{}
\begin{flushright}
\textit{La realización de este proyecto de grado \\
        blah blah blah blah blah blah blah blah \\
        blah blah blah blah blah blah blah blah \\
        blah blah blah blah blah blah blah blah \\ 
        blah blah blah blah blah blah blah blah \\
        }
    \end{flushright}
\newpage


  % Las páginas empiezan a contar desde aqui
  \setcounter{page}{1}
  \addcontentsline{toc}{chapter}{Ficha resumen}
\chapter*{Ficha resumen}
Se busca mejorar el proceso de deshidratación de alimentos, mediante la
conjunción de un sistema eléctrico/electrónico, sensores, controladores y
microcontroladores; que permitan tener un seguimiento automático del proceso de
deshidratado, con el fin de monitorear las variables referentes al secado de
diferentes tipos de alimentos; variables como temperatura, velocidad de flujo de
aire, peso, humedad del aire, humedad del producto. Y así encontrar el proceso
más adecuado para el alimento a secar.
\cleardoublepage

  \include{005-Indice01-IndiceGeneral}
  \include{006-Indice02-IndiceDeFiguras}
  \include{007-Indice03-IndiceDeTablas}
  \include{008-Introduccion}
  \include{009-Deshidratación}
  \include{010-ControlAutomático}
  \chapter{Herramientas}
\section{Arduino}

Es una plataforma de hardware libre, basada en una placa con un microcontrolador
y un entorno de desarrollo, diseñada para facilitar el uso de la electrónica en
proyectos multidisciplinares.

El hardware consiste en una placa con un microcontrolador Atmel AVR y puertos de
entrada/salida. Los microcontroladores más usados son los Atmega168, Atmega328,
Atmega1280, y Atmega8 por su sencillez y bajo coste, que permiten el desarrollo
de múltiples diseños. Por otro lado el software consiste en un entorno de
desarrollo que implementa el lenguaje de programación «Processing/Wiring» y el
cargador de arranque(bootloader) que es ejecutado en la placa. Se programa en el
ordenador para que la placa controle los componentes electrónicos.

Desde octubre de 2012, Arduino utiliza los microcontroladoras CortexM3 de ARM de
32 bits, que coexistirán con las más limitadas, pero también económicas AVR de 8
bits. ARM y AVR no son plataformas compatibles a nivel binario, pero se pueden
programar con el mismo IDE de Arduino y hacerse programas que compilen sin
cambios en las dos plataformas. Eso sí, los microcontroladores CortexM3 usan
3,3V, a diferencia de la mayoría de las placas con AVR, que generalmente usan
5V. Sin embargo, ya anteriormente se lanzaron placas Arduino con Atmel AVR a
3,3V como la Arduino Fio y existen compatibles de Arduino Nano y Pro como
Meduino en que se puede conmutar el voltaje.

De la placa Arduino, puede tomar información del entorno a través de sus
entradas analógicas y digitales, puede controlar luces, motores y otros
actuadores. El microcontrolador en la placa Arduino se programa mediante el
lenguaje de programación Arduino (basado en Wiring) y el entorno de desarrollo
Arduino (basado en Processing). Los proyectos hechos con Arduino pueden
ejecutarse sin necesidad de conectar a un ordenador.

También cuenta con su propio software que se puede descargar de su página
oficial que ya incluye los drivers de todas las tarjetas disponibles lo que hace
más fácil la carga de códigos desde el computador.

También se puede utilizar para desarrollar objetos interactivos autónomos o
puede ser conectado a software tal como Adobe Flash, Processing, Max/MSP, Pure
Data. Una tendencia tecnológica es utilizar Arduino como tarjeta de adquisición
de datos desarrollando interfaces en software como JAVA, Visual Basic y LabVIEW
6 . Las placas se pueden montar a mano o adquirirse. El entorno de desarrollo
integrado libre se puede descargar gratuitamente.

\subsection{Arduino Mega 2560}
El Arduino Mega 2560 esta basado en el Atmega2560. Cuenta con 54 pines digitales
de entrada / salida (de los cuales 15 se pueden utilizar como salidas PWM), 16
entradas analógicas, 4 UARTs (hardware puertos serie), un oscilador de 16 MHz,
una conexión USB, un conector de alimentación, un conector ICSP(In Cirtuit
Serial Programmer), y un botón de reinicio. Contiene todo lo necesario para
apoyar el microcontrolador; simplemente conectarlo a un ordenador con un cable
USB o el poder con un adaptador de CA o la batería a CC para empezar. El
conector Mega 2560 es compatible con la mayoría de los escudos diseñados para el
Arduino Uno y los antiguos tableros Duemilanove o Diecimila.

El Mega 2560 es una actualización de la Arduino Mega, que sustituye.

\subsubsection{Caracteristicas}
\begin{description} \itemsep0pt \parskip0pt \parsep0pt
  \item[Microcontrolador] Atmega2560
  \item[Voltaje de funcionamiento] 5V
  \item[Voltaje de entrada (recomendado)] 7-12V
  \item[Voltaje de entrada (límite)] 6-20V
  \item[Pines Digitales E/S] 54 (de los cuales 15 proporcionan salida PWM)
  \item[Entradas analógicas] 16
  \item[Corriente DC por cada pin E/S] 20 mA
  \item[Corriente DC de el pin 3.3V] 50 mA
  \item[Memoria Flash] 256 KB de los cuales 8 KB utilizado por el gestor de arranque
  \item[SRAM] 8 KB
  \item[EEPROM] 4 KB
  \item[Frecuencia] 16 MHz
  \item[Longitud] 101,52 mm
  \item[Ancho] 53.3 mm
  \item[Peso] 37g
\end{description}

\subsection{Processing} Es un lenguaje de programación y entorno de desarrollo
integrado de código abierto basado en Java, de fácil utilización, y que sirve
como medio para la enseñanza y producción de proyectos multimedia e interactivos
de diseño digital. Fue iniciado por Ben Fry y Casey Reas a partir de reflexiones
en el Aesthetics and Computation Group del MIT Media Lab dirigido por John
Maeda.

Se distribuye bajo la licencia GNU GPL.

\subsubsection{Alcance}

Al estar basado en Java, puede heredar todas sus funcionalidades, convirtiéndose
en una herramienta poderosa a la hora de encarar proyectos complejos.

%\begin{table}
%\centering
%  \begin{tabular}{r r r r}
%\textbf{Pin} & \textbf{Descripción} & \textbf{Pin} & \textbf{Descripción} \\
%\hline
%1 & \texttt{3.3v} & 2 & \texttt{5v} \\
%3 & \texttt{SDA0*} & 4 & \texttt{5v} \\
%5 & \texttt{SCL0*} & 6 & \texttt{GND} \\
%7 &  \texttt{GPIO\_GCLK} & 8 & \texttt{TXD0*} \\
%9 &  \texttt{GND} & 10 & \texttt{RXD0*} \\
%11 &  \texttt{GPIO\_GEN0} & 12 & \texttt{GPIO\_GEN1} \\
%13 &  \texttt{GPIO\_GEN2} & 14 & \texttt{GND} \\
%15 &  \texttt{GPIO\_GEN3} & 16 & \texttt{GPIO\_GEN4} \\
%17 &  \texttt{3.3v} & 18 & \texttt{GPIO\_GEN5} \\
%19 &  \texttt{SPI\_MOSI*} & 20 & \texttt{GND} \\
%21 &  \texttt{SPI\_MISO*} & 22 & \texttt{GPIO\_GEN6} \\
%23 &  \texttt{SPI\_SCLK*} & 24 & \texttt{SPI\_CEO\_N*} \\
%25 &  \texttt{GND} & 26 & \texttt{SPI\_CE1\_N*} \\
%  \end{tabular}
%  \caption{Descripción de los pines de GPIO}
%  \label{table:gpio_descr}
%\end{table}

\section{Conclusiones}

Se opto por la tecnología de Arduino por la simplicidad de desarrollo y las
capacidades que exceden a los requerimientos actuales. Se escogió el Arduino
Mega 2560, por su bajo coste y la cantidad extendida de pines, tanto analógicos
como digitales

  \chapter{Metodología}

\section{Pila de productos (product backlog)}
%\section{Pila del producto}
En el presente proyecto se utilizará la metodología de desarrollo de software
SCRUM \cite{scrum_book}.

La siguiente tabla es una lista priorizada de requisitos (características) para
ser implementadas en el proyecto. Cada una de estas características se
escribieron de acuerdo a los objetivos específicos, que fueron definidos en el
Capitulo \ref{cap:intro}.

\def\arraystretch{2}
\newcommand{\pbtemp}{Como investigador deseo conocer la temperatura más adecuada para realizar el deshidratado. }


\begin{longtable}{|c|p{12.5cm}|c|} %p{1.8cm}|p{1.9cm}
\hline
\textbf{N°} & \textbf{Características } & \textbf{Prioridad} \\ \hline
\hline

1 & \pbtemp & Alta  \\ \hline
\end{longtable}

\section{Estimación de historias de usuario}
La estimación de esfuerzo de las historias de usuario se realizaran utilizando
la técnica de Planning Poker. Se utilizara el rango de 1, 2, 3, 5, 8, 13, 21, 34.

\section{Ciclo Inicial(Sprint A) }
El ciclo inicial empiezo el día Viernes 26 de Septiembre del 2014. Este ciclo
consta de 2 semanas para completar las historias que serán definidas en la
planificación del sprint. El ciclo concluirá el Jueves 9 de Octubre del 2014.

%\begin{landscape}
\subsection{Pila del Ciclo (Sprint backlog)}
En este ciclo se eligió trabajar en las características 1 y 2 de la pila de
productos, por el hecho de tener alta prioridad. Cada característica se divide
en historias de usuario, y estas son estimadas con puntos de historia. En los
Cuadros \ref{table:eus1} y \ref{table:eus2} se definen las estimaciones. Para
cada historia de usuario se definen criterios de aceptación, como se muestra en
el Cuadro~\ref{table:criteriosSA}.

Cada historia de usuario se divide en tareas, y se definen en los Cuadros
\ref{table:tareasUS1}, \ref{table:tareasUS2} y \ref{table:tareasUS3_1}.

%--------------------------------------------------------------------
\begin{table}[ht]
\centering
\begin{tabular}{|l|p{6cm}|c|p{5cm}|r|}
\hline
\textbf{No.} & \textbf{Feature} & \textbf{Id.} & \textbf{Historia de usuario} & \textbf{Estimación} \\
\hline
1 & \pbtemp & US1 & Controlar Temperatura. & 3 \\
\hline
\end{tabular}
\caption{Estimación de US1}
\label{table:eus1}
\end{table}

%--------------------------------------------------------------------

\subsection{Demostración de fin de sprint}


\subsection{Gráfico burn down del sprint}
La Figura~\ref{fig:sprintA} muestra el gráfico \emph{burn down} del presente
sprint.

\subsection{Retrospectiva del sprint}
\begin{itemize}
  \item ¿Qué salió bien?
    \begin{itemize}
        \item Se completaron la mayoria de las tareas de la iteración.
        \item Se escribieron pruebas de unidad para el código desarrollado.
    \end{itemize}

  \item ¿Qué podría haber sido mejor?
    \begin{itemize}
        \item La estimación de la historia US3 no fue buena.
    \end{itemize}

  \item ¿Qué  se puede mejorar en el futuro?
    \begin{itemize}
        \item Mejorar las estimaciones de las historias, si la historia es muy larga es mejor dividirla.
    \end{itemize}
\end{itemize}





  %\input{018-Conclusiones}

  \begin{thebibliography}{99}
%\bibitem{Libro_ejemplo} Apellido, Nombre.Nombre texto.Editorial.Año.# de pag
\bibitem{burndown} Cohn, Mike . User Stories Applied, for Agile Software Development . Addison-Wesley . 2004 . Página 121

%\bibitem{} . . . . Página

%\bibitem{PHP} PHP Web-Seite: \url{http://www.php.net}

%\bibitem{robotica_wiki} Robótica - Wikipedia: \url{http://es.wikipedia.org/wiki/Robótica}. 25 de Marzo de 2014
\bibitem{scrum_book} Sutherland, Jeff . The Scrum Handbook . Scrum, Inc . 2001 . Página 6
\bibitem{raspberry_pi} Página oficial de Raspberry Pi: \url{http://www.raspberrypi.org/}. 27 de Marzo de 2014
\bibitem{raspberry_pi_wiki} Componentes de Raspberry Pi: \url{https://www.raspberrypi.org/blog/new-graphic/}. 8 de Septiembre de 2015
\end{thebibliography}

  %\input{020-Anexos-Anexos}

  \bookmark[page=13,level=-1]{Índice de cuadros}

\end{document}
