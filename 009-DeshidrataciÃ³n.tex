\chapter{Deshidratación}
\section{Introducción}

El secado o deshidratación, es un método de conservación de alimentos que
consistente en extraer el agua de estos, lo que inhibe la proliferación de
microorganismos y dificulta la putrefacción. El secado de alimentos mediante el
sol y el viento para evitar su deterioro ha sido practicado desde la
antigüedad. El agua suele eliminarse por evaporación (secado al aire, al sol,
ahumado o al viento) pero, en el caso de la liofilización, los alimentos se
congelan en primer lugar y luego se elimina el agua por sublimación.

Las bacterias, levaduras y hongos necesitan agua en el alimento para crecer. El
secado les impide efectivamente sobrevivir en él.


\section{Proceso de Deshidratación de Alimentos}

La desecación de alimentos por paso de aire caliente sobre superficie húmeda,
representa la forma mas común de secado de vegetales destinados a la
alimentación, siendo uno de los procesos mas antiguos de conservación que se
conocen.

Mediante este método es posible preservar el color, los principios activos otras
características de calidad que se pierden por acción de las encimas después de
la cosecha, permitiendo de esta manera, una mejor conservación por tiempos
relativamente largos sin perdidas de estas características.

A través del secado, el agua contenida en el interior de la especie vegetal es
reducida, bajo ciertas condiciones de temperatura y humedad hasta cierto grado,
el cual permite almacenar sin que exista deterioro. En el proceso de secado, el
agua es removida desde el interior y desde los espacios celulares del vegetal,
al principio la perdida de agua es rápida pero a medida que el vegetal se seca
el agua se remueve mas lentamente.

Para un buen almacenamiento el secado debe realizarse hasta que el contenido de
humedad sea del 11\% a 15\% para el follaje y de 10\% al 12\% para las semillas
y/o frutos.

El exceso de humedad contenida por los materiales puede eliminarse por métodos
mecánicos (sedimentación, filtración, centrifugación). Sin embargo, la
eliminación más completa de la humedad se obtiene por evaporación y eliminación
de los vapores formados, es decir, mediante el secado térmico, ya sea empleando
una corriente gaseosa o sin la ayuda del gas para extraer el vapor(Knoule, 1968).

Esta operación se utiliza ampliamente en la tecnología química y es muy común
que sea la última operación en la producción precedente a la salida del producto
resultante(Kasatkin, 1985)(Treybal, 1965). Es evidente que la eliminación de
agua o en general de líquidos existentes en sólidos es más económica por acción
mecánica que por acción térmica. La dificultad de los medios mecánicos surge
cuando los productos finales y gran número de productos intermedios deben
cumplir especificaciones rigurosas en cuanto a la humedad final. Habitualmente
una centrífuga trabajando con grandes cargas de sólidos húmedos, dejará
humedades en torno al 10-20\%, aunque en casos excepcionales como la sal común o
cloruro sódico se puede alcanzar el 1\%. La operación de secado es una operación
de transferencia de masa de contacto gas-sólido, donde la humedad contenida en
el sólido, se transfiere pro evaporación, hacia la fase gaseosa, en base a la
diferencia entre la presión de vapor ejercida por el sólido húmedo y la presión
parcial de vapor de la corriente gaseosa. Cuando estas dos presiones se igualan,
se dice que el sólido y el gas están en equilibrio y el proceso de secado cesa.

%\cite{comp_cv}


\section{Conclusiones}

Se usara el sistema de secado por el paso de aire caliente, por ser el mas común
y con un conjunto de variables que hacen necesario el uso de un sistema
informático para su análisis.
