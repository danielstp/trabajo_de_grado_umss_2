\chapter{Herramientas}
\section{Arduino}

Es una plataforma de hardware libre, basada en una placa con un microcontrolador
y un entorno de desarrollo, diseñada para facilitar el uso de la electrónica en
proyectos multidisciplinares.

El hardware consiste en una placa con un microcontrolador Atmel AVR y puertos de
entrada/salida. Los microcontroladores más usados son los Atmega168, Atmega328,
Atmega1280, y Atmega8 por su sencillez y bajo coste, que permiten el desarrollo
de múltiples diseños. Por otro lado el software consiste en un entorno de
desarrollo que implementa el lenguaje de programación «Processing/Wiring» y el
cargador de arranque(bootloader) que es ejecutado en la placa. Se programa en el
ordenador para que la placa controle los componentes electrónicos.

Desde octubre de 2012, Arduino utiliza los microcontroladoras CortexM3 de ARM de
32 bits, que coexistirán con las más limitadas, pero también económicas AVR de 8
bits. ARM y AVR no son plataformas compatibles a nivel binario, pero se pueden
programar con el mismo IDE de Arduino y hacerse programas que compilen sin
cambios en las dos plataformas. Eso sí, los microcontroladores CortexM3 usan
3,3V, a diferencia de la mayoría de las placas con AVR, que generalmente usan
5V. Sin embargo, ya anteriormente se lanzaron placas Arduino con Atmel AVR a
3,3V como la Arduino Fio y existen compatibles de Arduino Nano y Pro como
Meduino en que se puede conmutar el voltaje.

De la placa Arduino, puede tomar información del entorno a través de sus
entradas analógicas y digitales, puede controlar luces, motores y otros
actuadores. El microcontrolador en la placa Arduino se programa mediante el
lenguaje de programación Arduino (basado en Wiring) y el entorno de desarrollo
Arduino (basado en Processing). Los proyectos hechos con Arduino pueden
ejecutarse sin necesidad de conectar a un ordenador.

También cuenta con su propio software que se puede descargar de su página
oficial que ya incluye los drivers de todas las tarjetas disponibles lo que hace
más fácil la carga de códigos desde el computador.

También se puede utilizar para desarrollar objetos interactivos autónomos o
puede ser conectado a software tal como Adobe Flash, Processing, Max/MSP, Pure
Data. Una tendencia tecnológica es utilizar Arduino como tarjeta de adquisición
de datos desarrollando interfaces en software como JAVA, Visual Basic y LabVIEW
6 . Las placas se pueden montar a mano o adquirirse. El entorno de desarrollo
integrado libre se puede descargar gratuitamente.

\subsection{Arduino Mega 2560}
El Arduino Mega 2560 esta basado en el Atmega2560. Cuenta con 54 pines digitales
de entrada / salida (de los cuales 15 se pueden utilizar como salidas PWM), 16
entradas analógicas, 4 UARTs (hardware puertos serie), un oscilador de 16 MHz,
una conexión USB, un conector de alimentación, un conector ICSP(In Cirtuit
Serial Programmer), y un botón de reinicio. Contiene todo lo necesario para
apoyar el microcontrolador; simplemente conectarlo a un ordenador con un cable
USB o el poder con un adaptador de CA o la batería a CC para empezar. El
conector Mega 2560 es compatible con la mayoría de los escudos diseñados para el
Arduino Uno y los antiguos tableros Duemilanove o Diecimila.

El Mega 2560 es una actualización de la Arduino Mega, que sustituye.

\subsubsection{Caracteristicas}
\begin{description} \itemsep0pt \parskip0pt \parsep0pt
  \item[Microcontrolador] Atmega2560
  \item[Voltaje de funcionamiento] 5V
  \item[Voltaje de entrada (recomendado)] 7-12V
  \item[Voltaje de entrada (límite)] 6-20V
  \item[Pines Digitales E/S] 54 (de los cuales 15 proporcionan salida PWM)
  \item[Entradas analógicas] 16
  \item[Corriente DC por cada pin E/S] 20 mA
  \item[Corriente DC de el pin 3.3V] 50 mA
  \item[Memoria Flash] 256 KB de los cuales 8 KB utilizado por el gestor de arranque
  \item[SRAM] 8 KB
  \item[EEPROM] 4 KB
  \item[Frecuencia] 16 MHz
  \item[Longitud] 101,52 mm
  \item[Ancho] 53.3 mm
  \item[Peso] 37g
\end{description}

\subsection{Processing} Es un lenguaje de programación y entorno de desarrollo
integrado de código abierto basado en Java, de fácil utilización, y que sirve
como medio para la enseñanza y producción de proyectos multimedia e interactivos
de diseño digital. Fue iniciado por Ben Fry y Casey Reas a partir de reflexiones
en el Aesthetics and Computation Group del MIT Media Lab dirigido por John
Maeda.

Se distribuye bajo la licencia GNU GPL.

\subsubsection{Alcance}

Al estar basado en Java, puede heredar todas sus funcionalidades, convirtiéndose
en una herramienta poderosa a la hora de encarar proyectos complejos.

%\begin{table}
%\centering
%  \begin{tabular}{r r r r}
%\textbf{Pin} & \textbf{Descripción} & \textbf{Pin} & \textbf{Descripción} \\
%\hline
%1 & \texttt{3.3v} & 2 & \texttt{5v} \\
%3 & \texttt{SDA0*} & 4 & \texttt{5v} \\
%5 & \texttt{SCL0*} & 6 & \texttt{GND} \\
%7 &  \texttt{GPIO\_GCLK} & 8 & \texttt{TXD0*} \\
%9 &  \texttt{GND} & 10 & \texttt{RXD0*} \\
%11 &  \texttt{GPIO\_GEN0} & 12 & \texttt{GPIO\_GEN1} \\
%13 &  \texttt{GPIO\_GEN2} & 14 & \texttt{GND} \\
%15 &  \texttt{GPIO\_GEN3} & 16 & \texttt{GPIO\_GEN4} \\
%17 &  \texttt{3.3v} & 18 & \texttt{GPIO\_GEN5} \\
%19 &  \texttt{SPI\_MOSI*} & 20 & \texttt{GND} \\
%21 &  \texttt{SPI\_MISO*} & 22 & \texttt{GPIO\_GEN6} \\
%23 &  \texttt{SPI\_SCLK*} & 24 & \texttt{SPI\_CEO\_N*} \\
%25 &  \texttt{GND} & 26 & \texttt{SPI\_CE1\_N*} \\
%  \end{tabular}
%  \caption{Descripción de los pines de GPIO}
%  \label{table:gpio_descr}
%\end{table}

\section{Conclusiones}

Se opto por la tecnología de Arduino por la simplicidad de desarrollo y las
capacidades que exceden a los requerimientos actuales. Se escogió el Arduino
Mega 2560, por su bajo coste y la cantidad extendida de pines, tanto analógicos
como digitales
